\providecommand{\HINCLUDE}{NE}
\if ^\HINCLUDE^
\else
\def\HINCLUDE{}
\global\newdimen\Okraje
\global\Okraje =4cm
\input{$HOME/souteze/_hlavicka/h-.tex}

%\definecolor{colorV}{RGB}{255,127,0}
%\definecolor{colorPoz}{RGB}{153,51,0}
%\definecolor{orangeV}{RGB}{255,127,0}
%\definecolor{colorPr}{RGB}{0,5,255}
%\definecolor{colorDef}{RGB}{0.255,0}

\usepackage[shortlabels]{enumitem}
\setlength{\marginparsep}{2pt}
\setlength{\marginparwidth}{35pt}

\def\st{{\rm st}}
\def\P{{\rm P}}

\def\ISENUM{}
\def\inMargin#1{\End
		
		\hskip0pt \marginpar{{{#1}}}}
\newcounter{V}[section] 
\newcommand{\V}[1][]{\stepcounter{V}\inMargin{\textcolor{green}{V.\arabic{section}.\theV.:}}\ifx^#1^\else\textcolor{green}{\underline{{#1}:}}\addcontentsline{toc}{subsubsection}{V.\arabic{section}.\theV.:$\quad$ {#1}}\\\fi}
\def\Def{\inMargin{\textcolor{red}{Def:}}}
\def\Poz{{\inMargin{\textcolor{brown}{Pozn:}}}}
\def\Pr{{\inMargin{\textcolor{blue}{Př:}}}}
\def\Pozenum
{
	\begin{enumerate}[1)]%, left = 0pt ]
		\item\inMargin{\textcolor{brown}{Pozn:}}\def\ISENUM{a}}
\def\End
{
	\if	^\ISENUM^
	\else \end{enumerate}
	\fi
	\def\ISENUM{}
}
\reversemarginpar

\makeatletter
\renewcommand\thesection{§\arabic{section}.}
\renewcommand\thesubsection{\Alph{subsection})}
\renewcommand\thesubsubsection{\alph{subsubsection})}
\newcounter{chapter}
\setcounter{chapter}{0}
\renewcommand\thechapter{\Alph{chapter})}
\newcounter{roman}
\setcounter{roman}{0}
\renewcommand\theroman{\Roman{roman}.}
\makeatother
\def\sectionnum#1{\setcounter{section}{#1}\addtocounter{section}{-1}}
\def\subsectionnum#1{\setcounter{subsection}{#1}\addtocounter{subsection}{-1}}
\def\subsubsectionnum#1{\setcounter{subsubsection}{#1}\addtocounter{subsubsection}{-1}}
\def\chapternum#1{\setcounter{chapter}{#1}\addtocounter{chapter}{-1}}
\def\chapter#1{

	\addtocounter{chapter}{1}\sectionnum{1}
	\addcontentsline{toc}{section}{\large{\thechapter$\quad${#1}}}
	
	{\LARGE  \textbf{\begin{minipage}[t]{0.1\textwidth}\thechapter\end{minipage}\begin{minipage}[t]{0.95\textwidth}#1\end{minipage}}}

}
\def\ROM{}
\def\Rom#1#2{\setcounter{roman}{#1}\renewcommand\ROM{#2}}

\Rom{6}{Funkce}
\title{\Huge\textbf{\theroman\quad \ROM}}
\author{Jiří Kalvoda}

\newcounter{countOfBegin}
\setcounter{countOfBegin}{0}
\newcommand{\BeginDoc}[1][]
{
	\ifnum\value{countOfBegin}=0
	\begin{document}
		#1
		\fi
	\addtocounter{countOfBegin}{1}
		
}
\def\EndDoc
{
	\addtocounter{countOfBegin}{-1}
	\ifnum\value{countOfBegin}=0
	\end{document}
	\fi
}

\fi
\BeginDoc{}
\section{Kružnice}
\V Je li v rovině dána kartézká soustava souřadnic, pak platí:
\begin{itemize}
	\item Každou kružnici se středem $S[m,n]$ a poloměrem $r>0$ lze analiticky vyjádřit právě jednou rovnicí $(x-m)^2 + (y-n)^2 = r^2$.
		Každá rovnice $(x-m)^2 + (y-n)^2 = r^2$, kde $r>0$, analyticky vyjadřuje právě jednu kružnici se středem $S[m,n]$ a poloměrem $r$.
\end{itemize}
\Def
Rovnice $(x-m)^2 + (y-n)^2 = r^2$, kde $r>0$ se nazývá \emph{středový tvar rovnice kružnice}.
\Poz
Jestliže rozepíšeme mocnny dvojčlenů ve středovém tvaru rovnice kružnice a získané členy uspořádáme sestupně, dostaneme 
$x^2+y^2-2mx-2ny+m^2+n^2 - r^2 = 0$, což je rovnice typu $x^2+y^2+ax+by + c =0$

\Def
Pokud rovnice $x^2+y^2+ax+bx+c = 0$ vyjadřuje některou kružnici $k$, nazývá se \emph{obecný tvar rovnice kružnice} $k$.
\Pr Rozhodněte, da rovnice $x^2+y^2 + 4x - 6y  14 = 0$ vyjadřuje kružnici.\\
\begin{eqnarray*}
	(x^2+4x)+(y^2-6y) + 14 &=& 0 \\
	(x+2)^2-4 + (y-3)^2-9 + 14 &=& 0 \\
	(x+2)^2 + (y-3)^2 &=& -1 \\
\end{eqnarray*}
Rovnce neodpovídá kružnici.

\Pr Rozhodněte, da rovnice $x^2+y^2 + ax + by + c = 0$ vyjadřuje kružnici.\\
\begin{eqnarray*}
	\(x^2+\f a2\)-\f{a^2}4 + (y^2+\f b 2) -\f{b^2}4 + c &=& 0 \\
	(x+\f a 2)^2 + (y-\f b 2)^2 &=& \f{a^2}4+\f{b^2}4 -c \\
\end{eqnarray*}
Musí tedy platiti $\f{a^2}4+\f{b^2}4 -c > 0$.

\Pr
Určete rovnice všech kružnic, které prochází body $A[-1;3];B[0;2];C[-1;-1]$:

\begin{eqnarray}
	1+9-a+3b+c &=& 0 \\
	4+2b+c &=& 0 \\
	1+1-a-b+c &=& 0 \\
\end{eqnarray}
Toto upravím na:
\begin{eqnarray}
	a-3b-c &=& 10\\
	2b+c &=& -4 \\
	a+b-c &=& 2 \\
\end{eqnarray}
Soustava jediné má řešení $[4;-2;0]$, které odpovídá rovnici $x^2 + y^2 + 4x -2y =0$
Po úpravé: $(x+2)^2 v+ (y-1)^2 = 5$

Jedná se o kružnici $k(S[-2;1],r=\sqrt 5)$.

\Pr 210/5:
\begin{itemize}

\item $(x-2)^2+y^2 = 5+4$ Střed $[2;0]$, poloměr $3$.
\item $(x+1)^2+(y-1)^2 = 7+1+1$ Kruh: Střed $[-1;1]$, poloměr $3$.
\item $(x+\f 5 2)^2+(y-\f 7 2)^2 = 2.5+\f{25}4+\f{49}4$ Střed $[-\f 52;\f 72]$, poloměr $\sqrt{21}$.
\item $(x+\f 5 2)^2+(y-\f 3 2)^2 = \f{83}2+\f{25}4+\f 9 4$ Kruh: Střed $[-\f 5 2;\f 3 2]$, poloměr $\sqrt{50}$.
\end{itemize}
\Pr 210/6:
\begin{eqnarray}
9+3a+c &=& 0 \\
4+2a+4-2b+c &=& 0 \\
36+6a+36+6b+c &=& 0 \\
\end{eqnarray}
\begin{eqnarray}
3a+c &=& -9 \\
2a-2b+c &=& -8 \\
6a+6b+c &=& -72 \\
\end{eqnarray}
 $ \bar{A} = \begin{pmatrix}
3 &0 &1 &\vline& -9 \\
2 &-2 &1 &\vline& -8 \\
6 &6 &1 &\vline& -72 \\
\end{pmatrix}
\sim
\begin{pmatrix}
2 &-2 &1 &\vline& -8 \\
3 &0 &1 &\vline& -9 \\
6 &6 &1 &\vline& -72 \\
\end{pmatrix}
\sim
\begin{pmatrix}
2 &-2 &1 &\vline& -8 \\
0 &6 &-1 &\vline& 6 \\
0 &12 &-2 &\vline& -48 \\
\end{pmatrix}
\sim
\begin{pmatrix}
2 &-2 &1 &\vline& -8 \\
0 &6 &-1 &\vline& 6 \\
0 &6 &-1 &\vline& -24 \\
\end{pmatrix}
\sim
\begin{pmatrix}
2 &-2 &1 &\vline& -8 \\
0 &6 &-1 &\vline& 6 \\
0 &0 &0 &\vline& -30 \\
\end{pmatrix}
\sim
\begin{pmatrix}
2 &-2 &1 &\vline& -8 \\
0 &6 &-1 &\vline& 6 \\
0 &0 &0 &\vline& 1 \\
\end{pmatrix}
 $
 Zádně řešení $\imp$ kružnice neexistuje, protože body jsou kolineárí.

 \Pr Určete rovnice vešech kružnic, které prochází bodem $A[1;2]$, dotýká se osy $y$ a mají střed na přímce $p$, která má rovnici $y+x = 4$:

 Hledám $(x-m)^2 + (y-n)^2 = r^2$\\
$k$ se dotýká $y$ $\imp$ $m^2 = r^2$\\
$S \in P \imp m + n  = 4$\\
$A \in k \imp (1-m)^2 + (2-n)^2 = r^2$

$m+n = 4$\\
$(11-m)^2 + (2-n)^2 = m^2 $\\

Řešení:
$(x-1)^2 + (y-3)^2 = 1$
a
$(x-5)^2 + (y+1)^2 = 25$

\Pr 210/7:
Hledám $(x-m)^2 + (y-n)^2 = r^2$:

Dotýká se $x$ $\imp$ $r^2 = m^2$.\\
Dotýká se $x$ $\imp$ $r^2 = n^2$.\\
$K \in k \imp (9-n)^2 + (2-m)^2 = r^2 $ 

Když $m=n=\pm r$
$K \in k \imp (9-n)^2 + (2-n)^2 = n^2 \imp x^2-22x + 85 =0$ 

$m=n=r=5 \lor m=n=r = 17$

$$ (x+5)^2 + (y+5) = 5^2 $$
$$ (x+17)^2 + (y+17) = 5^2 $$

Když $m=-n=\pm r$
$K \in k \imp (9+n)^2 + (2-n)^2 = n^2 \imp x^2-15x + 85 =0 \imp D=255-4\*85 = -115 < 0$ 

\Pr 210/8:
Hledám $(x-m)^2 + (y-n)^2 = r^2$:

$m+3n-6 = 0 \imp m = 6-3n$\\
$r=5$\\
$M[6;9] \in k \imp (6-m)^2 + (9-n)^2 = 25 \imp (6-6+3n)^2 + (9-n)^2 = 25 \imp 9n^2 -18n+56 = 0 \imp D = 18^2 - 4 \* 9 \* 56 = -1692 < 0$.

Neexistuje řešení.

\Pr 210/9/a:

Osa přímek, na které musí náležet střed je buď $x=0$ nebo $y=0$:

Jelikož $p\perp q$, průsečík přímek, body doteku a střed tvoří čtverec o straně $r$, tedy $|[0;0]S| = 2\sqrt 2$:

Řešením tedy jsou:
$$ (x-2\sqrt 2)^2 + y^2 = 2 $$
$$ x^2 + (y-2\sqrt 2)^2 = 2 $$
$$ (x+2\sqrt 2)^2 + y^2 = 2 $$
$$ x^2 + (y+2\sqrt 2)^2 = 2 $$

\Pr 210/9/a:

$(m-4)^2 = 4  \imp m = 2 \lor m = 6$\\
$S \in r \parallel p$
$\rho(r,q) = 2 \imp r:x-y+2\pm 2\sqrt 2=0  \imp n = 2\pm 2\sqrt 2  +m $

$$ (x-2)^2 + (y-4-2 \sqrt 2)^2 = 2 $$
$$ (x-2)^2 + (y-4+2 \sqrt 2)^2 = 2 $$
$$ (x-2)^2 + (y-8-2 \sqrt 2)^2 = 2 $$
$$ (x-2)^2 + (y-8+2 \sqrt 2)^2 = 2 $$

\EndDoc
