\providecommand{\HINCLUDE}{NE}
\if ^\HINCLUDE^
\else
\def\HINCLUDE{}
\global\newdimen\Okraje
\global\Okraje =4cm
\input{$HOME/souteze/_hlavicka/h-.tex}

%\definecolor{colorV}{RGB}{255,127,0}
%\definecolor{colorPoz}{RGB}{153,51,0}
%\definecolor{orangeV}{RGB}{255,127,0}
%\definecolor{colorPr}{RGB}{0,5,255}
%\definecolor{colorDef}{RGB}{0.255,0}

\usepackage[shortlabels]{enumitem}
\setlength{\marginparsep}{2pt}
\setlength{\marginparwidth}{35pt}

\def\st{{\rm st}}
\def\P{{\rm P}}

\def\ISENUM{}
\def\inMargin#1{\End
		
		\hskip0pt \marginpar{{{#1}}}}
\newcounter{V}[section] 
\newcommand{\V}[1][]{\stepcounter{V}\inMargin{\textcolor{green}{V.\arabic{section}.\theV.:}}\ifx^#1^\else\textcolor{green}{\underline{{#1}:}}\addcontentsline{toc}{subsubsection}{V.\arabic{section}.\theV.:$\quad$ {#1}}\\\fi}
\def\Def{\inMargin{\textcolor{red}{Def:}}}
\def\Poz{{\inMargin{\textcolor{brown}{Pozn:}}}}
\def\Pr{{\inMargin{\textcolor{blue}{Př:}}}}
\def\Pozenum
{
	\begin{enumerate}[1)]%, left = 0pt ]
		\item\inMargin{\textcolor{brown}{Pozn:}}\def\ISENUM{a}}
\def\End
{
	\if	^\ISENUM^
	\else \end{enumerate}
	\fi
	\def\ISENUM{}
}
\reversemarginpar

\makeatletter
\renewcommand\thesection{§\arabic{section}.}
\renewcommand\thesubsection{\Alph{subsection})}
\renewcommand\thesubsubsection{\alph{subsubsection})}
\newcounter{chapter}
\setcounter{chapter}{0}
\renewcommand\thechapter{\Alph{chapter})}
\newcounter{roman}
\setcounter{roman}{0}
\renewcommand\theroman{\Roman{roman}.}
\makeatother
\def\sectionnum#1{\setcounter{section}{#1}\addtocounter{section}{-1}}
\def\subsectionnum#1{\setcounter{subsection}{#1}\addtocounter{subsection}{-1}}
\def\subsubsectionnum#1{\setcounter{subsubsection}{#1}\addtocounter{subsubsection}{-1}}
\def\chapternum#1{\setcounter{chapter}{#1}\addtocounter{chapter}{-1}}
\def\chapter#1{

	\addtocounter{chapter}{1}\sectionnum{1}
	\addcontentsline{toc}{section}{\large{\thechapter$\quad${#1}}}
	
	{\LARGE  \textbf{\begin{minipage}[t]{0.1\textwidth}\thechapter\end{minipage}\begin{minipage}[t]{0.95\textwidth}#1\end{minipage}}}

}
\def\ROM{}
\def\Rom#1#2{\setcounter{roman}{#1}\renewcommand\ROM{#2}}

\Rom{6}{Funkce}
\title{\Huge\textbf{\theroman\quad \ROM}}
\author{Jiří Kalvoda}

\newcounter{countOfBegin}
\setcounter{countOfBegin}{0}
\newcommand{\BeginDoc}[1][]
{
	\ifnum\value{countOfBegin}=0
	\begin{document}
		#1
		\fi
	\addtocounter{countOfBegin}{1}
		
}
\def\EndDoc
{
	\addtocounter{countOfBegin}{-1}
	\ifnum\value{countOfBegin}=0
	\end{document}
	\fi
}

\fi
\BeginDoc{}
\section{Elipsa}
\Poz
Termín \emph{elipsa} už známe, máme představu o eliptickém tvaru např vodní hladiny v šikmo postavené válcové nádobě.
V geometrii se elipsa definuje pomocí součtu vzdáleností.
\Def Nechť jsou dány dva různé body $F,G$ v rovině a číslo $2a > |FG|$.
Množinu všech bodů $X$ roviny, pro ktterá platí $|FX| + |GX| = 2a$ nazýváme \emph{elipa s ohnisky $F,G$ a s hlavní osou o velikosti $2a$.}
Stručně ji značíme $E(F,G,2a)$

\Def
U elipsy používáme tyto pojmy:\\
\begin{minipage}{0.5\textwidth}
	\begin{tabular}{|c|l|}\hline
		$F,G$ & ohniska \\\hline
		$A_1,A_2$ & hlavní vrcholy \\\hline
		$B_1,B_2$ & vedlejší vrcholy \\\hline
		$S$ & střed \\\hline
		$a$ & velikost hlavní poloosy \\\hline
		$b$ & velikost vedejší poloosy \\\hline
		$e$ & excentricita \\\hline
	\end{tabular}
\end{minipage}
\begin{minipage}{0.5\textwidth}
	\pdf{10-1.pdf}
\end{minipage}

\Pr Odvoďte analitické vyjádření elipsy $E(F,G,2a)$:

Zvolíme souřadnice $F[-e,0];G[e,0];X[x,y]$.
Přitom $|FG| = 2e < 2a; a^2 - e^2 = b^2$.
Analitické vyjádříme $|FX| + |GX| = 2a$:
\begin{eqnarray*}
	\sqrt{(x+e)^2+y^2} +\sqrt{(x-e)^2+y^2} &=& 2a\\
	{(x+e)^2+y^2} +{(x-e)^2+y^2} + 2 \sqrt{(x^2+e^2 + y^2 +2ex)(x^2+e^2 + y^2 -2ex)} &=& 4a^2\\
\end{eqnarray*}\\[-53px]
\begin{eqnarray*}
	(x^2 + e^2 + y^2)^2 - 4 e^2 x^2 &=& 4a^2 + (x^2 + e^2 + y^2)^2 - 4a^2 (x^2 + e^2 + y^2)\\
	x^2(a^2-e^2) + a^2 y^2 &=& a^2 (a^2 - e^2)\\
	\f{x^2}{a^2} + \f{y^2}{b^2} = 1 
\end{eqnarray*}

\Poz
Rovnice $\f{x^2}9+\f{y^2}4 ; \f{x^2}4 + \f{y^2}9 = 1$. vyjadřují elipsy s velikostmi polooos rovnými 2, resp. 3.
Elipsy se liší polohou ohnisek na osách $x,y$.

Ohniska leží na té ose souřadnic, kde je poloosa s větší velikostí.
\pdf{10-2.pdf}

\Poz
Když je jmenovatel prvníhgho zlomku menší než druhého, tak si $x,y$ prohodí role.
\Poz
Proto někdy raději značíme jmenovatele $p,q$, aby neinformovali o tom, která je hlavní poloosa.
\V[Analytické vyjádření elipsy]
Každá elipsa, která má osy rovnoběžné s osami $x,y$ a střed $S[m,n]$ má právě jednu rovnici typu
$$ \f{(x-m)^2}{p^2} + \f{(y-n)^2}{q^2} = 1$$
kde $p,q>0$.

\V Každá rovnice tohoto typu vyjadřuje právě jednu elipsu se středem $S[m,n]$.
Je-li $p>q$, je $2a = 2p$ a hlavní osa elipsy leží na $y=n$.
Je-li $p<q$, je $2a = 2q$ a hlavní osa elipsy leží na $x=m$.
Je-li $p=q$, je elipsa kružnicí s poloměrem $r=p=q$.

\Pr 249/3:
Zakreslete střed, vrcholy a ohniska elipsy dané rovnicí
$$ 5(x+2)^2 + 3(y-4)^2 - 30 =0 $$

Z rovnce určíme $S[-2;-4]$. Dále upravím na $\f{(x+2)}6 + \f{(y-4)^2}{10} = 1$.
Tedy $q^2 = a^2 = 10$ a $p^2 = b^2 = 6$. Hlavní osa s ohnisky leží na přímce rovnoběžné s osou $y$. Dále určíme excentricitu $e = \sqrt{a^2 - b^2} = 2$
Tedy:
$$
\begin{array}{r@{\ [}c@{;}c@{]}}
	A_1 & -2 & 4-\sqrt{10}\\
	A_2 & -2 & 4+\sqrt{10}\\
	B_1 & -5-\sqrt 6 & 4 \\
	B_1 & -5+\sqrt 6 & 4 \\
	F&-5&6\\
	G&-2&2\\
\end{array}
$$
\Pr 250/4:

Určete společné body elipsy a přímky $KL$, kde $K[3;-1]$ a $L[1;6]$.
Elipsa má rovnici $2(x+4)^2 + 3(y+1)^2 = 10$.

$\pri{KL} = \zs{[3-2t;-1+7t]|t\in\R}$
Dosadím:
$$2(7-2t)^2 + 3 (7t)^2 = 10$$
$$155t^2 - 56t+88 = 0$$
$D = 56^2 - 4\* 155 \* 88 < 0 \imp $ Není řešení, tedy není průsečík.

\Pr 250/18:
\begin{enumerate}[a)]
	\item Ze symetrie dle osy platí $|FB_1| = |GB_1|$.
		Ovšem jelikož $B_1$ leží na elipse, tak $|FB_1| + |GB_1| = 2|FB_1| =2a \imp a = |FB_1| = |GB_1|$. \emph{QED}

		Pro $B_2$ analogicky (nebo dle symetrie dle hlavní osy). \emph{QED}

	\item Z kolmosti os: $b^2 + e^2 = |B_1F|^2 = a^2$. \emph{QED}

		Upravíme na $e^2= a^2 - b^2$. \emph{QED}

		Jelikož $SA_1$ a $SA_2$ jsou hlavní poloosy, tak $|SA_1|=a=|SA_2|$. \emph{QED}

		Jelikož $A_1SA_2$ jsou kolineární v tomto pořadí, tak $|A_1A_2| = |A_1S| + |A_2S| = 2a$. \emph{QED}

\end{enumerate}
\Pr 251/19:
\begin{enumerate}[a)]
	\item
		Evidentně $S[0,0]$.
		Hlavní poloosa ve směru osy $y$ délky $a = 4$, vedlejší $b = 2$, tedy $e = \sqrt{16 - 4} = \sqrt{12}  =  2 \sqrt 3$.
$$
\begin{array}{r@{\ [}c@{;}c@{]}}
	A_1 &  0 & 4\\
	A_2 &  0 & -4\\
	B_1 &  2 & 0 \\
	B_1 & -2 & 0 \\
	F   &  0 & 2\sqrt 3\\
	G   &  0 & -2\sqrt 3\\
\end{array}
$$
		\pdf{10-3.pdf}
\item [b,c,d,e)] Analogicky. Střed je vždy stejný a pouze se mění směr a velikost poloos.
	V některých bodech je potřeba rovnici vydělit číslem na pravé straně.
\end{enumerate}

\Pr 251/20:
\begin{enumerate}[a)]
	\item
		Evidentně $S[-3,4]$.
		Hlavní poloosa ve směru osy $x$ délky $a = \sqrt{20}$, vedlejší $b = 2$, tedy $e = \sqrt{20 - 4} = \sqrt{16}  =  4$.
$$
\begin{array}{r@{\ [}c@{;}c@{]}}
	A_1 &  -3+\sqrt{20} & 4\\
	A_2 &  -3-\sqrt{20} & 4\\
	B_1 &  -3 & 6 \\
	B_1 &  -3 & 2 \\
	F   &  1 & 4\\
	G   &  -7 & 4\\
\end{array}
$$
		\pdf{10-4.pdf}
\item [b,c,d)] Analogicky. Pouze se mění střed, směr a velikost poloos. 
	V některých bodech je potřeba rovnici vydělit číslem na pravé straně.
\end{enumerate}

\Pr 251/21:
\begin{enumerate}[a)]
	\item
		$\ve{AB} = (2;-5) \imp \pri{AB} = \zs{[3+2t;-5t]|t\in\R}$\\
		Dosadím:  $\f{(3+2t)^2}4+\f{(-5t)^2}{16} = 1 \imp 21t^2 + 48 t + 36 = 0 \imp t = \f{-24 \pm 2\sqrt{39}}{21}$\\
		$\[\f{15-4\sqrt{39}}{21};\f{120+10\sqrt{39}}{21}\]$\\
		$\[\f{15+4\sqrt{39}}{21};\f{120-10\sqrt{39}}{21}\]$

		$\ve{AC} = (6;-3) \imp \ppri{AC}= \zs{[1+2t;5-t]|t\in\R^+_0}$\\
		Dosadím:  $\f{(1+2t)^2}4+\f{(5-t)^2}{16} = 1 \imp 17t^2 + 6t + 13 = 0 \imp D 36 - 4\*17\* 13 < 0$\\
		Průsečík není.

		$\ve{BC} = (4;2) \imp {BC} = \zs{[3+2t;t]|t\in\<0;2\>}$
		Dosadím:  $\f{(1+2t)^2}4+\f{(5-t)^2}{16} = 1 \imp 17t^2 + 48t + 20 = 0 \imp t = \f{-24\pm2\sqrt{59}}{17} < 0$\\
		Průsečík je pouze s přímkou, nikoliv úsečkou.


\item [b,c,d,e)] Analogicky. Vždyť je to jenom dosazení toho samého do jiné rovnice a výpočet kvadratické rovnice. 
	Já nemám zájem celý den dosazovat a počítat kvadratické rovnice.
\end{enumerate}


\EndDoc
