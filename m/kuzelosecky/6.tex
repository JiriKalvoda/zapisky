\providecommand{\HINCLUDE}{NE}
\if ^\HINCLUDE^
\else
\def\HINCLUDE{}
\global\newdimen\Okraje
\global\Okraje =4cm
\input{$HOME/souteze/_hlavicka/h-.tex}

%\definecolor{colorV}{RGB}{255,127,0}
%\definecolor{colorPoz}{RGB}{153,51,0}
%\definecolor{orangeV}{RGB}{255,127,0}
%\definecolor{colorPr}{RGB}{0,5,255}
%\definecolor{colorDef}{RGB}{0.255,0}

\usepackage[shortlabels]{enumitem}
\setlength{\marginparsep}{2pt}
\setlength{\marginparwidth}{35pt}

\def\st{{\rm st}}
\def\P{{\rm P}}

\def\ISENUM{}
\def\inMargin#1{\End
		
		\hskip0pt \marginpar{{{#1}}}}
\newcounter{V}[section] 
\newcommand{\V}[1][]{\stepcounter{V}\inMargin{\textcolor{green}{V.\arabic{section}.\theV.:}}\ifx^#1^\else\textcolor{green}{\underline{{#1}:}}\addcontentsline{toc}{subsubsection}{V.\arabic{section}.\theV.:$\quad$ {#1}}\\\fi}
\def\Def{\inMargin{\textcolor{red}{Def:}}}
\def\Poz{{\inMargin{\textcolor{brown}{Pozn:}}}}
\def\Pr{{\inMargin{\textcolor{blue}{Př:}}}}
\def\Pozenum
{
	\begin{enumerate}[1)]%, left = 0pt ]
		\item\inMargin{\textcolor{brown}{Pozn:}}\def\ISENUM{a}}
\def\End
{
	\if	^\ISENUM^
	\else \end{enumerate}
	\fi
	\def\ISENUM{}
}
\reversemarginpar

\makeatletter
\renewcommand\thesection{§\arabic{section}.}
\renewcommand\thesubsection{\Alph{subsection})}
\renewcommand\thesubsubsection{\alph{subsubsection})}
\newcounter{chapter}
\setcounter{chapter}{0}
\renewcommand\thechapter{\Alph{chapter})}
\newcounter{roman}
\setcounter{roman}{0}
\renewcommand\theroman{\Roman{roman}.}
\makeatother
\def\sectionnum#1{\setcounter{section}{#1}\addtocounter{section}{-1}}
\def\subsectionnum#1{\setcounter{subsection}{#1}\addtocounter{subsection}{-1}}
\def\subsubsectionnum#1{\setcounter{subsubsection}{#1}\addtocounter{subsubsection}{-1}}
\def\chapternum#1{\setcounter{chapter}{#1}\addtocounter{chapter}{-1}}
\def\chapter#1{

	\addtocounter{chapter}{1}\sectionnum{1}
	\addcontentsline{toc}{section}{\large{\thechapter$\quad${#1}}}
	
	{\LARGE  \textbf{\begin{minipage}[t]{0.1\textwidth}\thechapter\end{minipage}\begin{minipage}[t]{0.95\textwidth}#1\end{minipage}}}

}
\def\ROM{}
\def\Rom#1#2{\setcounter{roman}{#1}\renewcommand\ROM{#2}}

\Rom{6}{Funkce}
\title{\Huge\textbf{\theroman\quad \ROM}}
\author{Jiří Kalvoda}

\newcounter{countOfBegin}
\setcounter{countOfBegin}{0}
\newcommand{\BeginDoc}[1][]
{
	\ifnum\value{countOfBegin}=0
	\begin{document}
		#1
		\fi
	\addtocounter{countOfBegin}{1}
		
}
\def\EndDoc
{
	\addtocounter{countOfBegin}{-1}
	\ifnum\value{countOfBegin}=0
	\end{document}
	\fi
}

\fi
\BeginDoc{}
\section{Vyšetřování množin bodů metodou souřadnic}
\Pr
Vyšetřetřete množinu bodů $X$ roviny, pro které platí $\rho(X,p) \ge 3 \* \rho(X,q)$, kde $p,q$ jsou dvě kolmé přímky  v  rovině.

Zvolme ortonormální soustavu souřadnic tak, aby $p$ byla osou $x$ a $q$ osou $y$.

$p: y = 0, X[x,y]$\\
$q: x = 0, |y|| \ge|X|$\\

Útvar $U$, který má toto analytické vyjádření, sestrojíme na základě znalostí, které máme o grafech funkcí.
Diskuzí o hodnotách proměnné $y$ získám soustavu nerovnic:\\
$y\ge 0 \land y \ge 3 |x|$ nebo $y\le 0 \land -y\ge 3 |x|$\\
Část roviny \uv{nad grafem} funkce $y=3|x|$ a 
Část roviny \uv{pod grafem} funkce $y=-3|x|$ 

Útvar $U$ je sjednocením dvou vrcholových úhlů, jejíž osy leží na $q$.
Pro $\alpha$ platí $\tg \f 1 2 \alpha = \f 1 3$, tj $\alpha \doteq 36\d$

\Pr Vyšetřete množinu bodů $M$ všech bodů $X$ roviny, pro které platí $|AX| \ge 2 |BX|$.
Body $A,B$ jsou dva různé body roviny.

Nechť $A[-3;0];B[3;0];X[x,y]\in M$:

\begin{eqnarray*}
	|AX| \ge 2 |BX| \\
	\sqrt{(x+3)^ + y^2} \ge 2 \sqrt{(x-3)^2+y^2} \\
	(x+3)^2 + y^2 \ge 4 (x-3)^2+4y^2 \\
	-9 \ge x^2 - 10 x + y^2 \\
	4^2 = 16 \ge (x-5)^2 + y^2 \\
\end{eqnarray*}

Každý bod, který splňuje první nerovnici splňuje i tu poslední vyjadřující $K([5;0],4)$.

Obrácením postupu úprav prokážeme, že každý bod tohoto kruhu $K$ má charakteristickou vlastnost vyšetřované množiny $M$.

\Pr
227/32:A\\
\begin{enumerate}
	\item
Nechť $A[-1,0];B[1,0];X[x,y]$:
		$(x+1)^2 + y^2 + (x-1)^2 + y^2 = 2^2$\\
		$2y^2 + 2x^2 - 2x + 2x + 2 = 4$\\
		$2y^2 + 2x^2  = 2$\\
		$y^2 + x^2  = 2$\\

		$k([0;0],1)$
	\item

Nechť $A[-1,0];B[1,0];X[x,y]$:
		$\sqrt{(x+1)^2 + y^2} = 2 \sqrt{(x-1)^2 + y^2}$\\
		$(x+1)^2 + y^2 = 4 (x-1)^2 + 4y^2$\\
		$x^2+2x+1 + y^2 = 4x^2-8x+4 + 4y^2$\\
		$0 = 3x^2-10x+3 + 3y^2$\\
		$0 = x^2-\f{10}3 x+1 + y^2$\\
		$0 = \(x-\f{5}3\)^2 -\f{25}{9}+1 + y^2$\\
		$\(\f 43\)^2 = \f{16}9 = \(x-\f{5}3\)^2 + y^2$\\

		$k([\f 5 3;0],\f 4 3)$
	\item
Nechť $A[-1,0];B[1,0];X[x,y]$:
		$(x+1)^2 + y^2 + (x-1)^2 + y^2 = 4\*2^2$\\
		$2y^2 + 2x^2 - 2x + 2x + 2 = 16$\\
		$2y^2 + 2x^2  = 14$\\
		$y^2 + x^2  = 7$\\

		$k([0;0],\sqrt 7)$
Nechť $A[-1,0];B[1,0];X[x,y]$:
		$(x+1)^2 + y^2 + 2((x-1)^2 + y^2) = 3\*2^2$\\
		$3y^2 + 3x^2 -2x+3 = 3\*4$\\
		$y^2 + x^2 -\f 2 3x = 3$\\
		$y^2 + \(x -\f 1 3\)^2  = 3+\f 1 9 = \f{28}9$

		$k\(\[\f 1 3;0\],\f{28}9\)$
\end{enumerate}
\EndDoc
