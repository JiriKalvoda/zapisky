\providecommand{\HINCLUDE}{NE}
\if ^\HINCLUDE^
\else
\def\HINCLUDE{}
\global\newdimen\Okraje
\global\Okraje =4cm
\input{$HOME/souteze/_hlavicka/h-.tex}

%\definecolor{colorV}{RGB}{255,127,0}
%\definecolor{colorPoz}{RGB}{153,51,0}
%\definecolor{orangeV}{RGB}{255,127,0}
%\definecolor{colorPr}{RGB}{0,5,255}
%\definecolor{colorDef}{RGB}{0.255,0}

\usepackage[shortlabels]{enumitem}
\setlength{\marginparsep}{2pt}
\setlength{\marginparwidth}{35pt}

\def\st{{\rm st}}
\def\P{{\rm P}}

\def\ISENUM{}
\def\inMargin#1{\End
		
		\hskip0pt \marginpar{{{#1}}}}
\newcounter{V}[section] 
\newcommand{\V}[1][]{\stepcounter{V}\inMargin{\textcolor{green}{V.\arabic{section}.\theV.:}}\ifx^#1^\else\textcolor{green}{\underline{{#1}:}}\addcontentsline{toc}{subsubsection}{V.\arabic{section}.\theV.:$\quad$ {#1}}\\\fi}
\def\Def{\inMargin{\textcolor{red}{Def:}}}
\def\Poz{{\inMargin{\textcolor{brown}{Pozn:}}}}
\def\Pr{{\inMargin{\textcolor{blue}{Př:}}}}
\def\Pozenum
{
	\begin{enumerate}[1)]%, left = 0pt ]
		\item\inMargin{\textcolor{brown}{Pozn:}}\def\ISENUM{a}}
\def\End
{
	\if	^\ISENUM^
	\else \end{enumerate}
	\fi
	\def\ISENUM{}
}
\reversemarginpar

\makeatletter
\renewcommand\thesection{§\arabic{section}.}
\renewcommand\thesubsection{\Alph{subsection})}
\renewcommand\thesubsubsection{\alph{subsubsection})}
\newcounter{chapter}
\setcounter{chapter}{0}
\renewcommand\thechapter{\Alph{chapter})}
\newcounter{roman}
\setcounter{roman}{0}
\renewcommand\theroman{\Roman{roman}.}
\makeatother
\def\sectionnum#1{\setcounter{section}{#1}\addtocounter{section}{-1}}
\def\subsectionnum#1{\setcounter{subsection}{#1}\addtocounter{subsection}{-1}}
\def\subsubsectionnum#1{\setcounter{subsubsection}{#1}\addtocounter{subsubsection}{-1}}
\def\chapternum#1{\setcounter{chapter}{#1}\addtocounter{chapter}{-1}}
\def\chapter#1{

	\addtocounter{chapter}{1}\sectionnum{1}
	\addcontentsline{toc}{section}{\large{\thechapter$\quad${#1}}}
	
	{\LARGE  \textbf{\begin{minipage}[t]{0.1\textwidth}\thechapter\end{minipage}\begin{minipage}[t]{0.95\textwidth}#1\end{minipage}}}

}
\def\ROM{}
\def\Rom#1#2{\setcounter{roman}{#1}\renewcommand\ROM{#2}}

\Rom{6}{Funkce}
\title{\Huge\textbf{\theroman\quad \ROM}}
\author{Jiří Kalvoda}

\newcounter{countOfBegin}
\setcounter{countOfBegin}{0}
\newcommand{\BeginDoc}[1][]
{
	\ifnum\value{countOfBegin}=0
	\begin{document}
		#1
		\fi
	\addtocounter{countOfBegin}{1}
		
}
\def\EndDoc
{
	\addtocounter{countOfBegin}{-1}
	\ifnum\value{countOfBegin}=0
	\end{document}
	\fi
}

\fi
\BeginDoc{}
\section{Analytické vyjádření kružnice a kruhu}
\Def
\section{Vzájemná poloha kružnic, kruhů a lineárních útvarů}
\Poz Úloha požadující určení průniku dvou útvarů vede k řešení soustavy rovnic či nerovnic, ve kterých  jsou zahrnuta anaitická vyjádření těchto útvarů.

\Pr

Je dáná kružnice $k(S[2;3,r=5)$ a body $A[-3;-4]$ a $B[1;6]$.
Určete průsečík $k$ s:
\begin{enumerate}
	\item s úsečkou $AB$
	\item s polopřímkou $AB$
	\item s přímkou $AB$
\end{enumerate}

Kružnici vyjádřím: $(x-2)^2 + (y-3)^2 = 25$.\\
Přímku vyjádřím parametricky: $x=-3+4t; y=-4+10t | t\in\R$.

Dosadím: $(-5+4t)^2 + (-7+10t)^2 = 25$\\
$116t^2-180t+49=0$\\
$t_{1,2}=\f{180\pm\sqrt{9664}}{232} $

\begin{enumerate}
	\item Úsečka: $t\in\<0;1\>$. Zde leží pouze $t_2$:
		tedy $\zs{[-3+4t_2;-4+10t_2]}$
	\item Polopříka: $t>0$: Zde leží obě dvě hodnoty:
		tedy $\zs{[-3+4t_1;-4+10t_1],[-3+4t_2;-4+10t_2]}$
	\item Přímka:\\
		tedy $\zs{[-3+4t_1;-4+10t_1],[-3+4t_2;-4+10t_2]}$
\end{enumerate}
\pdf[0.5]{3-1.pdf}
\Pr 214/10:\\
\begin{enumerate}
	\item $(x-2)^2 + (y+1)^2 = 100$\\
		$x=0: y^2-6y = 0 \imp y^2 -95=0 \imp y=-1\pm 4 \sqrt 6$\\
		$A_1[0;-1+4\sqrt 6]$\\
		$A_2[0;-1-4\sqrt 6]$\\
		$y=0:  ix^2-2x+4+1= 100 \imp y^2-4x-95=0 \imp y=2\pm 3 \sqrt{11}$\\
		$B_1[2+3\sqrt{11};0]$\\
		$B_2[2-3\sqrt{11};0]$
	\item $(x-4)^2+(y-3)^2 = 25$\\
		$x=0: y^2-6y = 0 \imp y=0 \lor y=6$\\
		$A_1[0;0]$\\
		$A_2[0;6]$\\
		$y=0: x^2 - 8x = 0 \imp y=0 \lor y=8$\\
		$B_1[0;0]$\\
		$B_2[8;0]$
	\item $(x+3)^2+(y+4)^2 = 16$\\
		$x=0:y^2+8y+16+9=16\imp y = -4 \pm \sqrt 7$\\
		$A_1 [0;-4+\sqrt 7]$\\
		$A_2 [0;-4-\sqrt 7]$\\
		$y=0: (x+3)^2=0 \imp x=3$
		$B [3;0]$.
\end{enumerate}

\Pr 214/13:\\
$K: x^2 + (y-3)^2 \le= 25$\\

\begin{enumerate}
\item
$\pri{HL} = \zs{[-4+4t;8t]|t\in\R}$
$(-4+4t)^2 + (8t-3)^2 \le 25 \imp 80t^2 - 80 t \le 0 \imp t\in\<0;1\>$
$$\zs{[-4+4t;8t]|t\in\<0;1\>}$$
\item
$\pri{HM} = \zs{[-4+12t;4t]|t\in\R}$
$(-4+12t)^2 + (4t-3)^2 \le 25 \imp 160 t^2 - 120 t \le 0 \imp t\in\<0;\f 34\>$
$$\zs{[-4+12t;4t]|t\in\<0;\f 3 4\>i}$$
\item Analogicky
\end{enumerate}

\Pr 214/14:\\
$y= 2x + c$\\
$ (x-3)^2 + (2x + c +1)^2 = 4$\\
$ x^2-6x+9 + c^2 + 4cx + 2c + 4x^2 + 4x + 1 = 4$\\
$ x^2-6x+9 + c^2 + 4cx + 2c + 4x^2 + 4x + 1 = 4$\\
$ 5 x^2 +(4c-2)x + (2c+c^2+6) =0$

$D = (4c-2)^2-4\*5 ( 2c+c^2 + 6) = -4 c^2 - 56 c - 116 = (c-(-7+2\sqrt 5))(c-(-7-2 \sqrt 5))$

\begin{itemize}
	\item prázdný: $c \in (-\infty;-7-2\sqrt 5) \cup (-7+2\sqrt 5;\infty)$
	\item jednobodový: $c \in \zs{-7-2\sqrt 5 ; -7+2\sqrt 5}$
	\item výcebodový: $c \in (-7-2\sqrt 5 ; -7+2\sqrt 5)$
\end{itemize}

\EndDoc
