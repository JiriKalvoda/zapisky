\providecommand{\HINCLUDE}{NE}
\if ^\HINCLUDE^
\else
\def\HINCLUDE{}
\global\newdimen\Okraje
\global\Okraje =4cm
\input{$HOME/souteze/_hlavicka/h-.tex}

%\definecolor{colorV}{RGB}{255,127,0}
%\definecolor{colorPoz}{RGB}{153,51,0}
%\definecolor{orangeV}{RGB}{255,127,0}
%\definecolor{colorPr}{RGB}{0,5,255}
%\definecolor{colorDef}{RGB}{0.255,0}

\usepackage[shortlabels]{enumitem}
\setlength{\marginparsep}{2pt}
\setlength{\marginparwidth}{35pt}

\def\st{{\rm st}}
\def\P{{\rm P}}

\def\ISENUM{}
\def\inMargin#1{\End
		
		\hskip0pt \marginpar{{{#1}}}}
\newcounter{V}[section] 
\newcommand{\V}[1][]{\stepcounter{V}\inMargin{\textcolor{green}{V.\arabic{section}.\theV.:}}\ifx^#1^\else\textcolor{green}{\underline{{#1}:}}\addcontentsline{toc}{subsubsection}{V.\arabic{section}.\theV.:$\quad$ {#1}}\\\fi}
\def\Def{\inMargin{\textcolor{red}{Def:}}}
\def\Poz{{\inMargin{\textcolor{brown}{Pozn:}}}}
\def\Pr{{\inMargin{\textcolor{blue}{Př:}}}}
\def\Pozenum
{
	\begin{enumerate}[1)]%, left = 0pt ]
		\item\inMargin{\textcolor{brown}{Pozn:}}\def\ISENUM{a}}
\def\End
{
	\if	^\ISENUM^
	\else \end{enumerate}
	\fi
	\def\ISENUM{}
}
\reversemarginpar

\makeatletter
\renewcommand\thesection{§\arabic{section}.}
\renewcommand\thesubsection{\Alph{subsection})}
\renewcommand\thesubsubsection{\alph{subsubsection})}
\newcounter{chapter}
\setcounter{chapter}{0}
\renewcommand\thechapter{\Alph{chapter})}
\newcounter{roman}
\setcounter{roman}{0}
\renewcommand\theroman{\Roman{roman}.}
\makeatother
\def\sectionnum#1{\setcounter{section}{#1}\addtocounter{section}{-1}}
\def\subsectionnum#1{\setcounter{subsection}{#1}\addtocounter{subsection}{-1}}
\def\subsubsectionnum#1{\setcounter{subsubsection}{#1}\addtocounter{subsubsection}{-1}}
\def\chapternum#1{\setcounter{chapter}{#1}\addtocounter{chapter}{-1}}
\def\chapter#1{

	\addtocounter{chapter}{1}\sectionnum{1}
	\addcontentsline{toc}{section}{\large{\thechapter$\quad${#1}}}
	
	{\LARGE  \textbf{\begin{minipage}[t]{0.1\textwidth}\thechapter\end{minipage}\begin{minipage}[t]{0.95\textwidth}#1\end{minipage}}}

}
\def\ROM{}
\def\Rom#1#2{\setcounter{roman}{#1}\renewcommand\ROM{#2}}

\Rom{6}{Funkce}
\title{\Huge\textbf{\theroman\quad \ROM}}
\author{Jiří Kalvoda}

\newcounter{countOfBegin}
\setcounter{countOfBegin}{0}
\newcommand{\BeginDoc}[1][]
{
	\ifnum\value{countOfBegin}=0
	\begin{document}
		#1
		\fi
	\addtocounter{countOfBegin}{1}
		
}
\def\EndDoc
{
	\addtocounter{countOfBegin}{-1}
	\ifnum\value{countOfBegin}=0
	\end{document}
	\fi
}

\fi
\BeginDoc{}
\section{Středové kuželosečky a jejich tečny}
\Def
Kružnice, elipsy a hyperboly nazýváme středové křivky 2. stupně neboli \emph{středové kuželosečky}.
Jejich rovnice ve  kterých vystupují souřadnice středu, nazýváme rovnice ve středovém tvaru.
\Poz
Zapíšeme rovnice z předchozích definic:
$$
\begin{array}{ccc}
	(x-m)^2 + (y-n)^2 = r^2 & \f{(x-m)^2}{a^2}+\f{(y-n)^2}{b^2}= 1 & \f{(x-m)^2}{a^2}-\f{(y-n)^2}{b^2} = 1 \\
	{(x-m)^2}{r^2} + {(y-n)^2}{r^2} = 1 & \f{(x-m)^2}{b^2}+\f{(y-n)^2}{a^2}= 1 & -\f{(x-m)^2}{a^2}+\f{(y-n)^2}{b^2} = 1 \\
\end{array}
$$
Všechny rovnice jsou tedy tvaru
$$\pm p^2(x-m)^2 \pm q^2 (y-n)^1 = \pm s^2$$
\Poz
\pdf[0.3]{12-1.pdf}
Název kuželosečky vystihuje možnost její vytvoření jako průniku rotační kružních kuželových ploch a roviny (viz obrázek).

Povšimneme si \emph{vnitřku kuželové plochy} a jejího tečkoveného průniku s rovinou kuželesečky.
U kružnice a paraboly zřejmě dostaneme útvary, kter jsem nazvali vnitřními oblastmi.
Obdobný pojem zavedeme i pro středové kuželosečky.
Vidíme, že zatímco elipsa má jednu vnitřní oblast hyperbola má dvě.
Vislovíme však definici jen dle vzdáleností bodů v rovině:

\Def
\emph{Vnitřní oblstí elipsy} s ohnisky $F,G$ a s hlavní poloosou $a$ nezveme množinu všech bosů $X$ roviny, pro které platí $|FX|+|GX| < 2a$
\Def \emph{Vnitřní oblastí jedné větve hyperboly} $H(F,G,2a)$ se nazývá množina všech bodů $X$ roviny, pro které platí $|FX|-|GX" > 2a$ a druhé větve $|GX|-|FX|>2a$.
\V
\begin{enumerate}
	\item Má-li elipsa rovnici $\f{x^2}{a^2} + \f{y^2}{b^2} = 1$, pak její oblast má v téže soustavě souřadnic analytické vyjádření $\f{x^2}{a^2}+\f{y^2}{b^2} < 1$
	\item Má-li hyperbola rovnici $\f{x^2}{a^2}-\f{y^2}{b^2} = 1$, pak sjednocení vnitřních oblastí jejích větví má alalytické vyjádření $\f{x^2}{a^2}-\f{y^2}{b^2} > 1$
\end{enumerate}
[Dk: Při odvozování se zachovává znaménko]

\Def
\emph{Tečnou středové kuželosečky} se nazývá přímka, která obsahuje jediný bod kuželosečky a neobsahuje žádný bod její vnitřní oblasti.

\V
Má-li středová kuželosečka rovnici
$$\pm p^1 (x-m)^2 \pm q^2(y-n)^2 \pm s^2$$
pak v její tečna $t$ v $T[x_0,y_0]$ má rovnici
$$\pm p^2 (x-m)(x_0-m) \pm q^2 (y-n)(y_0-n) = \pm s^2$$

\Pr 262/1:\\
Je dána hyperbola s rovnicí $16(x+2)^2-5(y-5)^2 = 80$.
Určete rovnice všech tečen hyperboly, která má směrnici $k=2$.

Tečna v $[x_0;y_0]$:
$$ 16(x+2)(x_0+2) - 5(y-5)(y_0-5) = 80 $$
Směrnicovy tvarurčíme pokud $y_0-5 \neq 0$:
$$y-5 = \f{16(x_0+2)}{5(y_0-5)}(x+2) - \f{80}{5(y_0-5)}$$
Protože $k=\f{16(x_0+2)}{5(y_0-5)} = 2$, platí pro hledané souřadnice $x_0,y_0$ dvě rovnice:

\begin{eqnarray*} 
	  16(x_0+2) &=& 10 (y_0-5) \\
	  16(x_0+2)^2 -5 (y_0-5)^2 = 80 \\
\end{eqnarray*}
Dosadíme $5(y_0-5)$ na místo $8(x_0+2)$:
\begin{eqnarray*} 
	  [5(y_0-5)]^2 - 20 (y_0 - 5)^2 &=& 320\\
	  (y_0-5)^2 = 64 
\end{eqnarray*}
Tedy $y_0-5 = 5$ nebo $y_0'-5=-8$, tedy $y_0=13$ nebo $y_0=-7$.
Dosazeím $x_0=3$ resp. $x_0=-7$

$$ t_1: 2x-y + 7 = 0$$
$$ t_2: 2x-y + 11 = 0 $$

\Pr Je dána elipsa $E: x^2+5y^0-5  = 0$ a bod $M[5;1]$.

Tečna v bodě $T[x_0,y_0]$:
$$ xx_0 + 5yy_0 - 5 = 0 $$
Budeme hledat všechny $T_0$, pro něž tečne prochází $M$:


\begin{eqnarray} 
	  x_0^2 + 5y_0^2 -5 &=& 0\\
	  5x_0  + 5y_0^2 -5 &=& 0
\end{eqnarray}
Dosadím:
\begin{eqnarray*} 
	  (1-y_0)^2 + 5y_0^2 - 5 &=& 0\\
	  1-2y_0+y_0^2 + 5y_0^2 - 5 &=& 0\\
	  3y_0^2 - y_0 -2 &=& 0\\
\end{eqnarray*}
$y_0 = 1 \lor y_0 = - \f 23$

$T[0;1];T'[\f 53;-\f 23]$

$\ve{TM} = (5;0) \sim (1;0)$\\
$\ve{TM} = \(\f{10}3;\f 53\) \sim (2;1)$

$$\cos \phi = \f{2}{\sqrt 1 \* \sqrt 5 } =  \f 2 5 \sqrt 5$$
\Pr 265/30:
Kolmice jsou tvaru $3x+2y+c=0$
\begin{enumerate}
	\item[b)]
		  $$4xx_0-9yy_0 = 36$$		    
		    Musí tedy platit $(4x_0;-9y_0) \sim (3;2) \imp 8x_0= -27y_0$
		    \begin{eqnarray*} 
				4(-\f{27}8 y_0)^2 - 9 y_0 &=& 36 \\
				65y_0^2 = 64
		    \end{eqnarray*}
		    Tedy $y_0 = \pm \sqrt{\f{64}{65}} = \f{8\sqrt{65}}{65}$ a $x_0 = \f{-27}{8}\*\(\pm \sqrt{\f{64}{65}}\) = \mp \sqrt{\f{729}{65}} = \mp \f{27\sqrt{65}}{65}$
		    
		    $$t_1: 4x \f{27\sqrt{65}}{65}  + 9y \f{8\sqrt{65}}{65} = 36$$
		    $$t_2: 4x \f{27\sqrt{65}}{65}  + 9y \f{8\sqrt{65}}{65} = - 36$$
	  \item  Zavedeme posunuté souřadnice $x' = x-3$ a $y'=y-4$
		    Tedy $x^2 + 2 y^2 = 4$. Kolmost se posunem nezmění.

		    Dále tedy analogicky určím tečny a finálně je posunu do kůvodních souřadic.
\end{enumerate}

\Pr 265/32:\\
\begin{enumerate}
	  \item[a)]
		    Tečna bodem $T:-xx_0+yy_0=9$.

		    Musí procházet $M$: $6x_0+3y_0 = 9 \imp y_0 = 3-2x_0$\\
		    A $T$ musí náležet kuželosečce: $- x_0^2 + y_0^2 = 9$

		    Dosadím: $-x_0^2 + (3-2x_0)^2 = 9 \imp 3x_0^2 - 12 x_0 = 0 \imp x=0 \land x=4$.
		    $$t_1: 3y= 9$$
		    $$t_2: -4x - 5y = 9$$

	  \item[c)] Upravím na $2(x-2)^2 - 3(y-1)^2 - 30 = 0$.
		    Zavedením posunutých souřadnic $x' = x-2 ;y' = y-1$ získám $2x^2 - 3y^2 = 30$ a $M[3;9]$.

		    Dále analogicky.
\end{enumerate}

\subsection{Obecná rovnice kuželosečky a zakreslení množiny bodů obecné kvadratické rovnice se 2 neznámými bez členu $xy$}
\Poz 
$$Ax^2 + By^2 + Cx + Dy + E = 0$$
kde $(A,B)\neq \ve 0$

Obě proměnné upravíme na čtverec a upravíme na středovou rovnici kuželosečky (parabolu na vrcholovou rovnici).

\Pr
Zakreslete množinu bodů danou rovnicí $3x^2-2y^2-12x-4y-2=0$:

Upravíme:
\begin{eqnarray*} 
	3(x^2-4x)-2(y^2+2y)-2&=&0\\
	3(x-2)^2-2(y+1)^2 &=& 2 + 4\*3 - 1 \*2 = 12\\
	\f{(x-2)^2}{2} - \f{(y+1)^2}{6} &=& 1 \\
\end{eqnarray*}

Zakreslíme hperbolu se středem $S[2;-1]$, s hlavní osou na rovnoběžce s osou $x$, s poloosami $a=2;b=\sqrt 6$ a exentricitou $e=\sqrt{10}$ a s asymptotami:
$$y+1 = \f{\sqrt 6}2(x-2)$$
$$y+1 = -\f{\sqrt 6}2(x-2)$$

\Poz
Doplnění výrazů $Ax^2+Dx;Cy^2+Ey$ na druhé mocniny dvojčlenů poskytuje středové tvary rovnic kuželeoseček a tím umožnuje jejich zakreslení.

\Pr 
\begin{enumerate}[a)]
	\item 
		$$ \f{x^2}{12} - \f{y^2}{3} = 1$$
		\pdf[0.5]{12-p1.pdf}
		$$
		\begin{array}{r@{[}c@{;}c@{]}}
			S&0&0\\
			A_1 &-\sqrt{12}&0\\
			A_2 &+\sqrt{12}&0\\
			F   &-\sqrt{15}&0\\
			G   &+\sqrt{15}&0\\
		\end{array}
		$$
		$$a: y=\f x2$$
		$$a': y=-\f x2$$
	\item 
		$$ \f{x^2}{3} - \f{y^2}{9} = 1$$
		\pdf[0.5]{12-p2.pdf}
		$$
		\begin{array}{r@{[}c@{;}c@{]}}
			S&0&0\\
			A_1 &0&-3\\
			A_2 &0&+3\\
			B_1 &-\sqrt 3&0\\
			B_2 &+\sqrt 3&0\\
			F   &0&-\sqrt 6\\
			G   &0&+\sqrt 6\\
		\end{array}
		$$
	\item 
		$$ \f{(x^2+1)^2}{27} - \f{y^2}{9} = 1$$
		\pdf[0.5]{12-p3.pdf}
		$$
		\begin{array}{r@{[}c@{;}c@{]}}
			S&-1&0\\
			A_1 &-1-\sqrt{27}&0\\
			A_2 &-1+\sqrt{27}&0\\
			F   &-1-6&0\\
			G   &-1+6&0\\
		\end{array}
		$$
		$$a: y=\f {x+1}{\sqrt 3}$$
		$$a': y=-\f {x+1}{\sqrt 3}$$
	\item 
		$$ \(x^2-\f{13}2\)^2 - \(y^2-\f45\) = \f{65}2$$
		\pdf[0.4]{12-p4.pdf}
		$$
		\begin{array}{r@{[}c@{;}c@{]}}
			S&\f{13}2&\f 9 2\\
		\end{array}
		$$
	\item 
		$$16\(x-\f5 4\)^2 + 9\(y+\f 13\)^2 = -25+{25}+1 = 1$$
		\pdf[0.5]{12-p5.pdf}
		$$
		\begin{array}{r@{[}c@{;}c@{]}}
			S&\f 54&\f 13\\
			A_1 &\f 54&0\\
			A_2 &\f 54&-\f 2 3\\
			B_1 &1&-\f 13\\
			B_2 &1.5&-\f 13\\
			F   &\f 54&\f 13 - \f{\sqrt 7}12\\
			G   &\f 54&\f 13 + \f{\sqrt 7}12\\
		\end{array}
		$$
\end{enumerate}


\Poz
Kritéria vzniku jednotlivých útvarů:

Předpokládejme, že vznikne kuželosečka.

Kružnice vznikne právě tehdy když $A=B$

Elipsa vznikne právě tehdy když ${\rm sqn}(A)={\rm sqn}(B) = \pm 1$

Hyperbola vznikne právě tehdy když ${\rm sqn}(A)=-{\rm sqn}(B) = \pm 1$

Parabola vznikne právě tehdy když $AB = 0 \land A+B \neq 0$



\EndDoc

