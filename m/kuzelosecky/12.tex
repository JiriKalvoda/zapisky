\providecommand{\HINCLUDE}{NE}
\if ^\HINCLUDE^
\else
\def\HINCLUDE{}
\global\newdimen\Okraje
\global\Okraje =4cm
\input{$HOME/souteze/_hlavicka/h-.tex}

%\definecolor{colorV}{RGB}{255,127,0}
%\definecolor{colorPoz}{RGB}{153,51,0}
%\definecolor{orangeV}{RGB}{255,127,0}
%\definecolor{colorPr}{RGB}{0,5,255}
%\definecolor{colorDef}{RGB}{0.255,0}

\usepackage[shortlabels]{enumitem}
\setlength{\marginparsep}{2pt}
\setlength{\marginparwidth}{35pt}

\def\st{{\rm st}}
\def\P{{\rm P}}

\def\ISENUM{}
\def\inMargin#1{\End
		
		\hskip0pt \marginpar{{{#1}}}}
\newcounter{V}[section] 
\newcommand{\V}[1][]{\stepcounter{V}\inMargin{\textcolor{green}{V.\arabic{section}.\theV.:}}\ifx^#1^\else\textcolor{green}{\underline{{#1}:}}\addcontentsline{toc}{subsubsection}{V.\arabic{section}.\theV.:$\quad$ {#1}}\\\fi}
\def\Def{\inMargin{\textcolor{red}{Def:}}}
\def\Poz{{\inMargin{\textcolor{brown}{Pozn:}}}}
\def\Pr{{\inMargin{\textcolor{blue}{Př:}}}}
\def\Pozenum
{
	\begin{enumerate}[1)]%, left = 0pt ]
		\item\inMargin{\textcolor{brown}{Pozn:}}\def\ISENUM{a}}
\def\End
{
	\if	^\ISENUM^
	\else \end{enumerate}
	\fi
	\def\ISENUM{}
}
\reversemarginpar

\makeatletter
\renewcommand\thesection{§\arabic{section}.}
\renewcommand\thesubsection{\Alph{subsection})}
\renewcommand\thesubsubsection{\alph{subsubsection})}
\newcounter{chapter}
\setcounter{chapter}{0}
\renewcommand\thechapter{\Alph{chapter})}
\newcounter{roman}
\setcounter{roman}{0}
\renewcommand\theroman{\Roman{roman}.}
\makeatother
\def\sectionnum#1{\setcounter{section}{#1}\addtocounter{section}{-1}}
\def\subsectionnum#1{\setcounter{subsection}{#1}\addtocounter{subsection}{-1}}
\def\subsubsectionnum#1{\setcounter{subsubsection}{#1}\addtocounter{subsubsection}{-1}}
\def\chapternum#1{\setcounter{chapter}{#1}\addtocounter{chapter}{-1}}
\def\chapter#1{

	\addtocounter{chapter}{1}\sectionnum{1}
	\addcontentsline{toc}{section}{\large{\thechapter$\quad${#1}}}
	
	{\LARGE  \textbf{\begin{minipage}[t]{0.1\textwidth}\thechapter\end{minipage}\begin{minipage}[t]{0.95\textwidth}#1\end{minipage}}}

}
\def\ROM{}
\def\Rom#1#2{\setcounter{roman}{#1}\renewcommand\ROM{#2}}

\Rom{6}{Funkce}
\title{\Huge\textbf{\theroman\quad \ROM}}
\author{Jiří Kalvoda}

\newcounter{countOfBegin}
\setcounter{countOfBegin}{0}
\newcommand{\BeginDoc}[1][]
{
	\ifnum\value{countOfBegin}=0
	\begin{document}
		#1
		\fi
	\addtocounter{countOfBegin}{1}
		
}
\def\EndDoc
{
	\addtocounter{countOfBegin}{-1}
	\ifnum\value{countOfBegin}=0
	\end{document}
	\fi
}

\fi
\BeginDoc{}
\section{Středové kuželosečky a jejich tečny}
\Def
Kružnice, elipsy a hyperboly nazýváme středové křivky 2. stupně neboli \emph{středové kuželosečky}.
Jejich rovnice ve  kterých vystupují souřadnice středu, nazýváme rovnice ve středovém tvaru.
\Poz
Zapíšeme rovnice z předchozích definic:
$$
\begin{array}{ccc}
	(x-m)^2 + (y-n)^2 = r^2 & \f{(x-m)^2}{a^2}+\f{(y-n)^2}{b^2}= 1 & \f{(x-m)^2}{a^2}-\f{(y-n)^2}{b^2} = 1 \\
	{(x-m)^2}{r^2} + {(y-n)^2}{r^2} = 1 & \f{(x-m)^2}{b^2}+\f{(y-n)^2}{a^2}= 1 & -\f{(x-m)^2}{a^2}+\f{(y-n)^2}{b^2} = 1 \\
\end{array}
$$
Všechny rovnice jsou tedy tvaru
$$\pm p^2(x-m)^2 \pm q^2 (y-n)^1 = \pm s^2$$
\Poz
\pdf[0.3]{12-1.pdf}
Název kuželosečky vystihuje možnost její vytvoření jako průniku rotační kružních kuželových ploch a roviny (viz obrázek).

Povšimneme si \emph{vnitřku kuželové plochy} a jejího tečkoveného průniku s rovinou kuželesečky.
U kružnice a paraboly zřejmě dostaneme útvary, kter jsem nazvali vnitřními oblastmi.
Obdobný pojem zavedeme i pro středové kuželosečky.
Vidíme, že zatímco elipsa má jednu vnitřní oblast hyperbola má dvě.
Vislovíme však definici jen dle vzdáleností bodů v rovině:

\Def
\emph{Vnitřní oblstí elipsy} s ohnisky $F,G$ a s hlavní poloosou $a$ nezveme množinu všech bosů $X$ roviny, pro které platí $|FX|+|GX| < 2a$
\Def \emph{Vnitřní oblastí jedné větve hyperboly} $H(F,G,2a)$ se nazývá množina všech bodů $X$ roviny, pro které platí $|FX|-|GX" > 2a$ a druhé větve $|GX|-|FX|>2a$.
\V
\begin{enumerate}
	\item Má-li elipsa rovnici $\f{x^2}{a^2} + \f{y^2}{b^2} = 1$, pak její oblast má v téže soustavě souřadnic analytické vyjádření $\f{x^2}{a^2}+\f{y^2}{b^2} < 1$
	\item Má-li hyperbola rovnici $\f{x^2}{a^2}-\f{y^2}{b^2} = 1$, pak sjednocení vnitřních oblastí jejích větví má alalytické vyjádření $\f{x^2}{a^2}-\f{y^2}{b^2} > 1$
\end{enumerate}
[Dk: Při odvozování se zachovává znaménko]

\Def
\emph{Tečnou středové kuželosečky} se nazývá přímka, která obsahuje jediný bod kuželosečky a neobsahuje žádný bod její vnitřní oblasti.

\V
Má-li středová kuželosečka rovnici
$$\pm p^1 (x-m)^2 \pm q^2(y-n)^2 \pm s^2$$
pak v její tečna $t$ v $T[x_0,y_0]$ má rovnici
$$\pm p^2 (x-m)(x_0-m) \pm q^2 (y-n)(y_0-n) = \pm s^2$$
\EndDoc
