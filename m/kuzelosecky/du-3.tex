\providecommand{\HINCLUDE}{NE}
\if ^\HINCLUDE^
\else
\def\HINCLUDE{}
\global\newdimen\Okraje
\global\Okraje =4cm
\input{$HOME/souteze/_hlavicka/h-.tex}

%\definecolor{colorV}{RGB}{255,127,0}
%\definecolor{colorPoz}{RGB}{153,51,0}
%\definecolor{orangeV}{RGB}{255,127,0}
%\definecolor{colorPr}{RGB}{0,5,255}
%\definecolor{colorDef}{RGB}{0.255,0}

\usepackage[shortlabels]{enumitem}
\setlength{\marginparsep}{2pt}
\setlength{\marginparwidth}{35pt}

\def\st{{\rm st}}
\def\P{{\rm P}}

\def\ISENUM{}
\def\inMargin#1{\End
		
		\hskip0pt \marginpar{{{#1}}}}
\newcounter{V}[section] 
\newcommand{\V}[1][]{\stepcounter{V}\inMargin{\textcolor{green}{V.\arabic{section}.\theV.:}}\ifx^#1^\else\textcolor{green}{\underline{{#1}:}}\addcontentsline{toc}{subsubsection}{V.\arabic{section}.\theV.:$\quad$ {#1}}\\\fi}
\def\Def{\inMargin{\textcolor{red}{Def:}}}
\def\Poz{{\inMargin{\textcolor{brown}{Pozn:}}}}
\def\Pr{{\inMargin{\textcolor{blue}{Př:}}}}
\def\Pozenum
{
	\begin{enumerate}[1)]%, left = 0pt ]
		\item\inMargin{\textcolor{brown}{Pozn:}}\def\ISENUM{a}}
\def\End
{
	\if	^\ISENUM^
	\else \end{enumerate}
	\fi
	\def\ISENUM{}
}
\reversemarginpar

\makeatletter
\renewcommand\thesection{§\arabic{section}.}
\renewcommand\thesubsection{\Alph{subsection})}
\renewcommand\thesubsubsection{\alph{subsubsection})}
\newcounter{chapter}
\setcounter{chapter}{0}
\renewcommand\thechapter{\Alph{chapter})}
\newcounter{roman}
\setcounter{roman}{0}
\renewcommand\theroman{\Roman{roman}.}
\makeatother
\def\sectionnum#1{\setcounter{section}{#1}\addtocounter{section}{-1}}
\def\subsectionnum#1{\setcounter{subsection}{#1}\addtocounter{subsection}{-1}}
\def\subsubsectionnum#1{\setcounter{subsubsection}{#1}\addtocounter{subsubsection}{-1}}
\def\chapternum#1{\setcounter{chapter}{#1}\addtocounter{chapter}{-1}}
\def\chapter#1{

	\addtocounter{chapter}{1}\sectionnum{1}
	\addcontentsline{toc}{section}{\large{\thechapter$\quad${#1}}}
	
	{\LARGE  \textbf{\begin{minipage}[t]{0.1\textwidth}\thechapter\end{minipage}\begin{minipage}[t]{0.95\textwidth}#1\end{minipage}}}

}
\def\ROM{}
\def\Rom#1#2{\setcounter{roman}{#1}\renewcommand\ROM{#2}}

\Rom{6}{Funkce}
\title{\Huge\textbf{\theroman\quad \ROM}}
\author{Jiří Kalvoda}

\newcounter{countOfBegin}
\setcounter{countOfBegin}{0}
\newcommand{\BeginDoc}[1][]
{
	\ifnum\value{countOfBegin}=0
	\begin{document}
		#1
		\fi
	\addtocounter{countOfBegin}{1}
		
}
\def\EndDoc
{
	\addtocounter{countOfBegin}{-1}
	\ifnum\value{countOfBegin}=0
	\end{document}
	\fi
}

\fi
\BeginDoc{}


\Pr
Zakreslete množinu bodů danou rovnicí $3x^2-2y^2-12x-4y-2=0$:

Upravíme:
\begin{eqnarray*} 
	3(x^2-4x)-2(y^2+2y)-2&=&0\\
	3(x-2)^2-2(y+1)^2 &=& 2 + 4\*3 - 1 \*2 = 12\\
	\f{(x-2)^2}{2} - \f{(y+1)^2}{6} &=& 1 \\
\end{eqnarray*}

Zakreslíme hperbolu se středem $S[2;-1]$, s hlavní osou na rovnoběžce s osou $x$, s poloosami $a=2;b=\sqrt 6$ a exentricitou $e=\sqrt{10}$ a s asymptotami:
$$y+1 = \f{\sqrt 6}2(x-2)$$
$$y+1 = -\f{\sqrt 6}2(x-2)$$

\Poz
Doplnění výrazů $Ax^2+Dx;Cy^2+Ey$ na druhé mocniny dvojčlenů poskytuje středové tvary rovnic kuželeoseček a tím umožnuje jejich zakreslení.

\Pr 
\begin{enumerate}[a)]
	\item 
		$$ \f{x^2}{12} - \f{y^2}{3} = 1$$
		\pdf[0.5]{12-p1.pdf}
		$$
		\begin{array}{r@{[}c@{;}c@{]}}
			S&0&0\\
			A_1 &-\sqrt{12}&0\\
			A_2 &+\sqrt{12}&0\\
			F   &-\sqrt{15}&0\\
			G   &+\sqrt{15}&0\\
		\end{array}
		$$
		$$a: y=\f x2$$
		$$a': y=-\f x2$$
	\item 
		$$ \f{x^2}{3} - \f{y^2}{9} = 1$$
		\pdf[0.5]{12-p2.pdf}
		$$
		\begin{array}{r@{[}c@{;}c@{]}}
			S&0&0\\
			A_1 &0&-3\\
			A_2 &0&+3\\
			B_1 &-\sqrt 3&0\\
			B_2 &+\sqrt 3&0\\
			F   &0&-\sqrt 6\\
			G   &0&+\sqrt 6\\
		\end{array}
		$$
	\item 
		$$ \f{(x^2+1)^2}{27} - \f{y^2}{9} = 1$$
		\pdf[0.5]{12-p3.pdf}
		$$
		\begin{array}{r@{[}c@{;}c@{]}}
			S&-1&0\\
			A_1 &-1-\sqrt{27}&0\\
			A_2 &-1+\sqrt{27}&0\\
			F   &-1-6&0\\
			G   &-1+6&0\\
		\end{array}
		$$
		$$a: y=\f {x+1}{\sqrt 3}$$
		$$a': y=-\f {x+1}{\sqrt 3}$$
	\item 
		$$ \(x^2-\f{13}2\)^2 - \(y^2-\f45\) = \f{65}2$$
		\pdf[0.4]{12-p4.pdf}
		$$
		\begin{array}{r@{[}c@{;}c@{]}}
			S&\f{13}2&\f 9 2\\
		\end{array}
		$$
	\item 
		$$16\(x-\f5 4\)^2 + 9\(y+\f 13\)^2 = -25+{25}+1 = 1$$
		\pdf[0.5]{12-p5.pdf}
		$$
		\begin{array}{r@{[}c@{;}c@{]}}
			S&\f 54&\f 13\\
			A_1 &\f 54&0\\
			A_2 &\f 54&-\f 2 3\\
			B_1 &1&-\f 13\\
			B_2 &1.5&-\f 13\\
			F   &\f 54&\f 13 - \f{\sqrt 7}12\\
			G   &\f 54&\f 13 + \f{\sqrt 7}12\\
		\end{array}
		$$
\end{enumerate}


\Poz
Kritéria vzniku jednotlivých útvarů:

Předpokládejme, že vznikne kuželosečka.

Kružnice vznikne právě tehdy když $A=B$

Elipsa vznikne právě tehdy když ${\rm sqn}(A)={\rm sqn}(B) = \pm 1$

Hyperbola vznikne právě tehdy když ${\rm sqn}(A)=-{\rm sqn}(B) = \pm 1$

Parabola vznikne právě tehdy když $AB = 0 \land A+B \neq 0$



\EndDoc

