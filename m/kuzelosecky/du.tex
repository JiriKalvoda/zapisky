\providecommand{\HINCLUDE}{NE}
\if ^\HINCLUDE^
\else
\def\HINCLUDE{}
\global\newdimen\Okraje
\global\Okraje =4cm
\input{$HOME/souteze/_hlavicka/h-.tex}

%\definecolor{colorV}{RGB}{255,127,0}
%\definecolor{colorPoz}{RGB}{153,51,0}
%\definecolor{orangeV}{RGB}{255,127,0}
%\definecolor{colorPr}{RGB}{0,5,255}
%\definecolor{colorDef}{RGB}{0.255,0}

\usepackage[shortlabels]{enumitem}
\setlength{\marginparsep}{2pt}
\setlength{\marginparwidth}{35pt}

\def\st{{\rm st}}
\def\P{{\rm P}}

\def\ISENUM{}
\def\inMargin#1{\End
		
		\hskip0pt \marginpar{{{#1}}}}
\newcounter{V}[section] 
\newcommand{\V}[1][]{\stepcounter{V}\inMargin{\textcolor{green}{V.\arabic{section}.\theV.:}}\ifx^#1^\else\textcolor{green}{\underline{{#1}:}}\addcontentsline{toc}{subsubsection}{V.\arabic{section}.\theV.:$\quad$ {#1}}\\\fi}
\def\Def{\inMargin{\textcolor{red}{Def:}}}
\def\Poz{{\inMargin{\textcolor{brown}{Pozn:}}}}
\def\Pr{{\inMargin{\textcolor{blue}{Př:}}}}
\def\Pozenum
{
	\begin{enumerate}[1)]%, left = 0pt ]
		\item\inMargin{\textcolor{brown}{Pozn:}}\def\ISENUM{a}}
\def\End
{
	\if	^\ISENUM^
	\else \end{enumerate}
	\fi
	\def\ISENUM{}
}
\reversemarginpar

\makeatletter
\renewcommand\thesection{§\arabic{section}.}
\renewcommand\thesubsection{\Alph{subsection})}
\renewcommand\thesubsubsection{\alph{subsubsection})}
\newcounter{chapter}
\setcounter{chapter}{0}
\renewcommand\thechapter{\Alph{chapter})}
\newcounter{roman}
\setcounter{roman}{0}
\renewcommand\theroman{\Roman{roman}.}
\makeatother
\def\sectionnum#1{\setcounter{section}{#1}\addtocounter{section}{-1}}
\def\subsectionnum#1{\setcounter{subsection}{#1}\addtocounter{subsection}{-1}}
\def\subsubsectionnum#1{\setcounter{subsubsection}{#1}\addtocounter{subsubsection}{-1}}
\def\chapternum#1{\setcounter{chapter}{#1}\addtocounter{chapter}{-1}}
\def\chapter#1{

	\addtocounter{chapter}{1}\sectionnum{1}
	\addcontentsline{toc}{section}{\large{\thechapter$\quad${#1}}}
	
	{\LARGE  \textbf{\begin{minipage}[t]{0.1\textwidth}\thechapter\end{minipage}\begin{minipage}[t]{0.95\textwidth}#1\end{minipage}}}

}
\def\ROM{}
\def\Rom#1#2{\setcounter{roman}{#1}\renewcommand\ROM{#2}}

\Rom{6}{Funkce}
\title{\Huge\textbf{\theroman\quad \ROM}}
\author{Jiří Kalvoda}

\newcounter{countOfBegin}
\setcounter{countOfBegin}{0}
\newcommand{\BeginDoc}[1][]
{
	\ifnum\value{countOfBegin}=0
	\begin{document}
		#1
		\fi
	\addtocounter{countOfBegin}{1}
		
}
\def\EndDoc
{
	\addtocounter{countOfBegin}{-1}
	\ifnum\value{countOfBegin}=0
	\end{document}
	\fi
}

\fi
\BeginDoc{}
 \Pr Určete rovnice vešech kružnic, které prochází bodem $A[1;2]$, dotýká se osy $y$ a mají střed na přímce $p$, která má rovnici $y+x = 4$:

 Hledám $(x-m)^2 + (y-n)^2 = r^2$\\
$k$ se dotýká $y$ $\imp$ $m^2 = r^2$\\
$S \in P \imp m + n  = 4$\\
$A \in k \imp (1-m)^2 + (2-n)^2 = r^2$

$m+n = 4$\\
$(11-m)^2 + (2-n)^2 = m^2 $\\

Řešení:
$(x-1)^2 + (y-3)^2 = 1$
a
$(x-5)^2 + (y+1)^2 = 25$

\Pr 210/7:
Hledám $(x-m)^2 + (y-n)^2 = r^2$:

Dotýká se $x$ $\imp$ $r^2 = m^2$.\\
Dotýká se $x$ $\imp$ $r^2 = n^2$.\\
$K \in k \imp (9-n)^2 + (2-m)^2 = r^2 $ 

Když $m=n=\pm r$
$K \in k \imp (9-n)^2 + (2-n)^2 = n^2 \imp x^2-22x + 85 =0$ 

$m=n=r=5 \lor m=n=r = 17$

$$ (x+5)^2 + (y+5) = 5^2 $$
$$ (x+17)^2 + (y+17) = 5^2 $$

Když $m=-n=\pm r$
$K \in k \imp (9+n)^2 + (2-n)^2 = n^2 \imp x^2-15x + 85 =0 \imp D=255-4\*85 = -115 < 0$ 

\Pr 210/8:
Hledám $(x-m)^2 + (y-n)^2 = r^2$:

$m+3n-6 = 0 \imp m = 6-3n$\\
$r=5$\\
$M[6;9] \in k \imp (6-m)^2 + (9-n)^2 = 25 \imp (6-6+3n)^2 + (9-n)^2 = 25 \imp 9n^2 -18n+56 = 0 \imp D = 18^2 - 4 \* 9 \* 56 = -1692 < 0$.

Neexistuje řešení.

\Pr 210/9/a:

Osa přímek, na které musí náležet střed je buď $x=0$ nebo $y=0$:

Jelikož $p\perp q$, průsečík přímek, body doteku a střed tvoří čtverec o straně $r$, tedy $|[0;0]S| = 2\sqrt 2$:

Řešením tedy jsou:
$$ (x-2\sqrt 2)^2 + y^2 = 2 $$
$$ x^2 + (y-2\sqrt 2)^2 = 2 $$
$$ (x+2\sqrt 2)^2 + y^2 = 2 $$
$$ x^2 + (y+2\sqrt 2)^2 = 2 $$

\Pr 210/9/b:

$(m-4)^2 = 4  \imp m = 2 \lor m = 6$\\
$S \in r \parallel p$
$\rho(r,q) = 2 \imp r:x-y+2\pm 2\sqrt 2=0  \imp n = 2\pm 2\sqrt 2  +m $

$$ (x-2)^2 + (y-4-2 \sqrt 2)^2 = 2 $$
$$ (x-2)^2 + (y-4+2 \sqrt 2)^2 = 2 $$
$$ (x-2)^2 + (y-8-2 \sqrt 2)^2 = 2 $$
$$ (x-2)^2 + (y-8+2 \sqrt 2)^2 = 2 $$

\EndDoc
