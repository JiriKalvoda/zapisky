\providecommand{\HINCLUDE}{NE}
\if ^\HINCLUDE^
\else
\def\HINCLUDE{}
\global\newdimen\Okraje
\global\Okraje =4cm
\input{$HOME/souteze/_hlavicka/h-.tex}

%\definecolor{colorV}{RGB}{255,127,0}
%\definecolor{colorPoz}{RGB}{153,51,0}
%\definecolor{orangeV}{RGB}{255,127,0}
%\definecolor{colorPr}{RGB}{0,5,255}
%\definecolor{colorDef}{RGB}{0.255,0}

\usepackage[shortlabels]{enumitem}
\setlength{\marginparsep}{2pt}
\setlength{\marginparwidth}{35pt}

\def\st{{\rm st}}
\def\P{{\rm P}}

\def\ISENUM{}
\def\inMargin#1{\End
		
		\hskip0pt \marginpar{{{#1}}}}
\newcounter{V}[section] 
\newcommand{\V}[1][]{\stepcounter{V}\inMargin{\textcolor{green}{V.\arabic{section}.\theV.:}}\ifx^#1^\else\textcolor{green}{\underline{{#1}:}}\addcontentsline{toc}{subsubsection}{V.\arabic{section}.\theV.:$\quad$ {#1}}\\\fi}
\def\Def{\inMargin{\textcolor{red}{Def:}}}
\def\Poz{{\inMargin{\textcolor{brown}{Pozn:}}}}
\def\Pr{{\inMargin{\textcolor{blue}{Př:}}}}
\def\Pozenum
{
	\begin{enumerate}[1)]%, left = 0pt ]
		\item\inMargin{\textcolor{brown}{Pozn:}}\def\ISENUM{a}}
\def\End
{
	\if	^\ISENUM^
	\else \end{enumerate}
	\fi
	\def\ISENUM{}
}
\reversemarginpar

\makeatletter
\renewcommand\thesection{§\arabic{section}.}
\renewcommand\thesubsection{\Alph{subsection})}
\renewcommand\thesubsubsection{\alph{subsubsection})}
\newcounter{chapter}
\setcounter{chapter}{0}
\renewcommand\thechapter{\Alph{chapter})}
\newcounter{roman}
\setcounter{roman}{0}
\renewcommand\theroman{\Roman{roman}.}
\makeatother
\def\sectionnum#1{\setcounter{section}{#1}\addtocounter{section}{-1}}
\def\subsectionnum#1{\setcounter{subsection}{#1}\addtocounter{subsection}{-1}}
\def\subsubsectionnum#1{\setcounter{subsubsection}{#1}\addtocounter{subsubsection}{-1}}
\def\chapternum#1{\setcounter{chapter}{#1}\addtocounter{chapter}{-1}}
\def\chapter#1{

	\addtocounter{chapter}{1}\sectionnum{1}
	\addcontentsline{toc}{section}{\large{\thechapter$\quad${#1}}}
	
	{\LARGE  \textbf{\begin{minipage}[t]{0.1\textwidth}\thechapter\end{minipage}\begin{minipage}[t]{0.95\textwidth}#1\end{minipage}}}

}
\def\ROM{}
\def\Rom#1#2{\setcounter{roman}{#1}\renewcommand\ROM{#2}}

\Rom{6}{Funkce}
\title{\Huge\textbf{\theroman\quad \ROM}}
\author{Jiří Kalvoda}

\newcounter{countOfBegin}
\setcounter{countOfBegin}{0}
\newcommand{\BeginDoc}[1][]
{
	\ifnum\value{countOfBegin}=0
	\begin{document}
		#1
		\fi
	\addtocounter{countOfBegin}{1}
		
}
\def\EndDoc
{
	\addtocounter{countOfBegin}{-1}
	\ifnum\value{countOfBegin}=0
	\end{document}
	\fi
}

\fi
\BeginDoc{}
\section{Tečna kružnice}
\Poz
Jak víme, tečna je přímka, která leží v rovině kružnice a má tyto vlastnosti:
\begin{itemize}
	\item Obsahuje právě jeden bod kružnice
	\item Střed kružnice má od ní vzdálenost poloměru kružnice.
	\item Je kolmá k poloměru kružnice, který obsahuje bod dotyku
\end{itemize}
\Poz Rovnice tečny v daném bodě:

Je dána $k(S[m,n],r)$ a $T[x_0,y_0]\in k$.
$X[x,y]$ je libovolným bode tečny $t$.

$\ve{ST} = (x_0-m,y_0 - n)$\\
$\ve{SX} = (x  -m,y   - n)$\\

Pro každý bod $X\neq T$ existuje pravoúhlý trojúhelník $STX$,
přitom $|ST| = r$ a $|SX| \cos \alpha = r$, tedy $\ve{ST} = \ve {SX} = |ST|\*|SX|\*cos\alpha = r^2$.
oKaždá tečna kružnice má tedy rovnici:
$$(x_0 - m)(x-m) + (y_0-n)(y-n) = r^2 $$

\V Rovnice $(x_0 - m) (x-m) + (y_0 - n) (y-n)$ je analitickým řešením tečny kružnice $(x-m)^2 + (y-n)^2 = r^2$ v jejím bodě $[x_0,y_0]$.

\Pr Je dána kružnice $k$ s rovnicí $(x-3)^2 + (y+12)^2 = 100$ a body $L[9;-4]$ a $M[5;2]$.
Určete tečny ke $k$ procházející $L$ resp $M$.

\begin{itemize}
	\item Daosazením $L$ do rovnice $k$ zjistime, že $l\in k$.
		Tedy tečna $l:(9-3)(x-3) + (-4+12)(y+12) = 100$.
		Po úpraavě: $3x+4y-11=0$

		Dosazením $M$ do $k$ zjistíme, že $M$ leží ve vnějsí oblasti.

		Hledám bod dotyku:\\
		$T \in k \imp (x_0-3)^2 (y_0+12)^2 = 100$\\
		$T\in t \imp (x_0-3)(x-3) + (y_0-12)(y+12) = 100$\\
		$M \in t \imp (x_0-3)(5-3) + (y_0+12)(2+12) = 100$

		$(x_0-3)^2 + (y_0+12)^2 = 100$\\
		$x_0+7y+0  + 31 = 0$\\

		Řešení: $[-3;-4]$ a $[11,-6]$.

		$$t_1: -6(x-3) + 8(y+12) = 100 \dots -3x+4y+7=0$$
		$$t_1: -8(x-3) + 6(y+12) = 100 \dots 7x+3y-26=0$$
\end{itemize}

\Pr 217/16:
$x = \f{4y+7}3$

$(\f{4y+7}3-3)^2 + (y+12)^2 = 100$\\
$\f{25}9 y^2 + \f{200}9 t + \f{400}{9} =0$\\
$ D  = 200^2 - 4 \* 25 \* 400 = 0 $

Je tečnou

$x = \f{-4y+26}3$

$(\f{-4y+26}3-3)^2 + (y+12)^2 = 100$\\
$\f{25}9 y^2 + \f{80}9 t + \f{1859}{9} =0$\\
$ D  = 80^2 - 4 \* 25 \* 1859 = -179500 $
 
 Není tečnou

 \Pr 218/19:

 \begin{enumerate}
	 \item $(x+2)^2 + (y-2)^2 = 5$
		 $(x+2)(3+2) + (y+2)(7+2) = 5 \imp 5x+5y-5 = 0 \imp y = 1-x$

		 $(x+2)^2 + (1-x-2)^2 = 5 \imp 2x^2+6x=0 \imp x_1 = 0 \land x_2  = -3$

		 $T_1[0,1];T_2[-3,r4]$

		 $\ve u= (3,6); \ve v = (6,3)$

		 $\alpha = \arccos \f{|3\*6+6\*3|}{\sqrt{3^2 + 6^2}\*\sqrt{6^2+3^2}} = \arccos \f 45 = 36.87 \d$
 \end{enumerate}




\EndDoc
