\providecommand{\HINCLUDE}{NE}
\if ^\HINCLUDE^
\else
\def\HINCLUDE{}
\global\newdimen\Okraje
\global\Okraje =4cm
\input{$HOME/souteze/_hlavicka/h-.tex}

%\definecolor{colorV}{RGB}{255,127,0}
%\definecolor{colorPoz}{RGB}{153,51,0}
%\definecolor{orangeV}{RGB}{255,127,0}
%\definecolor{colorPr}{RGB}{0,5,255}
%\definecolor{colorDef}{RGB}{0.255,0}

\usepackage[shortlabels]{enumitem}
\setlength{\marginparsep}{2pt}
\setlength{\marginparwidth}{35pt}

\def\st{{\rm st}}
\def\P{{\rm P}}

\def\ISENUM{}
\def\inMargin#1{\End
		
		\hskip0pt \marginpar{{{#1}}}}
\newcounter{V}[section] 
\newcommand{\V}[1][]{\stepcounter{V}\inMargin{\textcolor{green}{V.\arabic{section}.\theV.:}}\ifx^#1^\else\textcolor{green}{\underline{{#1}:}}\addcontentsline{toc}{subsubsection}{V.\arabic{section}.\theV.:$\quad$ {#1}}\\\fi}
\def\Def{\inMargin{\textcolor{red}{Def:}}}
\def\Poz{{\inMargin{\textcolor{brown}{Pozn:}}}}
\def\Pr{{\inMargin{\textcolor{blue}{Př:}}}}
\def\Pozenum
{
	\begin{enumerate}[1)]%, left = 0pt ]
		\item\inMargin{\textcolor{brown}{Pozn:}}\def\ISENUM{a}}
\def\End
{
	\if	^\ISENUM^
	\else \end{enumerate}
	\fi
	\def\ISENUM{}
}
\reversemarginpar

\makeatletter
\renewcommand\thesection{§\arabic{section}.}
\renewcommand\thesubsection{\Alph{subsection})}
\renewcommand\thesubsubsection{\alph{subsubsection})}
\newcounter{chapter}
\setcounter{chapter}{0}
\renewcommand\thechapter{\Alph{chapter})}
\newcounter{roman}
\setcounter{roman}{0}
\renewcommand\theroman{\Roman{roman}.}
\makeatother
\def\sectionnum#1{\setcounter{section}{#1}\addtocounter{section}{-1}}
\def\subsectionnum#1{\setcounter{subsection}{#1}\addtocounter{subsection}{-1}}
\def\subsubsectionnum#1{\setcounter{subsubsection}{#1}\addtocounter{subsubsection}{-1}}
\def\chapternum#1{\setcounter{chapter}{#1}\addtocounter{chapter}{-1}}
\def\chapter#1{

	\addtocounter{chapter}{1}\sectionnum{1}
	\addcontentsline{toc}{section}{\large{\thechapter$\quad${#1}}}
	
	{\LARGE  \textbf{\begin{minipage}[t]{0.1\textwidth}\thechapter\end{minipage}\begin{minipage}[t]{0.95\textwidth}#1\end{minipage}}}

}
\def\ROM{}
\def\Rom#1#2{\setcounter{roman}{#1}\renewcommand\ROM{#2}}

\Rom{6}{Funkce}
\title{\Huge\textbf{\theroman\quad \ROM}}
\author{Jiří Kalvoda}

\newcounter{countOfBegin}
\setcounter{countOfBegin}{0}
\newcommand{\BeginDoc}[1][]
{
	\ifnum\value{countOfBegin}=0
	\begin{document}
		#1
		\fi
	\addtocounter{countOfBegin}{1}
		
}
\def\EndDoc
{
	\addtocounter{countOfBegin}{-1}
	\ifnum\value{countOfBegin}=0
	\end{document}
	\fi
}

\fi
\BeginDoc{}
\section{Kulová plocha, koule}

\Def
Nechť je dán bod $S$ a kladné reálné číslo $r$.
\emph{Kulovou plochu se středem $S$ a poloměrem $r$} nazýváme množinu všech bodů $X$ prostoru, pro které platí $|SX| = r$
\emph{Koulí se středem $S$ a poloměrem $r$} se nazývá množina všech bodů $X$ prostoru, pro která platí, že $|SX| \le r^2$.

\Poz Při analitickém vyjadřování těchto útvarů uplatníme ekvivalenci charakteristických vlastností:
$$ |SX| = r \Ekv  |SX|^2 = r^2 $$
$$ |SX|\le r \Ekv  |SX|^2\le r^2 $$
\V Má-li bod $S$ souřadnice $[m,n,q]$ a bod $X[x,y,z]$, pak kulová plocha $k(S,r)$
je analyticka vyjádřena rovnicí:
$$ (x-m)^2 + (y-n)^2 + (z-q)^2 = r^2 $$
a koule $K(S,r)$ nerovnicí:
$$ (x-m)^2 + (y-n)^2 + (z-q)^2 \le r^2 $$

\Poz
Úloha o vzájemné poloze přímky a kulové plochy povede obdobným způsobem ke kvadratické rovnici. Může tedy mít následující typy výsledků:
\begin{itemize}
	\item prázdná množina
	\item jednobodová množina
	\item dvoubodová množina
\end{itemize}

\Pr
Určete hodnotu parametru $k$, pro kterou má rovina $\rho : 2x-3y+z+k = 0$ neprázdný průnik s kulovou plochou $k$ středu $S[2;-3;1]$ a poloměru 3.

$\rho \cap k \neq \emptyset \Ekv \rho(S,\rho) \le 3$

Nerovnici vyjádřím analiticky:
\begin{eqnarray*}
	\f{|2\*2-3(-3)+1\*1+k|}{\sqrt{4+9+1}} &\le& 3\\
	|14+k| &\le& 3\* 14 
\end{eqnarray*}

Hledané hodnoty parametru $k$ patří do intervalu $\< - 14 - 3 \sqrt{14};-14+3\sqrt{14} \>$.

\Poz
Jak víme ze stereometrie, tečnou rovinu kulové plochy můžeme charakterizovat kteroukoliv z těchto tří vlastností:
\begin{itemize}
	\item Má s kulovou plochou jednobodový průnik
	\item Má od středu vzdálenost rovnou poloměru
	\item je kolmá k poloměru kulové plochy obsahující bod dotyku
\end{itemize}

Bod $X$ je libovolným bodem roviny $\tau, X \neq T$. Z rovnosti $\ve{ST} \* \ve {SX} = r^2$ můžeme odvodit rovnici tečné kruhové roviny $\tau$ kulové plochy $K$ zcela obdobně jako rovnici tečny kružnice.

\V
Rovnice $(x_0-m)(x-m) + (y_0-n)(t-n) + (z_0-q)(z-q) = r^2$
je analytickým vyjádřením tečné roviny kulové plochy $K:(x-m)^2 + (y-n)^2 + (z-q)^2 = r^2$ v jejím bodě $T[x_0,y_0,z_0]$.

\Poz Úpravy obecné rovnice kulové plochy na  středový tvar se provádí stejnými kroky jako úpravy rovnice kružnice.
Platí také, že ne každá rovnice typu $x^2 + y^2 + z^2 + ax + by + cz + d = 0$ vyjadřuje kulovou plochu.

\Pr 222/21:
$\ve{SA} = (0,-2,3) \imp |\ve{SA}| = \sqrt {13}$
$\ve{SB} = (0,-5,3) \imp |\ve{SA}| = \sqrt {34}$
$\ve{SC} = (5,0,4) \imp |\ve{SA}| = \sqrt {41}$

$r=\sqrt{41}$

$$ (x-1)^2 + (y-2)^2 + (z+3)^2 \le 41 $$

\Pr 22/24:
$ (1-t-1)^2 + (2+t)^2 + (3-2t-1)^2 = r^2 $\\
$ (1-t)^2 + (2+t)^2 + (3-2t-6)^2 = r^2 $

$ (1-t-1)^2 + (2+t)^2 + (3-2t-1)^2 = 
 (1-t)^2 + (2+t)^2 + (3-2t-6)^2  $\\
 $6t^2-4t+8 = 6t^2 + 14t + 14$\\
 $18t=6$\\
 $t=-\f 1 3 $

$S\[\f 2 3; \f 7 5 ; \f 7 5 \]$\\
$ r^2 = 10$

$$ \(x-\f 2 3\)^2 + \(y-\f 75\)^2 + \(z-\f 7 5\)^2 = 10$$

\Pr 222/25:

$\ve {AB} = (1;-2;2)$

$\pri{AB} = \zs{[4+t;5-2t;6+2t]|t\in\R}$

$$ (4+t-2)^2 + (5-2t-3)^2 + (6+2t-5)^2 = 36$$
$$ 9t^2 - 27 = 0$$
$t=\pm \sqrt 3$

\begin{enumerate}
	\item $\emptyset$
	\item $\zs{[4+\sqrt3;5-2\sqrt 3;6+2\sqrt 3]}$
	\item $\zs{[4-\sqrt3;5+2\sqrt 3;6-2\sqrt 3];[4+\sqrt3;5-2\sqrt 3;6+2\sqrt 3]}$
\end{enumerate}
\EndDoc
