\providecommand{\HINCLUDE}{NE}
\if ^\HINCLUDE^
\else
\def\HINCLUDE{}
\global\newdimen\Okraje
\global\Okraje =4cm
\input{$HOME/souteze/_hlavicka/h-.tex}

%\definecolor{colorV}{RGB}{255,127,0}
%\definecolor{colorPoz}{RGB}{153,51,0}
%\definecolor{orangeV}{RGB}{255,127,0}
%\definecolor{colorPr}{RGB}{0,5,255}
%\definecolor{colorDef}{RGB}{0.255,0}

\usepackage[shortlabels]{enumitem}
\setlength{\marginparsep}{2pt}
\setlength{\marginparwidth}{35pt}

\def\st{{\rm st}}
\def\P{{\rm P}}

\def\ISENUM{}
\def\inMargin#1{\End
		
		\hskip0pt \marginpar{{{#1}}}}
\newcounter{V}[section] 
\newcommand{\V}[1][]{\stepcounter{V}\inMargin{\textcolor{green}{V.\arabic{section}.\theV.:}}\ifx^#1^\else\textcolor{green}{\underline{{#1}:}}\addcontentsline{toc}{subsubsection}{V.\arabic{section}.\theV.:$\quad$ {#1}}\\\fi}
\def\Def{\inMargin{\textcolor{red}{Def:}}}
\def\Poz{{\inMargin{\textcolor{brown}{Pozn:}}}}
\def\Pr{{\inMargin{\textcolor{blue}{Př:}}}}
\def\Pozenum
{
	\begin{enumerate}[1)]%, left = 0pt ]
		\item\inMargin{\textcolor{brown}{Pozn:}}\def\ISENUM{a}}
\def\End
{
	\if	^\ISENUM^
	\else \end{enumerate}
	\fi
	\def\ISENUM{}
}
\reversemarginpar

\makeatletter
\renewcommand\thesection{§\arabic{section}.}
\renewcommand\thesubsection{\Alph{subsection})}
\renewcommand\thesubsubsection{\alph{subsubsection})}
\newcounter{chapter}
\setcounter{chapter}{0}
\renewcommand\thechapter{\Alph{chapter})}
\newcounter{roman}
\setcounter{roman}{0}
\renewcommand\theroman{\Roman{roman}.}
\makeatother
\def\sectionnum#1{\setcounter{section}{#1}\addtocounter{section}{-1}}
\def\subsectionnum#1{\setcounter{subsection}{#1}\addtocounter{subsection}{-1}}
\def\subsubsectionnum#1{\setcounter{subsubsection}{#1}\addtocounter{subsubsection}{-1}}
\def\chapternum#1{\setcounter{chapter}{#1}\addtocounter{chapter}{-1}}
\def\chapter#1{

	\addtocounter{chapter}{1}\sectionnum{1}
	\addcontentsline{toc}{section}{\large{\thechapter$\quad${#1}}}
	
	{\LARGE  \textbf{\begin{minipage}[t]{0.1\textwidth}\thechapter\end{minipage}\begin{minipage}[t]{0.95\textwidth}#1\end{minipage}}}

}
\def\ROM{}
\def\Rom#1#2{\setcounter{roman}{#1}\renewcommand\ROM{#2}}

\Rom{6}{Funkce}
\title{\Huge\textbf{\theroman\quad \ROM}}
\author{Jiří Kalvoda}

\newcounter{countOfBegin}
\setcounter{countOfBegin}{0}
\newcommand{\BeginDoc}[1][]
{
	\ifnum\value{countOfBegin}=0
	\begin{document}
		#1
		\fi
	\addtocounter{countOfBegin}{1}
		
}
\def\EndDoc
{
	\addtocounter{countOfBegin}{-1}
	\ifnum\value{countOfBegin}=0
	\end{document}
	\fi
}

\fi
\BeginDoc{}
\section{Parabola}
\Poz Paraboly známe jako grafy kvadratických funkcí $y=ax^2 + bx + c$, kde $a\neq 0$.
Graf každé kvadratick funkce lze získat vhodným posunutím grfu $y=ax^2$.
K těmto poznatkům dojdeme opět znovu uplatněním obecného analitického vyjádření parabol.

Geometrická definice paraboly pomocí vzdáleností jejíc bodů od dané přímky a daného bodu.

\Def
Nechť je dána přímka $d$ a bod $F\not \in d$.
Množinu všech bodů $X$ rovniny $dF$, pro která platí $\rho (X,d) = |FX|$, nazýváme \emph{parabola s ohnismek $F$ a řídící přímkou $d$.}
Označme ji $P(F,d)$.

\Poz
\pdf[0.7]{8-1.pdf}
$F$ -- ohnisko\\
$d$ -- řídící přímka\\
$o$ -- osa\\
$V$ -- vrchol

\Poz Vliv $\rho(F,d)$ na tvar paraboly budeme zkoumat pomocí sítě na obrázku, kde jsou ke ružnici se středem $F$ přípsaná čísla udávající vzdálenost jejich bodů od $F$.

Návod k práci s obrázkem:
\begin{enumerate}
	\item Položte průsvitku a vyznačte $F$ a jednu z přímek jako $d$.
		Na prímky v polorovniě $\ve{dF}$ připište vzdálenosti od $d$.
	\item Vyznačte body, které leží na kružnici a přímce se steným číslem.
	\item Zhustěte síť a opakujte
	\item spojete body hladkým obloukem.
\end{enumerate}
\pdf[0.5]{8-2.pdf}

\Poz Odvozujeme analitické vyjádření parabol $P(F,d)$ pro $V[0;0]$ a osu $y$:
Pomocí vzorců pro vzdálenosti vyjádřime charakteristicku vlastnost bodů parabol.

\begin{eqnarray*}
	\rho(X,d) &=& |FX|\\
	|y\pm 0.5 q| &=& \sqrt{x^2 + (y-0.5q)^2}\\
	y^2\pm qy + 0.25 0.5 q^2 &=& x^2 + (y-0.5q)^2\\
	y^2 \pm qy + 0.25 q^2 &=& x^2 + y^2 \mp qy + 0.25 q^2\\
	\pm 2qy &=& x^2
\end{eqnarray*}

Snadno ověríme, že každý bod $X$, který svými souřadnicemi splňuje poslední rovnici, má charakteristickou vlastnost bodu paraboly.

\V 
Každá parabola $P(F,d)$, která má vrchol $V[0;0]$ a svou osu v ose $y$, má analitické vyjádření $2py = x^2$, přitom $F[0;2.5p], d:y=-0.5p$.

[Dokázáno výše]

\V Každá parabola, která má rovnici $2py = x2$, kde $a\neq 0$ je grafem kvadratické funkce $y=sx^2$.
Zároveň graf každé kvadratické funkce $y=ax^2$ je parabolou o rovnici $2py=x^2$, přitom $p = \f 1{2a},a=\f1{2p}$.

[Dk: plyne z toho, že rovnice $y=ax^2, 2py = x^2$ jsou ekvivalentní při uvedeném vzttau mezi $a,p$.].

\Pr Určete rovnici všech parabl $P$, které mají osu rovnoběžnou s osou $x$ a procházi body $A[-4;-2];B[4;2];C[2;4]$.

Z podmínky rovnoběžnosti plyne, že se jedná o kvadratickou funkci $x = f(y)$, tedy $x = a y^2 + by + c$.

Dosadíme:
\begin{eqnarray}
	-4 &=& 4a -2 b +c \\
	 4 &=& 4a +2 b +c\\
	 2 &=& 16a + 4b +c\\
\end{eqnarray}

$ 8 =  4b \imp 2 = b$\\
$ 4a = - c$\\
$ 2 =  - 4c + 8 + c \imp 3c = 6 \imp c = 3  \imp a  = - \f 3 4$

$$ \f 3 4 y^2 + 2 y + 3 = x$$

\Pr 239/6:
\begin{enumerate}[a)]
	\item $V[3,0]$, $q = 2 \imp 4(y)= (x-3)^2$y
	\item $V[-3,-6]$, $q = -8 \imp -16(y+6)= (x+3)^2$y
	\item $V[4,1]$, $q = -6 \imp -12(y-1)= (x-4)^2$y
	\item $V[-2,\f 1 2]$, $q = -3 \imp -6(y-\f 12)= (x+2)^2$
\end{enumerate}

\Pr 239/8:
$y = \pm x^2 \imp a = 1\imp q = \f 1 2 \imp p: y = \mp \f 1 4 \land F[0,\pm \f 1 4]$

\begin{enumerate}[a)]
	\item $p:y = -\f 1 4$ $F[3;\f 1 4]$
	\item $p:y = 5-\f 1 4$ $F[0;5\f 1 4]$
	\item $p:y = 2+\f 1 4$ $F[0;2-\f 1 4]$
	\item $y-3 = (x-2)^2$
		$p:y = -\f 1 4$ $F[0;\f 1 4]$
	\item $ y - 13  =  - (x+2)^2 $
		$p:y = 13+\f 1 4$ $F[-2;13-\f 1 4]$
	\item $y + 4 = -4(x-1)$
		$p:y = -4+\f 1 4$ $F[1;-4-\f 1 4]$
	%\item $p:y = -\f 1 4$ $F[0;\f 1 4]$
\end{enumerate}

\Pr 240/10:
\Pr $y  =  a x^2 + b x + c $

\begin{eqnarray}
	-2 = 16 a - 4 b + c\\
	2 = 16 a + 4 b + c \\
	4 = 4 a + 2 b  + c
\end{eqnarray}

$4 = 8 b \imp b = \f 12$\\
$-16a = c$\\
$4 = 4a +1 -16 a \imp 12 a = -3 \imp a = -\f 14 \imp c=4$
$$ y = - \f 14 x^2 + \f 12 x + 4$$
\EndDoc
