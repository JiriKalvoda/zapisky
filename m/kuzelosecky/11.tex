\providecommand{\HINCLUDE}{NE}
\if ^\HINCLUDE^
\else
\def\HINCLUDE{}
\global\newdimen\Okraje
\global\Okraje =4cm
\input{$HOME/souteze/_hlavicka/h-.tex}

%\definecolor{colorV}{RGB}{255,127,0}
%\definecolor{colorPoz}{RGB}{153,51,0}
%\definecolor{orangeV}{RGB}{255,127,0}
%\definecolor{colorPr}{RGB}{0,5,255}
%\definecolor{colorDef}{RGB}{0.255,0}

\usepackage[shortlabels]{enumitem}
\setlength{\marginparsep}{2pt}
\setlength{\marginparwidth}{35pt}

\def\st{{\rm st}}
\def\P{{\rm P}}

\def\ISENUM{}
\def\inMargin#1{\End
		
		\hskip0pt \marginpar{{{#1}}}}
\newcounter{V}[section] 
\newcommand{\V}[1][]{\stepcounter{V}\inMargin{\textcolor{green}{V.\arabic{section}.\theV.:}}\ifx^#1^\else\textcolor{green}{\underline{{#1}:}}\addcontentsline{toc}{subsubsection}{V.\arabic{section}.\theV.:$\quad$ {#1}}\\\fi}
\def\Def{\inMargin{\textcolor{red}{Def:}}}
\def\Poz{{\inMargin{\textcolor{brown}{Pozn:}}}}
\def\Pr{{\inMargin{\textcolor{blue}{Př:}}}}
\def\Pozenum
{
	\begin{enumerate}[1)]%, left = 0pt ]
		\item\inMargin{\textcolor{brown}{Pozn:}}\def\ISENUM{a}}
\def\End
{
	\if	^\ISENUM^
	\else \end{enumerate}
	\fi
	\def\ISENUM{}
}
\reversemarginpar

\makeatletter
\renewcommand\thesection{§\arabic{section}.}
\renewcommand\thesubsection{\Alph{subsection})}
\renewcommand\thesubsubsection{\alph{subsubsection})}
\newcounter{chapter}
\setcounter{chapter}{0}
\renewcommand\thechapter{\Alph{chapter})}
\newcounter{roman}
\setcounter{roman}{0}
\renewcommand\theroman{\Roman{roman}.}
\makeatother
\def\sectionnum#1{\setcounter{section}{#1}\addtocounter{section}{-1}}
\def\subsectionnum#1{\setcounter{subsection}{#1}\addtocounter{subsection}{-1}}
\def\subsubsectionnum#1{\setcounter{subsubsection}{#1}\addtocounter{subsubsection}{-1}}
\def\chapternum#1{\setcounter{chapter}{#1}\addtocounter{chapter}{-1}}
\def\chapter#1{

	\addtocounter{chapter}{1}\sectionnum{1}
	\addcontentsline{toc}{section}{\large{\thechapter$\quad${#1}}}
	
	{\LARGE  \textbf{\begin{minipage}[t]{0.1\textwidth}\thechapter\end{minipage}\begin{minipage}[t]{0.95\textwidth}#1\end{minipage}}}

}
\def\ROM{}
\def\Rom#1#2{\setcounter{roman}{#1}\renewcommand\ROM{#2}}

\Rom{6}{Funkce}
\title{\Huge\textbf{\theroman\quad \ROM}}
\author{Jiří Kalvoda}

\newcounter{countOfBegin}
\setcounter{countOfBegin}{0}
\newcommand{\BeginDoc}[1][]
{
	\ifnum\value{countOfBegin}=0
	\begin{document}
		#1
		\fi
	\addtocounter{countOfBegin}{1}
		
}
\def\EndDoc
{
	\addtocounter{countOfBegin}{-1}
	\ifnum\value{countOfBegin}=0
	\end{document}
	\fi
}

\fi
\BeginDoc{}
\section{Hyperbola}
\Def Mějme dány dva ruzné body $F,G$ a takové číslo $2a$, že $0<2a<|FG|$.
Množinu všech bodů roviny $X$, pro než platí $||fx|-|GX||=2a$ nazýváme i\emph{hyperbolou s ohnisy $FG$ a s hlavní osou $2a$.}
Stručně ji ozmačujeme $H(F,G,2a)$.
\emph{Větví hyperboly} nazýváme množinu všech bodů $X$ roviny, pro které platí $|FX| - |GX|   = 2a$.
Stejně jako množinu, pro kterou platí $|GX|  - |FX| = 2a$.

\Poz
Zvláštní tvar hyperboly působí nesnáze; hyperboly nemají na ose úsečky $FG$ zádné body, proto nemají vedlejší vrcholy.

Číslo $b>0$ se definuje jinak: Pomocí vztahu $b^2 = e^2 - a^2$ a stále se nazývá velikostí vedlejši poloosy.
Ale jelikož hyperbola nemá vedlejší vrcholy, nemá ani žádnou vedlejší poloosu.

\Pr
Odvoďte analitické vyjádření hyperboly $H(F,G,2a)$, jejíž ohniska $F,G,$ leží na ose $x$.


Zvolíme $F[-e;0];G[e;0];X[x,y]$, přitom $|FG| = 2e > 2a, b^2 = e^2 - a^2$.
Vyjádřím charakteristickou vlastnost jednoho bodu:

\begin{eqnarray*}
	||FX|-|GX|| &=& 2a\\
	|\sqrt{(x-e)^2+y^2}+\sqrt{(x+e)^2+y^2} & = & 2a\\
	\sqrt{(x^2+e^2+y^2) - 4e^2x^2} &=& (x^2+e^2+y^2) - 2a^2\\
	a^2x^2 + a^2 y^2 - e^2x^2 &=& a^4-a^2e^2\\
	x^2(a^2-e^2) + a^2y^2 &=& a^2(a^2-e^2)\\
	-b^2x^2 + a^2y^2  &=& -a^2-b^2\\
	\f{x^2}{a^2} - \f{y^2}{b^2} = 1
\end{eqnarray*}

Poslední rovnice je ekvivalentní s první, proto slouží jako analytické vyjádření hyperboly, jejíž osy leží na osách soustavy souřadnic a ohniska na ose $x$.

\Pr
Ověřte, zda pro každou dvojici $a,b$ kladných reálných čísel muže rovnice odvozená v příkladě 1 vyjadřovat hyperbolu s ohnisky $F,G$ na ose $x$ a s hlavní osou $2a$.

Je-li dána rovnice $\f{x^2}{s^2} - \f{y^2}{b^2}$, kde $a>0;b>0$, můžeme sestrojit $A_1[-a;0];A_2[a;0];K[a;b]$.
Přepona $OK$ udává $e>a$, kružnice $k(O,e)$ protíná osu $x$ v bodech $F[-e;];G[e;0]$. Snadno ověříme (viz předhozí příkjlad), že hyperbola $H(F,G,2a)$ má analitické vyjádření, které bylo dáno.

\Poz
Na rozdíl od elips nerozhoduje nerovnost mezi $a,b$ o tom, na které z os leži ohniska hyperboly.
O tom rozhoduje prohození členů rozdílu.

\Def Směry, jejichž každá přímka má s hyperbolou nejvýše jeden spolecný bod, se nazyvají \emph{asymptotické směry hyperboly}.
Přímky těchto směrů, které neobsahují žádný bod hyperboly, nazyváme \emph{asymptoty hyperboly}

\Def
U hyperboly používáme následující termíny:

\begin{minipage}{0.5\textwidth}
	\begin{tabular}{|c|l|}\hline
		$F,G$ & ohniska \\\hline
		$A_1,A_2$ & (hlavní) vrcholy\\\hline
		$S$ nebo $O$ & střed \\\hline
		$a$ & velikost hlavní poloosy\\\hline
		$b$ & velikost vedlejší poloosy \\\hline
		$e$ & výstřednost \\\hline
		$a_1,a_2$ & asymptoty\\\hline
	\end{tabular}
\end{minipage}
\begin{minipage}{0.5\textwidth}
	\pdf{11-1.pdf}
\end{minipage}

\Pr 
Najdět analytické vyjádření rovnoosých hyperbol $H(F,G,2a)$, jejicž kolmé asymptoty jsou rovnoběžné s $x,y$.

Pracujme s hyperbolami, které mají přímo osy $x,y$  jako asymptoty. Střed je tedy $[0;0]$ a $F,G$ mají na osách 1. a 3. kvadrantu.
Protože $a=b$, je $e = \sqrt{a^2 + b^2} = a\sqrt 2$.
Ohniska mají pak souřadnice $F[-a;-a];G[a,a];X[x,y]$.

\begin{eqnarray*}
	||FX|-|GX|| &=& 2a\\
	|\sqrt{(x+a)^2 + (y+a)^2} - \sqrt{(x-a)^2+(y-a)^2}| &=& 2a\\
	y &=& cx^{-1}
\end{eqnarray*}

\V
Uvažme $H(F,G,2a)$, které mají střed $S[m,n]$, excentricitu $e=\f{|FG|}{2}$, velikost vedlejší poloosy $b=e^2 - a^2$.
\begin{enumerate}
	\item Každá hyperbola, jejíž osa $FG$ je rvnoběžná s $x$ má právě jednu rovnici
		$$\f{(x-m)^2}{a^2} - \f{(y-n)^2}{b^2} = 1$$
	\item Každá hyperbola, jejíž osa $FG$ je rvnoběžná s $y$ má právě jednu rovnici
		$$\f{(y-m)^2}{a^2} - \f{(x-n)^2}{b^2} = 1$$
	\item Každá rovnoosá hyperbla, která má asymptoty rovoběžné s $x,y$ má právě jednu rovnici
		$$ 2 (x-m) (y-n) = a^2$$
\end{enumerate}
Každá z rovnic vyjadřuje právě jednu hyperbolu v poloze výše popsané.

\Pr 257/22:
\begin{enumerate}
	\item $\f{x^2}{4} - \f{y^2}{5} = 1$
	\item $\f{y^2}{9} - \f{x^2}7 = 1$
	\item $\f{x^2}7 - \f{y^2}9 = 1$
	\item $\f{y^2}{9} - \f{x^2}{16} = 1$
\end{enumerate}
\Pr 257/23:
\begin{enumerate}[a)]
	\item
		\begin{minipage}{0.4\textwidth}
			\pdf[0.3]{11-2.pdf}
		\end{minipage}
		\begin{minipage}{0.4\textwidth}
			$$
			\begin{array}{r@{\ [}c@{;}c@{]}}
				F & -2\sqrt 2 & 0 \\
				G & 2\sqrt 2 & 0 \\
				S& 0 & 0 \\
				A_1 & -2 & 0 \\
				A_2 & 2 & 0\\
			\end{array}
			$$
			$a: x=y$\\
			$a': x=-y$
		\end{minipage}
	\item[d)]
		\begin{minipage}{0.4\textwidth}
			\pdf[0.3]{11-3.pdf}
		\end{minipage}
		\begin{minipage}{0.4\textwidth}
			$$
			\begin{array}{r@{\ [}c@{;}c@{]}}
				F & 0 & -\sqrt {10} \\
				G & 0 & \sqrt{10} \\
				S& 0 & 0 \\
				A_1 & 0 & -3 \\
				A_2 & 0 & 3\\
			\end{array}
			$$
			$a: x=3y$\\
			$a': x=-3y$
		\end{minipage}
\end{enumerate}

\Pr 258/24:
		$p:y=2x+3$\\
		$\ppri{UV} = \zs{[3-t;t]|t\in\R^+_0}$
\begin{enumerate}[a)]
	\item $x^2-y^2 = 4$:

		$x^2 - (2x+3)^2 = 4$\\
		$0=3x^2+12x+13 $\\
		$D = 12^2 - 4\*3\*13 < 0$\\
		$$H\cap p=\emptyset$$

		$(3-t)^2 - t^2 = 4$\\
		$5-6t = 0$\\
		$t = \f 56$

		$$F\cap\ppri{UV} = \zs{\[\f{13}6;\f 56\]}$$

	\item[d)] $y^2 - 9x^2 = 9$
		$x^2 - 9(2x+3)^2 = 9$\\
		$0=35x^2+108x+90 $\\
		$D = 108^2 - 4\*35\*90 < 0$\\
		$$H\cap p=\emptyset$$

		$(3-t)^2 - 9t^2 = 9$\\
		$ 0 = 8t^2 + 6t$\\
		$t = 0 \lor t = -\f 3 4 < 0$
		$$F\cap\ppri{UV} = \zs{\[3;0\]}$$
\end{enumerate}


\EndDoc
