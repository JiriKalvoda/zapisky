\providecommand{\HINCLUDE}{NE}
\if ^\HINCLUDE^
\else
\def\HINCLUDE{}
\global\newdimen\Okraje
\global\Okraje =4cm
\input{$HOME/souteze/_hlavicka/h-.tex}

%\definecolor{colorV}{RGB}{255,127,0}
%\definecolor{colorPoz}{RGB}{153,51,0}
%\definecolor{orangeV}{RGB}{255,127,0}
%\definecolor{colorPr}{RGB}{0,5,255}
%\definecolor{colorDef}{RGB}{0.255,0}

\usepackage[shortlabels]{enumitem}
\setlength{\marginparsep}{2pt}
\setlength{\marginparwidth}{35pt}

\def\st{{\rm st}}
\def\P{{\rm P}}

\def\ISENUM{}
\def\inMargin#1{\End
		
		\hskip0pt \marginpar{{{#1}}}}
\newcounter{V}[section] 
\newcommand{\V}[1][]{\stepcounter{V}\inMargin{\textcolor{green}{V.\arabic{section}.\theV.:}}\ifx^#1^\else\textcolor{green}{\underline{{#1}:}}\addcontentsline{toc}{subsubsection}{V.\arabic{section}.\theV.:$\quad$ {#1}}\\\fi}
\def\Def{\inMargin{\textcolor{red}{Def:}}}
\def\Poz{{\inMargin{\textcolor{brown}{Pozn:}}}}
\def\Pr{{\inMargin{\textcolor{blue}{Př:}}}}
\def\Pozenum
{
	\begin{enumerate}[1)]%, left = 0pt ]
		\item\inMargin{\textcolor{brown}{Pozn:}}\def\ISENUM{a}}
\def\End
{
	\if	^\ISENUM^
	\else \end{enumerate}
	\fi
	\def\ISENUM{}
}
\reversemarginpar

\makeatletter
\renewcommand\thesection{§\arabic{section}.}
\renewcommand\thesubsection{\Alph{subsection})}
\renewcommand\thesubsubsection{\alph{subsubsection})}
\newcounter{chapter}
\setcounter{chapter}{0}
\renewcommand\thechapter{\Alph{chapter})}
\newcounter{roman}
\setcounter{roman}{0}
\renewcommand\theroman{\Roman{roman}.}
\makeatother
\def\sectionnum#1{\setcounter{section}{#1}\addtocounter{section}{-1}}
\def\subsectionnum#1{\setcounter{subsection}{#1}\addtocounter{subsection}{-1}}
\def\subsubsectionnum#1{\setcounter{subsubsection}{#1}\addtocounter{subsubsection}{-1}}
\def\chapternum#1{\setcounter{chapter}{#1}\addtocounter{chapter}{-1}}
\def\chapter#1{

	\addtocounter{chapter}{1}\sectionnum{1}
	\addcontentsline{toc}{section}{\large{\thechapter$\quad${#1}}}
	
	{\LARGE  \textbf{\begin{minipage}[t]{0.1\textwidth}\thechapter\end{minipage}\begin{minipage}[t]{0.95\textwidth}#1\end{minipage}}}

}
\def\ROM{}
\def\Rom#1#2{\setcounter{roman}{#1}\renewcommand\ROM{#2}}

\Rom{6}{Funkce}
\title{\Huge\textbf{\theroman\quad \ROM}}
\author{Jiří Kalvoda}

\newcounter{countOfBegin}
\setcounter{countOfBegin}{0}
\newcommand{\BeginDoc}[1][]
{
	\ifnum\value{countOfBegin}=0
	\begin{document}
		#1
		\fi
	\addtocounter{countOfBegin}{1}
		
}
\def\EndDoc
{
	\addtocounter{countOfBegin}{-1}
	\ifnum\value{countOfBegin}=0
	\end{document}
	\fi
}

\fi
\BeginDoc{}
\section{ Analytické vyjádření obrazu útvaru}
\V Předpokládejme, že posunutí $T$ je dáno rovnicí $x' = x+m, y'=y+n$ a zobrazuje $U$ na $U'$ Potom platí:

Má li útvar $U$ rovnici $V(x,y)=0$, pak útvar $U'$ má v téže soustavě souřadnic
rovnici $V(x-m,y-n) = 0$.

[Dk: Jde o rovnici, v níž dvojčleny $x-m,y-n$ nahrazují $x,y$ na všech místech, kde se vyskytují $x,y$ v původní rovnici útvaru $U$.]

\Poz
Aplikace předchozí věty:
\begin{enumerate}
	\item Přímka, která prochází počátkem a má směrnici $k$, má také velmi jednoduchou rovnici $y=kx$\\
		Přímka, která prochází bodem $[x_1,y_1]$ a má směrnici $k$ se dá považovat za obraz první v posunutí $x'=x+x_1, y'=y+y_1$, má rovnici  $y-y_0 = k(x-x_0)$.
	\item Kružnice, která má střed v počátku a poloměr $r$, má rovnici $x^2 + y^2 = r^2$.\\
		Kružnice se středem $S[m,n]$ a poloměrem $r$ se dá považovat za obraz první kružnice v posunutí
		$x'=x+m,y' = y+n$ a má rovnici $(x-m)^2 + (x-n)^2 = r^2$.

	\item Tečna kružnice $k(O,r)$ v jejím bodě $T[x_0,y_0]$ má rovnici $xx_0+yy_0 = r^2$.
		Tečna kružnice $k(S,r)$ je obrazem tečny první kružnice v posunutí $x'=x+m,y'=y+n$ a má rovnici
		$(x-m)(x_0-m) + (y-n)(y_0-n) = r^2$.
\end{enumerate}

\V
Má li útvar $U$ rovnici $V(x,y)=0$, pak jeho obraz $U'$ v souměrnosti, která vyměňuje kladné poloosy $x,y$ má rovnici $V(y,x) = 0$.

[Dk: Jde o rovnici, v níž je každé $x$ nahrazeno za $y$ a $x$ je nahrazeno za $y$.]

\Pr
Kružnice $k(S,r)$ zobrazíme v souměrnosti podle osy 1. kvadrantu na kružnici $k'(S',r)$;
tečna $t$ kružnice v bodě $T$ se zobrazí na tečnu $t'$ kružnice $k'$ v bodě $T'$.
Určete rovnici kružnice $k'$ a tečny $t'$, je-li dáno:
$$ S[-3,6], r = 5, T[0;10]$$

Umíme zapsat rovnici kružnice $k$: $(x+3)^2 + (y-6)^2 = 25$ a rovnice tečny $t$ v jejím bodě $T$:
$(x+3)3 + (y-6) 4  25$, tj $3x+4y-40=0$

Zobrazením v osové souměrnosti získáme $S'[6;-3],T'[10;0]$.
Rovnice kružnice $k'$: $(x-6)^2 + (y+3)^2 = 25$.
rovnice tečny $t'$ v jejím bodě $T'$: $4x+3y - 0 =0$.

\Pr 233/2:\\
Pro rovnici $ax+by = 0$ je posunutí $a(x-x_1)+b(y-y_1) = 0$.

\Pr 233/4:\\
Má li útvar $U$ rovnici $V(x,y,z)=0$, pak v posunutí $[a,b,c]$ má rovnici $V(x-a,y-b,z-c) = 0$.


\EndDoc
