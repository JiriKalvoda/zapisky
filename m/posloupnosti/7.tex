\providecommand{\HINCLUDE}{NE}
\if ^\HINCLUDE^
\else
\def\HINCLUDE{}
\global\newdimen\Okraje
\global\Okraje =4cm
\input{$HOME/souteze/_hlavicka/h-.tex}

%\definecolor{colorV}{RGB}{255,127,0}
%\definecolor{colorPoz}{RGB}{153,51,0}
%\definecolor{orangeV}{RGB}{255,127,0}
%\definecolor{colorPr}{RGB}{0,5,255}
%\definecolor{colorDef}{RGB}{0.255,0}

\usepackage[shortlabels]{enumitem}
\setlength{\marginparsep}{2pt}
\setlength{\marginparwidth}{35pt}

\def\st{{\rm st}}
\def\P{{\rm P}}

\def\ISENUM{}
\def\inMargin#1{\End
		
		\hskip0pt \marginpar{{{#1}}}}
\newcounter{V}[section] 
\newcommand{\V}[1][]{\stepcounter{V}\inMargin{\textcolor{green}{V.\arabic{section}.\theV.:}}\ifx^#1^\else\textcolor{green}{\underline{{#1}:}}\addcontentsline{toc}{subsubsection}{V.\arabic{section}.\theV.:$\quad$ {#1}}\\\fi}
\def\Def{\inMargin{\textcolor{red}{Def:}}}
\def\Poz{{\inMargin{\textcolor{brown}{Pozn:}}}}
\def\Pr{{\inMargin{\textcolor{blue}{Př:}}}}
\def\Pozenum
{
	\begin{enumerate}[1)]%, left = 0pt ]
		\item\inMargin{\textcolor{brown}{Pozn:}}\def\ISENUM{a}}
\def\End
{
	\if	^\ISENUM^
	\else \end{enumerate}
	\fi
	\def\ISENUM{}
}
\reversemarginpar

\makeatletter
\renewcommand\thesection{§\arabic{section}.}
\renewcommand\thesubsection{\Alph{subsection})}
\renewcommand\thesubsubsection{\alph{subsubsection})}
\newcounter{chapter}
\setcounter{chapter}{0}
\renewcommand\thechapter{\Alph{chapter})}
\newcounter{roman}
\setcounter{roman}{0}
\renewcommand\theroman{\Roman{roman}.}
\makeatother
\def\sectionnum#1{\setcounter{section}{#1}\addtocounter{section}{-1}}
\def\subsectionnum#1{\setcounter{subsection}{#1}\addtocounter{subsection}{-1}}
\def\subsubsectionnum#1{\setcounter{subsubsection}{#1}\addtocounter{subsubsection}{-1}}
\def\chapternum#1{\setcounter{chapter}{#1}\addtocounter{chapter}{-1}}
\def\chapter#1{

	\addtocounter{chapter}{1}\sectionnum{1}
	\addcontentsline{toc}{section}{\large{\thechapter$\quad${#1}}}
	
	{\LARGE  \textbf{\begin{minipage}[t]{0.1\textwidth}\thechapter\end{minipage}\begin{minipage}[t]{0.95\textwidth}#1\end{minipage}}}

}
\def\ROM{}
\def\Rom#1#2{\setcounter{roman}{#1}\renewcommand\ROM{#2}}

\Rom{6}{Funkce}
\title{\Huge\textbf{\theroman\quad \ROM}}
\author{Jiří Kalvoda}

\newcounter{countOfBegin}
\setcounter{countOfBegin}{0}
\newcommand{\BeginDoc}[1][]
{
	\ifnum\value{countOfBegin}=0
	\begin{document}
		#1
		\fi
	\addtocounter{countOfBegin}{1}
		
}
\def\EndDoc
{
	\addtocounter{countOfBegin}{-1}
	\ifnum\value{countOfBegin}=0
	\end{document}
	\fi
}

\fi
\BeginDoc{}
\def\posloup{$\zs{a_n}_{n=1}^{\infty}$}
\def\pos#1{\zs{#1}_{n=1}^{\infty}}
\def\li{\lim_{n\rightarrow\infty}}
\def\sup{{\rm sup\ }}
\def\sciwinfup{{\rm inf\ }}
\def\su{\sum_{n=1}^{\infty}}
\section{Nekonečné řady}
\Def Nechť $\pos{a_n}$ je posloupnost reálých čísel. Číslo
$$ S_n = a_1 + a_2 + a_3 + \dots a_n = \sum_{i=1}^n a_i; \ \ \  \ \ n\in\N$$

Nazývejme \emph{$n$-tým částečným součtem posloupnosti} \posloup.
Posloupnost $\pos{S_n}$ nazýváme \emph{posloupností částečných součtů řady $\sum_{n=1}^{\infty} a_n$} (\emph{posloupnosti $\pos{a_n}$}).


\emph{Nekonečnu řadu} (číselnou) nazýváme posloupnost částečných součtů
$\pos{S_n}$ a značíme stručně:
$$ \sum_{n=1}^\infty a_n $$
Čísla $a_n; n\in\N$ nazýváme \emph{členy řady}, čísla $S_n;n\in\N$ nazýváme \emph{částečné součty řasy}.

\Poz Symbolem $\sum_{n=1}^{\infty}$ označujeme jak nekonečnou řadu, tak její součet.

\Def Má-li posloupnost částečných součtů $\pos{S_n}$ limitu $S$, řekneme, že
\emph{nekonečná řada konverguje} a číslo $S$ nazveme jejím součtem (zapisujeme
$\sum_{n=1}^{\infty} a_n = S$).

Je-li $\pos{S_n}$ posloupností divergentní, řekneme, že \emph{nekonečná řada diverguje}.

\Pozenum
\emph{Chováním řady} budeme rozumět to, zda řada konverguje či diverguje.
\item Někdy se rozlišují 2 možnosti divergence řady (posloupnosti):
	\begin{enumerate}
		\item \emph{Řada diverguje}, jestliže má nevlastní limitu
			($\li S_n = \pm \infty$).
		\item \emph{Řada osciluje}, jestliže nemá vlastní ani nevlastní limitu.
	\end{enumerate}
\End
\Pr Určete chování řady:

$$\sum_{n=1}^{infty} a_n = 1 + 2 + 3 + 0 + 0 + 0 +\dots + 0 + \dots = 6$$
$S_1 = 1;S_2=3;S_3=6;S_4=6;\dots$\\
$\li S_n = S = 6$ řada konverguje.

\Pozenum
Je-li posloupnost $\pos{a_n}$ divergentní, pak také řada $\su a_n$ je divergentní.
\item Je-li posloupnost $\pos{a_n}$ konvergentní, může být řada $\su a_n$ je divergentní i kovnergentní.
\End

\V Nechť je řada $\su a_n$ konvergentní, pak $\li a_n = 0$.

\Poz Jedná se pouze o padmínku nutnou nikoliv dostačující.

\V Dvě řady lišící se pouze v \emph{konečném} počtu členů se chovají stejně (i když mohou mít
jiné částečné součty i jiný součet), zejména se chování řady nezmění, jestliže z ní
vyškrtneme konečně mnoho členů, přidáme k ní konečně mnoho členů nebo změníme
konečně mnoho členů.

\Pr $$\su 0 = 0$$

$$\sum \f 1 {n(n+1)}$$

$S_n = \f{1}{1\*2}+\f 1{2\*3} + \f 1 {3\*4} + \dots + \f 1{n(n+1)}$

$S_n = \(1-\f 1 2\) + \(\f 12 - \f 13\) + \(\f 1 3 - \f 14\) + \cdots + \(\f 1n -\f 1 {n+1}\)$

$S_n = 1 - \f  1 {n+1}$

$\li S_n = 1$

\V
\begin{enumerate}
	\item Jestliže konvergují řady $\su a_n,\su b_n$. pak konvergují i řady $\su (a_n+b_n)$ a $\su (a_n-b_n)$ a platí:
		$$\su \(a_n\pm b_n\) = \su a_n \pm \su b_n$$
	\item Jestliže konverguje řada $\su a_n$. pak konvergují i řada $\su (a_n+b_n)$ a $\su (a_n-b_n)$ a platí:
		$$\su \(a_n\pm b_n\) = \su a_n \pm \su b_n$$
\end{enumerate}
\Pozenum Jestliže z řad $\su a_n;\su b_n$ konverguje pouze jedna, pak řada $\su(a_n+b_n)$ nekonverguje, chová se jako druhá řada.
\item Nekonverguje-li žádná z řad $\su a_n$, $\su b_n$ může jejich součet konvergovat.
	(Např. $\su A_n = \su n, \su b_n = \su (-n)$ -- jejich součet konverguje k nule.)
\End

\Pr Určete chování řady:
\begin{enumerate}
	\item $\su a_n = 1+1+1+1+1+\cdots$

		$S_i = i$  $\li S_n = \infty $ řada diverguje.

	\item $\su a_n = 1-1+1-1+1-1+\dots$
		$S_1 = 1;S_2 = 0 ;S_3 = 1;S_4 = 0;\dots$
		řada osciluje
\end{enumerate}

\Pr Určete součt řady:
$$\su \f{1}{3n-2)(3n+1)} = \su \(\f A{3n-2} + \f B{3n+1}\)$$

$1=A(3n+1)+B(3n-2)$\\
$n^0: 1 = A-2B$\\
$n^1: 0=3A+3B$

$A=\f 13 ; B = -\f13$

$$\su \f{1}{3n-2)(3n+1)} = \su \(\f 1{9n-6} - \f 1{9n+3}\) =$$$$= \f 1 3 \su \f 1 {3n-2} - \f 1 3 \su \f1{3n+1} = \f 13 \(1+\f14 + \f 17+\cdots - \f 14+\f 17 - \cdots\) = \f 13 $$

\Pr Určete chování řady:
$$\sum (-1)^k = 1-1+1-1+1-1+1-1+1-1+\dots$$

Jevidentně rouzdíl každých dvou po sobě jdoucích členů v posloupnosti součtů je $\pm 1$, tedy řada evidentně osciluje.

\Pr Dokažtte, že aritmetrická posloupnost s $d\neq 0$ diverguje.

Pro dostatečně vysoké $n_0<n$ evidentně platí $a_n \not\in (-1;1)$, tedy $\li a_n \neq 0$, což je nutnou podmínkou konvergence.

\Pr $$\f 3{1\*2\*3}+\f 5{2\*3\*4}+\f 7{3\*4\*5}+\f 9{4*5*6} + \dots$$

$$\su \f{(2k+1}{k(k+1)(k+2)} = \su \(\f 1{k(k+2)}+\f 1{k(k+1)}\) =
\su\( \f{1/2}k - \f{1/2}{k+2} + \f 1 k - \f1{k+1} \) = \f{1/2}1 + \f{1/2}2 + \f 1 1 = \f7 4$$


\EndDoc
