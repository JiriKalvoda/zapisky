\providecommand{\HINCLUDE}{NE}
\if ^\HINCLUDE^
\else
\def\HINCLUDE{}
\global\newdimen\Okraje
\global\Okraje =4cm
\input{$HOME/souteze/_hlavicka/h-.tex}

%\definecolor{colorV}{RGB}{255,127,0}
%\definecolor{colorPoz}{RGB}{153,51,0}
%\definecolor{orangeV}{RGB}{255,127,0}
%\definecolor{colorPr}{RGB}{0,5,255}
%\definecolor{colorDef}{RGB}{0.255,0}

\usepackage[shortlabels]{enumitem}
\setlength{\marginparsep}{2pt}
\setlength{\marginparwidth}{35pt}

\def\st{{\rm st}}
\def\P{{\rm P}}

\def\ISENUM{}
\def\inMargin#1{\End
		
		\hskip0pt \marginpar{{{#1}}}}
\newcounter{V}[section] 
\newcommand{\V}[1][]{\stepcounter{V}\inMargin{\textcolor{green}{V.\arabic{section}.\theV.:}}\ifx^#1^\else\textcolor{green}{\underline{{#1}:}}\addcontentsline{toc}{subsubsection}{V.\arabic{section}.\theV.:$\quad$ {#1}}\\\fi}
\def\Def{\inMargin{\textcolor{red}{Def:}}}
\def\Poz{{\inMargin{\textcolor{brown}{Pozn:}}}}
\def\Pr{{\inMargin{\textcolor{blue}{Př:}}}}
\def\Pozenum
{
	\begin{enumerate}[1)]%, left = 0pt ]
		\item\inMargin{\textcolor{brown}{Pozn:}}\def\ISENUM{a}}
\def\End
{
	\if	^\ISENUM^
	\else \end{enumerate}
	\fi
	\def\ISENUM{}
}
\reversemarginpar

\makeatletter
\renewcommand\thesection{§\arabic{section}.}
\renewcommand\thesubsection{\Alph{subsection})}
\renewcommand\thesubsubsection{\alph{subsubsection})}
\newcounter{chapter}
\setcounter{chapter}{0}
\renewcommand\thechapter{\Alph{chapter})}
\newcounter{roman}
\setcounter{roman}{0}
\renewcommand\theroman{\Roman{roman}.}
\makeatother
\def\sectionnum#1{\setcounter{section}{#1}\addtocounter{section}{-1}}
\def\subsectionnum#1{\setcounter{subsection}{#1}\addtocounter{subsection}{-1}}
\def\subsubsectionnum#1{\setcounter{subsubsection}{#1}\addtocounter{subsubsection}{-1}}
\def\chapternum#1{\setcounter{chapter}{#1}\addtocounter{chapter}{-1}}
\def\chapter#1{

	\addtocounter{chapter}{1}\sectionnum{1}
	\addcontentsline{toc}{section}{\large{\thechapter$\quad${#1}}}
	
	{\LARGE  \textbf{\begin{minipage}[t]{0.1\textwidth}\thechapter\end{minipage}\begin{minipage}[t]{0.95\textwidth}#1\end{minipage}}}

}
\def\ROM{}
\def\Rom#1#2{\setcounter{roman}{#1}\renewcommand\ROM{#2}}

\Rom{6}{Funkce}
\title{\Huge\textbf{\theroman\quad \ROM}}
\author{Jiří Kalvoda}

\newcounter{countOfBegin}
\setcounter{countOfBegin}{0}
\newcommand{\BeginDoc}[1][]
{
	\ifnum\value{countOfBegin}=0
	\begin{document}
		#1
		\fi
	\addtocounter{countOfBegin}{1}
		
}
\def\EndDoc
{
	\addtocounter{countOfBegin}{-1}
	\ifnum\value{countOfBegin}=0
	\end{document}
	\fi
}

\fi
\BeginDoc{}
\def\posloup{$\zs{a_n}_{n=1}^{\infty}$}
\def\pos#1{\zs{#1}_{n=1}^{\infty}}
\def\li{\lim_{n\rightarrow\infty}}
\def\sup{{\rm sup\ }}
\def\sciwinfup{{\rm inf\ }}
\section{Suprémum a infimum množiny, konvergence omezených
monotónních posloupností}

\Def Nechť $M \neq \emptyset , M\subset \R$ je množina.\\
Číslo $a\in\R$ (pokud existuje) nazveme \emph{horní závorou množiny} $M\ekv \forall x \in M : x \le a$.\\
Číslo $b\in\R$ (pokud existuje) nazveme \emph{dolní  závorou množiny} $M\ekv \forall x \in M : x \ge b$.\\

\Pozenum
Zřejmě platí: číselná množina $M$ je shora omezená alespoň 1 její horní závora.
\item
	Číslo $c\in\R$  není horní závorou množiny $M\ekv \exists x \in M : x>c$
\End

\Def
Nechť $M\neq \emptyset; M\subset \R$ je množina.
Číslo $s\in\R$ nazýváme \emph{suprémem množiny} $M$, jestliže je její nejmenší horní závorou,
tzn. že platí:
\begin{enumerate}
	\item $\forall x \in M : x\le s$
	\item $\forall t \in \R,\forall x\in M: x \le t \imp t \ge s$
\end{enumerate}
Zapisujeme $s=\sup M$.

Číslo $t\in\R$ nazýváme \emph{infimem množiny} $M$, jestliže je její nejmenší horní závorou,
tzn. že platí:
\begin{enumerate}
	\item $\forall x \in M : x\ge t$
	\item $\forall j \in \R,\forall x\in M: x \ge j \imp j \le i$
\end{enumerate}
Zapisujeme $s=\inf M$.

\Pozenum
Nechť $s=\sup M \imp \forall p\in\R, p<s:\exists x \in M: x>p$\\
Nechť $t=\inf M \imp \forall p\in\R, p>t:\exists x \in M: x<p$\\
\item  Každá neprázdná množina má nejvýše 1 suprémum a 1 infimum.
\End
\V[Věta o suprému a infimu]
\begin{enumerate}
	\item Každá neprázdná shora omezená množina reálných čísel má suprémum.
	\item Každá neprázdná zdola omezená množina reálných čísel má infimum.
\end{enumerate}
\Poz V.6.1. se v některých teoriích pokládá za axiom množinyx . Jinde je toto tvrzení
důsledkem tzv. Dedekindova axiomu.

\Poz \underline{Dedekindův axiom}\\
Nechť $X,Y$ jsou dvě neprázdné podmnožiny $\R$ s těmito vlastnostmi:
\begin{enumerate}
	\item $X\cup Y = \R$
	\item $\forall x \in X , \forall y \in Y: x \le y$
\end{enumerate}
pak k nim $\exists z \in\R : x \le z \le y$ pro $\forall x \in X; \forall y\in Y$.

\V Každá shora omezená neklesající posloupnost reálných čísel je konvergentní.

Každá zdola omezená nerostoucí posloupnost reálných čísel je konvergentní.

Dusledek: Každá omezená monotónní posloupnost reálných čísel je konvergentní.

\Pr
Najděte suprémum a infimum následujících množin:

\begin{enumerate}
	\item $M_1 = \(-2;3\>$

		$\sup M_1 = 3$\\
		$\inf M_1 = -2$
	\item $M_2 = \(-\infty;1\>$

		$\sup M_1 = 1$\\
		$\inf M_1$ Neexistuje
	\item $M_3 = \R^+$

		$\sup M_1$ Neexistuje\\
		$\inf M_1 = 0$
	\item $M_4 = \zs{-1,0,1}$

		$\sup M_1 = 1$\\
		$\inf M_1 = -1$
\end{enumerate}

\Pr 43/7.1:

Ukažte, že každé reálné číslo je limitou neklesající posloupnosti.

Nechť $a+n\in\R;a\in \<0,1\),n\in\Z$ je libovolné reálné číslo.

Dekadický zápis $a$ je $a = a_1 10^{-1} + a_2 10^{-2} + a_3 10^{-3} + \dots$.

Limitou posloupnosti $\[n;n+a_1 10^{-1};n+a_1 10^{-1}+a_2^{-2};n+a_1 10^{-1}+a_2 10^{-2}+a_3 10^{-3}; \dots \]$.

\Pr 44/2:
Limitou je evidentně supremum.

\Pr 44/3:
Rozložme součet na úseky pro každé $n$: $\[\f{1}{2^n};\f{1}{2^n+1};\dots\f{1}{2^{n-1}-1}\].$

Součet každého úseku je evidentně $> 2^n \* \f 1{2^{n-1}-1} > \f 12$. 

Pro každé $x\in\R^+$ tedy existuje $n > 2x$,
Ovšem součet prvních $2^{n+1}$ členů posloupnosti je $> n \f 12 \ge x $. \emph{QED}
\EndDoc
