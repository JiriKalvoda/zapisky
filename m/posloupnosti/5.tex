\providecommand{\HINCLUDE}{NE}
\if ^\HINCLUDE^
\else
\def\HINCLUDE{}
\global\newdimen\Okraje
\global\Okraje =4cm
\input{$HOME/souteze/_hlavicka/h-.tex}

%\definecolor{colorV}{RGB}{255,127,0}
%\definecolor{colorPoz}{RGB}{153,51,0}
%\definecolor{orangeV}{RGB}{255,127,0}
%\definecolor{colorPr}{RGB}{0,5,255}
%\definecolor{colorDef}{RGB}{0.255,0}

\usepackage[shortlabels]{enumitem}
\setlength{\marginparsep}{2pt}
\setlength{\marginparwidth}{35pt}

\def\st{{\rm st}}
\def\P{{\rm P}}

\def\ISENUM{}
\def\inMargin#1{\End
		
		\hskip0pt \marginpar{{{#1}}}}
\newcounter{V}[section] 
\newcommand{\V}[1][]{\stepcounter{V}\inMargin{\textcolor{green}{V.\arabic{section}.\theV.:}}\ifx^#1^\else\textcolor{green}{\underline{{#1}:}}\addcontentsline{toc}{subsubsection}{V.\arabic{section}.\theV.:$\quad$ {#1}}\\\fi}
\def\Def{\inMargin{\textcolor{red}{Def:}}}
\def\Poz{{\inMargin{\textcolor{brown}{Pozn:}}}}
\def\Pr{{\inMargin{\textcolor{blue}{Př:}}}}
\def\Pozenum
{
	\begin{enumerate}[1)]%, left = 0pt ]
		\item\inMargin{\textcolor{brown}{Pozn:}}\def\ISENUM{a}}
\def\End
{
	\if	^\ISENUM^
	\else \end{enumerate}
	\fi
	\def\ISENUM{}
}
\reversemarginpar

\makeatletter
\renewcommand\thesection{§\arabic{section}.}
\renewcommand\thesubsection{\Alph{subsection})}
\renewcommand\thesubsubsection{\alph{subsubsection})}
\newcounter{chapter}
\setcounter{chapter}{0}
\renewcommand\thechapter{\Alph{chapter})}
\newcounter{roman}
\setcounter{roman}{0}
\renewcommand\theroman{\Roman{roman}.}
\makeatother
\def\sectionnum#1{\setcounter{section}{#1}\addtocounter{section}{-1}}
\def\subsectionnum#1{\setcounter{subsection}{#1}\addtocounter{subsection}{-1}}
\def\subsubsectionnum#1{\setcounter{subsubsection}{#1}\addtocounter{subsubsection}{-1}}
\def\chapternum#1{\setcounter{chapter}{#1}\addtocounter{chapter}{-1}}
\def\chapter#1{

	\addtocounter{chapter}{1}\sectionnum{1}
	\addcontentsline{toc}{section}{\large{\thechapter$\quad${#1}}}
	
	{\LARGE  \textbf{\begin{minipage}[t]{0.1\textwidth}\thechapter\end{minipage}\begin{minipage}[t]{0.95\textwidth}#1\end{minipage}}}

}
\def\ROM{}
\def\Rom#1#2{\setcounter{roman}{#1}\renewcommand\ROM{#2}}

\Rom{6}{Funkce}
\title{\Huge\textbf{\theroman\quad \ROM}}
\author{Jiří Kalvoda}

\newcounter{countOfBegin}
\setcounter{countOfBegin}{0}
\newcommand{\BeginDoc}[1][]
{
	\ifnum\value{countOfBegin}=0
	\begin{document}
		#1
		\fi
	\addtocounter{countOfBegin}{1}
		
}
\def\EndDoc
{
	\addtocounter{countOfBegin}{-1}
	\ifnum\value{countOfBegin}=0
	\end{document}
	\fi
}

\fi
\BeginDoc{}
\def\posloup{$\zs{a_n}_{n=1}^{\infty}$}
\def\pos#1{\zs{#1}_{n=1}^{\infty}}
\def\li{\lim_{n\rightarrow\infty}}
\section{Limita posloupnosti}

\Pr Určete několik prvních členů posloupnosti $\zs{\f{n+1}n}_{n=1}^\infty$. nakreslete její graf a určete, jak se posloupnost chová pro vzrůstající $n$:
\pd{5-1.pdf}

\Def Nechť \posloup je posloupnost, $A\in\R$ číslo. Řekneme, že \emph{posloupnost \posloup má limitu} rovnu číslu $A$, jestliže $\forall \epsilon \in \R^+:\exists n_0\in N:\forall n\ge n_0:|a_n-A|<\epsilon$, zapisujeme $\lim_{n\rightarrow\infty}a_n=A$.

\Poz $|a_n-A| < s \ekv a_n\in(A-s;A+s)$
\Def Má-li posloupnost limitu, pak se nazývá \emph{konvergentní}, v opačném případě
\emph{divergentní}. 

\V Každá posloupnost má nejvýše jednu limitu. 

[Dk: Sporem: Nechť má posloupnsost \posloup limitu $A$ a $B$, $A < B$.

Položme $\epsilon = \f{B-A}2$. Musí platit:\\
$a_n\in(A-\epsilon;A+\epsilon) \cap a_n\in(B-\epsilon;B+\epsilon) \imp a_n \in \emptyset$, což je spor.]

\V Každá konvergentní posloupnost je omezená. 

\Pr Obrácení předchozí věty neplatí: $\zs{(-1)^n}_{n=1}^{\infty}$.
\Poz Důsledek:  Jestliže posloupnost není omezená, pak je divergentní.
\Pr 
Určete limitu posloupnosti $\pos{\f{n+1}n}$, hypotéza z předchozího příkladu:
Máme dokázat:
$$ \forall \epsilon \in \R^+: \exists n_0\in\N:\forall n\ge n_0:|a_n-A|<\epsilon$$

$|a_n-A|< \epsilon \ekv |\f{n+n}n - 1 | < \epsilon \ekv |\f 1 n| < s\epsilon\ekv \f 1n < \epsilon$, neboť $n\in\N \ekv \f 1\epsilon<n \imp n_0 = \[\f 1\epsilon\]+1$.

\V Každá nekonečná posloupnost vybraná z konvergentní posloupnosti je konvergentní a
má stejnou limitu. 

\Poz  Pokud lze vybrat z posloupnosti \posloup dvě konvergentní posloupnosti s různou
limitou, je posloupnost \posloup divergentní. (např.: $\pos{(-1)^n}$)

\V Nechť \posloup a $\pos{b_n}$ jsou posloupnosti takové, že 
$\forall n \in \N: 0 \le a_n \le b_n \cap \li b_n=0$, pak $\li a_n = 0$

[Dk: $
\forall \epsilon\in\R^+:\exists n_0\in \N: \forall n\ge n_0: b_n<\epsilon
\imp 
\forall \epsilon\in\R^+:\exists n_0\in \N: \forall n\ge n_0: 0\le a_n\le b_n<\epsilon$
]

\Poz Předpoklady předchozí věty lze zeslabit, nerovnosti nemusejí platit pro konečný počet
členů posloupnosti. 

\V[Věta o třech limitách]
Nechť $\pos{a_n},\pos{b_n}$ a $\pos{c_n}$ jsou tři posloupnosti takové, že
$\exists n_0:\forall n>n_0: a_n\le b_n \le c_n \cap \li a_n = \li b_n =A$.
Pak $\li b_n = A$.

\Pr
\begin{enumerate}
	\item $\pos 1$

		$\forall \epsilon >0 : \forall n: a_n=1 \imp \li{1}=1$.
	\item $\pos{\f 1 n}$

		$\forall \epsilon>0:\forall n >\f1\epsilon: 0 < a_n = \f 1 n < \f 1{1/\epsilon} = 0+ \epsilon \Imp \li{\f 1 n} = 0$.
	\item $\pos {1+(-1)^{n+1}\f 1 n}$

		$\forall \epsilon>0:\forall n >\f1\epsilon: 1-\epsilon = 1-\f 1 {1/\epsilon} = 1-\f 1 n \le 1+(-1)^n \f 1 n=a_n=1+(-1)^n \f 1 n \le 1+\f 1 n< \f 1{1/\epsilon} = 1+ \epsilon \Imp \li{\f 1 n} = 0$.
	\item $\pos{(-1)^n}$

		Na sudých členech $\lim (-1)^{2n} = \lim 1 = 1$.
		Na lichých členech $\lim (-1)^{2n+1} = \lim -1 = -1$.

		Diverguje!

		\V Nechť $\pos{a_n}$ a $\pos{b_n}$ jsou dvě posloupnosti. Nechť $\li a_n=A \li b_n = B$ a nechť $c\in\R$:
		\begin{enumerate}
			\item $\li(a_n+b_n) = A+B$
			\item $\li(a_n-b_n) = A-B$
			\item $\li(c\*a_n) = c\*A$
			\item $\li(a_n\*b_n) = A\*B$
			\item $\li\(\f{a_n}{b_n}\) = \f AB$  ($\forall n \in \N: b_n\neq 0;B \neq 0$)
		\end{enumerate}

		\Pr: Vypočítejte:

		$$\li\(\f{n+1}n\)=\li\(1+\f 1 n\) = \li 1 + \li \f 1 n = 1 + 0 = 0$$
		\Poz
		 Konvergence AP a GP:
		 \begin{enumerate}
			 \item  AP je konvergentní $\ekv d=0$
			 \item GP je konvergentní $\ekv q\in(-1,1)\cap a_1 =0$.
		 \end{enumerate}

		 \Poz Kromě limit zavedených v 1. definici tohoto paragrafu (tyto limity nazýváme \emph{vlastní limity}) existují i tzv. \emph{nevlastní limity} $\pm\infty$. 
		 \Def Nechť $\pos{a_n}$ je posloupnost. 
		 Řekněme, že \emph{posloupnost} $\pos{a_n}$:
		 \begin{enumerate}
			 \item \emph{Má nevlastní limitu $+\infty$} (diverguje k $+\infty?$) $\Ekv \forall K\in \R:\exists n_0\in\N:\forall n\ge n_0,n\in\N: a_n>K$. Zapisujeme $\lim a_n = \infty$.
			 \item \emph{Má nevlastní limitu $+\infty$} (diverguje k $+\infty?$) $\Ekv \forall K\in \R:\exists n_0\in\N:\forall n\ge n_0,n\in\N: a_n<K$. Zapisujeme $\lim a_n = -\infty$.
		 \end{enumerate}

		 \Poz Nechť $R(x) = \f{P(x)}{Q(x)} = \f{a_nx^n+a_{n-1}x^{n-1}+\dots+a_0}{b_mx^m+b_{m-1}x^{m-1}+\dots+b_0}$ je racionální lomaná funkce.
		 Pak platí:
		 \begin{enumerate}
			 \item $st\ P(x) > st\ Q(x) \imp \li R(n) = \pm \infty$
			 \item $st\ P(x) = st\ Q(x) \imp \li R(n) = \f{a_n}{b_n}$
			 \item $st\ P(x) < st\ Q(x) \imp \li R(n) = 0$
		 \end{enumerate}

		\V Nechť $\pos{a_n}$ a $\pos{b_n}$ jsou dvě posloupnosti takové, že$\li a_n=0$ a $\pos{b_n}$ je omezená.
		Pak $\lim\(a_nb_n\)=0$.

		\Pr
		$$\li\(\f{(n+1)^2}{2n^2}\) = \li\(\f{n^2+2n+1}{2n^2}\) = \f 1 2$$
		\Pr $$\li(2n^2 - 3) = +\infty$$



\end{enumerate}



\EndDoc
