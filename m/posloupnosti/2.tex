\providecommand{\HINCLUDE}{NE}
\if ^\HINCLUDE^
\else
\def\HINCLUDE{}
\global\newdimen\Okraje
\global\Okraje =4cm
\input{$HOME/souteze/_hlavicka/h-.tex}

%\definecolor{colorV}{RGB}{255,127,0}
%\definecolor{colorPoz}{RGB}{153,51,0}
%\definecolor{orangeV}{RGB}{255,127,0}
%\definecolor{colorPr}{RGB}{0,5,255}
%\definecolor{colorDef}{RGB}{0.255,0}

\usepackage[shortlabels]{enumitem}
\setlength{\marginparsep}{2pt}
\setlength{\marginparwidth}{35pt}

\def\st{{\rm st}}
\def\P{{\rm P}}

\def\ISENUM{}
\def\inMargin#1{\End
		
		\hskip0pt \marginpar{{{#1}}}}
\newcounter{V}[section] 
\newcommand{\V}[1][]{\stepcounter{V}\inMargin{\textcolor{green}{V.\arabic{section}.\theV.:}}\ifx^#1^\else\textcolor{green}{\underline{{#1}:}}\addcontentsline{toc}{subsubsection}{V.\arabic{section}.\theV.:$\quad$ {#1}}\\\fi}
\def\Def{\inMargin{\textcolor{red}{Def:}}}
\def\Poz{{\inMargin{\textcolor{brown}{Pozn:}}}}
\def\Pr{{\inMargin{\textcolor{blue}{Př:}}}}
\def\Pozenum
{
	\begin{enumerate}[1)]%, left = 0pt ]
		\item\inMargin{\textcolor{brown}{Pozn:}}\def\ISENUM{a}}
\def\End
{
	\if	^\ISENUM^
	\else \end{enumerate}
	\fi
	\def\ISENUM{}
}
\reversemarginpar

\makeatletter
\renewcommand\thesection{§\arabic{section}.}
\renewcommand\thesubsection{\Alph{subsection})}
\renewcommand\thesubsubsection{\alph{subsubsection})}
\newcounter{chapter}
\setcounter{chapter}{0}
\renewcommand\thechapter{\Alph{chapter})}
\newcounter{roman}
\setcounter{roman}{0}
\renewcommand\theroman{\Roman{roman}.}
\makeatother
\def\sectionnum#1{\setcounter{section}{#1}\addtocounter{section}{-1}}
\def\subsectionnum#1{\setcounter{subsection}{#1}\addtocounter{subsection}{-1}}
\def\subsubsectionnum#1{\setcounter{subsubsection}{#1}\addtocounter{subsubsection}{-1}}
\def\chapternum#1{\setcounter{chapter}{#1}\addtocounter{chapter}{-1}}
\def\chapter#1{

	\addtocounter{chapter}{1}\sectionnum{1}
	\addcontentsline{toc}{section}{\large{\thechapter$\quad${#1}}}
	
	{\LARGE  \textbf{\begin{minipage}[t]{0.1\textwidth}\thechapter\end{minipage}\begin{minipage}[t]{0.95\textwidth}#1\end{minipage}}}

}
\def\ROM{}
\def\Rom#1#2{\setcounter{roman}{#1}\renewcommand\ROM{#2}}

\Rom{6}{Funkce}
\title{\Huge\textbf{\theroman\quad \ROM}}
\author{Jiří Kalvoda}

\newcounter{countOfBegin}
\setcounter{countOfBegin}{0}
\newcommand{\BeginDoc}[1][]
{
	\ifnum\value{countOfBegin}=0
	\begin{document}
		#1
		\fi
	\addtocounter{countOfBegin}{1}
		
}
\def\EndDoc
{
	\addtocounter{countOfBegin}{-1}
	\ifnum\value{countOfBegin}=0
	\end{document}
	\fi
}

\fi
\BeginDoc{}
\section{Analytické vyjádření kružnice a kruhu}
\V {\huge\dots}
\V Nechť $\zs{a_n}_{n=1}^\infty$ je AP s diferencí $d$ nechť $S_n$ je součet prvních $n$ členů. Pak platí:

$$\forall n\in\N: S_n = \f 12 n (a_1+a_n)$$

[Dk: sečteme po dvojicích $x$-tého prvku od začátku a $x$-tého prvku od konce. Všechny dvojice mají součet $a_1+a_n$ a případný zbylý člen je $\f{a_1+a_n}2$. ]

\Pr určete součet prvnich 100 lichych přirozených čísel:

$$\f{100\*(1+199)}2 = 10000$$

Součet lihých přirozencyh císel do $2n-1$:

$$\f {n(2n-1+1)}2 = n^2$$

\V Nechť $\zs{a_n}_{n=1}^\infty$ je AP. Pak platí:
$$\forall n\in\N: a_n = \f{a_{n-1}+a_{n+1}}2$$

[Dk: $a_n = a_n-1+d \land a_n=a_{n+1}-d \imp a_n = \f{a_{n-1}+a_{n+1}}2$]

\Pozenum Vyjádření členu předchozí věty je vyjádřením \emph{aritmetrického průměru} čísel $a_{n-1},a_{n+1}$.
\item platí i obrácení V.2.3.

\V Nechť $\zs{a_n}_{n=1}^\infty$ je AP. Pak platí:
\begin{enumerate}
\item $\zs{a_n}_{n=1}^\infty$ je rostoucí $\ekv$ $d>0$
\item $\zs{a_n}_{n=1}^\infty$ je klesající $\ekv$ $d<0$
\item $\zs{a_n}_{n=1}^\infty$ je konstantní $\ekv$ $d=0$
\end{enumerate}

\V Nechť $\zs{a_n}_{n=1}^\infty$ je AP. Pak platí:
\begin{enumerate}
\item $\zs{a_n}_{n=1}^\infty$ je zdola omezená $\ekv$ $d\ge 0$
\item $\zs{a_n}_{n=1}^\infty$ je shora omezená $\ekv$ $d\le 0$
\item $\zs{a_n}_{n=1}^\infty$ je omezená $\ekv$ $d=0$
\end{enumerate}

\Pr 17/2:
$$S_{10} = \f{10*(3+60)}2 = 315$$
\Pr 17/7:

Všimnu si, že součet dvojice po sobě jdoucích čísel, kde první je na liché a druhé na sudé pozici, je $-1$.

V $2n$ členech je právě $n$ takovýchto dvojic. Tedy součet je $-n$.

\Pr 17/10:

$a_1 = 3\*1^2$
$3+a_2 = 3*2^2 \imp a_2=9$

Tedy posloupnost musí mít $d=6$ a tedy $a_n=-3+6n$

$$S_n=\f{n(3-3+6n)}2 = 3n^2$$
\Pr 17/12:
\begin{enumerate}
	\item $a_1,a_3,a_5,a_7,\dots$
	\item $a_1,a_3,a_4,a_5,\dots$
\end{enumerate}

\Pr
$$a_1+a_4 = 12 $$
$$a_2 - a_6 = -8$$

$$2a_1 + 3d = 12$$
$$a_1+d-a_1-5d = - 8 \imp d = 2$$

$$2a_1+3\*2 = 12 \imp a_1 = 3$$

AP: $a_0=1;d=2$

\Pr Délky stran pravoúhelého trojúhelníku  tvoří AP.
Vypočítejte strany, když $S=6$

$(x+d)^2 = x^2 + (x-d)^2$\\
$x^2+2xd+d^2 = x^2 + x^2 - 2xd + d^2$\\
$4xd = x^2$\\
$d = \f x4$

$x(x-d) = x\f34 x = 2\*6\imp x^2 = 16 \imp x=4$

$a=3;b=4;c=5$

\Pr Kde platí $S_5=S_6 = 60$
$a_6 = 60$
$$S_5 = a_6-d+a_6-2d+a_-35+a_6-4d+a_6+5d = -15 d = 60 \imp d=-4$$
AP:$a_1=20,d=-4$

\Pr 18/13: 
$$24^2 = (24-d)^2 + (24-2d)^2$$
$$576 = 5 d^2 - 144 d + 1152$$
$$0 = (d-24) (d-\f{24}5)$$

\Pr 18/14:
jedinou pitagorijskou trojicí, která je zároveň aritmetrickou posloupností je $3,4,5$,
tedy délky stran  musí být $3k,4k,5k;k\in\R$.

Všimneme si, že tyto trojůhelníky jsou podobné. Jelikož ppoloměr kružnice vepsané trojůhelníku o stranách $3,4,5$ je 1, tak se musí jednat o 
$21,28,35$.

\Pr 18/15:
$$(100-85+1)\f{100+85}{2} = 1480$$
\Pr 18/17:
$$ t \f{c-\f a 2 +c-\f a 2 - (t-1)a}{2} = t(c-\f 12 ta) = tc-\f 12t^2a$$



\EndDoc
