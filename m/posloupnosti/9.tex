\providecommand{\HINCLUDE}{NE}
\if ^\HINCLUDE^
\else
\def\HINCLUDE{}
\global\newdimen\Okraje
\global\Okraje =4cm
\input{$HOME/souteze/_hlavicka/h-.tex}

%\definecolor{colorV}{RGB}{255,127,0}
%\definecolor{colorPoz}{RGB}{153,51,0}
%\definecolor{orangeV}{RGB}{255,127,0}
%\definecolor{colorPr}{RGB}{0,5,255}
%\definecolor{colorDef}{RGB}{0.255,0}

\usepackage[shortlabels]{enumitem}
\setlength{\marginparsep}{2pt}
\setlength{\marginparwidth}{35pt}

\def\st{{\rm st}}
\def\P{{\rm P}}

\def\ISENUM{}
\def\inMargin#1{\End
		
		\hskip0pt \marginpar{{{#1}}}}
\newcounter{V}[section] 
\newcommand{\V}[1][]{\stepcounter{V}\inMargin{\textcolor{green}{V.\arabic{section}.\theV.:}}\ifx^#1^\else\textcolor{green}{\underline{{#1}:}}\addcontentsline{toc}{subsubsection}{V.\arabic{section}.\theV.:$\quad$ {#1}}\\\fi}
\def\Def{\inMargin{\textcolor{red}{Def:}}}
\def\Poz{{\inMargin{\textcolor{brown}{Pozn:}}}}
\def\Pr{{\inMargin{\textcolor{blue}{Př:}}}}
\def\Pozenum
{
	\begin{enumerate}[1)]%, left = 0pt ]
		\item\inMargin{\textcolor{brown}{Pozn:}}\def\ISENUM{a}}
\def\End
{
	\if	^\ISENUM^
	\else \end{enumerate}
	\fi
	\def\ISENUM{}
}
\reversemarginpar

\makeatletter
\renewcommand\thesection{§\arabic{section}.}
\renewcommand\thesubsection{\Alph{subsection})}
\renewcommand\thesubsubsection{\alph{subsubsection})}
\newcounter{chapter}
\setcounter{chapter}{0}
\renewcommand\thechapter{\Alph{chapter})}
\newcounter{roman}
\setcounter{roman}{0}
\renewcommand\theroman{\Roman{roman}.}
\makeatother
\def\sectionnum#1{\setcounter{section}{#1}\addtocounter{section}{-1}}
\def\subsectionnum#1{\setcounter{subsection}{#1}\addtocounter{subsection}{-1}}
\def\subsubsectionnum#1{\setcounter{subsubsection}{#1}\addtocounter{subsubsection}{-1}}
\def\chapternum#1{\setcounter{chapter}{#1}\addtocounter{chapter}{-1}}
\def\chapter#1{

	\addtocounter{chapter}{1}\sectionnum{1}
	\addcontentsline{toc}{section}{\large{\thechapter$\quad${#1}}}
	
	{\LARGE  \textbf{\begin{minipage}[t]{0.1\textwidth}\thechapter\end{minipage}\begin{minipage}[t]{0.95\textwidth}#1\end{minipage}}}

}
\def\ROM{}
\def\Rom#1#2{\setcounter{roman}{#1}\renewcommand\ROM{#2}}

\Rom{6}{Funkce}
\title{\Huge\textbf{\theroman\quad \ROM}}
\author{Jiří Kalvoda}

\newcounter{countOfBegin}
\setcounter{countOfBegin}{0}
\newcommand{\BeginDoc}[1][]
{
	\ifnum\value{countOfBegin}=0
	\begin{document}
		#1
		\fi
	\addtocounter{countOfBegin}{1}
		
}
\def\EndDoc
{
	\addtocounter{countOfBegin}{-1}
	\ifnum\value{countOfBegin}=0
	\end{document}
	\fi
}

\fi
\BeginDoc{}
\def\posloup{$\zs{a_n}_{n=1}^{\infty}$}
\def\pos#1{\zs{#1}_{n=1}^{\infty}}
\def\li{\lim_{n\rightarrow\infty}}
\def\sup{{\rm sup\ }}
\def\sciwinfup{{\rm inf\ }}
\def\su{\sum_{n=1}^{\infty}}
\def\sun{\sum_{n=0}^{\infty}}
\section{Nekonečná geometrická řada}
\Def Nekonečnou řadou $\su a_n$ kde $\pos{a_n}$ je geometrickou posloupností s kvocientem $q$ nazýváme \emph{nakonečnou geometrickou řadou}.

Číslo $q$ nazýváme \emph{kvocientem} geometrické řady.

\V Nechť řada $\su a_n$ je geometrická řada s kvocientem $q$, kde $|q|<1$.
Pak je tato řada konvergentní a má součet $S=\f{a_n}{1-q}$.

[Dk:
$$S = \lim S_n = \lim a_1 \f{q^n-1}{q-1} = \f{a_1}{q-1} \lim (q^n-1) = \f{a_1}{1-q}$$
]

\Poz Chování nekonečné geometrické řady $a_1 \neq 0$:
\begin{enumerate}
	\item $-1<q<1$\dots řada konverguje
	\item $q\ge 1$\dots řada diverguje
	\item $q\le 1$ řada osciluje
\end{enumerate}

\Pr Určete součty řad:
\begin{enumerate}
	\item $\su \f{1}{2^n} = \f{1/2}{1/2} = 1$
	\item $\su \(-\f 2 3\)^{n-1} = \f{1}{5/3} = \f 35$

\end{enumerate}

\Pr V oboru přirozenych čísel řešte rovnici:
$$1+\f 2x+\f4{x^2}+\f8{x^3}+\cdots = \f{4x-3}{3x-4}$$

$\f{2/x}{1-2x} = \f{4x-3}{3x-4}$\\
$\f{1}{1-2/x} = \f{4x-3}{3x-4}$\\
$\f{x}{x-2} = \f{4x-3}{3x-4}$\\
$3x^2-4x = 4x^2-11x+6$\\
$x^2-7x+6 =0$\\
$(x-6)(x-1)=0$

Zk: $1+2+4+\dots \neq \f{1}{-1}$

Zk: $\su \(\f 2 6\)^{n-1} \neq \f{1}{-1}$

$$p=\zs{6}$$

\Pr
Řešte:
$$\f 53=x+3x^2+x^3+3x^4+x^5+\dots$$
$$\f 53=\(x+x^2+x^3+x^4+x^5\)+\dots + 2\(x^2+x^4+x^6\)$$
$$\f 53=\(x+x^2+x^3+x^4+x^5\)+\dots + 2\(x^2+x^4+x^6\)$$

$$\f 53 = \f{x}{1-x} + 2 \f{x^2}{1-x^2}$$
$$\f 53 = \f{x(x+1)}{(1-x)(x+1)} + 2 \f{x^2}{1-x^2}$$
$$\f 53-\f53x^2= x^2+ x + x^2$$
$$5-5x^2= 6x^2+ 3x$$
$$0= 14x^2 + 3x - 5$$

$$P=\zs{-\f 57;\f 12}$$


\Pr Je dán čtverec o straně $a$. Spojnice středů jeho stran tvoří strany dalšího čtverce,...
Určete součet obvodů a obsahů všech takto vzniklých čtverců.

$o = 4\(a + \f a {\sqrt{2}} + \f a{\sqrt{2}^2}\) = 4a  \su \(2^{-1/2}\)^{n-1} = \f{4a}{1-2^{-1/2}} = 8a+4a\sqrt 2$ 

$S = \su \(\f{a}{\sqrt{2}^{n-1}}\)^2 = \f{a^2}{2^{n-1}} = \f{a^2}{1-1/2} = 2a^2$

\Pr Vyjádřete $\pi$ pomocí limity posloupností obvodů pravidelného $n$-úhelníku.

$$ o = \li n \* 2r \sin \f\pi n \imp  \pi = \li n\sin\f\pi n = \lim \pi = \pi$$

$$ o = \li n \* 2r \tan \f\pi n \imp  \pi = \li n\tan\f\pi n = \lim \pi = \pi$$

\Pr 56/3:
\begin{enumerate}
	\item $\sun \(\f2 3 \)^n = \f1{1-2/3} = 3$
	\item $\sun \(\f{\sqrt 3}{\(\sqrt 3\)^n} \)^n = \f{\sqrt 3}{1-\sqrt{1/3}} = \f{\sqrt 3\(1+\sqrt {1/3}\)}{2/3} = 3\f{\sqrt 3 + 1}2$
	\item $\sun \f 1{a^{2n}} \(1-\f 1a\) = \f{1-1/a}{1-1/a^2} = \f{1}{1+1/a} = \f a{1+a}$
	\item $\sun \(\sin^3 a\)^n = \f{1}{1-\sin^3 a}$
	\item Evidetně $\sqrt 2 = q>1$ tedy diverguje!
\end{enumerate}

\Pr 56/4:
$$\f{a}{1-1/3} = 10$$
$$a = 10\f2 3 = \f{20}3$$

\Pr 56/5:
$$\su \f{n}{2^n}  = \su \sum_{m=n}^\infty \f{1}{2^m} =\su \f{1/2^n}{1-1/2} = \su \f{2}{2^n} = \f{1}{1-1/2} = 2 $$

\Pr 56/7:

\begin{enumerate}
	\item $100a=a+13$\\
		$99 a = 13$\\
		$a=\f{13}{99}$

		$a = 13 \* \su\f 1{100^n} = \f{0.13}{1-1/100} = \f{13}{99}$
	\item $1000(a-3)=(a-3)+142$\\
		$999(a-3)=142$\\
		$a=3+\f{142}{999} = \f{3139}{999}$

		$a=2+142\* \su\f 1{1000^n} = \f{0.142}{1-1/1000} =3+ \f{142}{999} = \f{3139}{999}$
	\item $100(a-5.137)=(a-5.137)+0.081$\\
		$99 (a-5.137) = 0.00081$\\
		$a=5.137\f{0.081}{99}= \f{14129}{2750}$

		$a = 5.137 + 0.00001 \* \sum\f {81}{100^n} = \f{5137}{100}+\f{1}{100000}\*\f{80}{1-1/100} = \f{14129}{2750}$
\end{enumerate}

\EndDoc
