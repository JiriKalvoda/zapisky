\providecommand{\HINCLUDE}{NE}
\if ^\HINCLUDE^
\else
\def\HINCLUDE{}
\global\newdimen\Okraje
\global\Okraje =4cm
\input{$HOME/souteze/_hlavicka/h-.tex}

%\definecolor{colorV}{RGB}{255,127,0}
%\definecolor{colorPoz}{RGB}{153,51,0}
%\definecolor{orangeV}{RGB}{255,127,0}
%\definecolor{colorPr}{RGB}{0,5,255}
%\definecolor{colorDef}{RGB}{0.255,0}

\usepackage[shortlabels]{enumitem}
\setlength{\marginparsep}{2pt}
\setlength{\marginparwidth}{35pt}

\def\st{{\rm st}}
\def\P{{\rm P}}

\def\ISENUM{}
\def\inMargin#1{\End
		
		\hskip0pt \marginpar{{{#1}}}}
\newcounter{V}[section] 
\newcommand{\V}[1][]{\stepcounter{V}\inMargin{\textcolor{green}{V.\arabic{section}.\theV.:}}\ifx^#1^\else\textcolor{green}{\underline{{#1}:}}\addcontentsline{toc}{subsubsection}{V.\arabic{section}.\theV.:$\quad$ {#1}}\\\fi}
\def\Def{\inMargin{\textcolor{red}{Def:}}}
\def\Poz{{\inMargin{\textcolor{brown}{Pozn:}}}}
\def\Pr{{\inMargin{\textcolor{blue}{Př:}}}}
\def\Pozenum
{
	\begin{enumerate}[1)]%, left = 0pt ]
		\item\inMargin{\textcolor{brown}{Pozn:}}\def\ISENUM{a}}
\def\End
{
	\if	^\ISENUM^
	\else \end{enumerate}
	\fi
	\def\ISENUM{}
}
\reversemarginpar

\makeatletter
\renewcommand\thesection{§\arabic{section}.}
\renewcommand\thesubsection{\Alph{subsection})}
\renewcommand\thesubsubsection{\alph{subsubsection})}
\newcounter{chapter}
\setcounter{chapter}{0}
\renewcommand\thechapter{\Alph{chapter})}
\newcounter{roman}
\setcounter{roman}{0}
\renewcommand\theroman{\Roman{roman}.}
\makeatother
\def\sectionnum#1{\setcounter{section}{#1}\addtocounter{section}{-1}}
\def\subsectionnum#1{\setcounter{subsection}{#1}\addtocounter{subsection}{-1}}
\def\subsubsectionnum#1{\setcounter{subsubsection}{#1}\addtocounter{subsubsection}{-1}}
\def\chapternum#1{\setcounter{chapter}{#1}\addtocounter{chapter}{-1}}
\def\chapter#1{

	\addtocounter{chapter}{1}\sectionnum{1}
	\addcontentsline{toc}{section}{\large{\thechapter$\quad${#1}}}
	
	{\LARGE  \textbf{\begin{minipage}[t]{0.1\textwidth}\thechapter\end{minipage}\begin{minipage}[t]{0.95\textwidth}#1\end{minipage}}}

}
\def\ROM{}
\def\Rom#1#2{\setcounter{roman}{#1}\renewcommand\ROM{#2}}

\Rom{6}{Funkce}
\title{\Huge\textbf{\theroman\quad \ROM}}
\author{Jiří Kalvoda}

\newcounter{countOfBegin}
\setcounter{countOfBegin}{0}
\newcommand{\BeginDoc}[1][]
{
	\ifnum\value{countOfBegin}=0
	\begin{document}
		#1
		\fi
	\addtocounter{countOfBegin}{1}
		
}
\def\EndDoc
{
	\addtocounter{countOfBegin}{-1}
	\ifnum\value{countOfBegin}=0
	\end{document}
	\fi
}

\fi
\BeginDoc{\sectionnum{9}}
\def\mc{\mathcal}
\section{Odchylka}
\Def
Nechť $\ve u, \ve v$ jsou dva nenulové vektory. \emph{Odchylkou dvou vektorů} $\ve u, \ve v$ označujeme $\varphi = |\angle \ve u,\ve v|$ a definujeme takto:
\begin{enumerate}[1.]
	\item Je-li $\ve u = k \ve v ; k \in \R^+ \imp |\angle \ve u,\ve v| = 0\d$
	\item Je-li $\ve u = k \ve v ; k \in \R^- \imp |\angle \ve u,\ve v| = 0180\d$
	\item Je-li $\ve u \neq k \ve v$ pro $\forall k \in \R-\zs{0}| \imp $ odchylkou vektorů $\ve u , \ve v$ rozumíme velikost konvexního úhlu, který oba vektory svírají.
\end{enumerate}
\Poz $0\d \le | \angle \ve u , \ve v| \le 180\d$

\V Nechť $\ve u,\ve v$ jsou dva nenulové vektory. Pak platí:
$$
\begin{array}{|c|}\hline\\[-8px]
\displaystyle	|\angle\ve u , \ve v| = \arccos \f{\ve u \* \ve v}{|\ve u|\*|\ve v|}
	\\[-8px]\\\hline

\end{array}
$$
[Dk: Plyne z definice skalárního součinu] 
\Poz Jestliže jsou dva podprostory $\mc A, \mc B \subset \E_3$ rovnoběžné, jejich odchylka je $0\d$. 

\subsection{Odchylka dvou přímek v $\E_2$}
\Poz Nechť $p,q$ jsou 2 různoběžné přímky. Odchylka přímek $p,q$ je velikost ostrého nebo pravého úhlu, který svírají.
$0\d \le |\angle p,q| \le 90\d$

\V Nechť $p(A,\ve u),q(B,\ve v)$ jsou dvě různoběžné přímky, pak platí:
$$
\begin{array}{|c|}\hline\\[-8px]
\displaystyle	|\angle p,q| = \arccos \f{|\ve u \* \ve v|}{|\ve u|\*|\ve v|}
	\\[-8px]\\\hline

\end{array}
$$
[Dk.: Plyne z V.9.1. a z toho, že funkce $y = \arccos x$ má pro definiční obor $\<0;1\>$ obor hodnot $\<0;\f\pi2\>$].

\V Nechť
$p: ax+by+c=0\quad ((a,b)\neq\ve0)$, 
$p: ex+fy+g=0\quad ((e,f)\neq\ve0)$ jsou 2 různoběžné přímky a nechť $\ve{n_p} = (a,b), \ve{n_q}=(e,f)$. Pak platí:
$$
\begin{array}{|c|}\hline\\[-8px]
	\displaystyle	|\angle p,q| = \arccos \f{|\ve{n_p} \* \ve{n_q}|}{|\ve u|\*|\ve v|}
	\\[-8px]\\\hline

\end{array}
$$
[Dk.: Plyne z V.9.2. a z toho, že normálové vektory přímek svírají stejný úhel jako přímky samé.]

\V Nechť
$p(A,\ve u)$, 
$p: ax+by+c=0\quad ((a,b)\neq\ve0)$ jsou 2 různoběžné přímky a nechť $\ve{n_q} = (a,b)$. Pak platí:
$$
\begin{array}{|c|}\hline\\[-8px]
	\displaystyle	|\angle p , q| = \arcsin \f{|\ve u \* \ve {n_q}|}{|\ve u|\*|\ve v|}
	\\[-8px]\\\hline

\end{array}
$$
[Dk.: Plyne z V.9.2., z toho, že odchylka normálového vektoru 1 přímky a směrového vektoru 2.přímky svírá úhel $\f\pi2-\varphi$ a $\cos\(\f\pi2-\varphi\) = \sin \varphi$ ($\varphi$ je odchylka obou přímek).]

\Pr Určete $\angle p,q|$:
\begin{enumerate}[a)]
	\item
		$p(A,\ve u),q(B,\ve v);A[1;0],B[3;1],\ve u (1;1),\ve (-1;0)$

		$$
		|\angle p,q| = \arccos \f{|-1|}{\sqrt 2+\*\sqrt 1} = \arccos \f{\sqrt 2 }2 = \f\pi 4
		$$
	\item
		$p:5x+3y-7=0;q:4x-y+5=0$

		$\ve{n_p}=(5,3);\ve{n_q}=(4;-1)$
		$$
		|\angle p,q| = \arccos \f{|20-3|}{\sqrt{25+9}+\*\sqrt{16+1}} = \arccos \f{17}{\sqrt{34}\sqrt{17}} = \arccos \f{\sqrt 2}2 = \f\pi 4
		$$

	\item 
		$p = \ve{AB};A[1;0];B[2;1];q:x+2y-6=0$

		$\ve u = (1;1); \ve{n_q}=(1,2)$
		$$
		|\angle p,q|=\arcsin\f{|1+2|}{\sqrt 2 \* \sqrt 5} = \arcsin \f{3\sqrt{10}}{10}
		$$
\end{enumerate}

\Pr 86/69:
\pdf[0.3]{9-cv69.pdf}
\begin{enumerate}[a)]
	\item $\ve{n_p} = (2;1); \ve{n_q} = (6;-2)$
		$$
		|\angle p,q| = \arccos\f{|12-2|}{\sqrt 5\*\sqrt{40}}=\arccos \f{\sqrt 2}2 = \f\pi 4
		$$
	\item $\ve{u} = (1;-1); \ve{n_q} = (2;4)\sim(1;2)$
		$$
		|\angle p,q| = \arcsin\f{|1-2|}{\sqrt 2\*\sqrt{5}}=\arcsin \f{\sqrt {10}}{10}
		$$

\end{enumerate}
\Pr 86/70:
\pdf[0.3]{9-cv70.pdf}
$\ve{n} = (1;-2)$

Otočím o $+30\d$:
$\ve{n_1} = \ve n \* e^{i\f\pi6} = (1-2i)\*\(\f{\sqrt 3}2+\f i 2\) = \(\f{\sqrt 3}2+1\)+\(\f12-\sqrt 3  \)i$\\
$p_1:(\sqrt 3 +2)x + (1-2\sqrt 3 )y + c_1=0$\\
$A[0;\f32]\in p_1 \imp -\f{3}{2}\(-2\sqrt 3 + 1\)  = c_1$
$$p_1 :(2\sqrt 3 +4)x + (-4\sqrt 3 +2) + \(6\sqrt 3 - 3\)=0$$

Otočím o $-30\d$:
$\ve{n_1} = \ve n \* e^{-i\f\pi6} = (1-2i)\*\(\f{\sqrt 3}2-\f i 2\) = \(\f{\sqrt 3}2-1\)+\(-\f12-\sqrt 3  \)i$\\
$p_1:(\sqrt 3 -2)x + (-1-2\sqrt 3 )y + c_2=0$\\
$A[0;\f32]\in p_1 \imp -\f{3}{2}\(-2\sqrt 3 - 1\)  = c_1$
$$p_2 :(2\sqrt 3 -4)x + (-4\sqrt 3 -2)y + \(6\sqrt 3 + 3\)=0$$



\Pr 86/71:
\pdf[0.3]{9-cv71.pdf}
$p:6x+4y+9$\\
$\ve{n_p} = (6;4) \sim (3,2)$\\
$6\*5+7\*4-9 \neq 0 \imp A\not\in p \imp \pri{BC} = p$

Najdu $\pri{AC}\perp{BC}$: $\pri{AC}: 2x-3y+c=0$\\
$A\in \pri{AC} \imp 10-21+c=0 \imp c = 11$
$$\pri{AC}: 2x-3y+11=0$$

Najdu $\pri{AB}$:

Otočím $p$ o $+45\d$:\\
$\ve{n_1} = \ve{n_p}\*e^{i \f\pi4} = (3+2i)(\f{\sqrt 2}2+\frac{\sqrt 2}2i) = \f{\sqrt 2}2+\f{5\sqrt 2}2i$\\
$\pri{BC}_1:  x + 5 y + d = 0$\\
$A\in\pri{AB}_1 \imp  5 + 5 \*7 + d = 0 \imp d=-40$\\
$$\pri{BC}_1:  x + 5 y -40 = 0$$\\

Otočím $p$ o $-45\d$:\\
$\ve{n_2} = \ve{n_p}\*e^{-i \f\pi4} = (3+2i)(\f{\sqrt 2}2-\frac{\sqrt 2}2i) = \f{5\sqrt 2}2-\f{\sqrt 2}2i$\\
$\pri{BC}_2:  5x - y + e = 0$\\
$A\in\pri{AB}_2 \imp  5\*5 - 7 + e = 0 \imp e=-18$\\
$$\pri{BC}_2:  5x - y -18 = 0$$\\



\Pr 86/73:
\pdf[0.3]{9-cv73.pdf}

$\ve{AB} = (-2,2) \sin (-1;1)$\\
$\ve{BC} = (4,-2) \sin (2;-1)$\\
$\ve{AC} = (2,0) \sin (1;0)$

$$
\cos \alpha = \f{-1}{\sqrt 2 \* \sqrt 1} = -\f{\sqrt 2}2
$$
$$
\cos \beta = \f{2+1}{\sqrt 2 \* \sqrt 5} = \f{3\sqrt {10}}{10}
$$
$$
\cos \gamma = \f{2}{\sqrt 5 \* \sqrt 1} = \f{2\sqrt 5}5
$$




\subsection{Odchylka dvou přímek v $\E_3$}
\Poz Nechť $p, q$ jsou 2 různoběžné přímky. Pak jejich odchylka je rovna velikosti ostrého nebo pravého úhlu, který svírají.\\
Nechť $p, q$ jsou 2 mimoběžné přímky. Pak jejich odchylka je rovna $|\angle p',q'|$, kde $p'\parallel p,q'\parallel q$ jsou různoběžky.\\
$0\d \le |\angle p,q| \le 90\d$

\V Nechť $p(A,\ve u),q(B,\ve v)$ jsou dvě různoběžné přímky, pak platí:
$$
\begin{array}{|c|}\hline\\[-8px]
\displaystyle	|\angle\ve p,q| = \arccos \f{|\ve u \* \ve v|}{|\ve u|\*|\ve v|}
	\\[-8px]\\\hline

\end{array}
$$
[Dk.: viz V.9.2].

\Pr Určete odchylku přímek $\ve{A'B},\ve{BC'}$ krychle $ABCDA'B'C'D'$:

\begin{minipage}{0.4\textwidth}
	\pdf[0.2]{9-B.pdf}
\end{minipage}
\begin{minipage}{0.6\textwidth}
	Analytické řešení:\\
	$A[1;0;0];A'[1;0;1],$\\
	$B[1;1;0];B'[1;1;1],$\\
	$C[0;1;0];C'[0;1;1],$\\
	$D[0;0;0];D'[0;0;1],$\\
	$\ve{A'B} = (0;1;-1); \ve{BC'}  = (-1;0;1)$\\
	$\underline{|\angle \pri{A'B},\pri{B,C'}|} = \arccos \f{-1}{\sqrt 2 \* \sqrt 2} = \arccos \f 12 = \underline{\f\pi3}$
\end{minipage}
Stereometrické řešení:\\
Celý problém se evidentně odehrává v rovině $A'BC'$:
Úsečky $A'B,BC'$ a $A'C'$ mají evidentně stejnou vzdálenost, protože se jedná o stěnové úhlopříčky. $\triangle A'BC'$ je tedy rovnostroný, pročež $|\angle \pri{A'B},\pri{BC'}| = |\angle A'BC'|= \f\pi3$.

\Pr 177/11:\\
$\ve u = (-1;1;-1)$\\
$\ve v = (1;1;1)$

$$ |\angle p,\pri{AB}|=\arccos\f{|\ve u \* \ve v|}{|\ve u|\*|\ve v|} = \f{|-1|}{\sqrt 3 \* \sqrt 3} = \arccos \f{1}3$$.

\Pr 177/12:\\
$\ve u (2;2;10) \sim (1;1;5)$

$|\angle \pri{AB},x| = \arccos \f{|1|}{\sqrt{1}\*\sqrt{27}} = \arccos \f{\sqrt3}9$

$|\angle \pri{AB},y| = \arccos \f{|1|}{\sqrt{1}\*\sqrt{27}} = \arccos \f{\sqrt3}9$

$|\angle \pri{AB},z| = \arccos \f{|5|}{\sqrt{1}\*\sqrt{27}} = \arccos \f{5\sqrt3}9$

$\f{3}{81}+\f3{81}+{75}{81} = \f{81}{81} = 1$ \emph{QED}.

\Pr 178/17:
\begin{enumerate}
	\item $\ve u (1,1,1)$\\
		$\ve v (-1,1,1)$
$$ |\angle p,q|=\arccos\f{|\ve u \* \ve v|}{|\ve u|\*|\ve v|} = \f{|1|}{\sqrt 3 \* \sqrt 3} = \arccos \f{1}3$$.
	\item $\ve u (1,1,1)$\\
		$\ve v (-1,1,0)$
$$ |\angle p,q|=\arccos\f{|\ve u \* \ve v|}{|\ve u|\*|\ve v|} = \f{|0|}{\sqrt 3 \* \sqrt 2} = \arccos 0 = \f\pi2$$.
\end{enumerate}

\Pr 178/18:
\begin{enumerate}
	\item $\pri {AB}$ a $\pri{CD}$:\\
		$\ve u (-6;5;0)$\\
		$\ve v (-3;-3;8)$\\
		$$ |\angle \pri{AB},\pri{CD}|=\arccos\f{|\ve u \* \ve v|}{|\ve u|\*|\ve v|} = \f{|18-15|}{\sqrt {36+25} \* \sqrt {9+9+64}} = \arccos \f{3\sqrt{5002}}{5002}$$

	\item $\pri {AC}$ a $\pri{BD}$:\\
		$\ve u (-1;6;0)$\\
		$\ve v (2;-2;8)\sim(1;-1;4)$\\
		$$ |\angle \pri{AC},\pri{BD}|=\arccos\f{|\ve u \* \ve v|}{|\ve u|\*|\ve v|} = \f{|-1-6|}{\sqrt {1+36} \* \sqrt {1+1+8}} = \arccos \f{7\sqrt{370}}{370}$$

	\item $\pri {AD}$ a $\pri{BC}$:\\
		$\ve u (-4;3;8)$\\
		$\ve v (5;1;0)$\\
		$$ |\angle \pri{AD},\pri{BC}|=\arccos\f{|\ve u \* \ve v|}{|\ve u|\*|\ve v|} = \f{|-20+3|}{\sqrt {16+9+64} \* \sqrt {25+1}} = \arccos \f{17\*\sqrt{2314}}{2314}$$
\end{enumerate}

\subsection{Odchylka přímky od roviny v $\E_3$}
\Poz Nechť $p,\alpha$ jsou přímka a rovina navzájem různoběžné. Pak platí:
$|\angle p,\alpha| = |\angle p,q|$, kde $q = \alpha \cap \beta \land \beta \perp \alpha \land p \subset \beta$.\\
$0\d \le |\angle p,\alpha| \le 90\d$

\V Nechť
$p(A,\ve u)$ je přímka a 
$\alpha: ax+by+cz+d=0\quad ((a,b,c)\neq\ve0)$ je rovina a nechť $\ve{n} = (a,b,c)$. Pak platí:
$$
\begin{array}{|c|}\hline\\[-8px]
\displaystyle	|\angle p , \alpha| = \arcsin \f{|\ve u \* \ve n|}{|\ve u|\*|\ve n|}
	\\[-8px]\\\hline

\end{array}
$$
[Dk.: Plyne z V.9.5., z toho, že odchylka normálového vektoru roviny  a směrového vektoru přímky svírá úhel $\f\pi2-\varphi$ a $\cos\(\f\pi2-\varphi\) = \sin \varphi$ ($\varphi$ je odchylka přímky od roviny).]

\Pr Určete $|\angle p,\alpha|$:\\
$p=\pri{AB};A[1;1;-2];B[-1;0;-1]$\\
$\alpha 2x-3y+z+4=0$

$$|\angle p,\alpha| = \arcsin \f{|-4+3+1|}{\sqrt 6 \* \sqrt{14}} = \arcsin 0 = 0 \imp p\parallel \alpha$$

\subsection{Odchylka dvou rovin v $\E_3$}
\Poz Nechť $\alpha,\beta$ jsou dvě různoběžné roviny. Pak platí:
$|\angle \alpha,\beta| = |\angle p,q|$, kde $p\subset r \alpha;q\subset \beta ;p\perp r;q\perp r;r = \alpha \cap \beta$.\\
$0\d \le |\angle \alpha,\beta| \le 90\d$

\V Nechť\\
$\alpha: ax+by+cz+d=0\quad ((a,b,c)\neq\ve0)$\\
$\beta:  ex+fy+gz+h=0\quad ((e,f,q)\neq\ve0)$\\
jsou dvě roviny. Nechť $\ve{n_\alpha}=(a,b,c)$ a $\ve{n_\beta}=(e,f,g)$. Pak platí:
$$
\begin{array}{|c|}\hline\\[-8px]
	\displaystyle	|\angle  \alpha,\beta| = \arccos \f{|\ve{n_\alpha} \* \ve {n_\beta}|}{|\ve{n_\alpha}|\*|\ve{n_\beta}|}
	\\[-8px]\\\hline

\end{array}
$$
[Dk.: Plyne z V.9.5. a z toho, že normálové vektory rovin svírají stejný úhel jako roviny samé.]

\Pr Určete $|\angle \alpha,\beta|$:\\
$\alpha: 2x+3y+z-5 = 0$\\
$\beta : x-y+z+12=0$\\

$$|\angle \alpha,\beta| = \arccos\f{|2-3+1|}{\sqrt{14}\*\sqrt{3}} = \arccos 0 = \f\pi2\imp\alpha\perp\beta$$

\Pr 192/34:

	$A[0;0;0];A'[0;0;1],$\\
	$B[0;1;0];B'[0;1;1],$\\
	$C[1;1;0];C'[1;1;1],$\\
	$D[1;0;0];D'[1;0;1],$\\

	$\pri{ACB'} = \zs{[s,s+t,t];s,t\in\R}$\\
	$\pri{ACB'}: x-y+z=0$\\
	$\pri{ACB} = \zs{[s,s+t,0];s,t\in\R}$\\
	$\pri{ACB}: 0x+0y+z=0$\\
$$|\angle \pri{ACB'},\pri{ACB}| = \arccos\f{|1|}{\sqrt{3}\*\sqrt{1}} = \arccos \f{\sqrt3}3 $$

\Pr 192/35:

\begin{enumerate}[a)]
	\item
		$\alpha:x+y+0z-2=0$\\
		$\beta:0x+9y+2z-16=0$\\
		$$|\angle \alpha,\beta| = \arccos\f{|0+9+0|}{\sqrt{2}\*\sqrt{85}} = \arccos\f{9\sqrt{170}}{170}$$

	\item
		$\alpha:-3x+y-2z+16=0$\\
		$\beta:0x+1y+4z+2=0$\\
		$$|\angle \alpha,\beta| = \arccos\f{|0+1-8|}{\sqrt{14}\*\sqrt{17}} = \arccos\f{\sqrt{238}}{34}$$
	\item
		$\alpha:13x-2y+5z-56=0$\\
		$\beta:3x+0y+2x-16=0$\\
		$$|\angle \alpha,\beta| = \arccos\f{|39+10|}{\sqrt{198}\*\sqrt{13}} = \arccos\f{49\sqrt{286}}{858}$$
	\item Analogicky.
	\item $p([3;-1;3],(1;-1;-3))$\\
	$\alpha:6x+3y+4z-32=0$\\
		$$|\angle \alpha,p| = \arcsin\f{|6-3-12|}{\sqrt{11}\*\sqrt{61}} = \arcsin\f{9\sqrt{671}}{671}$$
	\item
		$p([2;0;5],(4;2;-6)\sim(2;1;-3))$\\
	$\alpha:13x-2y+5z-56=0$\\
		$$|\angle \alpha,p| = \arcsin\f{|26-2-5|}{\sqrt{14}\*\sqrt{198}} = \arcsin\f{19\sqrt{77}}{462}$$
\end{enumerate}

\Pr \pdf[0.3]{cv75.pdf}
$q=\zs{[4+t;3+4t;1-3t]|t\in\R}$\\
$p\in\pri{[0;0;0]q} =\alpha= \zs{[4s+t;3s+4t;s-3t]|s,t\in\R}$\\
$\alpha: x-y-z=0$\\
$p=\zs{[at,bt,ct]|t\in\R}$\\
BŮNO $a=1$ (když $a=0$, tak $b=1\land c=-1$, což evidentně nesedí)\\
$p\in\alpha \imp a-b-c = 0 \imp c=1-b$
$\ve u=(1,b,1-b)$\\
$\ve v=(1,4,-3)$\\
$\cos 30\d = \f{|1+4b-3+3b|}{\sqrt{26}\sqrt{2x^2-2x+2}}$\\
$\sqrt 3 \sqrt{26}\sqrt{2x^2-2x+2} = 2\*|7b-2|$\\
$ 3 \*{26}\*(2b^2-2b+2) = 196b^2-112b+16$\\
$0 = 40b^2+44b-140$\\
$b = -\f52 \lor b=\f75$

\underline{$p_1 = \zs{[2t;5t;3t]|t\in\R}$}\\
\underline{$p_2 = \zs{[5t;7t;-2t]|t\in\R}$}

\Pr \pdf[0.3]{cv76.pdf}
$\ve n = (1;-1;3)$\\

$A[0;4;-1]\in p$\\
$A_0 = k\*\ve n + A \in \alpha$:\\
$k-(4-k)+3(-1+3k)+2=0 \imp k = \f 5{11}$\\
$A_0[0+1\*\f5{11};4-1\*\f5{11};-1+3\*\f5{11}] = [\f5{11};\f{39}{11};\f4{11}]$

$B[-4;1;1]\in p$\\
$B_0 = k\*\ve n + B \in \alpha$:\\
$-4+k-(1-k)+3(1+3k)+2=0 \imp k = 0$\\
$B_0 = B[-4;1;1]$\\

$\ve{B_0A_0} = \(\f{49}{11};\f{28}{11};\f{-7}{11}\) = ({49};{28};{-7})$

$$\underline{p_0 = \zs{[-4+49t;1+28t;1-7t]|t\in\R}}$$

\Pr \pdf[0.3]{cv77.pdf}
$p=\zs{[t;2t;-t]|t\in\R}$\\
$\ve v  = (1,2,-1)$\\
$\ve n = (2,1,1)$

$$|\angle p,\alpha|= \arcsin\f{|2+2-1|}{\sqrt{6}\sqrt{6}} = \arcsin\f36 = \underline{\f\pi6}$$
\Pr \pdf[0.3]{cv79.pdf}
$x=3t+3s$\\
$y=-t-s $\\
$z=2t-5s$\\
$\alpha:x+3y = 0$

$\beta:2x+y-\sqrt{5}z+9=0$
$$|\angle \alpha,\beta|= \arccos\f{|2+3|}{\sqrt{10}\sqrt{10}} = \arccos\f12 = \underline{\f\pi3}$$

\Pr  177/14:
$\ve a = (-2;1)$\\
$\ve b = (0.5;1) ~ (1,2)$\\
$\ve c = (-1;1)$\\
$\ve d = (0;1)$\\

\begin{enumerate}
	\item $|\angle \ve{a}\ve{b}| = \arccos \f{|-2+2|}{\sqrt{5}\*\sqrt 5} = \arccos \f{0}{5} = \f\pi2$
	\item $|\angle \ve{a}\ve{c}| = \arccos \f{|2+1|}{\sqrt{5}\*\sqrt 2} = \arccos \f{3\sqrt{10}}{10} $
	\item $|\angle \ve{a}\ve{d}| = \arccos \f{|1|}{\sqrt{5}\*\sqrt 1} = \arccos \f{\sqrt 5}{5}$
	\item $|\angle \ve{b}\ve{c}| = \arccos \f{|-1+2|}{\sqrt{5}\*\sqrt 2} = \arccos \f{\sqrt{10}}{10} $
	\item $|\angle \ve{b}\ve{d}| = \arccos \f{|+2|}{\sqrt{5}\*\sqrt 1} = \arccos \f{2\sqrt5}{5}$
	\item $|\angle \ve{c}\ve{d}| = \arccos \f{|1|}{\sqrt{2}\*\sqrt 1} = \arccos \f{\sqrt 2}{2} = \f\pi4$
\end{enumerate}

\Pr 177/15:
$\ve u = (2,9)$\\
$\ve v = (9,-2)$

$p:9x-2y+c  = 0 $\\
$A\in p : 9\*0-2(-5)+c = 0 \imp c = -10$
$$p:9x-2y-10  = 0 $$

\Pr 192/36:
$\ve u = (1,2,-1)$
\begin{enumerate}
	\item  $ \ve a = (0;1;-1) $
		$$|\angle \ve{u}\ve{a}| = \arcsin \f{|2+1|}{\sqrt{6}\*\sqrt 2} = \arcsin \f{\sqrt 3}{2} = \f\pi34$$
	\item  $ \ve b = (1;3;-5) $
		$$|\angle \ve{u}\ve{a}| = \arcsin \f{|1+6+5|}{\sqrt{6}\*\sqrt{35}} = \arcsin \f{2\sqrt{210}}{36}$$

	\item  $ \ve c = (8;-1;3) $
		$$|\angle \ve{u}\ve{a}| = \arcsin \f{|8-2-3|}{\sqrt{6}\*\sqrt{77}} = \arcsin \f{\sqrt 462}{154}$$

\end{enumerate}

\Pr 192/37:

$x=5-r+2s$\\
$y=-3+2r-5s$\\
$z=1-3r+3s$

$2x+y = 7 - s $\\
$3x-z = 14 + 3 s$

$3(2x+y)+(3x-z) = 9x+3y-z = 35$

$\ve v = (9;3;-1)$


\begin{enumerate}
	\item  $ \ve a = (0;1;-1) $
		$$|\angle \ve{u}\ve{a}| = \arcsin \f{|3+1|}{\sqrt{81}\*\sqrt 2} = \arcsin \f{\sqrt 3}{2} = \f\pi34$$
	\item  $ \ve b = (1;3;-5) $
		$$|\angle \ve{u}\ve{a}| = \arcsin \f{|9+9+5|}{\sqrt{81}\*\sqrt{35}} = \arcsin \f{2\sqrt{210}}{36}$$

	\item  $ \ve c = (8;-1;3) $
		$$|\angle \ve{u}\ve{a}| = \arcsin \f{|72-3-3|}{\sqrt{81}\*\sqrt{77}} = \arcsin \f{\sqrt 462}{154}$$

\end{enumerate}




\EndDoc
