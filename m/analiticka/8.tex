\providecommand{\HINCLUDE}{NE}
\if ^\HINCLUDE^
\else
\def\HINCLUDE{}
\global\newdimen\Okraje
\global\Okraje =4cm
\input{$HOME/souteze/_hlavicka/h-.tex}

%\definecolor{colorV}{RGB}{255,127,0}
%\definecolor{colorPoz}{RGB}{153,51,0}
%\definecolor{orangeV}{RGB}{255,127,0}
%\definecolor{colorPr}{RGB}{0,5,255}
%\definecolor{colorDef}{RGB}{0.255,0}

\usepackage[shortlabels]{enumitem}
\setlength{\marginparsep}{2pt}
\setlength{\marginparwidth}{35pt}

\def\st{{\rm st}}
\def\P{{\rm P}}

\def\ISENUM{}
\def\inMargin#1{\End
		
		\hskip0pt \marginpar{{{#1}}}}
\newcounter{V}[section] 
\newcommand{\V}[1][]{\stepcounter{V}\inMargin{\textcolor{green}{V.\arabic{section}.\theV.:}}\ifx^#1^\else\textcolor{green}{\underline{{#1}:}}\addcontentsline{toc}{subsubsection}{V.\arabic{section}.\theV.:$\quad$ {#1}}\\\fi}
\def\Def{\inMargin{\textcolor{red}{Def:}}}
\def\Poz{{\inMargin{\textcolor{brown}{Pozn:}}}}
\def\Pr{{\inMargin{\textcolor{blue}{Př:}}}}
\def\Pozenum
{
	\begin{enumerate}[1)]%, left = 0pt ]
		\item\inMargin{\textcolor{brown}{Pozn:}}\def\ISENUM{a}}
\def\End
{
	\if	^\ISENUM^
	\else \end{enumerate}
	\fi
	\def\ISENUM{}
}
\reversemarginpar

\makeatletter
\renewcommand\thesection{§\arabic{section}.}
\renewcommand\thesubsection{\Alph{subsection})}
\renewcommand\thesubsubsection{\alph{subsubsection})}
\newcounter{chapter}
\setcounter{chapter}{0}
\renewcommand\thechapter{\Alph{chapter})}
\newcounter{roman}
\setcounter{roman}{0}
\renewcommand\theroman{\Roman{roman}.}
\makeatother
\def\sectionnum#1{\setcounter{section}{#1}\addtocounter{section}{-1}}
\def\subsectionnum#1{\setcounter{subsection}{#1}\addtocounter{subsection}{-1}}
\def\subsubsectionnum#1{\setcounter{subsubsection}{#1}\addtocounter{subsubsection}{-1}}
\def\chapternum#1{\setcounter{chapter}{#1}\addtocounter{chapter}{-1}}
\def\chapter#1{

	\addtocounter{chapter}{1}\sectionnum{1}
	\addcontentsline{toc}{section}{\large{\thechapter$\quad${#1}}}
	
	{\LARGE  \textbf{\begin{minipage}[t]{0.1\textwidth}\thechapter\end{minipage}\begin{minipage}[t]{0.95\textwidth}#1\end{minipage}}}

}
\def\ROM{}
\def\Rom#1#2{\setcounter{roman}{#1}\renewcommand\ROM{#2}}

\Rom{6}{Funkce}
\title{\Huge\textbf{\theroman\quad \ROM}}
\author{Jiří Kalvoda}

\newcounter{countOfBegin}
\setcounter{countOfBegin}{0}
\newcommand{\BeginDoc}[1][]
{
	\ifnum\value{countOfBegin}=0
	\begin{document}
		#1
		\fi
	\addtocounter{countOfBegin}{1}
		
}
\def\EndDoc
{
	\addtocounter{countOfBegin}{-1}
	\ifnum\value{countOfBegin}=0
	\end{document}
	\fi
}

\fi
\BeginDoc{\sectionnum{8}}
\section{Vzdálenost}
\def\mc{\mathcal}
\Def Nechť $\mathcal A,\mathcal B\subset \E_3$
 jsou 2 podprostory Euklidovského vektorového prostoru.
 Vzdáleností podprostorů $\mathcal A,\mathcal B$ nazýváme nezáporné reálné číslo $\rho( \mathcal A,\mathcal B)$ definované
  takto: $\rho(\mathcal A,\mathcal B) =  \min\zs{ |AB| : A\in \mathcal A, B \in \mathcal B}$, kde $|AB|$ je délka úsečky $AB$.

  \Poz
  \underline{  Jiné zavedení vzdálenosti podprostorů:}\\
Nechť $M \in \R$ je množina. Pak číslo $i \in R$ nazýváme infimem množiny $M$ , je-li
 největší dolní závorou množiny $M$ , tj. jestliže platí:
 \begin{enumerate}
	 \item $\forall m \in M: i\le m$
	 \item $\forall r \in \R : i(\forall m \in M : r\le m) \imp i \ge r$
 \end{enumerate}

 Platí: Každá neprázdná zdola omezená množina reálných čísel má infimum. 

 Nechť i$\mc A,\mc B\in \E_3$
  jsou 2 podprostory Euklidovského vektorového prostoru a nechť
  $D=\zs{ AB : A\in\mc A, B\in\mc B}$. Pak vzdáleností podprostorů $\mc A,\mc B$ nazýváme infimum
  množiny $D$ .

  Platí: Má-li množina $D$ nejmenší prvek $n$, pak tento prvek je infimum
  $\imp n = \rho(A,B)$.
  
  \Poz Jestliže $\mc A,\mc B$ mají nějaký společný bod, pak $\rho( \mc A,\mc B) = 0$.

  \subsection{Vzdálenost 2 bodů v $\E_2,\E_3$}

  \Poz $$\rho(A,B) = |AB|  = |\ve{AB}|$$
  \V Nechť $A[a_1,a_2,a_3],B[b_1,b_2,b_3]$ jsou dva body.

  Pak platí $\rho(A,B) = \sqrt{(b_1-a_1)^2+(b_2-a_2)^2+(b_3-a_3)^2} = \sqrt{\sum_{j=1}^{3(2)}(b_j-a_j)^2}$ 
[Dk. viz V.7.3 a pozn. v §7 kap.X] 

\Pr Určete $\rho(A,B)$:
\begin{enumerate}
	\item $A[3;-1],B[0;2]$\\
		$\rho(A,B) = 3 \sqrt 2$
	\item $A[1;0;2];B[-1;2;1]$
		$\rho(A,B) = 3 $
\end{enumerate}
\subsection{Vzdálenost bodu od přímky v $\E_2$}

\Poz
$ \rho(A,p) = \rho(A,A_0)$, kde $A_0$ je kolmý průmět $A$ na $p$.\\
$p:ax+by+c=0; \ve n = (a,b)\ne \ve 0; A[a_1,a_2]$\\
$q\perp p$ -- $q:x=\zs{[a_1+ta;a_2+tb]|t\in\R}$\\
$A_0[a_1+t^*a,a_2+t^*b] \in p \cap q$

$a(a_1 + t^*a)+b(a_2+t^*b)+c=0$\\
$t^* = \f{-aa_1-b_a2-c}{a^2+b^2}$\\
$\rho(A,p) = \rho(A,A_0) = |\ve{AA+0}|$\\
$\ve{AA_0} = A_0 - A = (t^*a,t^*b)$
$$ |AA_0| = \sqrt{(t^*a)^2 (t^*b^2)} = |t^*|\*\sqrt{a^2+b^2} = \f{|aa_1+bb_2+c|}{\sqrt{a^2+b^2}}$$

\V Nechť $A[a_1;a_2]\in\E_2$ je bod, $p:ax+by+c = 0;[a,b]\ne[0;0]$ je přímka.
Pak platí:
$$ \rho(A,p) =  \f{|aa_1+bb_2+c|}{\sqrt{a^2+b^2}}$$

\Pr Určete $\rho(A,p)$:
$A[-1;2];p:x-2y+1=0$\\
$\rho(A,p) = \f{|-1 - 2\*2+1|}{\sqrt{1^2+2^2}} = \f4{\sqrt 5} = \f{4\sqrt 5}5$

\Pr 200/47:\\
$A[3;2];;p:3x-4y-7=0$\\
$\rho(A,p) = \f{|9 - 8-7|}{\sqrt{9+16}} = \f6{5}$
\Pr 200/48:\\
Určete velikosti výšek trojúhelníku $ABC$:\\
$A[0;0]$\\
$B[7;0]$\\
$C[4;5]$

$\ve{BC} = (-3,5) \imp a: 5x+3y+?=0; 35+0+?=0 \imp a: 5x+3y-35=0$\\
$\ve{AB} = (7,0) \imp c: 0x+y=0$\\
$\ve{AC} = (4,5) \imp b: 5x-4y+?=0; 0-0+?=0 \imp b: 5x-4y=0$\\

$\rho(A,a) = \f{|-35|}{\sqrt{25+9}} = \sqrt{34}\f{35}{34}$
$\rho(B,b) = \f{|35|}{\sqrt{25+16}} = \sqrt{41}\f{35}{41}$
$\rho(C,c) = \f{|5|}{\sqrt{25}} = 1$

\subsection{Vzdálenost bodu od přímky v $\E_3$}

\Poz
$ \rho(A,p) = \rho(A,A_0)$, kde $A_0$ je kolmý průmět $A$ na $p$.\\
\begin{enumerate}[I. způsob:]
	\item
		$p(P,\ve u);P[p_1,p_2,p_3];\ve u(u_1,u_2,u_3),A[a_1,a_2,a_3]$\\
		$A_0[p_1+t^*u_1+p_2+t^*u_2,p_3+t^*y_3]$\\
		$\ve n = \ve{AA_0} = (
			p_1-a_1+t^*u_1
			p_2-a_2+t^*u_2
			p_3-a_3+t^*u_3
			)$\\
			$\ve n \perp \ve u \imp \ve n\*\ve u = 0$.

			Jedna rovnice o jedné neznámé $t^*$, po určení $t^*$ určíme $A_0$.
		\item
			Určení rovnice roviny $\rho$, která prochází $A$ a je kolmá k přímce $p$.
			$$p\cap\rho = \{A_0\}$$
		\item
			Vyjádření $|\ve{AX}|$, kde $X$ je libovolný bod $p$ jako funkce proměnné $t$ (parametr přímky) a určení minima této funkce.

\end{enumerate}

\Pr Určete $\rho(A,p)$\\
$A[1;0;1],p={[2-t;t;0],t\in\R}$.
\begin{enumerate}[I. způsob:]
	\item $\ve{AA_0}=(1-t^*;t^*;-1)$\\
		$\ve u = (-1;1;0)$\\
		$\ve n\* \ve u = t^*-1+t^*$\\
		$t^*=\f12 \imp A_0 = \[\f32;\f12;0\]$
		$\ve{AA_0}=\(\f 12;\f 12;-1\)\imp\rho(A,A_0) = \f{\sqrt 6}2$

	\item
		$\ve{n_\rho} = \ve u = (-1;1;0)$\\
		$\rho:-x+y+d=0$\\
		$A\in\rho \imp -1+d=0 \imp d=1$\\
		$\rho: -x+y+1=0$\\
		$p \cap \rho: -(2-t)+t+1=0$\\
		$-2+2t+1=0 \imp t=\f 12$.
		(dále stejně jako v předchozím bodě)
	\item $X[2-t;t;0];\ve{AX}=(1-t;t;-1)$
		$$
		|\ve{AX}| = \sqrt{(1-t)^2 + t^2 + (-1)^2} = \sqrt{2\(t-\f12\)^2 + \f 32}
		$$
		minimum nastane pro $t=\f 12$ a nabývá hodnoty $\f{\sqrt 6}2$.
\end{enumerate}

\Pr 200/45\\
$p = \zs{[1-1t;2+3t;4+t]|t\in\R}; M[1;4;5]$
\begin{enumerate}[I. způsob:]
	\item $\ve{MM_0}=(-t;-2+3t;-1+t)$\\
		$\ve u = (-1;3;1)$\\
		$\ve n \* \ve u = t-6+9t-1+t \imp t = \f{7}{11}$\\
		$M_0 = \[\f4{11};\f{43}{11};\f{51}{11}\]$\\
		$|MM_0| = \sqrt{\(\f4{11}-1\)^2+\(\f{43}{11}-4\)^2+\(\f{51}{11}-5\)^2} = \f{2\sqrt{286}}{11}$


	\item
		$\ve{n_\rho} = \ve u = (-1;3;1)$\\
		$\rho:-x+3y+z+d=0$\\
		$M\in\rho \imp -1+12+5+d=0 \imp d=-16$\\
		$\rho: -x+3y+z-16=0$\\
		$p \cap \rho: -1+t+6+9t+4+t-16=0$\\
		$11t-7=0 \imp t=\f 7{11}$.
		(dále stejně jako v předchozím bodě)
	\item $X[2-t;t;0];\ve{AX}=(1-t;t;-1)$
		$$
		|\ve{MX}| = \sqrt{(-t)^2 + (3t-2)^2 + (t-1)^2} = \sqrt{t^2+9t^2-12t+4+t^2-2t+1} =$$$$= \sqrt{11t^2-14t+5} = \sqrt{11(t-\f{7}{11})^2-\f{49}{11}+5}
		$$
		minimum nastane pro $t=\f 7{11}$ a nabývá hodnoty $\f{2\sqrt {286}}{11}$.
\end{enumerate}
\Pr 200/46\\
$p=\zs{[5+5t;3-4t;2]|t\in\R}; C[4;12;4]$\\
$|CX|
= \sqrt{(1+5t)^2 + (9+4t)^2 + 2^2}
= \sqrt{1+10t+25t^2 + 81+72t+16t^2 + 4}
= \sqrt{41x^2+82x+86}
= \sqrt{41(x^2+1)^2+86r-41}
$
Minimum je $\sqrt{45}  = 3\sqrt 5$.

\Pr 200/49\\
$p=\zs{[t;1-t;2t]|t\in\R} ; A[1,0,5]$
$|AX|
= \sqrt{(t-1)^2 + (t-1)^2 + (2t-5)^2} =\\
= \sqrt{t^2-2t+1 + t^2-2t+1 + 4t^2-20t+25}
= \sqrt{6t^2-24t+27} 
= \sqrt{6(t^2-2)^2+3} 
$
Minimum je $\sqrt 3$.
\subsection{Vzdálenost bodu od roviny v $\E_3$}
\Poz $\rho(A,\alpha) = \rho(A,A_0)$, kde $A_0$ je kolmý průmět bodu $A$ do roviny $\alpha$.
\pdf[0.3]{8-d.pdf}
$\alpha: ax+by+cz+d = 0; [a,b,c] \ne [0,0,0]; A[a_1,a_2,a_3$

$q\perp \alpha \imp p={[a_1+ta;b_1+tb;c_1+tc]|t\in\R}$\\
$A_0[a_1+t^*a;b_1+t^*b;c_1+t^*c]\in\alpha\cap q$\\
$a(a_1+t^*a)+b(b_1+t^*b)c(c_1+t^*c)=0$\\
$t^* (a^2+b^2+c^2) = -aa_1 - ba_2 - ca_3 -d$
$t^*  = \f{-aa_1 - ba_2 - ca_3 -d}{a^2+b^2+c^2}$

$\rho(A,\alpha) = |\ve{AA_0}|$\\
$\ve{AA_0}=A_0-A = (t^*a,t^*b,y^*c)$\\
$|\ve{AA_0} = \sqrt{(t^*a)^2+(t^*b)^2+(y^*c)^2} = \f{|aa_1+ba_1+ca_1+d|}{\sqrt{a^2+b^2+c^2}}$

\V Nechť $A[a_1;a_2;a_3]$ je bod a $\alpha:ax+by+cz+d=0;(a,b,c)\ne\ve 0$ je rovina. Pak platí:
$$
\rho(A,\alpha) = \f{|aa_1+ba_1+ca_1+d|}{\sqrt{a^2+b^2+c^2}}
$$

\Pr Určete $\rho(A,\alpha)$\\
$A[1;-1;0],\alpha : x-y+2z -1 = 0$
$$ 
\rho(A,\alpha) = \f{|1+1+0-1|}{\sqrt{1+1+4}} = \f1{\sqrt 6}=\f{\sqrt 6}6
$$
\Pr 200/50
Určete $\rho(A,\alpha)$\\
$A[1;0;5];\alpha:12x+3y-4z = 0$
$$
\rho(A,\alpha)= \f{|12+0-20|}{\sqrt{144+9+16}} = \f{8}{\sqrt{169}}=\f8{13}
$$

\subsection{Vzdálenost dvou rovnoběžných přímek v $E_3$}
\Poz $\rho(p,q) = \rho(A,q)$, kde $A\in p$ je libovolný bod.\\
$p:ax+by+c_1 = 0$
$p:ax+by+c_2 = 0; (a,b)\neq \ve 0$

Zvolíme $A[a_1,a_2]\in p$:

$\rho(A,q) = \frac{|aa_1+ba_2+c_2|}{\sqrt{a^2+b^2}}$

$A\in p : aa_1+ba_2+c_1 = 0 \imp aa_1+ba2 = -c$

\underline{$\imp \rho(p,q) = \f{|c_2-c_1|}{\sqrt{a^2+b^2}}$}

\V Nechť $p:ax+by+c_1 = 0; q: ax+by+c_2 = 0$ jsou dvě rovnoběžky, $(a,b) \neq \ve 0$. Pak platí:

$$ \rho(p,q) = \f{|c_1-c_2|}{\sqrt{a^2+b^2}} $$

\Pr Určete $\rho(p,q)$:\\
$p: 4x-2y+1 = 0 \sim 2x-y+0.5=0$\\
$q: 2x-y+3=0$

$\rho(p,q) = \f{\left|3-\f12\right|}{\sqrt{2^2+(-1^2}} = \f{\sqrt 5}2$
 

\subsection{Vzdálenost dvou rovnoběžných přímek v $E_2$}
\Poz $\rho(p,q) = \rho(A,q)$, kde $A\in p$ je libovolný bod.\\

Na přímce $p$ zvolíme libovolný bod, dále viz C).

\Pr Určete $\rho(p,q)$:\\
$p  =\zs{[1+2t;-2t;-4t]|t\in\R}$\\
$q  =\zs{[1-s;s;2+2s]|s\in\R}$

$A[1;0;0]\in p$\\
$B\in q$

$\rho(A,B) = \sqrt{(1-s-1)^2+s^2+(2+2s)^2} = \sqrt{6s^2+8s+4=6(s+\f23)-\f49\*6+4 }=\sqrt{ \f 43}$.

$$\rho(A,q) = \rho(p,q) = \sqrt{\f 4 3}$$

\subsection{Vzdálenost přímky od roviny s ní rovnoběžné v $\E_3$}
\Poz
$\rho(p,\alpha) = \rho(A,\alpha)$, kde $A\in p$ je libovolný bod.

Na přímce $p$ zvolíme libovolný bod, dále viz D).

\Pr Určete $\rho(p,\alpha)$:\\
$p = \zs{[-1+2t;1-t;2+3t]|t\in\R}$\\
$\alpha : x+5y+z-3 =0$

Ověření $p\parallel \alpha \ekv \ve{u_p} \perp \ve{n_\alpha} \ekv\ve{u_p} \cdot \ve{n_\alpha}=0 $:\\
$(2;-1;3)\cdot (1;5;1) = 2 - 5 + 3 = 0 \imp$ platí.

$A[-1;1;2] \in p$:

$$ \rho(p,\alpha) = \rho(A,\alpha) = \f{|-1+5+2-3)}{\sqrt{1+25+1}} = \f{3}{\sqrt{27}} = \f{\sqrt 3}3$$

\subsection{Vzdálenost dvou rovnoběžných rovin v $\E_3$}

\Poz $\rho(\alpha,\beta) = \rho(A,\beta)$, kde $A\in\alpha$ je libovolný bod.
\pdf[0.2]{8-H.pdf}

$\alpha: ax+by+cz+d_1 = 0$\\
$\beta : ax+by+cz+d_2 = 0;\ (a,b,c) \neq \ve 0$


$A[a_1,a_2,a_3] \in \alpha$.

$\rho(A,\beta) = \f{|aa_1+ba_2+ca_3+d_2|}{\sqrt{a^2+b^2+c^2}}$

$A\in \alpha : aa_1 + ba_2 +ca_3+d_1 = 0 \imp aa_1 + ba_2 +ca_3=-d_1 $

$\imp$\underline{$\rho(\alpha,\beta)=\f{|d_2-d_1|}{\sqrt{a^2+b^2+c^2}}$}

\V Nechť 
$\alpha: ax+by+cz+d_1 = 0, 
\beta : ax+by+cz+d_2 = 0$
jsou dvé různé rovnoběžné roviny, $(a,b,c) \neq \ve 0$. Pak platí:

$$\rho (\alpha,\beta) = \f{|d_2-d_1|}{\sqrt{a^2+b^2+c^2}}$$

\Pr Určete $\rho(\alpha,\beta)$:\\
$\alpha : 3x-y+2z-1=0$\\
$\beta  : -\f32+\f12y-z+5=0 ~ 3x-y+2z-0=0$

$\rho(\alpha,\beta) = \f{|-10+1|}{\sqrt{14}} = \f{9}{\sqrt{14}}=\f{9\sqrt{14}}{14}$

\subsection{Vzdálenosti dvou mimoběžných přímek v $\E_3$}

\Poz $\rho(p,q) = \rho(\alpha,\beta)$, kde $p\subset \alpha; q\subset\beta; \alpha\parallel \beta$.

\begin{enumerate}[I. způsob:]
	\item Z definice:

		$p(A,\ve u);q(B,\ve v)$

		$\ve{n_\alpha} = \ve{n_\beta} = \ve{u}\times \ve v = (n_1,n_2,n_3)$\\
		$\alpha: n_1x+n_2y+n_3z+d_1=0$\\
		$A\in \alpha \imp$ dosazením do předchozí rovnice určíme $d_1$.\\
		$\beta: n_1x+n_2y+n_3z+d_2=0$\\
		$B\in \beta \imp$ dosazením do předchozí rovnice určíme $d_2$.

		Dále viz H).

	\item Pomocí osy mimoběžek:

		Nechť $o$ je osa mimoběžek $p,q$.\\
		Určíme průniky
		$o\cap p = \zs{P}, o\cap q = \zs{Q}$
\end{enumerate}

\Pr Určete $\rho(p,q)$:\\
$p = \zs{[9+4t;-2-3t;t]|t\in\R}$\\
$q = \zs{[-2r;-7+9r;2+2r]|r\in\R}$
\begin{enumerate}[I. zp.:]
	\item $\ve u = (4;-3;1); \ve v = (-2;9;2)$\\
		$\ve n = \ve u \times \ve v = \(-3\*2-1\*9;4\*(-2)-4\*;4\*9-(-3)\*(-2)\) = (-15;-10;30) \rightarrow (3;2;-6)$

		$\alpha: 3x+2y-6z+d_1=0$\\
		$A[9;2;-1]\in\alpha \imp 27-4+0+d_1 \imp d_1 = -23$.

		$\beta: 3x+2y-6z+d_2=0$\\
		$B[0;-7;2]\in\beta \imp 0-14-12+d_2 \imp d_1 = 26$.

		$\rho(p,q) = \rho(\alpha,\beta) = \f{|26+3|}{\sqrt{9+4+36}}=\f{49}{\sqrt{49}} = 7$
		
	\item
		Hleddáme $o(P,\ve w)$\\
		$\ve w = \ve u\times \ve v = \dots = (3;2;-6)$

		Příčka $p,q$ rovnoběžná s $\ve w$:\\
		$\ve{AB} = (-9;-5;2)$\\
		$
		\left.
		\begin{array}{c}
			P = A + k \ve u \\
			Q = B + l \ve v 
		\end{array}
		\right\} \ve{PQ} = B +l \ve v - A - k \ve u = x \ve w 
		$

		 $ \begin{pmatrix}
			 3 &4 &2 &\vline& -9 \\ 
			 2 &-3 &-9 &\vline& -5 \\ 
			 -6 &1 &-2 &\vline& 2 \\ 
		 \end{pmatrix}
		 \sim
		 \begin{pmatrix}
			 3 &4 &2 &\vline& -9 \\ 
			 2 &-3 &-9 &\vline& -5 \\ 
			 6 &-1 &2 &\vline& -2 \\ 
		 \end{pmatrix}
		 \sim
		 \begin{pmatrix}
			 2 &-3 &-9 &\vline& -5 \\ 
			 3 &4 &2 &\vline& -9 \\ 
			 6 &-1 &2 &\vline& -2 \\ 
		 \end{pmatrix}
		 \sim
		 \begin{pmatrix}
			 2 &-3 &-9 &\vline& -5 \\ 
			 0 &17 &31 &\vline& -3 \\ 
			 0 &8 &29 &\vline& 13 \\ 
		 \end{pmatrix}
		 \sim
		 \begin{pmatrix}
			 2 &-3 &-9 &\vline& -5 \\ 
			 0 &8 &29 &\vline& 13 \\ 
			 0 &17 &31 &\vline& -3 \\ 
		 \end{pmatrix}
		 \sim
		 \begin{pmatrix}
			 2 &-3 &-9 &\vline& -5 \\ 
			 0 &8 &29 &\vline& 13 \\ 
			 0 &0 &-245 &\vline& -245 \\ 
		 \end{pmatrix}
		 \sim
		 \begin{pmatrix}
			 2 &-3 &-9 &\vline& -5 \\ 
			 0 &8 &29 &\vline& 13 \\ 
			 0 &0 &1 &\vline& 1 \\ 
		 \end{pmatrix}
		 \sim
		 \begin{pmatrix}
			 16 &0 &15 &\vline& -1 \\ 
			 0 &8 &29 &\vline& 13 \\ 
			 0 &0 &1 &\vline& 1 \\ 
		 \end{pmatrix}
		 \sim
		 \begin{pmatrix}
			 16 &0 &0 &\vline& -16 \\ 
			 0 &8 &0 &\vline& -16 \\ 
			 0 &0 &1 &\vline& 1 \\ 
		 \end{pmatrix}
		 \sim
		 \begin{pmatrix}
			 1 &0 &0 &\vline& -1 \\ 
			 0 &1 &0 &\vline& -2 \\ 
			 0 &0 &1 &\vline& 1 \\ 
		 \end{pmatrix}
		  $ 

		  $ P = [9;-2;0] - 2 (4;-3;1) = [1;4;-2]$\\
		  $ Q = [0;-7;2] + (-2;9;2)   = [-2;2;4]$

		  $\imp |PQ| = \sqrt{(-2-1)^2 + (2-4)^2 + (4+2)^2} =\sqrt{49} = 7 \imp \rho (p,q) = 7$


\end{enumerate}

\Pr 96/100:
\pdf[0.3]{8-cv100.pdf}
$p = \zs{[7+t; 3+2t;9-t]|t\in\R}$\\
$q = \zs{[3-7s;1+2s;1+3s];s\in\R}$

$\ve u = ( 1; 2;-1)$\\
$\ve v = (-7; 2; 3)$\\
$\ve{AB}=(-4;-2;-8)$ 


$\ve w = (2\*3+1\*2;7-3;2+14) = (8;4;16) \sim (-4;2;8) \imp $ příčkou je $AB$.

$P = [7;3;9]$\\
$Q = [3;1;1]$

$|PQ| = \sqrt{4^2+2^2+8^2} = 2\sqrt{21}$






       \Pr 92/82
       \pdf[0.2]{8-cv82.pdf}
       Nechť hledaná přímka je tvaru $p:4x-3y+c=0$ (z rovnoběžnosti).
       Vzdálenost tedy je $\f{|4\*2-3\*3+c|}{\sqrt{16+9}} = \f{|-1+c|}5 = 5 \imp c = 5\*5+1 = 26 \lor c = 1-5\*5=-24$
       $$ p_1: 4x-3y+26$$
       $$ q_2: 4x-3y-24$$
       
       \Pr 92/83
       \pdf[0.2]{8-cv83.pdf}
       Hledám bod $A[0,y]$.
       Vzdálenost k počátku: $|y|$.\\
       Vzdálenost k přímce: $\f{|-4y+12|}{\sqrt{9+16}}= \f{4}{5}\*|y-3|$.\\

       Když $y\le0$: $-y=\f 45 (-y+3) \imp -5y = -4y+12\imp y=-12\in\(-\infty;0\>$\\
       Když $0\le y \le 3$: $y=\f 45 (-y+3) \imp 5y = -4y+12\imp 9y=12 \imp y = \f{9}{12} \in \<0,3\>$\\
       Když $3\le y$: $y=\f 45 (y-3) \imp 5y = 4y-12\imp y=-12\not\in \<3,\infty\)$\\

       $$A_0 = \[0;-12\]$$
       $$A_1 = \[0;\f9{12}\]$$

       \Pr 92/84
       \pdf[0.2]{8-cv84.pdf}
       Nechť hledaná přímka je tvaru $p:ax+by+c=0$.\\
       Když $a=0$: $A\in p \imp 1b+c=0 \imp p:0x+y-1=0 \imp \rho(B,p) = 0 \neq 3 $ $\imp$ spor.
       BÚNO tedy $a=1$.\\
       $A\in p\imp -2+b+c=0 \imp b+c=2$\\
       $\rho(p,B) = 4 \imp \f{|3+b+c|}{\sqrt{1+b^2}} = 4  \imp \f{5}{\sqrt{1+b^2}}=4 \imp \f{5}4 = \sqrt{1+b^2} \imp \f{25}{16} = 1+b^2 \imp b = \pm\sqrt{\f{9}{16}} = \pm\f34$.

       Když $b=\f34$: $c = \f 54$. Zkouška: $\rho(p,B) = \f{3+\f 34+\f 54}{\sqrt{1+\(\f 34 \)^2}} = \f{5}{\sqrt{\f{25}{16}}}=4$\\
       Když $b=-\f34$: $c = \f {11}4$. Zkouška: $\rho(p,B) = \f{3-\f 34+\f{11}4}{\sqrt{1+\(-\f 34 \)^2}} = \f{5}{\sqrt{\f{25}{16}}}=4$
       $$ p_1:x+\f34y+\f54=0 \ekv p_1:4x+3y+5=0$$
       $$ p_1:x-\f34y+\f{11}4=0 \ekv p_1:4x-3y+11=0$$
       
       \Pr 92/85
       \pdf[0.2]{8-cv85.pdf}
       Nechť hledaná přímka je tvaru $p:ax+by+c=0$.\\
       Když $b=0$: $A\in p \imp a+c=0 \imp p:x-1=0 \imp \rho(B,p) = 3 \neq 5 = \rho(C,p) $ $\imp$ spor.
       BÚNO tedy $b=1$.\\
       $A\in p \imp a+2+c=0\imp a+c = -2$\\
       $\rho(p,B) = \rho(p,C) \imp \f{|3a+3+c|}{\sqrt{a^2+b^2}} = \f{|5a+2+c|}{\sqrt{a^2+b^2}} \imp |3a+3+c|=|5a+2+c| \imp |2a+1| = |4a|$

       Když $a\le -\f12 \lor a\ge0$: $2a+1=4a \imp a = \f 12$\\[2px]
       Když $a\ge -\f12 \land a\le0$: $2a+1=-4a \imp a = -\f16$.
       $$p_1:\f 12 x + y - \f 52 =0 \Ekv p: x + 2y -5 =0$$
       $$p_2:-\f 16 x + y - \f {11}6 =0 \Ekv p: x - 6y +11 =0$$
\EndDoc
