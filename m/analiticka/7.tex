\providecommand{\HINCLUDE}{NE}
\if ^\HINCLUDE^
\else
\def\HINCLUDE{}
\global\newdimen\Okraje
\global\Okraje =4cm
\input{$HOME/souteze/_hlavicka/h-.tex}

%\definecolor{colorV}{RGB}{255,127,0}
%\definecolor{colorPoz}{RGB}{153,51,0}
%\definecolor{orangeV}{RGB}{255,127,0}
%\definecolor{colorPr}{RGB}{0,5,255}
%\definecolor{colorDef}{RGB}{0.255,0}

\usepackage[shortlabels]{enumitem}
\setlength{\marginparsep}{2pt}
\setlength{\marginparwidth}{35pt}

\def\st{{\rm st}}
\def\P{{\rm P}}

\def\ISENUM{}
\def\inMargin#1{\End
		
		\hskip0pt \marginpar{{{#1}}}}
\newcounter{V}[section] 
\newcommand{\V}[1][]{\stepcounter{V}\inMargin{\textcolor{green}{V.\arabic{section}.\theV.:}}\ifx^#1^\else\textcolor{green}{\underline{{#1}:}}\addcontentsline{toc}{subsubsection}{V.\arabic{section}.\theV.:$\quad$ {#1}}\\\fi}
\def\Def{\inMargin{\textcolor{red}{Def:}}}
\def\Poz{{\inMargin{\textcolor{brown}{Pozn:}}}}
\def\Pr{{\inMargin{\textcolor{blue}{Př:}}}}
\def\Pozenum
{
	\begin{enumerate}[1)]%, left = 0pt ]
		\item\inMargin{\textcolor{brown}{Pozn:}}\def\ISENUM{a}}
\def\End
{
	\if	^\ISENUM^
	\else \end{enumerate}
	\fi
	\def\ISENUM{}
}
\reversemarginpar

\makeatletter
\renewcommand\thesection{§\arabic{section}.}
\renewcommand\thesubsection{\Alph{subsection})}
\renewcommand\thesubsubsection{\alph{subsubsection})}
\newcounter{chapter}
\setcounter{chapter}{0}
\renewcommand\thechapter{\Alph{chapter})}
\newcounter{roman}
\setcounter{roman}{0}
\renewcommand\theroman{\Roman{roman}.}
\makeatother
\def\sectionnum#1{\setcounter{section}{#1}\addtocounter{section}{-1}}
\def\subsectionnum#1{\setcounter{subsection}{#1}\addtocounter{subsection}{-1}}
\def\subsubsectionnum#1{\setcounter{subsubsection}{#1}\addtocounter{subsubsection}{-1}}
\def\chapternum#1{\setcounter{chapter}{#1}\addtocounter{chapter}{-1}}
\def\chapter#1{

	\addtocounter{chapter}{1}\sectionnum{1}
	\addcontentsline{toc}{section}{\large{\thechapter$\quad${#1}}}
	
	{\LARGE  \textbf{\begin{minipage}[t]{0.1\textwidth}\thechapter\end{minipage}\begin{minipage}[t]{0.95\textwidth}#1\end{minipage}}}

}
\def\ROM{}
\def\Rom#1#2{\setcounter{roman}{#1}\renewcommand\ROM{#2}}

\Rom{6}{Funkce}
\title{\Huge\textbf{\theroman\quad \ROM}}
\author{Jiří Kalvoda}

\newcounter{countOfBegin}
\setcounter{countOfBegin}{0}
\newcommand{\BeginDoc}[1][]
{
	\ifnum\value{countOfBegin}=0
	\begin{document}
		#1
		\fi
	\addtocounter{countOfBegin}{1}
		
}
\def\EndDoc
{
	\addtocounter{countOfBegin}{-1}
	\ifnum\value{countOfBegin}=0
	\end{document}
	\fi
}

\fi
\BeginDoc{\sectionnum{7}}
\section{ Příčka mimoběžek}
\Def Nechť $p, q$ jsou 2 mimoběžné přímky. Přímka $r$, která je různoběžná s oběma
přímkami $p, q$, se nazývá \emph{příčka mimoběžek} $p, q$ . 

\Poz Nechť $p(A,\ve u);q(B,\ve v)$ jsou přímky.
Pak pro příčku mimoběžek $r(Q,\ve w)$ platí:
$\ve w \in \<\ve u,\ve w,\ve{AB}\>$.

\subsection{Nalezení příčky $r$ mimoběžek $p, q$, která je rovnoběžná s daným vektorem}
\Poz Dáno: $p(A,\ve u);q(B,\ve v);\ve w \ne \ve 0$

Rozbor:
\begin{enumerate}[1)]
	\item $\ve w  \in \<\ve u , \ve v\> \imp$ 0 řešení
	\item $\ve w  \not\in \<\ve u , \ve v\> \imp$ 1 řešení
\end{enumerate}
\pdf[0.2]{7-a.pdf}
$\left.\begin{array}{l} P = A + k \* \ve u\\ Q = B + l\* \ve v\end{array}\right\} \imp \ve{PQ} = Q - P = B + l \ve{v} - A - k\ve u$\\
	$\ve{PQ} = x \ve w $\\
	$\imp B + l \* \ve v - A - k \ve u = x \ve w$\\
	\underline{$\ve AB = x \ve w + k \ve u - l \ve v$}


	\Pr Jsou dány mymoněžky $p(A,\ve u),q(B,\ve v)$, a vektor $\ve w$:\\
	$A[1;-2;5]; B[-1;1;-5]; \ve u (1;3;-1), \ve v (1;1;2), \ve w (1;1;4)$\\
	Najděte příčku $p,q$, která je rovnoběžná s $\ve w$.

	$\ve{AB} = (-2;3;-10) \imp (-2;3;-10) = x (1;1;4) - l (1;1;2) + k (1;3;-1)$.
	Hledáme body $P,Q$, pro které platí:
	$P = A + k \* \ve u; Q = B + l\*\ve v$ a zároveň $\ve{PQ} = x \* \ve w$.\\
	$\imp x \*\ve w  = Q-P = B-A + l \ve v - k \ve u$\\
	$k\ve u-l\ve v + x \ve w=\ve {AB}= (-2,3;-10)$.

 $ \bar{A} = \begin{pmatrix}
	 1 &-1 &1 &\vline& -2 \\ 
	 3 &-1 &1 &\vline& 3 \\ 
	 -1 &-2 &4 &\vline& -10 \\ 
 \end{pmatrix}
 \sim
 \begin{pmatrix}
	 1 &-1 &1 &\vline& -2 \\ 
	 3 &-1 &1 &\vline& 3 \\ 
	 1 &2 &-4 &\vline& 10 \\ 
 \end{pmatrix}
 \sim
 \begin{pmatrix}
	 1 &-1 &1 &\vline& -2 \\ 
	 1 &2 &-4 &\vline& 10 \\ 
	 3 &-1 &1 &\vline& 3 \\ 
 \end{pmatrix}
 \sim
 \begin{pmatrix}
	 1 &-1 &1 &\vline& -2 \\ 
	 0 &3 &-5 &\vline& 12 \\ 
	 0 &2 &-2 &\vline& 9 \\ 
 \end{pmatrix}
 \sim
 \begin{pmatrix}
	 1 &-1 &1 &\vline& -2 \\ 
	 0 &2 &-2 &\vline& 9 \\ 
	 0 &3 &-5 &\vline& 12 \\ 
 \end{pmatrix}
 \sim
 \begin{pmatrix}
	 1 &-1 &1 &\vline& -2 \\ 
	 0 &2 &-2 &\vline& 9 \\ 
	 0 &0 &-4 &\vline& -3 \\ 
 \end{pmatrix}
 \sim
 \begin{pmatrix}
	 1 &-1 &1 &\vline& -2 \\ 
	 0 &2 &-2 &\vline& 9 \\ 
	 0 &0 &4 &\vline& 3 \\ 
 \end{pmatrix}
 \sim
 \begin{pmatrix}
	 2 &0 &0 &\vline& 5 \\ 
	 0 &2 &-2 &\vline& 9 \\ 
	 0 &0 &4 &\vline& 3 \\ 
 \end{pmatrix}
 \sim
 \begin{pmatrix}
	 2 &0 &0 &\vline& 5 \\ 
	 0 &4 &0 &\vline& 21 \\ 
	 0 &0 &4 &\vline& 3 \\ 
 \end{pmatrix}
  $ \\
   $ 
    k =\frac{5}{2}  \imp P = \[\f 72;\f{11}2;\f 52\] \imp \pri{PQ} = \zs{\[\f72+t;\f{11}2+t;\f 52+4t\]|t\in\R} $


	\Pr Cvičení 1:\\
	Jsou dány mymoněžky $p(A,\ve u),q(B,\ve v)$, a vektor $\ve w$:\\
	$A[10;-7;0]; B[-3;5;0]; \ve u (5;4;1), \ve v (2;1;1), \ve w (8;7;1)$\\
	Najděte příčku $p,q$, která je rovnoběžná s $\ve w$.

	$k\ve u-l\ve v + x \ve w\ve {AB}= (-13;12;0)$.
	 $ \bar{A} = \begin{pmatrix}
		 5 &-2 &8 &\vline& -13 \\ 
		 4 &-1 &7 &\vline& 12 \\ 
		 1 &-1 &1 &\vline& 0 \\ 
	 \end{pmatrix}
	 \sim
	 \begin{pmatrix}
		 1 &-1 &1 &\vline& 0 \\ 
		 4 &-1 &7 &\vline& 12 \\ 
		 5 &-2 &8 &\vline& -13 \\ 
	 \end{pmatrix}
	 \sim
	 \begin{pmatrix}
		 1 &-1 &1 &\vline& 0 \\ 
		 0 &3 &3 &\vline& 12 \\ 
		 0 &3 &3 &\vline& -13 \\ 
	 \end{pmatrix}
	 \sim
	 \begin{pmatrix}
		 1 &-1 &1 &\vline& 0 \\ 
		 0 &1 &1 &\vline& 4 \\ 
		 0 &3 &3 &\vline& -13 \\ 
	 \end{pmatrix}
	 \sim
	 \begin{pmatrix}
		 1 &-1 &1 &\vline& 0 \\ 
		 0 &1 &1 &\vline& 4 \\ 
		 0 &0 &0 &\vline& -25 \\ 
	 \end{pmatrix}
	 \sim
	 \begin{pmatrix}
		 1 &-1 &1 &\vline& 0 \\ 
		 0 &1 &1 &\vline& 4 \\ 
		 0 &0 &0 &\vline& 1 \\ 
	 \end{pmatrix}
	  $ 
	  Soustava nemá řešení $\imp$ hledaná příčka neexistuje.

    \subsection{Nalezení příčky $r$ mimoběžek $p, q$, která prochází bodem $M$}
\Poz Dáno: $p(A,\ve u);q(B,\ve v),$ bod $M$.

Rozbor:
\begin{enumerate}[1)]
	\item $M \in p\cap q \imp$ nekonečně mnoho řešení
	\item $M \not\in p\cap q \imp$ 
		\begin{enumerate}
			\item jedna přímka je rovnoběžná s rovinou, která je
 dána druhou přímkou a bodem $M$ $\imp$ 0 řešení.
			\item ani jedna přímka není rovnoběžná s rovinou, která je
 dána druhou přímkou a bodem $M$ $\imp$ 1 řešení.
		\end{enumerate}
\end{enumerate}
\pdf[0.2]{7-b.pdf}
$P = A + k \ve u$\\
$Q = B + l \ve v$\\
$\ve{MP} = x \* \ve{MQ}$\\
$A + k \ve u = x (\ve{MB} + l \ve v)$\\
$-k\ve u + m \ve v + x \ve{MB} =\ve {MA}$, kde $m = x\* l$

$\imp$ 3 rovnice o 3 neznámých $k,m,x$.
\Pr Jsou dány mimoběžky $p(A,u), q(B, v)$, bod $M$. Nalezněte příčku mimoběžek $p, q$, procházející bodem $M$.\\
$A[1;5;2];B[0;-1;1];M[0;1;-5];\ve u (1;2;1),\ve v(3;1;0)$\\

$\ve{MA} = (1; 4;7)$\\
$\ve{MB} = (0;-2;6)$

 $  \begin{pmatrix}
	 -1 &3 &0 &\vline& 1 \\ 
	 -2 &1 &-2 &\vline& 4 \\ 
	 -1 &0 &6 &\vline& 7 \\ 
 \end{pmatrix}
 \sim
 \begin{pmatrix}
	 1 &-3 &0 &\vline& -1 \\ 
	 2 &-1 &2 &\vline& -4 \\ 
	 1 &0 &-6 &\vline& -7 \\ 
 \end{pmatrix}
 \sim
 \begin{pmatrix}
	 1 &-3 &0 &\vline& -1 \\ 
	 1 &0 &-6 &\vline& -7 \\ 
	 2 &-1 &2 &\vline& -4 \\ 
 \end{pmatrix}
 \sim
 \begin{pmatrix}
	 1 &-3 &0 &\vline& -1 \\ 
	 0 &3 &-6 &\vline& -6 \\ 
	 0 &5 &2 &\vline& -2 \\ 
 \end{pmatrix}
 \sim
 \begin{pmatrix}
	 1 &-3 &0 &\vline& -1 \\ 
	 0 &1 &-2 &\vline& -2 \\ 
	 0 &5 &2 &\vline& -2 \\ 
 \end{pmatrix}
 \sim
 \begin{pmatrix}
	 1 &-3 &0 &\vline& -1 \\ 
	 0 &1 &-2 &\vline& -2 \\ 
	 0 &0 &12 &\vline& 8 \\ 
 \end{pmatrix}
 \sim
 \begin{pmatrix}
	 1 &-3 &0 &\vline& -1 \\ 
	 0 &1 &-2 &\vline& -2 \\ 
	 0 &0 &3 &\vline& 2 \\ 
 \end{pmatrix}
 \sim
 \begin{pmatrix}
	 1 &0 &-6 &\vline& -7 \\ 
	 0 &1 &-2 &\vline& -2 \\ 
	 0 &0 &3 &\vline& 2 \\ 
 \end{pmatrix}
 \sim
 \begin{pmatrix}
	 1 &0 &0 &\vline& -3 \\ 
	 0 &3 &0 &\vline& -2 \\ 
	 0 &0 &3 &\vline& 2 \\ 
 \end{pmatrix}
  $ 
   $$ 
    \fce{P} = \zs{\zh{-3; - \frac{2}{3}; \frac{2}{3}}} 
     $$ 

  $$ 
  k = -3 ; m = -\f 23 ; x = \f 2 3
    $$ 
    $ P = [1;5;2] - 3(1;2;1) = [-2;-1;-1]$\\
    $ Q = [0;-1;1] - 1(3;1;0) = [-3;-2;1]$\\
    $ \pri{PQ} = \{[t;1+t;-5-2t] | t\in \R \}$.


\Pr Jsou dány mimoběžky $p(A,u), q(B, v)$, bod $M$. Nalezněte příčku mimoběžek $p, q$, procházející bodem $M$.\\
$A[3;1;2];B[-2;1;0];M[-\f 12;\f 52;0];\ve u (1;2;-1),\ve v(3;-1;1)$\\

$\ve{MA} = (\f72,-\f32;2)$\\
$\ve{MB} = (-\f32;-\f32;0)$

$P = A + k \ve u$\\
$Q = B + l \ve v$\\
$\ve{MP} = x \* \ve{MQ}$\\
$A + k \ve u = x (\ve{MB} + l \ve v)$\\
$-k\ve u + m \ve v + x \ve{MB} = \ve{MA}$, kde $m = x\* l$

$ \bar{A} = \begin{pmatrix}
	-2 &6 &-3 &\vline& 7 \\ 
	-4 &-2 &-3 &\vline& -3 \\ 
	-1 &1 &0 &\vline& 2 \\ 
\end{pmatrix}
\sim
\begin{pmatrix}
	2 &-6 &3 &\vline& -7 \\ 
	4 &2 &3 &\vline& 3 \\ 
	1 &-1 &0 &\vline& -2 \\ 
\end{pmatrix}
\sim
\begin{pmatrix}
	1 &-1 &0 &\vline& -2 \\ 
	2 &-6 &3 &\vline& -7 \\ 
	4 &2 &3 &\vline& 3 \\ 
\end{pmatrix}
\sim
\begin{pmatrix}
	1 &-1 &0 &\vline& -2 \\ 
	0 &-4 &3 &\vline& -3 \\ 
	0 &6 &3 &\vline& 11 \\ 
\end{pmatrix}
\sim
\begin{pmatrix}
	1 &-1 &0 &\vline& -2 \\ 
	0 &4 &-3 &\vline& 3 \\ 
	0 &6 &3 &\vline& 11 \\ 
\end{pmatrix}
\sim
\begin{pmatrix}
	1 &-1 &0 &\vline& -2 \\ 
	0 &4 &-3 &\vline& 3 \\ 
	0 &0 &15 &\vline& 13 \\ 
\end{pmatrix}
\sim
\begin{pmatrix}
	4 &0 &-3 &\vline& -5 \\ 
	0 &4 &-3 &\vline& 3 \\ 
	0 &0 &15 &\vline& 13 \\ 
\end{pmatrix}
\sim
\begin{pmatrix}
	20 &0 &0 &\vline& -12 \\ 
	0 &20 &0 &\vline& 28 \\ 
	0 &0 &15 &\vline& 13 \\ 
\end{pmatrix}
\sim
\begin{pmatrix}
	5 &0 &0 &\vline& -3 \\ 
	0 &5 &0 &\vline& 7 \\ 
	0 &0 &15 &\vline& 13 \\ 
\end{pmatrix}
 $ 
  $$ 
   k = - \frac{3}{5};m= \frac{7}{5};x= \frac{13}{15}
    $$ 

$P = [3-\f35;1-\f65;2+\f35]
 = [\f{12}5;-\f15;\f{13}5]$\\
 $Q = [-2+\f{21}5;1-\f75;0+\f75]
  =[\f{11}5;-\f25;\f75]$

  $\ve{QP} = [\f 15; \f 15;\f65] $
  $$ \pri{PQ} = \zs{\[[\f{12}5+k;-\f15+k;\f{13}5+6k\]|k\in\R}$$

  \subsection{ Nalezení osy o mimoběžek $p, q$}
  \Def Nechť $p, q$ jsou mimoběžné přímky. Pak příčka mimoběžek $o$, která je kolmá
k přímkám $p$ i $q$, se nazývá osa mimoběžek $p,q$.

\Pr nalezněte osu mimoběžek $p,q$.\\
$p = \zs{[8+t;5+2t;8-t]|t\in\R}$
$q = \zs{[-4-7r;3+2r;4+3r],r\in\R}$

Hledáme osu $o(P,\ve w);\ve w = (w_1,w_2,w_3)$, máme dáno:\\
$A[8;5;8], \ve u (1;2;-1)$\\
$B[-4;3;4], \ve u (-7;2;3)$

$p\perp o: \ve u \* \ve w = w_1+2w_2-w_3 = 0 \imp w_2=\f {w_1} 2$\\
$q\perp o: \ve v \* \ve w = -7w_1+2w_2+3w_3 = 0 \imp w_3=2 {w_1}$\\

$w_1$ -- libovolný (jedná se jen o násobek) $\imp \ve w =(2;1;4)$.

tento vektor lze také zistat jako vektorový součin $\ve u \times \ve v$.
Nyní hledáme příčku $p,q$ rovnoběžnou s $\ve w$.

$\left.\begin{array}{l} P = A + k \* \ve u\\ Q = B + l\* \ve v\end{array}\right\} \imp \ve{PQ} = Q - P = B + l \ve{v} - A - k\ve u$\\
	$\ve{PQ} = x \ve w $\\
	$\imp B + l \* \ve v - A - k \ve u = x \ve w$\\
	\underline{$\ve {AB} = x \ve w + k \ve u - l \ve v$}

	$k\ve u-l\ve v + x \ve w=\ve {AB}= (-12;-2;-4)$.

	 $ \bar{A} = \begin{pmatrix}
		 1 &7 &2 &\vline& -12 \\ 
		 2 &-2 &1 &\vline& -2 \\ 
		 -1 &-3 &4 &\vline& -4 \\ 
	 \end{pmatrix}
	 \sim
	 \begin{pmatrix}
		 1 &7 &2 &\vline& -12 \\ 
		 2 &-2 &1 &\vline& -2 \\ 
		 1 &3 &-4 &\vline& 4 \\ 
	 \end{pmatrix}
	 \sim
	 \begin{pmatrix}
		 1 &7 &2 &\vline& -12 \\ 
		 1 &3 &-4 &\vline& 4 \\ 
		 2 &-2 &1 &\vline& -2 \\ 
	 \end{pmatrix}
	 \sim
	 \begin{pmatrix}
		 1 &7 &2 &\vline& -12 \\ 
		 0 &-4 &-6 &\vline& 16 \\ 
		 0 &-16 &-3 &\vline& 22 \\ 
	 \end{pmatrix}
	 \sim
	 \begin{pmatrix}
		 1 &7 &2 &\vline& -12 \\ 
		 0 &2 &3 &\vline& -8 \\ 
		 0 &16 &3 &\vline& -22 \\ 
	 \end{pmatrix}
	 \sim
	 \begin{pmatrix}
		 1 &7 &2 &\vline& -12 \\ 
		 0 &2 &3 &\vline& -8 \\ 
		 0 &0 &-21 &\vline& 42 \\ 
	 \end{pmatrix}
	 \sim
	 \begin{pmatrix}
		 1 &7 &2 &\vline& -12 \\ 
		 0 &2 &3 &\vline& -8 \\ 
		 0 &0 &1 &\vline& -2 \\ 
	 \end{pmatrix}
	 \sim
	 \begin{pmatrix}
		 2 &0 &-17 &\vline& 32 \\ 
		 0 &2 &3 &\vline& -8 \\ 
		 0 &0 &1 &\vline& -2 \\ 
	 \end{pmatrix}
	 \sim
	 \begin{pmatrix}
		 2 &0 &0 &\vline& -2 \\ 
		 0 &2 &0 &\vline& -2 \\ 
		 0 &0 &1 &\vline& -2 \\ 
	 \end{pmatrix}
	 \sim
	 \begin{pmatrix}
		 1 &0 &0 &\vline& -1 \\ 
		 0 &1 &0 &\vline& -1 \\ 
		 0 &0 &1 &\vline& -2 \\ 
	 \end{pmatrix}
	  $ 

	  $k=1 \imp P = [7;3;9]$
	  $$ o = \pri{PQ} = \zs{[7+2t;3+t;9+4t]|t\in\R} $$

	  \Pr Úkol 3\\
	  Nalezněte příčku mimoběžek, která je
rovnoběžná s rovinami $\rho, \sigma$.

$p:A[-5;2;2];\ve u = (2;0;1)$\\
$q:z-2=0\land 5x-8y+9z+100=0 \imp B[\f{-118}5;0;2]; \ve v(8;5;0)$

$\rho \zs{[3+3r+s;2r;2s];r,s\in\R} \imp \rho: 2x-3y-3z = 6$\\
$\sigma: x-4u-3z+12=0$

 $ \bar{A} = \begin{pmatrix}
	 2 &-3 &-3 &\vline& 6 \\ 
	 1 &-4 &-3 &\vline& -12 \\ 
 \end{pmatrix}
 \sim
 \begin{pmatrix}
	 1 &-4 &-3 &\vline& -12 \\ 
	 2 &-3 &-3 &\vline& 6 \\ 
 \end{pmatrix}
 \sim
 \begin{pmatrix}
	 1 &-4 &-3 &\vline& -12 \\ 
	 0 &5 &3 &\vline& 30 \\ 
 \end{pmatrix}
 \sim
 \begin{pmatrix}
	 5 &0 &-3 &\vline& 60 \\ 
	 0 &5 &3 &\vline& 30 \\ 
 \end{pmatrix}
  $ 

$\imp \ve w(3;-3;5)$

$\left.\begin{array}{l} P = A + k \* \ve u\\ Q = B + l\* \ve v\end{array}\right\} \imp \ve{PQ} = Q - P = B + l \ve{v} - A - k\ve u$\\
	$\ve{PQ} = x \ve w $\\
	$\imp B + l \* \ve v - A - k \ve u = x \ve w$\\
	$\ve {AB} = x \ve w + k \ve u - l \ve v$

	$k\ve u-l\ve v + x \ve w=\ve {AB}= (\f{-93}5;-2;0)$.
 $ \bar{A} = \begin{pmatrix}
	 10 &-40 &15 &\vline& -93 \\ 
	 0 &-5 &-10 &\vline& -40 \\ 
	 15 &-93 &0 &\vline& -5 \\ 
 \end{pmatrix}
 \sim
 \begin{pmatrix}
	 10 &-40 &15 &\vline& -93 \\ 
	 0 &1 &2 &\vline& 8 \\ 
	 15 &-93 &0 &\vline& -5 \\ 
 \end{pmatrix}
 \sim
 \begin{pmatrix}
	 10 &-40 &15 &\vline& -93 \\ 
	 15 &-93 &0 &\vline& -5 \\ 
	 0 &1 &2 &\vline& 8 \\ 
 \end{pmatrix}
 \sim
 \begin{pmatrix}
	 10 &-40 &15 &\vline& -93 \\ 
	 0 &-66 &-45 &\vline& 269 \\ 
	 0 &1 &2 &\vline& 8 \\ 
 \end{pmatrix}
 \sim
 \begin{pmatrix}
	 10 &-40 &15 &\vline& -93 \\ 
	 0 &66 &45 &\vline& -269 \\ 
	 0 &1 &2 &\vline& 8 \\ 
 \end{pmatrix}
 \sim
 \begin{pmatrix}
	 10 &-40 &15 &\vline& -93 \\ 
	 0 &1 &2 &\vline& 8 \\ 
	 0 &66 &45 &\vline& -269 \\ 
 \end{pmatrix}
 \sim
 \begin{pmatrix}
	 10 &-40 &15 &\vline& -93 \\ 
	 0 &1 &2 &\vline& 8 \\ 
	 0 &0 &-87 &\vline& -797 \\ 
 \end{pmatrix}
 \sim
 \begin{pmatrix}
	 10 &-40 &15 &\vline& -93 \\ 
	 0 &1 &2 &\vline& 8 \\ 
	 0 &0 &87 &\vline& 797 \\ 
 \end{pmatrix}
 \sim
 \begin{pmatrix}
	 10 &0 &95 &\vline& 227 \\ 
	 0 &1 &2 &\vline& 8 \\ 
	 0 &0 &87 &\vline& 797 \\ 
 \end{pmatrix}
 \sim
 \begin{pmatrix}
	 870 &0 &0 &\vline& -55966 \\ 
	 0 &87 &0 &\vline& -898 \\ 
	 0 &0 &87 &\vline& 797 \\ 
 \end{pmatrix}
 \sim
 \begin{pmatrix}
	 435 &0 &0 &\vline& -27983 \\ 
	 0 &87 &0 &\vline& -898 \\ 
	 0 &0 &87 &\vline& 797 \\ 
 \end{pmatrix}
  $ 
  
  $k = \f{27983}{435} \imp P = [-5+2\f{27983}{435};2;2+\f{27983}{435}]$
  $$o = \pri{PQ} = \zs{\[-5+2\f{27983}{435}+3k;2-3k;2+\f{27983}{435}+5k\]|k\in\R}$$

  \Pr úkol 4:\\
  Nalezněte příčku mimoběžek, která leží v rovině $\rho$:

  $p: x+y = 2\land 2x + z = 5$.\\
  $q: x+2y = 1 \land -3y+z = 2$

  $\rho: x + 2y - x = -2 $

  Určím $P = p\cap\rho$:
   $ \bar{A} = \begin{pmatrix}
	   1 &2 &-1 &\vline& -2 \\ 
	   1 &1 &0 &\vline& 2 \\ 
	   2 &0 &1 &\vline& 5 \\ 
   \end{pmatrix}
   \sim
   \begin{pmatrix}
	   1 &2 &-1 &\vline& -2 \\ 
	   0 &-1 &1 &\vline& 4 \\ 
	   0 &-4 &3 &\vline& 9 \\ 
   \end{pmatrix}
   \sim
   \begin{pmatrix}
	   1 &2 &-1 &\vline& -2 \\ 
	   0 &1 &-1 &\vline& -4 \\ 
	   0 &4 &-3 &\vline& -9 \\ 
   \end{pmatrix}
   \sim
   \begin{pmatrix}
	   1 &2 &-1 &\vline& -2 \\ 
	   0 &1 &-1 &\vline& -4 \\ 
	   0 &0 &1 &\vline& 7 \\ 
   \end{pmatrix}
   \sim
   \begin{pmatrix}
	   1 &0 &1 &\vline& 6 \\ 
	   0 &1 &-1 &\vline& -4 \\ 
	   0 &0 &1 &\vline& 7 \\ 
   \end{pmatrix}
   \sim
   \begin{pmatrix}
	   1 &0 &0 &\vline& -1 \\ 
	   0 &1 &0 &\vline& 3 \\ 
	   0 &0 &1 &\vline& 7 \\ 
   \end{pmatrix}
    $ 

     $$ 
      P = {\zh{-1; 3; 7}} 
       $$ 
  Určím $Q = q\cap\rho$:
   $ \bar{A} = \begin{pmatrix}
	   1 &2 &-1 &\vline& -2 \\ 
	   1 &2 &0 &\vline& 1 \\ 
	   0 &-3 &1 &\vline& 2 \\ 
   \end{pmatrix}
   \sim
   \begin{pmatrix}
	   1 &2 &-1 &\vline& -2 \\ 
	   1 &2 &0 &\vline& 1 \\ 
	   0 &3 &-1 &\vline& -2 \\ 
   \end{pmatrix}
   \sim
   \begin{pmatrix}
	   1 &2 &-1 &\vline& -2 \\ 
	   0 &0 &1 &\vline& 3 \\ 
	   0 &3 &-1 &\vline& -2 \\ 
   \end{pmatrix}
   \sim
   \begin{pmatrix}
	   1 &2 &-1 &\vline& -2 \\ 
	   0 &3 &-1 &\vline& -2 \\ 
	   0 &0 &1 &\vline& 3 \\ 
   \end{pmatrix}
   \sim
   \begin{pmatrix}
	   3 &0 &-1 &\vline& -2 \\ 
	   0 &3 &-1 &\vline& -2 \\ 
	   0 &0 &1 &\vline& 3 \\ 
   \end{pmatrix}
   \sim
   \begin{pmatrix}
	   3 &0 &0 &\vline& 1 \\ 
	   0 &3 &0 &\vline& 1 \\ 
	   0 &0 &1 &\vline& 3 \\ 
   \end{pmatrix}
    $ 
     $$ 
      Q = {\zh{\frac{1}{3}; \frac{1}{3}; 3}} 
       $$ 
       $\ve{PQ} = (\f 43;-\f83;-4)$
       $$
       \pri{PQ} = \zs{\[-1+4t; 3-8t; 7-12t\]|t\in\R}
       $$




\EndDoc
