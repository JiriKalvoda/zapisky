\providecommand{\HINCLUDE}{NE}
\if ^\HINCLUDE^
\else
\def\HINCLUDE{}
\global\newdimen\Okraje
\global\Okraje =4cm
\input{$HOME/souteze/_hlavicka/h-.tex}

%\definecolor{colorV}{RGB}{255,127,0}
%\definecolor{colorPoz}{RGB}{153,51,0}
%\definecolor{orangeV}{RGB}{255,127,0}
%\definecolor{colorPr}{RGB}{0,5,255}
%\definecolor{colorDef}{RGB}{0.255,0}

\usepackage[shortlabels]{enumitem}
\setlength{\marginparsep}{2pt}
\setlength{\marginparwidth}{35pt}

\def\st{{\rm st}}
\def\P{{\rm P}}

\def\ISENUM{}
\def\inMargin#1{\End
		
		\hskip0pt \marginpar{{{#1}}}}
\newcounter{V}[section] 
\newcommand{\V}[1][]{\stepcounter{V}\inMargin{\textcolor{green}{V.\arabic{section}.\theV.:}}\ifx^#1^\else\textcolor{green}{\underline{{#1}:}}\addcontentsline{toc}{subsubsection}{V.\arabic{section}.\theV.:$\quad$ {#1}}\\\fi}
\def\Def{\inMargin{\textcolor{red}{Def:}}}
\def\Poz{{\inMargin{\textcolor{brown}{Pozn:}}}}
\def\Pr{{\inMargin{\textcolor{blue}{Př:}}}}
\def\Pozenum
{
	\begin{enumerate}[1)]%, left = 0pt ]
		\item\inMargin{\textcolor{brown}{Pozn:}}\def\ISENUM{a}}
\def\End
{
	\if	^\ISENUM^
	\else \end{enumerate}
	\fi
	\def\ISENUM{}
}
\reversemarginpar

\makeatletter
\renewcommand\thesection{§\arabic{section}.}
\renewcommand\thesubsection{\Alph{subsection})}
\renewcommand\thesubsubsection{\alph{subsubsection})}
\newcounter{chapter}
\setcounter{chapter}{0}
\renewcommand\thechapter{\Alph{chapter})}
\newcounter{roman}
\setcounter{roman}{0}
\renewcommand\theroman{\Roman{roman}.}
\makeatother
\def\sectionnum#1{\setcounter{section}{#1}\addtocounter{section}{-1}}
\def\subsectionnum#1{\setcounter{subsection}{#1}\addtocounter{subsection}{-1}}
\def\subsubsectionnum#1{\setcounter{subsubsection}{#1}\addtocounter{subsubsection}{-1}}
\def\chapternum#1{\setcounter{chapter}{#1}\addtocounter{chapter}{-1}}
\def\chapter#1{

	\addtocounter{chapter}{1}\sectionnum{1}
	\addcontentsline{toc}{section}{\large{\thechapter$\quad${#1}}}
	
	{\LARGE  \textbf{\begin{minipage}[t]{0.1\textwidth}\thechapter\end{minipage}\begin{minipage}[t]{0.95\textwidth}#1\end{minipage}}}

}
\def\ROM{}
\def\Rom#1#2{\setcounter{roman}{#1}\renewcommand\ROM{#2}}

\Rom{6}{Funkce}
\title{\Huge\textbf{\theroman\quad \ROM}}
\author{Jiří Kalvoda}

\newcounter{countOfBegin}
\setcounter{countOfBegin}{0}
\newcommand{\BeginDoc}[1][]
{
	\ifnum\value{countOfBegin}=0
	\begin{document}
		#1
		\fi
	\addtocounter{countOfBegin}{1}
		
}
\def\EndDoc
{
	\addtocounter{countOfBegin}{-1}
	\ifnum\value{countOfBegin}=0
	\end{document}
	\fi
}

\fi
\BeginDoc{\sectionnum{5}ccc}
\section{Vzájemná poloha dvou rovin}
\V Věta o vzájemné poloze dvou rovin daných obecnými rovnicemi
Nechť $\rho : ax + by + cz + d = 0, \sigma : ex + fy + gz + h = 0$ jsou roviny.
Pak platí:
\begin{itemize}
	\item $\rho = \sigma \ekv \exists k \in \R:( a, d,ci,d )=k \*(e,f,g,d)$
	\item $\rho \parallel \sigma \land \rho \neq \sigma \ekv \exists k \in \R: (a,b,c)=k\*(e,f,g) \land d\neq k\* h$
	\item $\rho \nparallel \sigma \ekv \forall k \in \R : (a,b,c) \neq k\*(e,f,g)$.
\end{itemize}
\Pr Určete vzájemnou polohu dvou rovin:\\
$\rho : 2x + 3y + 4z + 5 =0$\\
$\sigma :  x-y-z+1 = 0$ \\
$\ppri{n_\rho} = (2;3;4)$\\
$\ppri{n_\sigma} = (1;-1;-1)$\\
vektory jsou lin. nezávislé: $\rho \nparallel\sigma$

\underline{Určení průsečnice rovin} $\rho, \sigma$ (hledáme parametrickou rovnici přímky v $E_3$): \\
volíme $z=t;t\in\R$:\\
$2x+3y+4t+5=0$\\
$x-y - t + 1 =0$\\
$\imp 5x + t + 8 =0 \imp x = -\f85-\f t5 $\\
$y=x-t+1 \imp y = -\f 35 - \f 6 5 t$.\\
Průsečnice: $\left\{\[-\f 8 5 - \f 1 5 t ;-\f 35 - \f 65 t;t\]|t\in\R\right\}$

\V[Věta o vzájemné poloze dvou rovin daných parametrickými rovnicemi]
Nechť $\rho (A,\ve u, \ve v),\sigma(B,\ve k,\ve l)$ jsou roviny. Pak platí:
\begin{itemize}
	\item $\rho = \sigma \ekv \dim\<\ve u,\ve v, \ve k, \ve l\> = 2 \land \dim \<\ve u , \ve v, \ve k ,\ve l, \ve {AB}\> =2$/
	\item $\rho \parallel \sigma \land \rho \neq \sigma \ekv\dim\<\ve u,\ve v, \ve k, \ve l\>= 2 \land \dim \<\ve u , \ve v, \ve k ,\ve \, \ve {AB}\> =3$ 
	\item $\rho \nparallel \sigma \ekv\dim \<\ve u , \ve v, \ve k ,\ve l, \ve {AB}\> =3$.
\end{itemize}
\Pr Určete vzájemnou polohu rovin $\rho$ a $\sigma$ : 
$\rho = \{ [1+t_1 + 2t_2; 2t_1 + 3 t_2 ; -2 -2t_1 + t_2];t_1,t_2\in\R\}$\\
$\imp \rho(A=[1;0;-2];\ve u = (1;2;-2);\ve v = (2;3;1))$ \\
$\sigma = \{ [r_1;-3+r_2;1+4r_1-r_2];r_1,r_2\in\R\}$\\
$\imp \sigma(B=[0;-3;1];\ve k = (1;0;4);\ve l = (0;1;-1))$ \\
$\imp \ve {AB} = (-1;-3;3)$
$$
\begin{pmatrix}
	1&2&-2\\
	2&3&1\\
	1&0&4\\
	0&1&-1\\
	-1&-3&3
\end{pmatrix}
\sim
\begin{pmatrix}
	1&2&-2\\
	0&-1&5\\
	0&-2&6\\
	0&1&-1\\
	0&-1&1
\end{pmatrix}
\sim
\begin{pmatrix}
	1&2&-2\\
	0&-1&5\\
	0&0&1\\
\end{pmatrix}
$$
$\dim<\ve u ,\ve v,\ve k,\ve l> = 3 \imp $ roviny jsou různaběžné.

\underline{Rovnice průsečnice $\rho, \sigma$:}
-- porovnání souřadnic $\rho$ a $\sigma$:\\
$1+t_1 + 2t_2 = r_1$\\
$2t_1+3t_0 = +3 + r_2$\\
$-2 - 2t_1 + t_1 = 1 + 4r_1 - r_2$\\
soustava 3 rovnic o 4 neznámých, po vyjádření z 2. a 3.rovnice $t=r_1 = t$, odsud a z 1. rovnice $t_1 = -1-t,r_2=t+1$.\\
Dosazením do rovnice roviny $\rho$ :\\
$$p =\{[t; - 2 + t; 3t]|t \in \R\}$$

\Pr 182/19,20\\

Rozhodněte, jakou maji roviny vzájemnou polohu a určete průsečnice: 
\begin{eqnarray*}
	\rho&:& 2x - 3y + z - 4 = 0\\
	\sigma&:& 4x + y -5z +3 = 0\\
	\tau &:& x + 2y - z + 1 =0\\
	\varphi &:& -4x + 6y - 2z +  5 = 0 \\
	\alpha &:& 3x - y - x + 5 =0\\
	\beta &:& x + y + z -7 = 0
\end{eqnarray*}
\begin{itemize}
	\item \underline{$\rho$ a $\sigma$}: \\
		Nerovnoběžné:\\
		$ \bar{A} = \begin{pmatrix}
			2 &-3 &1 &\vline& 4 \\ 
			4 &1 &-5 &\vline& -3 \\ 
		\end{pmatrix}
		\sim
		\begin{pmatrix}
			2 &-3 &1 &\vline& 4 \\ 
			0 &7 &-7 &\vline& -11 \\ 
		\end{pmatrix}
		\sim
		\begin{pmatrix}
			14 &0 &-14 &\vline& -5 \\ 
			0 &7 &-7 &\vline& -11 \\ 
		\end{pmatrix}
		$

		Průsečnice:
		 $$ 
		   \zs{\zh{\frac{-5+14 a}{14}; \frac{-11+7 a}{7}; a}:a \in \mathbb{R}} 
		   $$ 
	\item \underline{$\rho$ a $\tau$}: \\
		Nerovnoběžné:\\
		 $  \begin{pmatrix}
			 2 &-3 &1 &\vline& 4 \\ 
			 1 &2 &-1 &\vline& -1 \\ 
		 \end{pmatrix}
		 \sim
		 \begin{pmatrix}
			 1 &2 &-1 &\vline& -1 \\ 
			 2 &-3 &1 &\vline& 4 \\ 
		 \end{pmatrix}
		 \sim
		 \begin{pmatrix}
			 1 &2 &-1 &\vline& -1 \\ 
			 0 &-7 &3 &\vline& 6 \\ 
		 \end{pmatrix}
		 \sim
		 \begin{pmatrix}
			 1 &2 &-1 &\vline& -1 \\ 
			 0 &7 &-3 &\vline& -6 \\ 
		 \end{pmatrix}
		 \sim
		 \begin{pmatrix}
			 7 &0 &-1 &\vline& 5 \\ 
			 0 &7 &-3 &\vline& -6 \\ 
		 \end{pmatrix}
		 $

		Průsečnice:
		  $$
			 \zs{\zh{\frac{5+1 a}{7}; \frac{-6+3 a}{7}; a}:a \in \mathbb{R}}
		 $$
	\item \underline{$\sigma$ a $\tau$}: \\
		Nerovnoběžné:\\
		 $ \begin{pmatrix}
			 4 &1 &-5 &\vline& -3 \\ 
			 1 &2 &-1 &\vline& -1 \\ 
		 \end{pmatrix}
		 \sim
		 \begin{pmatrix}
			 1 &2 &-1 &\vline& -1 \\ 
			 4 &1 &-5 &\vline& -3 \\ 
		 \end{pmatrix}
		 \sim
		 \begin{pmatrix}
			 1 &2 &-1 &\vline& -1 \\ 
			 0 &-7 &-1 &\vline& 1 \\ 
		 \end{pmatrix}
		 \sim
		 \begin{pmatrix}
			 1 &2 &-1 &\vline& -1 \\ 
			 0 &7 &1 &\vline& -1 \\ 
		 \end{pmatrix}
		 \sim
		 \begin{pmatrix}
			 7 &0 &-9 &\vline& -5 \\ 
			 0 &7 &1 &\vline& -1 \\ 
		 \end{pmatrix}
		 $


		Průsečnice:
		 $$ 
		  \zs{\zh{\frac{-5+9 a}{7}; \frac{-1-1 a}{7}; a}:a \in \mathbb{R}} 
		   $$ 
	\item \underline{$\rho$ a $\varphi$}: \\
		$(2 , -3,1) = -\f 12 (-4,3,-2) \land -4 \* \f {-1}2 = -2 \neq 5$\\
		Rovnoběžné.


	\item \underline{$\sigma$ a $\varphi$}: \\
		Nerovnoběžné:\\
		 $  \begin{pmatrix}
			 4 &1 &-5 &\vline& -3 \\ 
			 -4 &6 &-2 &\vline& -5 \\ 
		 \end{pmatrix}
		 \sim
		 \begin{pmatrix}
			 4 &1 &-5 &\vline& -3 \\ 
			 4 &-6 &2 &\vline& 5 \\ 
		 \end{pmatrix}
		 \sim
		 \begin{pmatrix}
			 4 &1 &-5 &\vline& -3 \\ 
			 0 &-7 &7 &\vline& 8 \\ 
		 \end{pmatrix}
		 \sim
		 \begin{pmatrix}
			 4 &1 &-5 &\vline& -3 \\ 
			 0 &7 &-7 &\vline& -8 \\ 
		 \end{pmatrix}
		 \sim
		 \begin{pmatrix}
			 28 &0 &-28 &\vline& -13 \\ 
			 0 &7 &-7 &\vline& -8 \\ 
		 \end{pmatrix}
		 $

		Průsečnice:
		 $$ 
		  \zs{\zh{\frac{-13+28 a}{28}; \frac{-8+7 a}{7}; a}:a \in \mathbb{R}} 
		   $$ 
	\item \underline{$\tau$ a $\varphi$}: \\
		Nerovnoběžné:\\
		 $ \begin{pmatrix}
			 1 &2 &-1 &\vline& -1 \\ 
			 -4 &6 &-2 &\vline& -5 \\ 
		 \end{pmatrix}
		 \sim
		 \begin{pmatrix}
			 1 &2 &-1 &\vline& -1 \\ 
			 4 &-6 &2 &\vline& 5 \\ 
		 \end{pmatrix}
		 \sim
		 \begin{pmatrix}
			 1 &2 &-1 &\vline& -1 \\ 
			 0 &-14 &6 &\vline& 9 \\ 
		 \end{pmatrix}
		 \sim
		 \begin{pmatrix}
			 1 &2 &-1 &\vline& -1 \\ 
			 0 &14 &-6 &\vline& -9 \\ 
		 \end{pmatrix}
		 \sim
		 \begin{pmatrix}
			 7 &0 &-1 &\vline& 2 \\ 
			 0 &14 &-6 &\vline& -9 \\ 
		 \end{pmatrix}
		 $


		Průsečnice:
		 $$ 
		 \zs{\zh{\frac{2+1 a}{7}; \frac{-9+6 a}{14}; a}:a \in \mathbb{R}} 
		   $$ 
	\item \underline{$\rho$ a $\alpha$}: \\
		Nerovnoběžné:\\
		 $ \begin{pmatrix}
			 2 &-3 &1 &\vline& 4 \\ 
			 3 &-1 &-1 &\vline& -5 \\ 
		 \end{pmatrix}
		 \sim
		 \begin{pmatrix}
			 2 &-3 &1 &\vline& 4 \\ 
			 0 &7 &-5 &\vline& -22 \\ 
		 \end{pmatrix}
		 \sim
		 \begin{pmatrix}
			 14 &0 &-8 &\vline& -38 \\ 
			 0 &7 &-5 &\vline& -22 \\ 
		 \end{pmatrix}
		 \sim
		 \begin{pmatrix}
			 7 &0 &-4 &\vline& -19 \\ 
			 0 &7 &-5 &\vline& -22 \\ 
		 \end{pmatrix}
		 $


		Průsečnice:
		$$ 
		 \zs{\zh{\frac{-19+4 a}{7}; \frac{-22+5 a}{7}; a}:a \in \mathbb{R}} 
		  $$ 
	\item \underline{$\sigma$ a $\alpha$}: \\
		Nerovnoběžné:\\
		 $  \begin{pmatrix}
			 4 &1 &-5 &\vline& -3 \\ 
			 3 &-1 &-1 &\vline& -5 \\ 
		 \end{pmatrix}
		 \sim
		 \begin{pmatrix}
			 3 &-1 &-1 &\vline& -5 \\ 
			 4 &1 &-5 &\vline& -3 \\ 
		 \end{pmatrix}
		 \sim
		 \begin{pmatrix}
			 3 &-1 &-1 &\vline& -5 \\ 
			 0 &7 &-11 &\vline& 11 \\ 
		 \end{pmatrix}
		 \sim
		 \begin{pmatrix}
			 21 &0 &-18 &\vline& -24 \\ 
			 0 &7 &-11 &\vline& 11 \\ 
		 \end{pmatrix}
		 \sim
		 \begin{pmatrix}
			 7 &0 &-6 &\vline& -8 \\ 
			 0 &7 &-11 &\vline& 11 \\ 
		 \end{pmatrix}
		 $


		Průsečnice:
		$$
			 \fce{P} = \zs{\zh{\frac{-8+6 a}{7}; \frac{11+11 a}{7}; a}:a \in \mathbb{R}} 
		 $$
	\item \underline{$\tau$ a $\alpha$}: \\
		Nerovnoběžné:\\
		 $ \begin{pmatrix}
			 1 &2 &-1 &\vline& -1 \\ 
			 3 &-1 &-1 &\vline& -5 \\ 
		 \end{pmatrix}
		 \sim
		 \begin{pmatrix}
			 1 &2 &-1 &\vline& -1 \\ 
			 0 &-7 &2 &\vline& -2 \\ 
		 \end{pmatrix}
		 \sim
		 \begin{pmatrix}
			 1 &2 &-1 &\vline& -1 \\ 
			 0 &7 &-2 &\vline& 2 \\ 
		 \end{pmatrix}
		 \sim
		 \begin{pmatrix}
			 7 &0 &-3 &\vline& -11 \\ 
			 0 &7 &-2 &\vline& 2 \\ 
		 \end{pmatrix}
		 $


		Průsečnice:
		 $$ 
		  \zs{\zh{\frac{-11+3 a}{7}; \frac{2+2 a}{7}; a}:a \in \mathbb{R}} 
		   $$ 
	\item \underline{$\varphi$ a $\alpha$}: \\
		Nerovnoběžné:\\
		 $ \begin{pmatrix}
			 -4 &6 &-2 &\vline& -5 \\ 
			 3 &-1 &-1 &\vline& -5 \\ 
		 \end{pmatrix}
		 \sim
		 \begin{pmatrix}
			 4 &-6 &2 &\vline& 5 \\ 
			 3 &-1 &-1 &\vline& -5 \\ 
		 \end{pmatrix}
		 \sim
		 \begin{pmatrix}
			 3 &-1 &-1 &\vline& -5 \\ 
			 4 &-6 &2 &\vline& 5 \\ 
		 \end{pmatrix}
		 \sim
		 \begin{pmatrix}
			 3 &-1 &-1 &\vline& -5 \\ 
			 0 &-14 &10 &\vline& 35 \\ 
		 \end{pmatrix}
		 \sim
		 \begin{pmatrix}
			 3 &-1 &-1 &\vline& -5 \\ 
			 0 &14 &-10 &\vline& -35 \\ 
		 \end{pmatrix}
		 \sim
		 \begin{pmatrix}
			 42 &0 &-24 &\vline& -105 \\ 
			 0 &14 &-10 &\vline& -35 \\ 
		 \end{pmatrix}
		 \sim
		 \begin{pmatrix}
			 14 &0 &-8 &\vline& -35 \\ 
			 0 &14 &-10 &\vline& -35 \\ 
		 \end{pmatrix}
		 $


		Průsečnice:
		 $$
			  \zs{\zh{\frac{-35+8 a}{14}; \frac{-35+10 a}{14}; a}:a \in \mathbb{R}}
		 $$
	\item \underline{$\rho$ a $\beta$}: \\
		Nerovnoběžné:\\
		 $ \begin{pmatrix}
			 2 &-3 &1 &\vline& 4 \\ 
			 1 &1 &1 &\vline& 7 \\ 
		 \end{pmatrix}
		 \sim
		 \begin{pmatrix}
			 1 &1 &1 &\vline& 7 \\ 
			 2 &-3 &1 &\vline& 4 \\ 
		 \end{pmatrix}
		 \sim
		 \begin{pmatrix}
			 1 &1 &1 &\vline& 7 \\ 
			 0 &-5 &-1 &\vline& -10 \\ 
		 \end{pmatrix}
		 \sim
		 \begin{pmatrix}
			 1 &1 &1 &\vline& 7 \\ 
			 0 &5 &1 &\vline& 10 \\ 
		 \end{pmatrix}
		 \sim
		 \begin{pmatrix}
			 5 &0 &4 &\vline& 25 \\ 
			 0 &5 &1 &\vline& 10 \\ 
		 \end{pmatrix}
		 $


		Průsečnice:
		 $$ 
		 \zs{\zh{\frac{25-4 a}{5}; \frac{10-1 a}{5}; a}:a \in \mathbb{R}} 
		   $$ 
	\item \underline{$\sigma$ a $\beta$}: \\
		Nerovnoběžné:\\
		 $ \begin{pmatrix}
			 4 &1 &-5 &\vline& -3 \\ 
			 1 &1 &1 &\vline& 7 \\ 
		 \end{pmatrix}
		 \sim
		 \begin{pmatrix}
			 1 &1 &1 &\vline& 7 \\ 
			 4 &1 &-5 &\vline& -3 \\ 
		 \end{pmatrix}
		 \sim
		 \begin{pmatrix}
			 1 &1 &1 &\vline& 7 \\ 
			 0 &-3 &-9 &\vline& -31 \\ 
		 \end{pmatrix}
		 \sim
		 \begin{pmatrix}
			 1 &1 &1 &\vline& 7 \\ 
			 0 &3 &9 &\vline& 31 \\ 
		 \end{pmatrix}
		 \sim
		 \begin{pmatrix}
			 3 &0 &-6 &\vline& -10 \\ 
			 0 &3 &9 &\vline& 31 \\ 
		 \end{pmatrix}
		 $


		Průsečnice:
		 $$ 
		  \zs{\zh{\frac{-10+6 a}{3}; \frac{31-9 a}{3}; a}:a \in \mathbb{R}} 
		   $$ 
	\item \underline{$\tau$ a $\beta$}: \\
		Nerovnoběžné:\\
		 $ \begin{pmatrix}
			 1 &2 &-1 &\vline& -1 \\ 
			 1 &1 &1 &\vline& 7 \\ 
		 \end{pmatrix}
		 \sim
		 \begin{pmatrix}
			 1 &2 &-1 &\vline& -1 \\ 
			 0 &-1 &2 &\vline& 8 \\ 
		 \end{pmatrix}
		 \sim
		 \begin{pmatrix}
			 1 &2 &-1 &\vline& -1 \\ 
			 0 &1 &-2 &\vline& -8 \\ 
		 \end{pmatrix}
		 \sim
		 \begin{pmatrix}
			 1 &0 &3 &\vline& 15 \\ 
			 0 &1 &-2 &\vline& -8 \\ 
		 \end{pmatrix}
		 $


		Průsečnice:
		 $$ 
		 \zs{\zh{15-3 a; -8+2 a; a}:a \in \mathbb{R}} 
		   $$ 
	\item \underline{$\varphi$ a $\beta$}: \\
		Nerovnoběžné:\\
		 $ \begin{pmatrix}
			 -4 &6 &-2 &\vline& -5 \\ 
			 1 &1 &1 &\vline& 7 \\ 
		 \end{pmatrix}
		 \sim
		 \begin{pmatrix}
			 4 &-6 &2 &\vline& 5 \\ 
			 1 &1 &1 &\vline& 7 \\ 
		 \end{pmatrix}
		 \sim
		 \begin{pmatrix}
			 1 &1 &1 &\vline& 7 \\ 
			 4 &-6 &2 &\vline& 5 \\ 
		 \end{pmatrix}
		 \sim
		 \begin{pmatrix}
			 1 &1 &1 &\vline& 7 \\ 
			 0 &-10 &-2 &\vline& -23 \\ 
		 \end{pmatrix}
		 \sim
		 \begin{pmatrix}
			 1 &1 &1 &\vline& 7 \\ 
			 0 &10 &2 &\vline& 23 \\ 
		 \end{pmatrix}
		 \sim
		 \begin{pmatrix}
			 10 &0 &8 &\vline& 47 \\ 
			 0 &10 &2 &\vline& 23 \\ 
		 \end{pmatrix}
		 $


		Průsečnice:
		$$ 
		 \zs{\zh{\frac{47-8 a}{10}; \frac{23-2 a}{10}; a}:a \in \mathbb{R}} 
		  $$ 
	\item \underline{$\alpha$ a $\beta$}: \\
		Nerovnoběžné:\\
		 $ \begin{pmatrix}
			 3 &-1 &-1 &\vline& -5 \\ 
			 1 &1 &1 &\vline& 7 \\ 
		 \end{pmatrix}
		 \sim
		 \begin{pmatrix}
			 1 &1 &1 &\vline& 7 \\ 
			 3 &-1 &-1 &\vline& -5 \\ 
		 \end{pmatrix}
		 \sim
		 \begin{pmatrix}
			 1 &1 &1 &\vline& 7 \\ 
			 0 &-4 &-4 &\vline& -26 \\ 
		 \end{pmatrix}
		 \sim
		 \begin{pmatrix}
			 1 &1 &1 &\vline& 7 \\ 
			 0 &2 &2 &\vline& 13 \\ 
		 \end{pmatrix}
		 \sim
		 \begin{pmatrix}
			 2 &0 &0 &\vline& 1 \\ 
			 0 &2 &2 &\vline& 13 \\ 
		 \end{pmatrix}
		 $


		Průsečnice:
		 $$ 
		  \zs{\zh{\frac{1}{2}; \frac{13-2 a}{2}; a}:a \in \mathbb{R}} 
		   $$ 
\end{itemize}
 \Pr 
 183/21
 $$\rho(A=[1,2,0],\ve r = (2,-1,1),\ve s = (-1,1,-1))$$
 $$\sigma(B=[2,3,-1],\ve t = (-2,2,-2),\ve u = (2,-2,-2))$$
 $$\tau(C=[4,3,2],\ve m = (-1,1,1),\ve n = (1,-2,-3))$$
 Což mohu ekvivalentně převést na:
 $$\rho(A=[1,2,0],\ve r = (2,-1,1),\ve s = (1,-1,1))$$
 $$\sigma(B=[2,3,-1],\ve t = (1,-1,1),\ve u = (1,-1,-1))$$
 $$\tau(C=[4,3,2],\ve m = (1,-1,-1),\ve n = (1,-2,-3))$$

 $ \ve{BA} = (-1,-1,1)$ \\
 $ \ve{AC} = (3,1,2)$ \\
 $ \ve{BC} = (2,0,3)$ \\
\begin{itemize}
	\item \underline{$\rho$ a $\phi$}: \\
		 $ \bar{A} = \begin{pmatrix}
			 2 &-1 & 1 \\ 
			 1 &-1 & 1 \\ 
			 1 &-1 & 1 \\ 
			 1 &-1 & -1 \\ 
		 \end{pmatrix}
		 \sim
		 \begin{pmatrix}
			 1 &-1 & 1 \\ 
			 1 &-1 & 1 \\ 
			 1 &-1 & -1 \\ 
			 2 &-1 & 1 \\ 
		 \end{pmatrix}
		 \sim
		 \begin{pmatrix}
			 1 &-1 & 1 \\ 
			 0 &0 & -2 \\ 
			 0 &1 & -1 \\ 
		 \end{pmatrix}
		 \sim
		 \begin{pmatrix}
			 1 &-1 & 1 \\ 
			 0 &0 & 1 \\ 
			 0 &1 & -1 \\ 
		 \end{pmatrix}
		 \sim
		 \begin{pmatrix}
			 1 &-1 & 1 \\ 
			 0 &1 & -1 \\ 
			 0 &0 & 1 \\ 
		 \end{pmatrix}
		  $ \\
		  $\imp \dim\<\ve r, \ve s,\ve t,\ve u\> = 3 \imp $ nerovnoběžné.

		  Průsečnice:\\
		   $ \begin{pmatrix}
			   2 &-1 &2 &-2 &\vline& 1 \\ 
			   -1 &1 &-2 &2 &\vline& 1 \\ 
			   1 &-1 &2 &2 &\vline& -1 \\ 
		   \end{pmatrix}
		   \sim
		   \begin{pmatrix}
			   2 &-1 &2 &-2 &\vline& 1 \\ 
			   1 &-1 &2 &-2 &\vline& -1 \\ 
			   1 &-1 &2 &2 &\vline& -1 \\ 
		   \end{pmatrix}
		   \sim
		   \begin{pmatrix}
			   1 &-1 &2 &-2 &\vline& -1 \\ 
			   1 &-1 &2 &2 &\vline& -1 \\ 
			   2 &-1 &2 &-2 &\vline& 1 \\ 
		   \end{pmatrix}
		   \sim
		   \begin{pmatrix}
			   1 &-1 &2 &-2 &\vline& -1 \\ 
			   0 &0 &0 &4 &\vline& 0 \\ 
			   0 &1 &-2 &2 &\vline& 3 \\ 
		   \end{pmatrix}
		   \sim
		   \begin{pmatrix}
			   1 &-1 &2 &-2 &\vline& -1 \\ 
			   0 &0 &0 &1 &\vline& 0 \\ 
			   0 &1 &-2 &2 &\vline& 3 \\ 
		   \end{pmatrix}
		   \sim
		   \begin{pmatrix}
			   1 &-1 &2 &-2 &\vline& -1 \\ 
			   0 &1 &-2 &2 &\vline& 3 \\ 
			   0 &0 &0 &1 &\vline& 0 \\ 
		   \end{pmatrix}
		   \sim
		   \begin{pmatrix}
			   1 &0 &0 &0 &\vline& 2 \\ 
			   0 &1 &-2 &2 &\vline& 3 \\ 
			   0 &0 &0 &1 &\vline& 0 \\ 
		   \end{pmatrix}
		   \sim
		   \begin{pmatrix}
			   1 &0 &0 &0 &\vline& 2 \\ 
			   0 &1 &-2 &0 &\vline& 3 \\ 
			   0 &0 &0 &1 &\vline& 0 \\ 
		   \end{pmatrix}
		    $ 
		    $$u=0 \land t\in\R \imp p=\{[2-2t;3+2t;-1-2t]|t\in\R\}$$

	\item \underline{$\rho$ a $\tau$}: \\
		 $ \begin{pmatrix}
			 2 &-1 & 1 \\ 
			 -1 &1 & -1 \\ 
			 -1 &1 & 1 \\ 
			 1 &-2 & -3 \\ 
		 \end{pmatrix}
		 \sim
		 \begin{pmatrix}
			 2 &-1 & 1 \\ 
			 1 &-1 & 1 \\ 
			 1 &-1 & -1 \\ 
			 1 &-2 & -3 \\ 
		 \end{pmatrix}
		 \sim
		 \begin{pmatrix}
			 1 &-1 & 1 \\ 
			 1 &-1 & -1 \\ 
			 1 &-2 & -3 \\ 
			 2 &-1 & 1 \\ 
		 \end{pmatrix}
		 \sim
		 \begin{pmatrix}
			 1 &-1 & 1 \\ 
			 0 &0 & -2 \\ 
			 0 &-1 & -4 \\ 
			 0 &1 & -1 \\ 
		 \end{pmatrix}
		 \sim
		 \begin{pmatrix}
			 1 &-1 & 1 \\ 
			 0 &0 & 1 \\ 
			 0 &1 & 4 \\ 
			 0 &1 & -1 \\ 
		 \end{pmatrix}
		 \sim
		 \begin{pmatrix}
			 1 &-1 & 1 \\ 
			 0 &1 & 4 \\ 
			 0 &1 & -1 \\ 
			 0 &0 & 1 \\ 
		 \end{pmatrix}
		 \sim
		 \begin{pmatrix}
			 1 &-1 & 1 \\ 
			 0 &1 & 4 \\ 
			 0 &0 & -5 \\ 
			 0 &0 & 1 \\ 
		 \end{pmatrix}
		 \sim
		 \begin{pmatrix}
			 1 &-1 & 1 \\ 
			 0 &1 & 4 \\ 
			 0 &0 & 1 \\ 
			 0 &0 & 1 \\ 
		 \end{pmatrix}
		 \sim
		 \begin{pmatrix}
			 1 &-1 & 1 \\ 
			 0 &1 & 4 \\ 
			 0 &0 & 1 \\ 
		 \end{pmatrix}
		  $ 
		  $\imp \dim\<\ve r, \ve s,\ve m,\ve n\> = 3 \imp $ nerovnoběžné.

		
	\item \underline{$\varphi$ a $\tau$}: \\
		 $\begin{pmatrix}
			 1 &-1 & 1 \\ 
			 1 &-1 & -1 \\ 
			 1 &-1 & -1 \\ 
			 1 &-2 & -3 \\ 
		 \end{pmatrix}
		 \sim
		 \begin{pmatrix}
			 1 &-1 & 1 \\ 
			 0 &0 & -2 \\ 
			 0 &0 & -2 \\ 
			 0 &-1 & -4 \\ 
		 \end{pmatrix}
		 \sim
		 \begin{pmatrix}
			 1 &-1 & 1 \\ 
			 0 &0 & 1 \\ 
			 0 &0 & 1 \\ 
			 0 &1 & 4 \\ 
		 \end{pmatrix}
		 \sim
		 \begin{pmatrix}
			 1 &-1 & 1 \\ 
			 0 &1 & 4 \\ 
			 0 &0 & 1 \\ 
			 0 &0 & 1 \\ 
		 \end{pmatrix}
		 \sim
		 \begin{pmatrix}
			 1 &-1 & 1 \\ 
			 0 &1 & 4 \\ 
			 0 &0 & 1 \\ 
		 \end{pmatrix}
		  $ 
		  $\imp \dim\<\ve t, \ve u,\ve m,\ve n\> = 3 \imp $ nerovnoběžné.
		



\end{itemize}



\EndDoc
