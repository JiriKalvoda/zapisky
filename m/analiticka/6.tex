\providecommand{\HINCLUDE}{NE}
\if ^\HINCLUDE^
\else
\def\HINCLUDE{}
\global\newdimen\Okraje
\global\Okraje =4cm
\input{$HOME/souteze/_hlavicka/h-.tex}

%\definecolor{colorV}{RGB}{255,127,0}
%\definecolor{colorPoz}{RGB}{153,51,0}
%\definecolor{orangeV}{RGB}{255,127,0}
%\definecolor{colorPr}{RGB}{0,5,255}
%\definecolor{colorDef}{RGB}{0.255,0}

\usepackage[shortlabels]{enumitem}
\setlength{\marginparsep}{2pt}
\setlength{\marginparwidth}{35pt}

\def\st{{\rm st}}
\def\P{{\rm P}}

\def\ISENUM{}
\def\inMargin#1{\End
		
		\hskip0pt \marginpar{{{#1}}}}
\newcounter{V}[section] 
\newcommand{\V}[1][]{\stepcounter{V}\inMargin{\textcolor{green}{V.\arabic{section}.\theV.:}}\ifx^#1^\else\textcolor{green}{\underline{{#1}:}}\addcontentsline{toc}{subsubsection}{V.\arabic{section}.\theV.:$\quad$ {#1}}\\\fi}
\def\Def{\inMargin{\textcolor{red}{Def:}}}
\def\Poz{{\inMargin{\textcolor{brown}{Pozn:}}}}
\def\Pr{{\inMargin{\textcolor{blue}{Př:}}}}
\def\Pozenum
{
	\begin{enumerate}[1)]%, left = 0pt ]
		\item\inMargin{\textcolor{brown}{Pozn:}}\def\ISENUM{a}}
\def\End
{
	\if	^\ISENUM^
	\else \end{enumerate}
	\fi
	\def\ISENUM{}
}
\reversemarginpar

\makeatletter
\renewcommand\thesection{§\arabic{section}.}
\renewcommand\thesubsection{\Alph{subsection})}
\renewcommand\thesubsubsection{\alph{subsubsection})}
\newcounter{chapter}
\setcounter{chapter}{0}
\renewcommand\thechapter{\Alph{chapter})}
\newcounter{roman}
\setcounter{roman}{0}
\renewcommand\theroman{\Roman{roman}.}
\makeatother
\def\sectionnum#1{\setcounter{section}{#1}\addtocounter{section}{-1}}
\def\subsectionnum#1{\setcounter{subsection}{#1}\addtocounter{subsection}{-1}}
\def\subsubsectionnum#1{\setcounter{subsubsection}{#1}\addtocounter{subsubsection}{-1}}
\def\chapternum#1{\setcounter{chapter}{#1}\addtocounter{chapter}{-1}}
\def\chapter#1{

	\addtocounter{chapter}{1}\sectionnum{1}
	\addcontentsline{toc}{section}{\large{\thechapter$\quad${#1}}}
	
	{\LARGE  \textbf{\begin{minipage}[t]{0.1\textwidth}\thechapter\end{minipage}\begin{minipage}[t]{0.95\textwidth}#1\end{minipage}}}

}
\def\ROM{}
\def\Rom#1#2{\setcounter{roman}{#1}\renewcommand\ROM{#2}}

\Rom{6}{Funkce}
\title{\Huge\textbf{\theroman\quad \ROM}}
\author{Jiří Kalvoda}

\newcounter{countOfBegin}
\setcounter{countOfBegin}{0}
\newcommand{\BeginDoc}[1][]
{
	\ifnum\value{countOfBegin}=0
	\begin{document}
		#1
		\fi
	\addtocounter{countOfBegin}{1}
		
}
\def\EndDoc
{
	\addtocounter{countOfBegin}{-1}
	\ifnum\value{countOfBegin}=0
	\end{document}
	\fi
}

\fi
\BeginDoc{\sectionnum{6}}
\section{Vzájemná poloha přímky a roviny}
\V [Věta o vzájemné poloze přímky a roviny dané parametrickými rovnicemi]
Nechť $p(A,\ve u)$ je přímka,$\rho(B,\ve v,\ve w)$ rovina. Pak platí:
\begin{itemize}
	\item $p \subset \phi \ekv \dim\<\ve v, \ve w,\ve u\> = 2 \land \dim \<\ve v,\ve w,\ve u,\ve{AB}\> = 2$
	\item $p \parallel \phi \ekv \dim\<\ve v, \ve w,\ve u\> = 2 \land \dim \<\ve v,\ve w,\ve u,\ve{AB}\> = 3$
	\item $p \nparallel \phi \ekv \dim\<\ve v, \ve w,\ve u\> = 3$
\end{itemize}

\Pr Rozhodněte vzájemnou polohu přímky a roviny:
$ p = \{[3+t;1+2t;2-t]|t\in\R\}  \imp p: A=[3;1;2],\ve u = (1;2;-1)$\\
$ \rho = \{[1-3r+s;2r-s;1+4r-s]|r,s\in\R\} \imp \rho: B = [1;0;1],\ve v = (-3;2;4),\ve w = (1;-1;-1)$\\
$\imp \ve{AB}  = (-2;-1;-1) $

 $\begin{pmatrix}
	 1 &-1 & -1 \\ 
	 -3 &2 & 4 \\ 
	 1 &2 & -1 \\ 
 \end{pmatrix}
 \sim
 \begin{pmatrix}
	 1 &-1 & -1 \\ 
	 3 &-2 & -4 \\ 
	 1 &2 & -1 \\ 
 \end{pmatrix}
 \sim
 \begin{pmatrix}
	 1 &-1 & -1 \\ 
	 1 &2 & -1 \\ 
	 3 &-2 & -4 \\ 
 \end{pmatrix}
 \sim
 \begin{pmatrix}
	 1 &-1 & -1 \\ 
	 0 &3 & 0 \\ 
	 0 &1 & -1 \\ 
 \end{pmatrix}
 \sim
 \begin{pmatrix}
	 1 &-1 & -1 \\ 
	 0 &1 & 0 \\ 
	 0 &1 & -1 \\ 
 \end{pmatrix}
 \sim
 \begin{pmatrix}
	 1 &-1 & -1 \\ 
	 0 &1 & 0 \\ 
	 0 &0 & -1 \\ 
 \end{pmatrix}
 \sim
 \begin{pmatrix}
	 1 &-1 & -1 \\ 
	 0 &1 & 0 \\ 
	 0 &0 & 1 \\ 
 \end{pmatrix}
  $ 

  $\dim\<\ve u, \ve v,\ve w\> = 3 \imp $ přímka je různoběžná.

  Určení průsečíku:\\
  porovnáme souřadnice $p$ a $\rho$:
  $ 3+t = 1 - 3r + s $\\
  $ 1 + 2t = 2r - s$\\
  $2-t = 1+4r - s $\\
  $\imp$ 3 rovnice o třech neznámích -- vyřešením dostaneme $t=-2$.
  Dosadím: \underline{$P = [1;-3;4]$}



\V[Věta o vzájemné poloze přímky a roviny dané obecnou rovnicí]
Nechť $p(A,\ve u)$ je přímka, $\rho: ax+by+cz+d=0,[a,b,c]\neq[0,0,0]$ rovina.
Nechť $\ve n = (a,b,c)$. Pak platí:
\begin{itemize}
	\item $p\subset\rho \ekv \ve u \* \ve n = 0 \land A \in \rho$
	\item $p\parallel\rho \ekv \ve u \* \ve n = 0 \land A \not\in \rho$
	\item $p\nparallel\rho \ekv \ve u \* \ve n \neq 0$
\end{itemize}

\Pr
Rozhodněte o vzájemné poloze přímky $p$ a roviny $\rho$:
\begin{itemize}
	\item $p = {[1-t,1+3t;-2]|t\in\R}, rho: 3x+y+5z+7 = 0$\\
		$\imp \ve u = (-1;3;0), \ve n = (3;1;5)$.\\
		$\pri u \* \pri n = -3 + 3 +0 = 0 \imp$ přímka je s rovinou rovnoběžná.
		Rozhodneme, jestli $p$ leží v rovině $\rho$, tzn. jestli $A\in \rho$:\\
		$A[1;1;-2]$\\
		$3+1-10+7=1\neq 0 \imp 0$ \underline{rovnoběžné různé.}
	\item
		$p = {[3+t;1-t;2t]|t\in\R}, \rho: x-2y+z-3 = 0$\\
		$\ve u = (1;-1;2)$\\
		$\ve n = (1;-2;1)$\\
		$\ve u \* \ve n  = 1+2+2 = 5 \ne 0 \imp p\nparallel \rho $

		Určení průsečíku:\\
		Dosadíme rovnici přímky do rovnice roviny $\rho$:
		$3+t-2+2t+2t-3 = 0 \imp t = \f 25$.
		$$ p = \[\f{17}5;\f 35 ; \f 45\] $$
\end{itemize}
\Pr 188/26:\\
$\ve u = \ve{PQ} = (1;0;2)$\\
$\ve n =  (2;1;1)$\\
$\ve n \* \ve u = 2+0+2 = 4 \imp $ nejsou rovnoběžné.

$2\*t + 0 + 2 t + 8 = 0 \imp t = -2 \imp \pri{PQ} \cap \rho = P+\ve{PQ}\*(-2) = \[-2;0;-4\]$

\Pr 188/27:\\
Určím rovnoběžnou rovinu procházející $A$:
$$ 2\*3-2-1+a = 0 \imp a = -3$$
Rovnicí tedy je $\varphi: 2x - y - z -3 = 0$.\\
Dosadím: $6 - y + 2 - 3 = 0 \imp \underline{y=5}$.

\Pr 189/28:
\begin{itemize}
	\item
$\ve t  = (-1,1,-3)$\\
$\ve n =  (-1;2;1)$\\
$\ve t \* \ve n = 1 + 2 - 3 = 0 \imp $ rovnoběžné\\

$A=[1,0,2]$. Dosadím: $-1+0+2-1 = 0$.
Průnikem je $p$.

\item
	$\ve t = (-1,3,1)$\\
		$\ve n = (1,1,-1)$\\
		$\ve t \* \ve n = -1 + 3 -1 = 1 \imp $ nejsou rovnoběžné.

		Průsečík:
		$2-t+3t-t = 4 \imp t = 2 \imp q\cap \sigma = {[-1,1,-4]}$.

	\item
		$\ve t = (3;-4;2) $\\
		$\ve n = (2;1;-1)$\\
		$\ve t \* \ve n = 6 -4 -2 = 0 \imp $ rovnoběžné.

		$A[2;1;0]$ dosadím: $4+1=5\neq0 \imp m \cap \tau = \emptyset$. 
\end{itemize}

\Pr 189/29:
Průsečnice $p$:
 $ \begin{pmatrix}
	 3 &-4 &1 &\vline& 7 \\ 
	 1 &3 &-2 &\vline& -8 \\ 
 \end{pmatrix}
 \sim
 \begin{pmatrix}
	 1 &3 &-2 &\vline& -8 \\ 
	 3 &-4 &1 &\vline& 7 \\ 
 \end{pmatrix}
 \sim
 \begin{pmatrix}
	 1 &3 &-2 &\vline& -8 \\ 
	 0 &-13 &7 &\vline& 31 \\ 
 \end{pmatrix}
 \sim
 \begin{pmatrix}
	 1 &3 &-2 &\vline& -8 \\ 
	 0 &13 &-7 &\vline& -31 \\ 
 \end{pmatrix}
 \sim
 \begin{pmatrix}
	 13 &0 &-5 &\vline& -11 \\ 
	 0 &13 &-7 &\vline& -31 \\ 
 \end{pmatrix}
  $ 
   $$ 
    p = \zs{\zh{\frac{-11+5 t}{13}; \frac{-31+7 t}{13}; a}:t \in \mathbb{R}}\imp p \(A\[-\f{11}{13};-\f{31}{13};0\];\ve u  = \(5;7;13\)\)
     $$ 
     $$
     \rho (B[5;3;1],\ve v = (-1;1;0),\ve w = (2,-1,5)
     $$

      $ \begin{pmatrix}
	      5 &7 & 13 \\ 
	      -1 &1 & 0 \\ 
	      2 &-1 & 5 \\ 
      \end{pmatrix}
      \sim
      \begin{pmatrix}
	      5 &7 & 13 \\ 
	      1 &-1 & 0 \\ 
	      2 &-1 & 5 \\ 
      \end{pmatrix}
      \sim
      \begin{pmatrix}
	      1 &-1 & 0 \\ 
	      2 &-1 & 5 \\ 
	      5 &7 & 13 \\ 
      \end{pmatrix}
      \sim
      \begin{pmatrix}
	      1 &-1 & 0 \\ 
	      0 &1 & 5 \\ 
	      0 &12 & 13 \\ 
      \end{pmatrix}
      \sim
      \begin{pmatrix}
	      1 &-1 & 0 \\ 
	      0 &1 & 5 \\ 
	      0 &0 & -47 \\ 
      \end{pmatrix}
      \sim
      \begin{pmatrix}
	      1 &-1 & 0 \\ 
	      0 &1 & 5 \\ 
	      0 &0 & 1 \\ 
      \end{pmatrix}
       $

       Nejsou rovnoběžné.

       Převedu $\rho$ na obecnou rovnici roviny:
       $x+y = 8 + s \imp 5x + 5y - z = 40-1 = 39$
       Spočítám průsečím všech rovnic:

 $ \bar{A} = \begin{pmatrix}
	 3 &-4 &1 &\vline& 7 \\ 
	 1 &3 &-2 &\vline& -8 \\ 
	 5 &5 &-1 &\vline& 39 \\ 
 \end{pmatrix}
 \sim
 \begin{pmatrix}
	 1 &3 &-2 &\vline& -8 \\ 
	 3 &-4 &1 &\vline& 7 \\ 
	 5 &5 &-1 &\vline& 39 \\ 
 \end{pmatrix}
 \sim
 \begin{pmatrix}
	 1 &3 &-2 &\vline& -8 \\ 
	 0 &-13 &7 &\vline& 31 \\ 
	 0 &-10 &9 &\vline& 79 \\ 
 \end{pmatrix}
 \sim
 \begin{pmatrix}
	 1 &3 &-2 &\vline& -8 \\ 
	 0 &13 &-7 &\vline& -31 \\ 
	 0 &10 &-9 &\vline& -79 \\ 
 \end{pmatrix}
 \sim
 \begin{pmatrix}
	 1 &3 &-2 &\vline& -8 \\ 
	 0 &10 &-9 &\vline& -79 \\ 
	 0 &13 &-7 &\vline& -31 \\ 
 \end{pmatrix}
 \sim
 \begin{pmatrix}
	 1 &3 &-2 &\vline& -8 \\ 
	 0 &10 &-9 &\vline& -79 \\ 
	 0 &0 &47 &\vline& 717 \\ 
 \end{pmatrix}
 \sim
 \begin{pmatrix}
	 10 &0 &7 &\vline& 157 \\ 
	 0 &10 &-9 &\vline& -79 \\ 
	 0 &0 &47 &\vline& 717 \\ 
 \end{pmatrix}
 \sim
 \begin{pmatrix}
	 470 &0 &0 &\vline& 2360 \\ 
	 0 &470 &0 &\vline& 2740 \\ 
	 0 &0 &47 &\vline& 717 \\ 
 \end{pmatrix}
 \sim
 \begin{pmatrix}
	 47 &0 &0 &\vline& 236 \\ 
	 0 &47 &0 &\vline& 274 \\ 
	 0 &0 &47 &\vline& 717 \\ 
 \end{pmatrix}
  $ 
   $$ 
    P = \zs{\zh{\frac{236}{47}; \frac{274}{47}; \frac{717}{47}}} 
     $$ 


       \Pr 
       Převedu $\gamma$ na obecnou rovnici:
       $x + z = -6 + r \imp x-y+z  = -5$

       Průsečík:
        $ \begin{pmatrix}
		3 &-1 &1 &\vline& -1 \\ 
		-1 &2 &-1 &\vline& 5 \\ 
		1 &-1 &1 &\vline& -5 \\ 
	\end{pmatrix}
	\sim
	\begin{pmatrix}
		3 &-1 &1 &\vline& -1 \\ 
		1 &-2 &1 &\vline& -5 \\ 
		1 &-1 &1 &\vline& -5 \\ 
	\end{pmatrix}
	\sim
	\begin{pmatrix}
		1 &-2 &1 &\vline& -5 \\ 
		1 &-1 &1 &\vline& -5 \\ 
		3 &-1 &1 &\vline& -1 \\ 
	\end{pmatrix}
	\sim
	\begin{pmatrix}
		1 &-2 &1 &\vline& -5 \\ 
		0 &1 &0 &\vline& 0 \\ 
		0 &5 &-2 &\vline& 14 \\ 
	\end{pmatrix}
	\sim
	\begin{pmatrix}
		1 &-2 &1 &\vline& -5 \\ 
		0 &1 &0 &\vline& 0 \\ 
		0 &0 &-2 &\vline& 14 \\ 
	\end{pmatrix}
	\sim
	\begin{pmatrix}
		1 &-2 &1 &\vline& -5 \\ 
		0 &1 &0 &\vline& 0 \\ 
		0 &0 &1 &\vline& -7 \\ 
	\end{pmatrix}
	\sim
	\begin{pmatrix}
		1 &0 &1 &\vline& -5 \\ 
		0 &1 &0 &\vline& 0 \\ 
		0 &0 &1 &\vline& -7 \\ 
	\end{pmatrix}
	\sim
	\begin{pmatrix}
		1 &0 &0 &\vline& 2 \\ 
		0 &1 &0 &\vline& 0 \\ 
		0 &0 &1 &\vline& -7 \\ 
	\end{pmatrix}
	 $ 
	  $$ 
	   P = \zs{\zh{2; 0; -7}} 
	    $$ 

	    Převedu $\delta$ na obecnou:
	    $x+3y = 12 +7t \imp x+3y+7z =19$
	    
	    $$ 1+0-49+a=0 \imp a=48$$
	    
	    \underline{$\epsilon: x+3y+7z+48=0$}
	    \Pr 32
	    Jelikož se roviny prtínají v právě jednom bodě, musí průsečnice protínat $\gamma$ v jednom bodě.

	    Spočítám dimenzi vektorového prostoru tvořeného normálovými vektory k rovinám:
	     $ \begin{pmatrix}
		     3 &-1 & 1 \\ 
		     -1 &2 & -1 \\ 
		     1 &3 & 7 \\ 
	     \end{pmatrix}
	     \sim
	     \begin{pmatrix}
		     3 &-1 & 1 \\ 
		     1 &-2 & 1 \\ 
		     1 &3 & 7 \\ 
	     \end{pmatrix}
	     \sim
	     \begin{pmatrix}
		     1 &-2 & 1 \\ 
		     1 &3 & 7 \\ 
		     3 &-1 & 1 \\ 
	     \end{pmatrix}
	     \sim
	     \begin{pmatrix}
		     1 &-2 & 1 \\ 
		     0 &5 & 6 \\ 
		     0 &5 & -2 \\ 
	     \end{pmatrix}
	     \sim
	     \begin{pmatrix}
		     1 &-2 & 1 \\ 
		     0 &5 & 6 \\ 
		     0 &0 & -8 \\ 
	     \end{pmatrix}
	     \sim
	     \begin{pmatrix}
		     1 &-2 & 1 \\ 
		     0 &5 & 6 \\ 
		     0 &0 & 1 \\ 
	     \end{pmatrix}
	      $ 
	      Jelikož dimenze je 3,roviny se protínají v právě jednom bodě.
	      Jelikož $\epsilon$ má stejný normálový vektor jako $\delta$, výsledk se nezmění.

\EndDoc
