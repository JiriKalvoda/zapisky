\providecommand{\HINCLUDE}{NE}
\if ^\HINCLUDE^
\else
\def\HINCLUDE{}
\global\newdimen\Okraje
\global\Okraje =4cm
\input{$HOME/souteze/_hlavicka/h-.tex}

%\definecolor{colorV}{RGB}{255,127,0}
%\definecolor{colorPoz}{RGB}{153,51,0}
%\definecolor{orangeV}{RGB}{255,127,0}
%\definecolor{colorPr}{RGB}{0,5,255}
%\definecolor{colorDef}{RGB}{0.255,0}

\usepackage[shortlabels]{enumitem}
\setlength{\marginparsep}{2pt}
\setlength{\marginparwidth}{35pt}

\def\st{{\rm st}}
\def\P{{\rm P}}

\def\ISENUM{}
\def\inMargin#1{\End
		
		\hskip0pt \marginpar{{{#1}}}}
\newcounter{V}[section] 
\newcommand{\V}[1][]{\stepcounter{V}\inMargin{\textcolor{green}{V.\arabic{section}.\theV.:}}\ifx^#1^\else\textcolor{green}{\underline{{#1}:}}\addcontentsline{toc}{subsubsection}{V.\arabic{section}.\theV.:$\quad$ {#1}}\\\fi}
\def\Def{\inMargin{\textcolor{red}{Def:}}}
\def\Poz{{\inMargin{\textcolor{brown}{Pozn:}}}}
\def\Pr{{\inMargin{\textcolor{blue}{Př:}}}}
\def\Pozenum
{
	\begin{enumerate}[1)]%, left = 0pt ]
		\item\inMargin{\textcolor{brown}{Pozn:}}\def\ISENUM{a}}
\def\End
{
	\if	^\ISENUM^
	\else \end{enumerate}
	\fi
	\def\ISENUM{}
}
\reversemarginpar

\makeatletter
\renewcommand\thesection{§\arabic{section}.}
\renewcommand\thesubsection{\Alph{subsection})}
\renewcommand\thesubsubsection{\alph{subsubsection})}
\newcounter{chapter}
\setcounter{chapter}{0}
\renewcommand\thechapter{\Alph{chapter})}
\newcounter{roman}
\setcounter{roman}{0}
\renewcommand\theroman{\Roman{roman}.}
\makeatother
\def\sectionnum#1{\setcounter{section}{#1}\addtocounter{section}{-1}}
\def\subsectionnum#1{\setcounter{subsection}{#1}\addtocounter{subsection}{-1}}
\def\subsubsectionnum#1{\setcounter{subsubsection}{#1}\addtocounter{subsubsection}{-1}}
\def\chapternum#1{\setcounter{chapter}{#1}\addtocounter{chapter}{-1}}
\def\chapter#1{

	\addtocounter{chapter}{1}\sectionnum{1}
	\addcontentsline{toc}{section}{\large{\thechapter$\quad${#1}}}
	
	{\LARGE  \textbf{\begin{minipage}[t]{0.1\textwidth}\thechapter\end{minipage}\begin{minipage}[t]{0.95\textwidth}#1\end{minipage}}}

}
\def\ROM{}
\def\Rom#1#2{\setcounter{roman}{#1}\renewcommand\ROM{#2}}

\Rom{6}{Funkce}
\title{\Huge\textbf{\theroman\quad \ROM}}
\author{Jiří Kalvoda}

\newcounter{countOfBegin}
\setcounter{countOfBegin}{0}
\newcommand{\BeginDoc}[1][]
{
	\ifnum\value{countOfBegin}=0
	\begin{document}
		#1
		\fi
	\addtocounter{countOfBegin}{1}
		
}
\def\EndDoc
{
	\addtocounter{countOfBegin}{-1}
	\ifnum\value{countOfBegin}=0
	\end{document}
	\fi
}

\fi
\BeginDoc{}
\section{Polynomy, kořeny polynomů}
\def\d{\rm d}
\pd{9-1.pdf}
Když $x\ge 0$:
$$\f{2}{2x-2} = \f{1}{x+1}$$
Když $x\le 0$:
$$\f{2}{0-2} = 1$$
\pd{9-1r.pdf}
\pd{9-2.pdf}
\def\dec{\left| 
\begin{array}{ccc}}
\def\dee{\end{array}
\right|}
\begin{enumerate}
	\item[a)]
		$$\int \f{\d x}{(x-2)^3} =  \int \f{\d(x-2)}{(x-2)^3} = \f 1{2(x-2)^2} +C$$
	\item[d)]
		$$2  \int\f{\d(2x-3)}{(2x-3)^3} = \f 2 {2 (2x-3)^2} + C$$
		$$2  \int\f{\d(x-a)}{(x-a)^n} = \f 1 {(n-1) (x-a)^{n-1}} + C$$
	\item[e)]/
\end{enumerate}

\pd{9-3.pdf}
Vždyť takové kkekely jsme se určitě neučili.
\pd{9-3r.pdf}
\pd{9-4.pdf}
\begin{enumerate}
	\item [e)]
		$$\f \pi 6$$
\end{enumerate}

\pd{9-5.pdf}
\begin{enumerate}
	\item [a)]
		$$\dec1&1&1\\1&p&0\\0&1&-1\dee = 2-p$$
		Když $p=2$:
		 $ \bar{A} = \begin{pmatrix}
1 &1 &1 &\vline& 6 \\
1 &2 &0 &\vline& 9 \\
0 &1 &-1 &\vline& -1 \\
\end{pmatrix}
\sim
\begin{pmatrix}
1 &1 &1 &\vline& 6 \\
0 &1 &-1 &\vline& 3 \\
0 &1 &-1 &\vline& -1 \\
\end{pmatrix}
\sim
\begin{pmatrix}
1 &1 &1 &\vline& 6 \\
0 &1 &-1 &\vline& 3 \\
0 &0 &0 &\vline& -4 \\
\end{pmatrix}
\sim
\begin{pmatrix}
1 &1 &1 &\vline& 6 \\
0 &1 &-1 &\vline& 3 \\
0 &0 &0 &\vline& 1 \\
\end{pmatrix}
 $
 $$
 \fce{P} = \emptyset
 $$


		$$\dec6&1&1\\9&p&0\\-1&1&-1\dee = 18-5p$$
		$$x=\f{18-5p}{2-p}$$
		$$\dec1&6&1\\1&9&0\\0&-1&-1\dee = -4$$
		$$y=\f{-4}{2-p}$$
		$$\dec1&1&6\\1&p&9\\0&1&-1\dee = -p-2$$
		$$z=\f{p+2}{p-2}$$

		$$\zs{\[\f{18-5p}{2-p};\f{-4}{2-p};\f{p+2}{p-2}\]}$$
	\item [b,c,d)] Analogicky.
\end{enumerate}

\pd{9-6.pdf}
\begin{enumerate}
	\item [a)]
		$$\dec p&1&1\\1&p&1\\1&p&1 \dee = p^3+3p+2 = (p-1)^2(p+2)$$
		Když $p=1$: $$x+y+z=1$$
		$$P=\zs{[x,y,1-x-y]|x,y\in\R}$$
		Když $p=-2$:
		 $ \bar{A} = \begin{pmatrix}
-2 &1 &1 &\vline& 1 \\
1 &-2 &1 &\vline& -2 \\
1 &1 &-2 &\vline& 4 \\
\end{pmatrix}
\sim
\begin{pmatrix}
2 &-1 &-1 &\vline& -1 \\
1 &-2 &1 &\vline& -2 \\
1 &1 &-2 &\vline& 4 \\
\end{pmatrix}
\sim
\begin{pmatrix}
1 &-2 &1 &\vline& -2 \\
1 &1 &-2 &\vline& 4 \\
2 &-1 &-1 &\vline& -1 \\
\end{pmatrix}
\sim
\begin{pmatrix}
1 &-2 &1 &\vline& -2 \\
0 &3 &-3 &\vline& 6 \\
0 &3 &-3 &\vline& 3 \\
\end{pmatrix}
\sim
\begin{pmatrix}
1 &-2 &1 &\vline& -2 \\
0 &1 &-1 &\vline& 2 \\
0 &1 &-1 &\vline& 1 \\
\end{pmatrix}
\sim
\begin{pmatrix}
1 &-2 &1 &\vline& -2 \\
0 &1 &-1 &\vline& 2 \\
0 &0 &0 &\vline& -1 \\
\end{pmatrix}
\sim
\begin{pmatrix}
1 &-2 &1 &\vline& -2 \\
0 &1 &-1 &\vline& 2 \\
0 &0 &0 &\vline& 1 \\
\end{pmatrix}
 $
 $$
 \fce{P} = \emptyset
 $$
 Jinak:
		$$\dec 1&1&1\\p&p&1\\p^2&p&1 \dee = -p^3+p^2+p-1$$
		$$x = \f{-p^3+p^2+p-1}{(p-1)^2(p+2)}$$
		$$\dec p&1&1\\1&p&1\\1&p^2&1 \dee = p^2-2p+1$$
		$$y = \f{p^2-2p+1}{(p-1)^2(p+2)}$$
		$$\dec p&1&1\\1&p&p\\1&p&p^2 \dee = p^4-2p^2+1$$
		$$z = \f{p^4-2p^2+1}{(p-1)^2(p+2)}$$

 $$
		\fce{P} = \zs{\[\f{-p^3+p^2+p-1}{(p-1)^2(p+2)};\f{p^2-2p+1}{(p-1)^2(p+2)};\f{p^4-2p^2+1}{(p-1)^2(p+2)}\]}
 $$

	\item [b)] Analogicky.
\end{enumerate}
\pd{9-7.pdf}
Nechť $b=a+1$:
$$\dec b&1&1\\1&b&1\\1&1&b\dee = = b^3+3b+2 = (b-1)^2(b+2)=a^2(a+3)$$
Když $a=0$:
$$3= x+y+z = 1$$
Nemá řešnení.
Když $a=-3$:
 $ \bar{A} = \begin{pmatrix}
-2 &1 &1 &\vline& -2 \\
1 &-2 &1 &\vline& 0 \\
1 &1 &-2 &\vline& 2 \\
\end{pmatrix}
\sim
\begin{pmatrix}
2 &-1 &-1 &\vline& 2 \\
1 &-2 &1 &\vline& 0 \\
1 &1 &-2 &\vline& 2 \\
\end{pmatrix}
\sim
\begin{pmatrix}
1 &-2 &1 &\vline& 0 \\
1 &1 &-2 &\vline& 2 \\
2 &-1 &-1 &\vline& 2 \\
\end{pmatrix}
\sim
\begin{pmatrix}
1 &-2 &1 &\vline& 0 \\
0 &3 &-3 &\vline& 2 \\
0 &3 &-3 &\vline& 2 \\
\end{pmatrix}
\sim
\begin{pmatrix}
1 &-2 &1 &\vline& 0 \\
0 &3 &-3 &\vline& 2 \\
\end{pmatrix}
\sim
\begin{pmatrix}
3 &0 &-3 &\vline& 4 \\
0 &3 &-3 &\vline& 2 \\
\end{pmatrix}
 $
 $$
 \zs{\zh{\frac{4+3 a}{3}; \frac{2+3 a}{3}; a}:a \in \mathbb{R}}
 $$

 Jinak:
	$$\dec b&1&1\\b+2&b&1\\-b&1&b\dee = b^3+b^2-4b+2 = a^3+4a^2+1$$
		$$ x = \f{a^2+4+1}{a(a+3)} $$
	$$\dec b&1&b\\1&b+2&1\\1&-2b&b\dee = p^3+3p^2-2p-2 = a^3+6a^2+7a$$
		$$ y = \f{a^2+6a+7}{a(a+3)} $$
	$$\dec b&1&b\\1&b&b+2\\1&1&-2b\dee = -2p^3-2p^2+2p+2 = -2a^3-8a^2-8a $$
		$$ z = \f{-2a^2-8a-8}{a^2(a+3)} $$

		$$\zs{\[\f{a^2+4+1}{a(a+3)};\f{a^2+6a+7}{a(a+3)};\f{-2a^2-8a-8}{a^2(a+3)}\]}$$



\pd{9-8.pdf}
$$\dec a&1&1\\1&b&1\\1&2b&1\dee ab+1+2b-b-2ab-1 = b(1-a)$$
\begin{enumerate}
	\item 
Když $b=0$:
$$3=x+z=4$$
Nemá řešení.

	\item 
Když $a=1$:
$$x+y+z=4=z+2by+z$$
\begin{enumerate}
	\item Když $b=\f 12$:

		 $ \bar{A} = \begin{pmatrix}
2 &2 &2 &\vline& 8 \\
2 &1 &2 &\vline& 6 \\
2 &2 &2 &\vline& 8 \\
\end{pmatrix}
\sim
\begin{pmatrix}
1 &1 &1 &\vline& 4 \\
2 &1 &2 &\vline& 6 \\
1 &1 &1 &\vline& 4 \\
\end{pmatrix}
\sim
\begin{pmatrix}
1 &1 &1 &\vline& 4 \\
1 &1 &1 &\vline& 4 \\
2 &1 &2 &\vline& 6 \\
\end{pmatrix}
\sim
\begin{pmatrix}
1 &1 &1 &\vline& 4 \\
0 &-1 &0 &\vline& -2 \\
\end{pmatrix}
\sim
\begin{pmatrix}
1 &1 &1 &\vline& 4 \\
0 &1 &0 &\vline& 2 \\
\end{pmatrix}
\sim
\begin{pmatrix}
1 &1 &1 &\vline& 4 \\
0 &1 &0 &\vline& 2 \\
\end{pmatrix}
\sim
\begin{pmatrix}
1 &0 &1 &\vline& 2 \\
0 &1 &0 &\vline& 2 \\
\end{pmatrix}
 $
 $$
 \fce{P} = \zs{\zh{2-1 z; 2; z}:z \in \mathbb{R}}
 $$
 \item Když $a\neq 1$:
	 $y=0$
		$$3=x+z=4$$

		Nemá řešení.

\end{enumerate}
\item Jinak
$$\dec 4&1&1\\3&b&1\\4&2b&1\dee = 1-2b $$
$$x=\f{1-2b}{b(1-a)}$$
$$\dec a&4&1\\1&3&1\\1&4b&1\dee = 1-a$$
$$x=\f{1}{b}$$
$$\dec a&1&4\\1&b&3\\1&2b41\dee = -2ab+7b-1$$
$$x=\f{-2ab+7b-1}{b(1-a)}$$

$$\zs{\[\f{1-2b}{b(1-a)};\f{1}{b};\f{-2ab+7b-1}{b(1-a)}\]}$$

\end{enumerate}

\EndDoc
