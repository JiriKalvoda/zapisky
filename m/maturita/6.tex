\providecommand{\HINCLUDE}{NE}
\if ^\HINCLUDE^
\else
\def\HINCLUDE{}
\global\newdimen\Okraje
\global\Okraje =4cm
\input{$HOME/souteze/_hlavicka/h-.tex}

%\definecolor{colorV}{RGB}{255,127,0}
%\definecolor{colorPoz}{RGB}{153,51,0}
%\definecolor{orangeV}{RGB}{255,127,0}
%\definecolor{colorPr}{RGB}{0,5,255}
%\definecolor{colorDef}{RGB}{0.255,0}

\usepackage[shortlabels]{enumitem}
\setlength{\marginparsep}{2pt}
\setlength{\marginparwidth}{35pt}

\def\st{{\rm st}}
\def\P{{\rm P}}

\def\ISENUM{}
\def\inMargin#1{\End
		
		\hskip0pt \marginpar{{{#1}}}}
\newcounter{V}[section] 
\newcommand{\V}[1][]{\stepcounter{V}\inMargin{\textcolor{green}{V.\arabic{section}.\theV.:}}\ifx^#1^\else\textcolor{green}{\underline{{#1}:}}\addcontentsline{toc}{subsubsection}{V.\arabic{section}.\theV.:$\quad$ {#1}}\\\fi}
\def\Def{\inMargin{\textcolor{red}{Def:}}}
\def\Poz{{\inMargin{\textcolor{brown}{Pozn:}}}}
\def\Pr{{\inMargin{\textcolor{blue}{Př:}}}}
\def\Pozenum
{
	\begin{enumerate}[1)]%, left = 0pt ]
		\item\inMargin{\textcolor{brown}{Pozn:}}\def\ISENUM{a}}
\def\End
{
	\if	^\ISENUM^
	\else \end{enumerate}
	\fi
	\def\ISENUM{}
}
\reversemarginpar

\makeatletter
\renewcommand\thesection{§\arabic{section}.}
\renewcommand\thesubsection{\Alph{subsection})}
\renewcommand\thesubsubsection{\alph{subsubsection})}
\newcounter{chapter}
\setcounter{chapter}{0}
\renewcommand\thechapter{\Alph{chapter})}
\newcounter{roman}
\setcounter{roman}{0}
\renewcommand\theroman{\Roman{roman}.}
\makeatother
\def\sectionnum#1{\setcounter{section}{#1}\addtocounter{section}{-1}}
\def\subsectionnum#1{\setcounter{subsection}{#1}\addtocounter{subsection}{-1}}
\def\subsubsectionnum#1{\setcounter{subsubsection}{#1}\addtocounter{subsubsection}{-1}}
\def\chapternum#1{\setcounter{chapter}{#1}\addtocounter{chapter}{-1}}
\def\chapter#1{

	\addtocounter{chapter}{1}\sectionnum{1}
	\addcontentsline{toc}{section}{\large{\thechapter$\quad${#1}}}
	
	{\LARGE  \textbf{\begin{minipage}[t]{0.1\textwidth}\thechapter\end{minipage}\begin{minipage}[t]{0.95\textwidth}#1\end{minipage}}}

}
\def\ROM{}
\def\Rom#1#2{\setcounter{roman}{#1}\renewcommand\ROM{#2}}

\Rom{6}{Funkce}
\title{\Huge\textbf{\theroman\quad \ROM}}
\author{Jiří Kalvoda}

\newcounter{countOfBegin}
\setcounter{countOfBegin}{0}
\newcommand{\BeginDoc}[1][]
{
	\ifnum\value{countOfBegin}=0
	\begin{document}
		#1
		\fi
	\addtocounter{countOfBegin}{1}
		
}
\def\EndDoc
{
	\addtocounter{countOfBegin}{-1}
	\ifnum\value{countOfBegin}=0
	\end{document}
	\fi
}

\fi
\BeginDoc{}
\def\sqn{{\rm sgn}}
\section{Kartézkský součin, binární relace, zobrazení}
\Pr \pd{6-1.pdf}
\begin{tabular}{|c||c|c|c|c|} \hline
	&$y=x-[x]$&$x=[|x|]$&$y=\sgn(\sin x)$&$y=\sqn(\log x)$\vrule width 0pt
	height 14pt depth 8pt\relax\\\hline\hline
	Definiční obor & $\R$ & $\R$ & $\R$ & $\R^+$ \\\hline
	Obor hodnot & $\<0;1\)$ & $\N_0$ & $\zs{0;\pm 1}$ & $\zs{0;\pm 1}$
	\\\hline
	Monotonost & neklesajíci & ne & ne & neklesající \\\hline
	Omezenost & ano & zdola & ano & ano \\\hline
	Minima & 0 & 0 pro $\(-1;1\)$ & $-1$ & $-1$ pro $(0,1)$ \\\hline
	Maxima & 1 & ne  & 1& 1 pro $(1,\infty)$ \\\hline
	Perioda & 1 & ne & $2\pi$ & ne \\\hline
\end{tabular}
\pd{6-1a.pdf}
\Pr \pd[0.4]{6-2.pdf}
\begin{eqnarray*} 
	\f{r+2-r+2}{r-2} &=& \f{3r^2+r+9-3(r-2)^2}{3(r^2-4)^2}\\
	\f{4}{r-2} &=& \f{3r^2+r+9-3r^2-12r+12}{3(r^2-4)^2}\\
	0 &=& \f{3r^2+r+9-3r^2-12r+12-12r-12}{3(r^2-4)^2}\\
	0 &=& \f{9-23r}{3(r^2-4)^2}\\
	r = \f{9}{23}
\end{eqnarray*}
\Pr \pd[0.4]{6-3.pdf}
\begin{eqnarray*} 
	\f{-96}{(x-4)(x+4)} &=& \f{\f{2x-1}x}{\f{x+4}x } - \f{\f{3x-1}{x}}{\f{4-x}{x}} - 5 \\
	\f{-96}{(x-4)(x+4)} &=& \f{2x-1}{x+4} - \f{3x-1}{4-x} - 5 \\
	\f{-96}{x^2-16} &=& \f{8x-4-2x^2+x}{x^2-16} - \f{3x^2-x+12x-4}{x^2-16} - \f{5x^2-80}{x^2-16}\\
	0 &=& \f{-10x^2 -2x+16}{x^2-16}\\
	x &=& \f{1\pm\sqrt{161}}{10}
\end{eqnarray*}
\Pr \pd{6-4.pdf}
\begin{eqnarray*} 
\f{1}{15}-\f 1{x}  &=& \f{1}{60} \\
\f{1}{15}-\f 1{60}&=& \f{1}{x} \\
	\f{3}{60}&=& \f{1}{x} \\
	20&=& x \\
\end{eqnarray*}
\Pr \pd[0.4]{6-5.pdf}
\begin{eqnarray*} 
	2\sqrt{x^2+5x+1} &=& -3x^2-15x+2 \\
	4x^2+20x+4 &=& 9x^4+90x^3+213x^2-60x+4\\
	4x^2+20x+4 &=& 9x^4+90x^3+213x^2-60x+4\\
	0 &=& 9x^4+90x^3+209x^2-80x\\
	0 &=& x (x+5)\(x-\f 13 \)\(x+\f{16}3\)\\
	0 &=& x (x+5)
\end{eqnarray*}
Zkouška evidentně u všech hodnot vyhoví.
\Pr \pd[0.4]{6-6.pdf}
\begin{eqnarray*} 
	25x^2-28x-8 &=& 25x^2-40x+16 \\
	12x -8 &=& 24 \\
	x  &=& 2 \\
\end{eqnarray*}
Zkouška evidentně vyhoví ($6=6$).
\Pr
\begin{eqnarray*} 
	\f{\sqrt{x}+1}{\sqrt{x}-1} &>& 2 \\
\end{eqnarray*}
Když $x> 1$:
\begin{eqnarray*} 
	\sqrt{x}+1 &>& 2\sqrt{x}-2 \\
	3 &>& \sqrt{x} \\
	9 &>& x \\
\end{eqnarray*}
Když $x< 1$:
\begin{eqnarray*} 
	\sqrt{x}+1 &>& -2\sqrt{x}+2 \\
	3\sqrt{x} &>& 1 \\
	x &>& \f 19 \\
\end{eqnarray*}

$$x\in\(-\f 19 ; 9\) -\zs{1}$$




\EndDoc
