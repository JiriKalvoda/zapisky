\providecommand{\HINCLUDE}{NE}
\if ^\HINCLUDE^
\else
\def\HINCLUDE{}
\global\newdimen\Okraje
\global\Okraje =4cm
\input{$HOME/souteze/_hlavicka/h-.tex}

%\definecolor{colorV}{RGB}{255,127,0}
%\definecolor{colorPoz}{RGB}{153,51,0}
%\definecolor{orangeV}{RGB}{255,127,0}
%\definecolor{colorPr}{RGB}{0,5,255}
%\definecolor{colorDef}{RGB}{0.255,0}

\usepackage[shortlabels]{enumitem}
\setlength{\marginparsep}{2pt}
\setlength{\marginparwidth}{35pt}

\def\st{{\rm st}}
\def\P{{\rm P}}

\def\ISENUM{}
\def\inMargin#1{\End
		
		\hskip0pt \marginpar{{{#1}}}}
\newcounter{V}[section] 
\newcommand{\V}[1][]{\stepcounter{V}\inMargin{\textcolor{green}{V.\arabic{section}.\theV.:}}\ifx^#1^\else\textcolor{green}{\underline{{#1}:}}\addcontentsline{toc}{subsubsection}{V.\arabic{section}.\theV.:$\quad$ {#1}}\\\fi}
\def\Def{\inMargin{\textcolor{red}{Def:}}}
\def\Poz{{\inMargin{\textcolor{brown}{Pozn:}}}}
\def\Pr{{\inMargin{\textcolor{blue}{Př:}}}}
\def\Pozenum
{
	\begin{enumerate}[1)]%, left = 0pt ]
		\item\inMargin{\textcolor{brown}{Pozn:}}\def\ISENUM{a}}
\def\End
{
	\if	^\ISENUM^
	\else \end{enumerate}
	\fi
	\def\ISENUM{}
}
\reversemarginpar

\makeatletter
\renewcommand\thesection{§\arabic{section}.}
\renewcommand\thesubsection{\Alph{subsection})}
\renewcommand\thesubsubsection{\alph{subsubsection})}
\newcounter{chapter}
\setcounter{chapter}{0}
\renewcommand\thechapter{\Alph{chapter})}
\newcounter{roman}
\setcounter{roman}{0}
\renewcommand\theroman{\Roman{roman}.}
\makeatother
\def\sectionnum#1{\setcounter{section}{#1}\addtocounter{section}{-1}}
\def\subsectionnum#1{\setcounter{subsection}{#1}\addtocounter{subsection}{-1}}
\def\subsubsectionnum#1{\setcounter{subsubsection}{#1}\addtocounter{subsubsection}{-1}}
\def\chapternum#1{\setcounter{chapter}{#1}\addtocounter{chapter}{-1}}
\def\chapter#1{

	\addtocounter{chapter}{1}\sectionnum{1}
	\addcontentsline{toc}{section}{\large{\thechapter$\quad${#1}}}
	
	{\LARGE  \textbf{\begin{minipage}[t]{0.1\textwidth}\thechapter\end{minipage}\begin{minipage}[t]{0.95\textwidth}#1\end{minipage}}}

}
\def\ROM{}
\def\Rom#1#2{\setcounter{roman}{#1}\renewcommand\ROM{#2}}

\Rom{6}{Funkce}
\title{\Huge\textbf{\theroman\quad \ROM}}
\author{Jiří Kalvoda}

\newcounter{countOfBegin}
\setcounter{countOfBegin}{0}
\newcommand{\BeginDoc}[1][]
{
	\ifnum\value{countOfBegin}=0
	\begin{document}
		#1
		\fi
	\addtocounter{countOfBegin}{1}
		
}
\def\EndDoc
{
	\addtocounter{countOfBegin}{-1}
	\ifnum\value{countOfBegin}=0
	\end{document}
	\fi
}

\fi
\BeginDoc{}
\section{Planimetrie}
\Pr
\pd{4-1.pdf}
Ukáži, že $2  t_a < b+c$:\\
Označme $v$ jako výšku na starnu $a$.
Dále nechť $x,y$ jsou po řadě orientované vzdálenosti od paty výšky z $A$ k $B$ a středu $AB$.
Požadovanou nerovnost pak lze vyjádřit takto:
$2 \sqrt{v^2+\(\f{x+y}{2}\)^2} < \sqrt{v^2+x^2}+\sqrt{v^2+y^2}$\\
$ 4v^2+x^2+y^2+4xy < 2v^2+x^2+y^2+2\sqrt{v^2+x^2}\sqrt{v^2+y^2} $\\
$ 2v^2+4xy < 2\sqrt{v^4+2v^2x^2+v^2y^2+x^2y^2}$\\
$ 2v^2+2x^2y^2+2v^2xy < 2v^4+4v^2x^2+2v^2y^2+2x^2y^2$\\
$ 0 < 2v^2$\\

Požadovanou nerovnst získáme součtem těchto nerovností přes všechny strany.
\emph{QED}
\pd{4-2.pdf}
Nechť $O$ je ortocentrum.
\begin{enumerate}
	\item $s<v_a+v_b+v_c$

		Ukáži, že $a<v_b+v_c$.
		Z trojúhelníkové nerovnosti $|BO|+|CO|<|BC|$. Jelikož je ale trojúhelník ostroúhlý, tak $O$ leží ve výškách, tedy $v_b+v_c<|BO|+|CO|<|BC|$.

		Součtem těchto nerovností přes všechny strany dostaneme:
		$v_b+v_c+v_c+v_a+v_a+v_b < a+b+c \ekv v_a+v_b+v_c < 2s$
	\item $v_a+v_b+v_c < 2s$
		Evidentně $a<v_b$, protože výška je jediná nejkratší spojnicí vrcholu a protější strany.
		A jelikož trojúhelník není pravoúhlý, tak vyška není současně stranou.

		Sečtením těchto nerovností přes starny dostaneme:
		$2s = a+b+c < v_b+v_c+v_a$
\end{enumerate}

\pd{4-3.pdf}
\pd{4-3a.pdf}
$|\angle PBM| = |\angle PBC| = |\angle BPM|$. Tedy trojúhelník $BPM$ je rovnoramenný.
Z toho plyne, že $|BM|=|MP|$.
Analogicky $|PN|=|NC|$.
V součtu tedy $|BM|+|NC|=|MP|+|PN|=|MN|$. \emph{QED}
\pd{4-4.pdf}
\pd{4-4a.pdf}
Dále budeme úhlit orientovaně modulo $\pi$:

Z tětivovosti platí: $\angle CEA = \angle CBA = \angle DBA = \angle DFA$. Tedy přímky $CD$ a $DF$ vzniknou z $EF$ otočením o stejný úhel, tedy musí být rovnoběžné. \emph{QED}
\pd{4-5.pdf}
\pd{4-5a.pdf}

Označme antirovnoběžku k $BA$ vzhledem k $CD$ procházející $A$ jako $x$. Dále nechť $\zs{X}=x\cap\pri{CD}$.

Platí, že $|AX|=|AD|$, protože $X$ vznikne z $D$ dle simetrice podle kolmice $CD$ prcházející $A$ ($\pri{AD}$ je obrazem $pri{AX}$ z antirovnoběžnosti a $DX$ je kolmé na osu.

Jelikož $AC$ a $AX$ jsou antirovnoběžky vzhledem k $CD$ s $BC$ a $BX$, tak $|\angle CAX|=|\angle CBD|$, tady trojúhelník $CAX$ je podobný s $CBD$.

Z podobnosti tedy $AX:AC = AD:AC = DB:BC$. \emph{QED}

Nejsou podobné, protože věta $uss$ nefunguje :-(


\subsection{Rovnice}


\pd{4-6.pdf}
$$r^2 = 15^2+(2x)^2$$
$$r^2 = 17^2+x^2$$

$$0 = 15^2-17^2+3x^2$$

$x^2=\f{64}3$

$r=\sqrt{17^2+\f{64}3} = \sqrt{\f{931}3} = \f{7\sqrt{57}}3$
\pd{4-7.pdf}
$$xu=15$$
$$(x+xu+85)\*(u+1) = 420$$

$$(x+100)\*(u+1) = 420$$
$$15+100u + x + 100 = 420$$
$$ 100 \f{15}x + x  = 305 $$
$$ 1500  + x^2  = 285x $$
$$(x-5)(x-300) = 0$$

$x_1=5 \imp u_1 = 3$\\
$x_2=300 \imp u_2 = \f1{20}$

\pd{4-8.pdf}
\begin{enumerate}
	\item[d)] $ D = (4-6i)^2 - 4 (10-20i) = -60+32 i = (2+8i)^2 $

		$x = \f{-(4-6i)+(2+8i)}{2} = -1+7i $\\
		$x = \f{-(4-6i)-(2+8i)}{2} = -3+i $
	\item[e)]
		$D = (2+i)^2-4(1+i)(3+i) -5-12i = (2-3i)^2$

		$x = \f{2+i+(2-3i)}{2(1+i)} = \f{2-i}{1+i} = \f12-\f32 i$
		$x = \f{2+i-(2-3i)}{2(1+i)} = \f{i}{1+i} = \f12+\f12 i$
\end{enumerate}



\EndDoc
