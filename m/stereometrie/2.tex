\providecommand{\HINCLUDE}{NE}
\if ^\HINCLUDE^
\else
\def\HINCLUDE{}
\global\newdimen\Okraje
\global\Okraje =4cm
\input{$HOME/souteze/_hlavicka/h-.tex}

%\definecolor{colorV}{RGB}{255,127,0}
%\definecolor{colorPoz}{RGB}{153,51,0}
%\definecolor{orangeV}{RGB}{255,127,0}
%\definecolor{colorPr}{RGB}{0,5,255}
%\definecolor{colorDef}{RGB}{0.255,0}

\usepackage[shortlabels]{enumitem}
\setlength{\marginparsep}{2pt}
\setlength{\marginparwidth}{35pt}

\def\st{{\rm st}}
\def\P{{\rm P}}

\def\ISENUM{}
\def\inMargin#1{\End
		
		\hskip0pt \marginpar{{{#1}}}}
\newcounter{V}[section] 
\newcommand{\V}[1][]{\stepcounter{V}\inMargin{\textcolor{green}{V.\arabic{section}.\theV.:}}\ifx^#1^\else\textcolor{green}{\underline{{#1}:}}\addcontentsline{toc}{subsubsection}{V.\arabic{section}.\theV.:$\quad$ {#1}}\\\fi}
\def\Def{\inMargin{\textcolor{red}{Def:}}}
\def\Poz{{\inMargin{\textcolor{brown}{Pozn:}}}}
\def\Pr{{\inMargin{\textcolor{blue}{Př:}}}}
\def\Pozenum
{
	\begin{enumerate}[1)]%, left = 0pt ]
		\item\inMargin{\textcolor{brown}{Pozn:}}\def\ISENUM{a}}
\def\End
{
	\if	^\ISENUM^
	\else \end{enumerate}
	\fi
	\def\ISENUM{}
}
\reversemarginpar

\makeatletter
\renewcommand\thesection{§\arabic{section}.}
\renewcommand\thesubsection{\Alph{subsection})}
\renewcommand\thesubsubsection{\alph{subsubsection})}
\newcounter{chapter}
\setcounter{chapter}{0}
\renewcommand\thechapter{\Alph{chapter})}
\newcounter{roman}
\setcounter{roman}{0}
\renewcommand\theroman{\Roman{roman}.}
\makeatother
\def\sectionnum#1{\setcounter{section}{#1}\addtocounter{section}{-1}}
\def\subsectionnum#1{\setcounter{subsection}{#1}\addtocounter{subsection}{-1}}
\def\subsubsectionnum#1{\setcounter{subsubsection}{#1}\addtocounter{subsubsection}{-1}}
\def\chapternum#1{\setcounter{chapter}{#1}\addtocounter{chapter}{-1}}
\def\chapter#1{

	\addtocounter{chapter}{1}\sectionnum{1}
	\addcontentsline{toc}{section}{\large{\thechapter$\quad${#1}}}
	
	{\LARGE  \textbf{\begin{minipage}[t]{0.1\textwidth}\thechapter\end{minipage}\begin{minipage}[t]{0.95\textwidth}#1\end{minipage}}}

}
\def\ROM{}
\def\Rom#1#2{\setcounter{roman}{#1}\renewcommand\ROM{#2}}

\Rom{6}{Funkce}
\title{\Huge\textbf{\theroman\quad \ROM}}
\author{Jiří Kalvoda}

\newcounter{countOfBegin}
\setcounter{countOfBegin}{0}
\newcommand{\BeginDoc}[1][]
{
	\ifnum\value{countOfBegin}=0
	\begin{document}
		#1
		\fi
	\addtocounter{countOfBegin}{1}
		
}
\def\EndDoc
{
	\addtocounter{countOfBegin}{-1}
	\ifnum\value{countOfBegin}=0
	\end{document}
	\fi
}

\fi
\let\braceru=\relax \let\bracelu=\relax 
\def\o#1{\setbox0=
	\hbox{$\kern2pt\overbrace{\kern-2pt#1\kern-2pt}\kern2pt$}\ht0=2.1ex\box0}
\def\to#1{\hbox{#1\rlap{\t{}}}}
\def\rad{\rm{rad}}
\def\f{\frac}
\BeginDoc{}
\section{Dvě přímky v rovině}
\Pr Je dána krychle $ABCDEFGH$, zobrazte ji ve volném rovnoběžném promítání a určete průniky přímek: $\pri{AC}$ a $\pri{BD}$, $\pri{EH}$ a $\pri{BC}$, $\pri{EF}$ a $\pri{BG}$.
% todo

$\pri{AC} \cap \pri{BD} = \{S\}$ -- jedná se o různoběžky.\\
$\pri{EH} \cap \pri{BG} = \{\}$ -- jedná se o různé rovnoběžky.\\
$\pri{EF} \cap \pri{BG} = \{\}$ -- jedná se o mimoběžky.\\

\Def Nechť $p,q \in P$ jsou dvě přimky. Jestliže platí:
\begin{enumerate}
	\item $p\cap q=\emptyset \land p,q$ jsou komplanární $\imp$ \emph{různé rovnoběžky}
	\item $p\cap q=\emptyset \land p,q$ jsou nekomplanární $\imp$ \emph{mimoběžky}
	\item $p\cap q = \{P\}$ $\imp$ \emph {různoběžky} a $P$ je \emph{průsečík}
	\item $p\cap q = p$ $\imp$ \emph{splívající (totožné) rovnoběžky}
\end{enumerate}

\begin{description}
	\item[$A_4$] Axiom rovnoběžnosti:\\
		Každým bodem v $\E_2$ lze vést ke každé přímce právě jednu rovnoběžku.

\end{description}
\V Každým bodem v $\E_3$ lze vést ke každé přimce právě jednu rovnoběžku.

\V[Tranzitivnost rovnoběžek]
$\forall a,b,c \subset \E_3 , a,b,c \in P: a || b \land b || c  \imp c||a$

Důsledek: Všechny přimky rovnoběžné s danou přímku jsou navzájem rovnoběžné a vytvářejí tzv. \emph{směr}.

\EndDoc
