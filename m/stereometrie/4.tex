\providecommand{\HINCLUDE}{NE}
\if ^\HINCLUDE^
\else
\def\HINCLUDE{}
\global\newdimen\Okraje
\global\Okraje =4cm
\input{$HOME/souteze/_hlavicka/h-.tex}

%\definecolor{colorV}{RGB}{255,127,0}
%\definecolor{colorPoz}{RGB}{153,51,0}
%\definecolor{orangeV}{RGB}{255,127,0}
%\definecolor{colorPr}{RGB}{0,5,255}
%\definecolor{colorDef}{RGB}{0.255,0}

\usepackage[shortlabels]{enumitem}
\setlength{\marginparsep}{2pt}
\setlength{\marginparwidth}{35pt}

\def\st{{\rm st}}
\def\P{{\rm P}}

\def\ISENUM{}
\def\inMargin#1{\End
		
		\hskip0pt \marginpar{{{#1}}}}
\newcounter{V}[section] 
\newcommand{\V}[1][]{\stepcounter{V}\inMargin{\textcolor{green}{V.\arabic{section}.\theV.:}}\ifx^#1^\else\textcolor{green}{\underline{{#1}:}}\addcontentsline{toc}{subsubsection}{V.\arabic{section}.\theV.:$\quad$ {#1}}\\\fi}
\def\Def{\inMargin{\textcolor{red}{Def:}}}
\def\Poz{{\inMargin{\textcolor{brown}{Pozn:}}}}
\def\Pr{{\inMargin{\textcolor{blue}{Př:}}}}
\def\Pozenum
{
	\begin{enumerate}[1)]%, left = 0pt ]
		\item\inMargin{\textcolor{brown}{Pozn:}}\def\ISENUM{a}}
\def\End
{
	\if	^\ISENUM^
	\else \end{enumerate}
	\fi
	\def\ISENUM{}
}
\reversemarginpar

\makeatletter
\renewcommand\thesection{§\arabic{section}.}
\renewcommand\thesubsection{\Alph{subsection})}
\renewcommand\thesubsubsection{\alph{subsubsection})}
\newcounter{chapter}
\setcounter{chapter}{0}
\renewcommand\thechapter{\Alph{chapter})}
\newcounter{roman}
\setcounter{roman}{0}
\renewcommand\theroman{\Roman{roman}.}
\makeatother
\def\sectionnum#1{\setcounter{section}{#1}\addtocounter{section}{-1}}
\def\subsectionnum#1{\setcounter{subsection}{#1}\addtocounter{subsection}{-1}}
\def\subsubsectionnum#1{\setcounter{subsubsection}{#1}\addtocounter{subsubsection}{-1}}
\def\chapternum#1{\setcounter{chapter}{#1}\addtocounter{chapter}{-1}}
\def\chapter#1{

	\addtocounter{chapter}{1}\sectionnum{1}
	\addcontentsline{toc}{section}{\large{\thechapter$\quad${#1}}}
	
	{\LARGE  \textbf{\begin{minipage}[t]{0.1\textwidth}\thechapter\end{minipage}\begin{minipage}[t]{0.95\textwidth}#1\end{minipage}}}

}
\def\ROM{}
\def\Rom#1#2{\setcounter{roman}{#1}\renewcommand\ROM{#2}}

\Rom{6}{Funkce}
\title{\Huge\textbf{\theroman\quad \ROM}}
\author{Jiří Kalvoda}

\newcounter{countOfBegin}
\setcounter{countOfBegin}{0}
\newcommand{\BeginDoc}[1][]
{
	\ifnum\value{countOfBegin}=0
	\begin{document}
		#1
		\fi
	\addtocounter{countOfBegin}{1}
		
}
\def\EndDoc
{
	\addtocounter{countOfBegin}{-1}
	\ifnum\value{countOfBegin}=0
	\end{document}
	\fi
}

\fi
\let\braceru=\relax \let\bracelu=\relax 
\def\o#1{\setbox0=
	\hbox{$\kern2pt\overbrace{\kern-2pt#1\kern-2pt}\kern2pt$}\ht0=2.1ex\box0}
\def\to#1{\hbox{#1\rlap{\t{}}}}
\def\rad{\rm{rad}}
\def\f{\frac}
\BeginDoc{\sectionnum{4}}
\section{Přímka a rovina}
\Poz Klasifikaci vzájemné polohy přímky a roviny provedeme podle jejich společňého průniku:
\begin{itemize}[a)]
	\item prázdná množina
	\item jednoprvková množina
	\item alespoň dva prvky, tedy celá přimka
\end{itemize}
\Def
Nechť $a\in\E_3,\alpha\subset \E_3$. Jestliže platí:
\begin{itemize}[a)]
	\item $a\cap \alpha = \emptyset$ $\imp$ Přímka $a$ je s rovinou $\alpha$ rovnoběžná.
	\item $a\cap \alpha = \{P\}$ $\imp$ Přímka $a$ je s rovinou $\alpha$ různoběžná, bod $P$ je průsečíkem.
	\item $a\cap \alpha = a$ $\imp$ Přímka $a$ leží v rovině $a$ ($a\subset\alpha$).
\end{itemize}

\Poz Je-li $a\in \alpha$, pokládáme přímku $a$ též za rovnoběžnou s rovinou $\alpha$

\V[Kritérium povnoběžnosti přímky a roviny]
Pro $\forall p \in P$ a $\forall \rho \in \E_3$ platí: $p\parallel\rho \ekv \exists q \in \rho: p \parallel q$

[Dk: 
\begin{enumerate}[1.]
	\item \uv{$\imp$}: 
\end{enumerate}
]

\V  Nechť $\forall p \in P, \forall \rho \subset \E_3 : p \parallel \rho \imp \forall \sigma \subset \E_3 : p \subset \sigma \nparallel \rho$: Pak rovina $\sigma$ protnw rovinu $\rho $ v průsečnici $q$, kde platí $p\parallel q$.

\Pr Setrojte průsečík přímky $\pri{PQ}$ a krychle $ABCDEFGH$, kde $P\in\pri{DB}$za bodem $B$, $Q\in \pri{DH}$ za bodem $H$.

\Poz Obecný způsob při stanovení průniku přímky $p$ a roviny $\rho$:
\begin{itemize}
	\item Přímkou $p$ proložíme libovolnou rovinu $\phi,\phi \nparallel \rho$.
	\item Sestrojíme přůsečnici obou rovin $q$.
	\item Průsečíkem $p\cap \phi$ je průsečík $p$ a $q$ (pokud existuje).\\
\end{itemize}


\Pr Je dána krychle $ABCDEFGH$, na jejíh hranách body $R,S,T$. Určete $\pri{FD}\cap\pri{RST}$
%TODO 2. načrt
\begin{itemize}
	\item $\phi = \pri{DBF}; DF\subset\pri{DBF}$
	\item $q=\pri{TX}; \pri{TX} \subset \pri{RST} \cap \phi; X\in \pri{DB} \cap \pri{RS}$
	\item $Y;Y\in \pri{TX}\cap \pri{DF} = \pri{DF} \cap \pri{RST}$
\end{itemize}

\Pr Dú: šestrojte průnik příkmy $\pri{PQ}$ a krychle $ABCDEFGH$:
\begin{itemize}
	\item $H=$ střed $DP$ a $B=$ střed $DQ$.
	\item $P\in\ppri{S_2SS_1}, Q\in\ppri{S_1S_3}$, kde $S_1,S_2,S_3$ jsou po řadě středy Hseček $AD,BC,BG$,
\end{itemize}
\EndDoc

