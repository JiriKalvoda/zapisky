\providecommand{\HINCLUDE}{NE}
\if ^\HINCLUDE^
\else
\def\HINCLUDE{}
\global\newdimen\Okraje
\global\Okraje =4cm
\input{$HOME/souteze/_hlavicka/h-.tex}

%\definecolor{colorV}{RGB}{255,127,0}
%\definecolor{colorPoz}{RGB}{153,51,0}
%\definecolor{orangeV}{RGB}{255,127,0}
%\definecolor{colorPr}{RGB}{0,5,255}
%\definecolor{colorDef}{RGB}{0.255,0}

\usepackage[shortlabels]{enumitem}
\setlength{\marginparsep}{2pt}
\setlength{\marginparwidth}{35pt}

\def\st{{\rm st}}
\def\P{{\rm P}}

\def\ISENUM{}
\def\inMargin#1{\End
		
		\hskip0pt \marginpar{{{#1}}}}
\newcounter{V}[section] 
\newcommand{\V}[1][]{\stepcounter{V}\inMargin{\textcolor{green}{V.\arabic{section}.\theV.:}}\ifx^#1^\else\textcolor{green}{\underline{{#1}:}}\addcontentsline{toc}{subsubsection}{V.\arabic{section}.\theV.:$\quad$ {#1}}\\\fi}
\def\Def{\inMargin{\textcolor{red}{Def:}}}
\def\Poz{{\inMargin{\textcolor{brown}{Pozn:}}}}
\def\Pr{{\inMargin{\textcolor{blue}{Př:}}}}
\def\Pozenum
{
	\begin{enumerate}[1)]%, left = 0pt ]
		\item\inMargin{\textcolor{brown}{Pozn:}}\def\ISENUM{a}}
\def\End
{
	\if	^\ISENUM^
	\else \end{enumerate}
	\fi
	\def\ISENUM{}
}
\reversemarginpar

\makeatletter
\renewcommand\thesection{§\arabic{section}.}
\renewcommand\thesubsection{\Alph{subsection})}
\renewcommand\thesubsubsection{\alph{subsubsection})}
\newcounter{chapter}
\setcounter{chapter}{0}
\renewcommand\thechapter{\Alph{chapter})}
\newcounter{roman}
\setcounter{roman}{0}
\renewcommand\theroman{\Roman{roman}.}
\makeatother
\def\sectionnum#1{\setcounter{section}{#1}\addtocounter{section}{-1}}
\def\subsectionnum#1{\setcounter{subsection}{#1}\addtocounter{subsection}{-1}}
\def\subsubsectionnum#1{\setcounter{subsubsection}{#1}\addtocounter{subsubsection}{-1}}
\def\chapternum#1{\setcounter{chapter}{#1}\addtocounter{chapter}{-1}}
\def\chapter#1{

	\addtocounter{chapter}{1}\sectionnum{1}
	\addcontentsline{toc}{section}{\large{\thechapter$\quad${#1}}}
	
	{\LARGE  \textbf{\begin{minipage}[t]{0.1\textwidth}\thechapter\end{minipage}\begin{minipage}[t]{0.95\textwidth}#1\end{minipage}}}

}
\def\ROM{}
\def\Rom#1#2{\setcounter{roman}{#1}\renewcommand\ROM{#2}}

\Rom{6}{Funkce}
\title{\Huge\textbf{\theroman\quad \ROM}}
\author{Jiří Kalvoda}

\newcounter{countOfBegin}
\setcounter{countOfBegin}{0}
\newcommand{\BeginDoc}[1][]
{
	\ifnum\value{countOfBegin}=0
	\begin{document}
		#1
		\fi
	\addtocounter{countOfBegin}{1}
		
}
\def\EndDoc
{
	\addtocounter{countOfBegin}{-1}
	\ifnum\value{countOfBegin}=0
	\end{document}
	\fi
}

\fi
\let\braceru=\relax \let\bracelu=\relax 
\def\o#1{\setbox0=
	\hbox{$\kern2pt\overbrace{\kern-2pt#1\kern-2pt}\kern2pt$}\ht0=2.1ex\box0}
\def\to#1{\hbox{#1\rlap{\t{}}}}
\def\rad{\rm{rad}}
\def\f{\frac}
\BeginDoc{\sectionnum{9}}

\Pr Vyřešte rovnici:
\begin{eqnarray*}
	3 \cos 2x + \cos x &=& 1-4'sin^2 x
\end{eqnarray*}

\Pr Vyřešte rovnici:
\begin{eqnarray*}
	\sin x + \sin 2x = \tg x
\end{eqnarray*}
\section{Vzdálenost}
\Def Nechť $A,B \in \E_3$. \emph{Vzdáleností dvou bodů} A,B nazýváme délku úsečky AB a označujeme ji $\rho(A,B)$.
\Poz
Vzdálenost bodů $A,B$ je tedy reálné číslo $\rho(A,B) = |AB|$.
\Poz
Vzdálenost $\rho(A,B)$ můžeme považovat za zobrazení $\rho: \E_3\times\E_3 \rightarrow \R$, které má vlastnosti:
$\forall A,B,C \in \E_3:$
\begin{enumerate}
	\item $\rho(A,B) \ge 0$, přičemž $\rho(A,B) = 0 \ekv A=B$
	\item $\rho(A,B)=\rho(B,A)$
	\item $\rho(A,B) + \rho(B,C) \ge \rho(A,C)$, přičemž rovnost nastává $\ekv$ $B\in AC$
\end{enumerate}
\Poz Uvedené vlastnosti se používají při axiomatické definici vzdálenosti.
\Def
Necť $A\in\E_3$ je bod $\alpha\subset \E_3$ je rovina.\\
\emph{Kolmým průmětem bodu $A$ do roviny} $\alpha$ nazýváme bod $A_0$ definovaný takto:
\begin{itemize}
\item $A\in\alpha \imp A_0=A$
\item $A\not\in\alpha \imp A_0\imp\cap\alpha, p\perp\alpha , A\in p$
\end{itemize}
\V Nechť $A\in\E_3$ je bod, $\alpha\subset\E_3$ je rovina. Pak platí:
$\rho (A,\alpha) = \min\{\rho{A,X},X\in\alpha\}$

\Pr Vypočtěte vzdálenost $V$ od podstavy pravidelného čtyřbokého jehlanu $ABCDV$, je li $|AB|=a, |\angle VAB| = \f \pi 3$:

$\f i{a\sqrt 2} 2 $

\Def
Necť $A\in\E_3$ je bod $p\subset \E_3$ je přímka.\\
\emph{Kolmým průmětem bodu $A$ na přímku} $p$ nazýváme bod $A_0$ definovaný takto:
\begin{itemize}
\item $A\in p \imp A_0=A$
\item $A\not\in p \imp A_0\in p\cap\alpha, p\perp\alpha , A\in p$
\end{itemize}
\V Nechť $A\in\E_3$ je bod, $p \subset\E_3$ je přímmka. Pak platí:
$\rho (A,\alpha) = \min\{\rho{A,X},X\in p\}$

\Pr Vypočítejte vzdálenost bodu $A$ od přímky $VC$ v pravidelném čtyřbokém jehlanu $ABCV$, je li $|AB|=a$, $|AV|=s$


\V Nechť $\alpha,\beta \subset \E_3$ jsou dvě rovnoběžné roviny.\\
Pak platí: $\forall A,B \in \alpha: \rho(A,\beta) = \rho(B,\beta)$
\Def Nechť $\alpha,\beta \subset \E_3$ jsou dvě rovnoběžné roviny.
Pak \emph{vzdáleností dvou rovnoběžných rovin} $\alpha,\beta)$ definujeme takto: $\rho(\alpha,\beta)$, $A\alpha$ je libovolný bod.
\Poz
\begin{itemize}
	\item $\alpha = \beta \imp \rho(\alpha,\beta) = 0 $
	\item Vzdálenosti různoběžných rovin klademe hodnotu 0.
\end{itemize}
\EndDoc


