\providecommand{\HINCLUDE}{NE}
\if ^\HINCLUDE^
\else
\def\HINCLUDE{}
\global\newdimen\Okraje
\global\Okraje =4cm
\input{$HOME/souteze/_hlavicka/h-.tex}

%\definecolor{colorV}{RGB}{255,127,0}
%\definecolor{colorPoz}{RGB}{153,51,0}
%\definecolor{orangeV}{RGB}{255,127,0}
%\definecolor{colorPr}{RGB}{0,5,255}
%\definecolor{colorDef}{RGB}{0.255,0}

\usepackage[shortlabels]{enumitem}
\setlength{\marginparsep}{2pt}
\setlength{\marginparwidth}{35pt}

\def\st{{\rm st}}
\def\P{{\rm P}}

\def\ISENUM{}
\def\inMargin#1{\End
		
		\hskip0pt \marginpar{{{#1}}}}
\newcounter{V}[section] 
\newcommand{\V}[1][]{\stepcounter{V}\inMargin{\textcolor{green}{V.\arabic{section}.\theV.:}}\ifx^#1^\else\textcolor{green}{\underline{{#1}:}}\addcontentsline{toc}{subsubsection}{V.\arabic{section}.\theV.:$\quad$ {#1}}\\\fi}
\def\Def{\inMargin{\textcolor{red}{Def:}}}
\def\Poz{{\inMargin{\textcolor{brown}{Pozn:}}}}
\def\Pr{{\inMargin{\textcolor{blue}{Př:}}}}
\def\Pozenum
{
	\begin{enumerate}[1)]%, left = 0pt ]
		\item\inMargin{\textcolor{brown}{Pozn:}}\def\ISENUM{a}}
\def\End
{
	\if	^\ISENUM^
	\else \end{enumerate}
	\fi
	\def\ISENUM{}
}
\reversemarginpar

\makeatletter
\renewcommand\thesection{§\arabic{section}.}
\renewcommand\thesubsection{\Alph{subsection})}
\renewcommand\thesubsubsection{\alph{subsubsection})}
\newcounter{chapter}
\setcounter{chapter}{0}
\renewcommand\thechapter{\Alph{chapter})}
\newcounter{roman}
\setcounter{roman}{0}
\renewcommand\theroman{\Roman{roman}.}
\makeatother
\def\sectionnum#1{\setcounter{section}{#1}\addtocounter{section}{-1}}
\def\subsectionnum#1{\setcounter{subsection}{#1}\addtocounter{subsection}{-1}}
\def\subsubsectionnum#1{\setcounter{subsubsection}{#1}\addtocounter{subsubsection}{-1}}
\def\chapternum#1{\setcounter{chapter}{#1}\addtocounter{chapter}{-1}}
\def\chapter#1{

	\addtocounter{chapter}{1}\sectionnum{1}
	\addcontentsline{toc}{section}{\large{\thechapter$\quad${#1}}}
	
	{\LARGE  \textbf{\begin{minipage}[t]{0.1\textwidth}\thechapter\end{minipage}\begin{minipage}[t]{0.95\textwidth}#1\end{minipage}}}

}
\def\ROM{}
\def\Rom#1#2{\setcounter{roman}{#1}\renewcommand\ROM{#2}}

\Rom{6}{Funkce}
\title{\Huge\textbf{\theroman\quad \ROM}}
\author{Jiří Kalvoda}

\newcounter{countOfBegin}
\setcounter{countOfBegin}{0}
\newcommand{\BeginDoc}[1][]
{
	\ifnum\value{countOfBegin}=0
	\begin{document}
		#1
		\fi
	\addtocounter{countOfBegin}{1}
		
}
\def\EndDoc
{
	\addtocounter{countOfBegin}{-1}
	\ifnum\value{countOfBegin}=0
	\end{document}
	\fi
}

\fi
\let\braceru=\relax \let\bracelu=\relax 
\def\o#1{\setbox0=
	\hbox{$\kern2pt\overbrace{\kern-2pt#1\kern-2pt}\kern2pt$}\ht0=2.1ex\box0}
\def\to#1{\hbox{#1\rlap{\t{}}}}
\def\rad{\rm{rad}}
\def\f{\frac}
\BeginDoc{\sectionnum{15}}
\section{Hranol, Válec}
\Def
Hranol
Mějme v prostoru rovinu $\rho$, v ní konvexní
mnohoúhelník $A_1 A_2 A_3 \dots A_n$ a nechť $A_1'$ je bod,
který v rovině $\rho$ neleží. Nechť $T : E_3 \rightarrow E_3$ je
takové posunutí, že $A_1' =T(A_1 )$. Při tomto
zobrazení se rovina $\rho$ zobrazí na rovinu $\rho'$ ,
tyto dvě roviny jsou rovnoběžné. Množinu
všech bodů $X$, všech úseček $BB'$ takových, že
$B \subset A_1 A_2 A_3 \dots A_n$ a $B'$ je obraz bodu $B$ v posunutí
$T$, nazýváme hranolem.
Mnohoúhelníky $A_1 A_2 A_3 ... A_n$ a $A_1' A_2' A_3' \dots A_n'$
nazýváme podstavami, rovnoběžníky $A_i A_{i+1}  A_{i+1}' A_i'$, kde ( $i \in \{ 1 , 2 ,\dots, n \}, n+1 \rightarrow 1 $).
nazýváme bočními stěnami hranolu. Všechny boční stěny tvoří plášť hranolu. Podstavy
spolu s bočními stěnami tvoří stěny hranolu. Úsečky $A_i A_i'$ (respektive $A_i A_{i+1}$ a $A_i'A_{i+1}'$ ) se nazývají
boční (respektive podstavné) hrany hranolu. Body $A_1 , A_2 , A_3 ,\dots, A_n$ a $A_1' , A_2' , A_3' ,\dots, A_n'$
se nazývají vrcholy.

\Poz
Podle hodnoty $n$ rozlišujeme hranoly na trojboký, čtyřboký, ...a $n$-boký hranol.

\Def
Je-li směr posunutí kolmý k rovině podstavy, mluvíme o hranolu kolmém, jinak jde o
hranol kosý. Kolmý hranol, jehož podstavami jsou pravidelné n-úhelníky, se nazývá
pravidelný
n - boký hranol. Hranol, jehož podstavy jsou rovnoběžníky, se nazývá rovnoběžnostěn.
Rovnoběžnostěn, jehož všechny stěny jsou pravoúhelníky (resp. čtverce), nazýváme kvádr
(respektive krychle).

\Poz
Čtyři zřejmé vlastnosti objemu $V(T)$ tělesa $T$:
\begin{enumerate}
	\item Dvě shodná tělesa mají tentýž objem.
	\item Skládá-li se těleso $T$ z nepřekrývajících se těles $T_1 ,T_2$, je objem tělesa $T$ součtem objemů
těles $T_1 ,T_2 : V(T)=V(T_1 )+V(T_2 )$.
\item Za jednotku objemu bereme objem krychle o hraně délky 1.
\item Cavalieriho princip:
Nechť tělesa $T_1 ,T_2$ leží mezi dvěma rovnoběžnými rovinami $\alpha_1$ ,$\alpha_2$ a každá rovina $\rho$
rovnoběžná s rovinami $\rho_1 , \rho_2$ protne tělesa $T_1 ,T_2$ v konvexních rovinných útvarech
s obsahy $P_1 , P_2$. Jestliže pro každou rovinu $\rho$ platí, že $P_1 = P_2$ , mají tělesa $T_1 ,T_2$ stejný
objem.
Jestliže pro každou rovinu $\rho$ , platí, že $P_1 = m\*P_2$ , kde $m$ je pevné číslo, nezávislé na
volbě roviny $\rho$, je objem tělesa $T_1$-násobkem objemu tělesa $T_2$: $V ( T_1 ) = m \* V ( T_2 )$
\end{enumerate}

\V Nechť P je obsah podstavy a Q je obsah pláště, v je vzdálenost rovin obou podstav, o je
obvod n-úhelníku podstavy, pak pro objem V a povch S hranolu platí:
V  P  v , S  2 P  Q (-pro kaţdý hranol, pro kolmý hranol: S  2 P  o  v ).

\Def Válec
Mějme v prostoru rovinu  , v ní kruh
K ohraničený kruţnicí k, na kruţnici
k leţí bod A a nechť A ́ je bod, který
v rovině  neleţí. Nechť T : E 3  E 3 je
takové posunutí, ţe A ́=T(A). Označme
T(  )    ́, T ( k )  k  ́, T ( K )  K  ́.
Všechny body všech úseček XX ́, kde
X  K a X ́je obraz bodu X, vytvoří
válec. Omezíme-li se pouze na body X
leţící na kruţnici k, dostaneme plášť
válce. Kruhy K, K ́tvoří podstavy válce. Je-li směr posunutí T kolmý k rovině  , mluvíme
o kolmém válci, v opačném případě o kosém válci.

\Pr
Vypočtěte obsah podstavy prav. $n$-bokého hranolu, o objemu $V$ v němž je výška stejná jako délka podstavné hrany ($a$).

	

\EndDoc


