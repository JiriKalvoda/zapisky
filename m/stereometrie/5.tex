\providecommand{\HINCLUDE}{NE}
\if ^\HINCLUDE^
\else
\def\HINCLUDE{}
\global\newdimen\Okraje
\global\Okraje =4cm
\input{$HOME/souteze/_hlavicka/h-.tex}

%\definecolor{colorV}{RGB}{255,127,0}
%\definecolor{colorPoz}{RGB}{153,51,0}
%\definecolor{orangeV}{RGB}{255,127,0}
%\definecolor{colorPr}{RGB}{0,5,255}
%\definecolor{colorDef}{RGB}{0.255,0}

\usepackage[shortlabels]{enumitem}
\setlength{\marginparsep}{2pt}
\setlength{\marginparwidth}{35pt}

\def\st{{\rm st}}
\def\P{{\rm P}}

\def\ISENUM{}
\def\inMargin#1{\End
		
		\hskip0pt \marginpar{{{#1}}}}
\newcounter{V}[section] 
\newcommand{\V}[1][]{\stepcounter{V}\inMargin{\textcolor{green}{V.\arabic{section}.\theV.:}}\ifx^#1^\else\textcolor{green}{\underline{{#1}:}}\addcontentsline{toc}{subsubsection}{V.\arabic{section}.\theV.:$\quad$ {#1}}\\\fi}
\def\Def{\inMargin{\textcolor{red}{Def:}}}
\def\Poz{{\inMargin{\textcolor{brown}{Pozn:}}}}
\def\Pr{{\inMargin{\textcolor{blue}{Př:}}}}
\def\Pozenum
{
	\begin{enumerate}[1)]%, left = 0pt ]
		\item\inMargin{\textcolor{brown}{Pozn:}}\def\ISENUM{a}}
\def\End
{
	\if	^\ISENUM^
	\else \end{enumerate}
	\fi
	\def\ISENUM{}
}
\reversemarginpar

\makeatletter
\renewcommand\thesection{§\arabic{section}.}
\renewcommand\thesubsection{\Alph{subsection})}
\renewcommand\thesubsubsection{\alph{subsubsection})}
\newcounter{chapter}
\setcounter{chapter}{0}
\renewcommand\thechapter{\Alph{chapter})}
\newcounter{roman}
\setcounter{roman}{0}
\renewcommand\theroman{\Roman{roman}.}
\makeatother
\def\sectionnum#1{\setcounter{section}{#1}\addtocounter{section}{-1}}
\def\subsectionnum#1{\setcounter{subsection}{#1}\addtocounter{subsection}{-1}}
\def\subsubsectionnum#1{\setcounter{subsubsection}{#1}\addtocounter{subsubsection}{-1}}
\def\chapternum#1{\setcounter{chapter}{#1}\addtocounter{chapter}{-1}}
\def\chapter#1{

	\addtocounter{chapter}{1}\sectionnum{1}
	\addcontentsline{toc}{section}{\large{\thechapter$\quad${#1}}}
	
	{\LARGE  \textbf{\begin{minipage}[t]{0.1\textwidth}\thechapter\end{minipage}\begin{minipage}[t]{0.95\textwidth}#1\end{minipage}}}

}
\def\ROM{}
\def\Rom#1#2{\setcounter{roman}{#1}\renewcommand\ROM{#2}}

\Rom{6}{Funkce}
\title{\Huge\textbf{\theroman\quad \ROM}}
\author{Jiří Kalvoda}

\newcounter{countOfBegin}
\setcounter{countOfBegin}{0}
\newcommand{\BeginDoc}[1][]
{
	\ifnum\value{countOfBegin}=0
	\begin{document}
		#1
		\fi
	\addtocounter{countOfBegin}{1}
		
}
\def\EndDoc
{
	\addtocounter{countOfBegin}{-1}
	\ifnum\value{countOfBegin}=0
	\end{document}
	\fi
}

\fi
\let\braceru=\relax \let\bracelu=\relax 
\def\o#1{\setbox0=
	\hbox{$\kern2pt\overbrace{\kern-2pt#1\kern-2pt}\kern2pt$}\ht0=2.1ex\box0}
\def\to#1{\hbox{#1\rlap{\t{}}}}
\def\rad{\rm{rad}}
\def\f{\frac}
\BeginDoc{\sectionnum{5}}
\section{Tři různé roviny}
\V[Tranzitivnost rovnoběžnosti rovin]
$$\forall \alpha,\beta,\gamma\subset\E_3:  \alpha || \beta \land \beta || \gamma \imp \alpha || \gamma$$

\V[Vzájemná poloha 3 různých rovin]
Nechť $\alpha,\beta,\gamma\subset \E_3$ jsou tři navzájem různé roviny, pak jejich  vzájemná poloha v prostoru je jedním z následujících 5 typů:
\begin{itemize}
	\item $\alpha||\beta||\gamma$ 
	\item Některé 2 roviny (BÚNO $\alpha,\beta$ jsou různoběžné)\\
		$a=\alpha\cap\beta$
		\begin{itemize}
			\item $a\cap \gamma = \emptyset$
				\begin{itemize}
					\item $\alpha || \gamma$ -- dvě různoběžné roviny protínají třetí.
					\item $\alpha \not || \gamma$ -- rovnoběžně průsečnice.
				\end{itemize}
			\item $a\cap\gamma = {P}$ -- trs.
			\item $a\cap \gamma = a$ -- společná přímka
		\end{itemize}
\end{itemize}

\Poz Dvě rovnoběžné roviny protíná třetí v dvou rovnoběžných přímkách.
\Poz Jsou li každé 2 ze 3 rovin různoběžné, a mají li tyto 3 roviny jediný společný bod, procházejí tímto společným bodem všechny jejich průsečnice.
\Pr Sestrojte řez krychle $ABCDEFGH$ rovinou $\rho = \pri{UVW}$, kde $V=$ střed $AE$, $W$ je střed $AB$ a $U$ leží v třetině $GC$.
\begin{enumerate}
	\item $VW$ leží ve stěně
	\item $\pri{ABFE} \parallel \pri{DCGH} =imp UZ \parallel VW$\\
		$Z;Z\in GH \land UZ \parallel VW$
	\item $\text{přední}\cap\text{pravá}\cap \rho = \{P_1\}$ \\
		$P_1 \in \pri{VW} \cap \pri{BF}\imp \rho \cap \text{pravá} = \pri{XU}\imp X;X\in\pri{P_1U}\cap BC$
	\item $WX$
	\item $Y; Y \in EH \land VY \parallel XY$
		(dk. viz 2)
	\item $YZ$
	\item 6-úhelník $VWXUZY$
\Pr Druhé řešení pomocí průsečnic:
\Pr Sestrojte řez krychle rovinou $\pri{PQR}$. Body volte dle obrázku:
\Pr Sestrojte řez pravidelného čtyřstěného jehlanu určenou $p;p\parallel AC;L\in p$, kde $L$ je středem $DV$ a bodem $K$, kde $K$ je střed $DV$.
\Pr Řez pravidelného čtyřbokého jehlanu $ABCDV$ rovinou $\rho = \pri{PQR}$, kde:\\
		$P=\text{středem} AV$\\
		$Q \in BV \land |BQ| : |QV| = 1:5$\\
		$R\in CV \cap |CR|:|RV| = 1:3 $
\end{enumerate}

\EndDoc

