\providecommand{\HINCLUDE}{NE}
\if ^\HINCLUDE^
\else
\def\HINCLUDE{}
\global\newdimen\Okraje
\global\Okraje =4cm
\input{$HOME/souteze/_hlavicka/h-.tex}

%\definecolor{colorV}{RGB}{255,127,0}
%\definecolor{colorPoz}{RGB}{153,51,0}
%\definecolor{orangeV}{RGB}{255,127,0}
%\definecolor{colorPr}{RGB}{0,5,255}
%\definecolor{colorDef}{RGB}{0.255,0}

\usepackage[shortlabels]{enumitem}
\setlength{\marginparsep}{2pt}
\setlength{\marginparwidth}{35pt}

\def\st{{\rm st}}
\def\P{{\rm P}}

\def\ISENUM{}
\def\inMargin#1{\End
		
		\hskip0pt \marginpar{{{#1}}}}
\newcounter{V}[section] 
\newcommand{\V}[1][]{\stepcounter{V}\inMargin{\textcolor{green}{V.\arabic{section}.\theV.:}}\ifx^#1^\else\textcolor{green}{\underline{{#1}:}}\addcontentsline{toc}{subsubsection}{V.\arabic{section}.\theV.:$\quad$ {#1}}\\\fi}
\def\Def{\inMargin{\textcolor{red}{Def:}}}
\def\Poz{{\inMargin{\textcolor{brown}{Pozn:}}}}
\def\Pr{{\inMargin{\textcolor{blue}{Př:}}}}
\def\Pozenum
{
	\begin{enumerate}[1)]%, left = 0pt ]
		\item\inMargin{\textcolor{brown}{Pozn:}}\def\ISENUM{a}}
\def\End
{
	\if	^\ISENUM^
	\else \end{enumerate}
	\fi
	\def\ISENUM{}
}
\reversemarginpar

\makeatletter
\renewcommand\thesection{§\arabic{section}.}
\renewcommand\thesubsection{\Alph{subsection})}
\renewcommand\thesubsubsection{\alph{subsubsection})}
\newcounter{chapter}
\setcounter{chapter}{0}
\renewcommand\thechapter{\Alph{chapter})}
\newcounter{roman}
\setcounter{roman}{0}
\renewcommand\theroman{\Roman{roman}.}
\makeatother
\def\sectionnum#1{\setcounter{section}{#1}\addtocounter{section}{-1}}
\def\subsectionnum#1{\setcounter{subsection}{#1}\addtocounter{subsection}{-1}}
\def\subsubsectionnum#1{\setcounter{subsubsection}{#1}\addtocounter{subsubsection}{-1}}
\def\chapternum#1{\setcounter{chapter}{#1}\addtocounter{chapter}{-1}}
\def\chapter#1{

	\addtocounter{chapter}{1}\sectionnum{1}
	\addcontentsline{toc}{section}{\large{\thechapter$\quad${#1}}}
	
	{\LARGE  \textbf{\begin{minipage}[t]{0.1\textwidth}\thechapter\end{minipage}\begin{minipage}[t]{0.95\textwidth}#1\end{minipage}}}

}
\def\ROM{}
\def\Rom#1#2{\setcounter{roman}{#1}\renewcommand\ROM{#2}}

\Rom{6}{Funkce}
\title{\Huge\textbf{\theroman\quad \ROM}}
\author{Jiří Kalvoda}

\newcounter{countOfBegin}
\setcounter{countOfBegin}{0}
\newcommand{\BeginDoc}[1][]
{
	\ifnum\value{countOfBegin}=0
	\begin{document}
		#1
		\fi
	\addtocounter{countOfBegin}{1}
		
}
\def\EndDoc
{
	\addtocounter{countOfBegin}{-1}
	\ifnum\value{countOfBegin}=0
	\end{document}
	\fi
}

\fi
\let\braceru=\relax \let\bracelu=\relax 
\def\o#1{\setbox0=
	\hbox{$\kern2pt\overbrace{\kern-2pt#1\kern-2pt}\kern2pt$}\ht0=2.1ex\box0}
\def\to#1{\hbox{#1\rlap{\t{}}}}
\def\rad{\rm{rad}}
\def\f{\frac}
\BeginDoc{\sectionnum{10}}
\section{Odchylka}
\Def Nechť $p,q\subset \E_3$ jsou  dvě komplanární přímky. Odchylku dvou komplanárních přímek
$p,q$ označujeme $|\angle p ,q|$ a definujeme takto:
\begin{enumerate}
	\item Je-li $p\parallel q \imp |\angle p,q| = 0\d$
	\item Je-li $p\not\parallel q \imp $ odchylkou rozumíme velikost ostrého nebo pravého úhlu, který svírají.
\end{enumerate}

\Poz Odchylka leží v intervalu $<0\d;90\d>$.
\V  Nechť $p',q', p, q \subset\E_3$ jsou takové přímky, že $p'\parallel p, q'\parallel q$ (tedy dvojice $p, p'$ a $q, q'$ jsou
komplanární). Pak platí: $|\angle p ,q| =  |\angle p’, q’|$.
\Def Nechť $p, q \subset \E_3$ jsou dvě mimoběžné přímky. Odchylku dvou mimoběžných přímek $p,q$
označujeme $|\angle p ,q|$ a definujeme takto: $|\angle p ,q| = |\angle p', q'|$, kde $p\parallel p$ a $q' \parallel q$. a $p',q'$
jsou komplanární a různoběžné.

\Def Nechť $p \subset\E_3$ je přímka, $\alpha \subset E_3$ je rovina. Pak odchylku přímky $p$ od roviny $\alpha$
označujeme $|\angle p , \alpha|$ a definujeme takto:
\begin{itemize}
	\item Je-li $p \parallel \alpha \imp |\angle p,\alpha| = 0\d$.
	\item Je-li $p\not\parallel \alpha \imp|\angle p,\alpha|=|\angle p,q|$, kde $q$ je průsečnice roviny $\alpha$ s rovinou, která je
\end{itemize}
kolmá na rovinu $\alpha$ a obsahuje přímku $p$.

\Pr Je dán pravidelný čtyřboký jehlan $ABCDV$
s podstavnou hranou $a$ a výškou $v$.
Body $M,N$ jsou po řadě středy úseček $VC$ a
$AB$. Určete $|\angle \pri{MN},\pri{ABC}|$ : 
\pdf{in/10-1.pdf}
$\tg \alpha = \f{|MM_0|}{|NM_0|}=\f{\f v 2}{\sqrt{\f{a^2}{16}+\f{9a^2}{16}}}=\f{v}{a}\*\sqrt{\f 2 5} \imp \alpha = \tg^{-1} \f{v}{a}\*\sqrt{\f 2 5}$
Konstrukčne:\\
Podstavu uděláme ve skutečné velikosti.\\
Překlopíme rovinu $\pri{ACV}$ podle $AC$. $MM_0$ je tedy ve skutečné velikosti.\\
PReklopíme rovinu $\pri{MN_0M}$ podle $NM_0$. Výslený $\triangle NM_0M$ je tedy ve skutečné velikosti.
\pdf[0.6]{in/10-2.pdf}

\Def 
Nechť $\alpha,\beta \subset \E_3$ jsou dvě roviny. Odchylku dvou rovin $\alpha,\beta$ označujeme $|\angle \alpha,\beta|$
a definujeme ji takto:
\begin{itemize}
	\item Je-li $\alpha \parallel \beta \imp |\angle \alpha,\beta|=0$
	\item Je-li $\alpha \not\parallel \beta| \imp |\angle \alpha,\beta|=|\angle p,q|$ kde $p\subset \alpha $ a $q\subset\beta$ a  obě přímky jsou kolmé k průsečnici $\alpha,\beta$.
\end{itemize}
\V Nechť $\alpha,\beta \subset \E_3$ jsou dvě různoběžné roviny. $\alpha\cap\beta = r$, pak platí:
$|\angle\alpha,\beta|=|\angle p,q|$, kde $p=\alpha\cap\gamma$, $q=\beta\cap\gamma$ a $\gamma\perp r$.
\V Posouvání roviny zachovává odchylku.

\Pr Je dán pravidelný jehlan $ABCDV$. (hrana $a$, výška $v$) Určete odchylku boční stěny od podstavy.

$\tan\alpha = \f v {\f a 2}\imp \alpha = \tan^{-1} \(2\f v a \)$

\Pr Je dán pravidelný jehlan $ABCDV$. (podstavná hrana $a$, boční hrana $b$) Určete odchylku boční stěny od podstavy.
\begin{itemize}
	\item Algebraicky:\\
		$\cos \alpha = \f{\f a 2}{\sqrt{b^2-\f {a^2} 4}} \imp \alpha = \cos^{-1}\( \f {a} {\sqrt{4b^2-a^2}}\)=\cos^{-1}\f{a\sqrt{4b^2+a^2}}{4b^2-a^2}$
	\item Konstrukčně $a=4,b=6$:
		\pdf[0.5]{in/10-3.pdf}
\end{itemize}

\Pr Je dán pravidelný šestiboký jehlan ($a=3.5\j{cm},v=6\j{cm}$) Nechť $M$j e střed $AB$. Určete $|\angle \pri{CM},\pri{ABC}|$.

$\tg \alpha = \f{\f v 2}{\sqrt{a^2+a^2\*\f{3}4}}=\f v a\*\f 1{\sqrt{7}}
\imp \alpha = \tg^-1 \f {a\sqrt 7}{v 7} = 32.94\d$.


\EndDoc


