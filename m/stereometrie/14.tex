\providecommand{\HINCLUDE}{NE}
\if ^\HINCLUDE^
\else
\def\HINCLUDE{}
\global\newdimen\Okraje
\global\Okraje =4cm
\input{$HOME/souteze/_hlavicka/h-.tex}

%\definecolor{colorV}{RGB}{255,127,0}
%\definecolor{colorPoz}{RGB}{153,51,0}
%\definecolor{orangeV}{RGB}{255,127,0}
%\definecolor{colorPr}{RGB}{0,5,255}
%\definecolor{colorDef}{RGB}{0.255,0}

\usepackage[shortlabels]{enumitem}
\setlength{\marginparsep}{2pt}
\setlength{\marginparwidth}{35pt}

\def\st{{\rm st}}
\def\P{{\rm P}}

\def\ISENUM{}
\def\inMargin#1{\End
		
		\hskip0pt \marginpar{{{#1}}}}
\newcounter{V}[section] 
\newcommand{\V}[1][]{\stepcounter{V}\inMargin{\textcolor{green}{V.\arabic{section}.\theV.:}}\ifx^#1^\else\textcolor{green}{\underline{{#1}:}}\addcontentsline{toc}{subsubsection}{V.\arabic{section}.\theV.:$\quad$ {#1}}\\\fi}
\def\Def{\inMargin{\textcolor{red}{Def:}}}
\def\Poz{{\inMargin{\textcolor{brown}{Pozn:}}}}
\def\Pr{{\inMargin{\textcolor{blue}{Př:}}}}
\def\Pozenum
{
	\begin{enumerate}[1)]%, left = 0pt ]
		\item\inMargin{\textcolor{brown}{Pozn:}}\def\ISENUM{a}}
\def\End
{
	\if	^\ISENUM^
	\else \end{enumerate}
	\fi
	\def\ISENUM{}
}
\reversemarginpar

\makeatletter
\renewcommand\thesection{§\arabic{section}.}
\renewcommand\thesubsection{\Alph{subsection})}
\renewcommand\thesubsubsection{\alph{subsubsection})}
\newcounter{chapter}
\setcounter{chapter}{0}
\renewcommand\thechapter{\Alph{chapter})}
\newcounter{roman}
\setcounter{roman}{0}
\renewcommand\theroman{\Roman{roman}.}
\makeatother
\def\sectionnum#1{\setcounter{section}{#1}\addtocounter{section}{-1}}
\def\subsectionnum#1{\setcounter{subsection}{#1}\addtocounter{subsection}{-1}}
\def\subsubsectionnum#1{\setcounter{subsubsection}{#1}\addtocounter{subsubsection}{-1}}
\def\chapternum#1{\setcounter{chapter}{#1}\addtocounter{chapter}{-1}}
\def\chapter#1{

	\addtocounter{chapter}{1}\sectionnum{1}
	\addcontentsline{toc}{section}{\large{\thechapter$\quad${#1}}}
	
	{\LARGE  \textbf{\begin{minipage}[t]{0.1\textwidth}\thechapter\end{minipage}\begin{minipage}[t]{0.95\textwidth}#1\end{minipage}}}

}
\def\ROM{}
\def\Rom#1#2{\setcounter{roman}{#1}\renewcommand\ROM{#2}}

\Rom{6}{Funkce}
\title{\Huge\textbf{\theroman\quad \ROM}}
\author{Jiří Kalvoda}

\newcounter{countOfBegin}
\setcounter{countOfBegin}{0}
\newcommand{\BeginDoc}[1][]
{
	\ifnum\value{countOfBegin}=0
	\begin{document}
		#1
		\fi
	\addtocounter{countOfBegin}{1}
		
}
\def\EndDoc
{
	\addtocounter{countOfBegin}{-1}
	\ifnum\value{countOfBegin}=0
	\end{document}
	\fi
}

\fi
\let\braceru=\relax \let\bracelu=\relax 
\def\o#1{\setbox0=
	\hbox{$\kern2pt\overbrace{\kern-2pt#1\kern-2pt}\kern2pt$}\ht0=2.1ex\box0}
\def\to#1{\hbox{#1\rlap{\t{}}}}
\def\rad{\rm{rad}}
\def\f{\frac}
\BeginDoc{\sectionnum{14}}
\section{Čtyřstěn}
\Def Zvolme v prostoru body $A,B,C,D$, které neleží
v jedné rovině. Čtyřstěnem $ABCD$ rozumíme
množinu bodů, které jsou ohraničené
trojúhelníky $\triangle ABC , \triangle ABD , \triangle ACD , \triangle BCD$. Tyto
trojúhelníky tvoří stěny čtyřstěnu $ABCD$, úsečky
$AB, AC, AD, BC, BD, CD$ jsou jeho hranami a
body $A,B,C,D$ jeho vrcholy.

\Poz Ty hrany čtyřstěnu, které nejsou různoběžné, ale
mimoběžné, se nazývají protější hrany čtyřstěnu. V čtyřstěnu ABCD jsou tři dvojice
protějších hran $(AB, CD), (AC,BD), (AD,BC)$.

\V Středy všech tří úseček, které spojují vždy středy dvou protějších hran čtyřstěnu,
splývají.

\Def Spojnice lib. vrcholu čtyřstěnu s těžištěm protější stěny, se nazývá těžnice čtyřstěnu.

\V Všechny čtyři těžnice čtyřstěnu procházejí jedním bodem (tzv. těžištěm čtyřstěnu), který
dělí úsečku s krajními body ve vrcholu čtyřstěnu a v těžišti protější stěny v poměru 3:1.

\Def Úsečka procházející vrcholem čtyřstěnu, která je kolmá na rovinu, v níž leží protější stěna
čtyřstěnu, se nazývá výška čtyřstěnu.

\V Výšky čtyřstěnu ABCD vedené body A, D jsou právě tehdy různoběžné, je-li hrana AD
kolmá k protější hraně BC. Je-li tato podmínka splněna, leží průsečík V výšek čtyřstěnu
vedených body A,D na přímce, která je s oběma přímkami AD, BC různoběžná a k nim
kolmá.

\Poz Výšky čtyřstěnu ABCD vedené body A, D jsou právě tehdy různoběžné, jsou-li
různoběžné výšky čtyřstěnu vedené body B,C. To nastane právě tehdy, když jsou přímky
AD a BC navzájem kolmé.

\V V čtyřstěnu mohou nastat právě tyto 3 navzájem se vylučující situace:
\begin{enumerate}
\item Žádné dvě protější hrany čtyřstěnu nejsou navzájem kolmé a každé dvě výšky
čtyřstěnu jsou mimoběžné.
\item Pouze jedna dvojice protějších hran čtyřstěnu je tvořena dvojicí navzájem kolmých
přímek, výšky čtyřstěnu vedené vrcholy na každé z těchto hran jsou různoběžné, každé
dvě jiné výšky čtyřstěnu jsou mimoběžné.
\item Každé dvě protější hrany jsou navzájem kolmé, všechny čtyři výšky čtyřstěnu
procházejí jedním bodem. Takovýto bod čtyřstěnu se nazývá ortocentrum a takovýto
čtyřstěn ortocentrický.
\end{enumerate}

\EndDoc


