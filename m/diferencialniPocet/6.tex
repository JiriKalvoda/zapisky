\providecommand{\HINCLUDE}{NE}
\if ^\HINCLUDE^
\else
\def\HINCLUDE{}
\global\newdimen\Okraje
\global\Okraje =4cm
\input{$HOME/souteze/_hlavicka/h-.tex}

%\definecolor{colorV}{RGB}{255,127,0}
%\definecolor{colorPoz}{RGB}{153,51,0}
%\definecolor{orangeV}{RGB}{255,127,0}
%\definecolor{colorPr}{RGB}{0,5,255}
%\definecolor{colorDef}{RGB}{0.255,0}

\usepackage[shortlabels]{enumitem}
\setlength{\marginparsep}{2pt}
\setlength{\marginparwidth}{35pt}

\def\st{{\rm st}}
\def\P{{\rm P}}

\def\ISENUM{}
\def\inMargin#1{\End
		
		\hskip0pt \marginpar{{{#1}}}}
\newcounter{V}[section] 
\newcommand{\V}[1][]{\stepcounter{V}\inMargin{\textcolor{green}{V.\arabic{section}.\theV.:}}\ifx^#1^\else\textcolor{green}{\underline{{#1}:}}\addcontentsline{toc}{subsubsection}{V.\arabic{section}.\theV.:$\quad$ {#1}}\\\fi}
\def\Def{\inMargin{\textcolor{red}{Def:}}}
\def\Poz{{\inMargin{\textcolor{brown}{Pozn:}}}}
\def\Pr{{\inMargin{\textcolor{blue}{Př:}}}}
\def\Pozenum
{
	\begin{enumerate}[1)]%, left = 0pt ]
		\item\inMargin{\textcolor{brown}{Pozn:}}\def\ISENUM{a}}
\def\End
{
	\if	^\ISENUM^
	\else \end{enumerate}
	\fi
	\def\ISENUM{}
}
\reversemarginpar

\makeatletter
\renewcommand\thesection{§\arabic{section}.}
\renewcommand\thesubsection{\Alph{subsection})}
\renewcommand\thesubsubsection{\alph{subsubsection})}
\newcounter{chapter}
\setcounter{chapter}{0}
\renewcommand\thechapter{\Alph{chapter})}
\newcounter{roman}
\setcounter{roman}{0}
\renewcommand\theroman{\Roman{roman}.}
\makeatother
\def\sectionnum#1{\setcounter{section}{#1}\addtocounter{section}{-1}}
\def\subsectionnum#1{\setcounter{subsection}{#1}\addtocounter{subsection}{-1}}
\def\subsubsectionnum#1{\setcounter{subsubsection}{#1}\addtocounter{subsubsection}{-1}}
\def\chapternum#1{\setcounter{chapter}{#1}\addtocounter{chapter}{-1}}
\def\chapter#1{

	\addtocounter{chapter}{1}\sectionnum{1}
	\addcontentsline{toc}{section}{\large{\thechapter$\quad${#1}}}
	
	{\LARGE  \textbf{\begin{minipage}[t]{0.1\textwidth}\thechapter\end{minipage}\begin{minipage}[t]{0.95\textwidth}#1\end{minipage}}}

}
\def\ROM{}
\def\Rom#1#2{\setcounter{roman}{#1}\renewcommand\ROM{#2}}

\Rom{6}{Funkce}
\title{\Huge\textbf{\theroman\quad \ROM}}
\author{Jiří Kalvoda}

\newcounter{countOfBegin}
\setcounter{countOfBegin}{0}
\newcommand{\BeginDoc}[1][]
{
	\ifnum\value{countOfBegin}=0
	\begin{document}
		#1
		\fi
	\addtocounter{countOfBegin}{1}
		
}
\def\EndDoc
{
	\addtocounter{countOfBegin}{-1}
	\ifnum\value{countOfBegin}=0
	\end{document}
	\fi
}

\fi
\BeginDoc{}
\def\posloup{$\zs{a_n}_{n=1}^{\infty}$}
\def\pos#1{\zs{#1}_{n=1}^{\infty}}
\def\li{\lim_{n\rightarrow\infty}}
\def\lix{\lim_{x\rightarrow x_0}}
\def\r{\rightarrow}
\def\sup{{\rm sup\ }}
\def\sciwinfup{{\rm inf\ }}
\def\su{\sum_{n=1}^{\infty}}
\section{Derivace}

\Def
Nechť $x_0 \in D(f)$. Existuje-li limita
$$ \lim_{x\r x_0} \f{f(x)-f(x_0)}{x-x_0} $$
značíme je $f'(x_0)$ a nazýváme derivací funkce $f$ v bodě $x_0$.

Je-li $f'(x_0)\in\R$, pak říkáme, že $f$ má v bodě $x_0$ vlasní derivaci.

Je-li $f'(x_0)\in\zs{\pm\infty}$, pak říkáme, že $f$ má v bodě $x_0$ nevlasní derivaci.

\Def
Nechť $x_0 \in D(f)$. Existuje-li limita
$$ \lim_{x\r x_0^+} \f{f(x)-f(x_0)}{x-x_0} $$
značíme je $f'_+(x_0)$ a nazýváme derivací funkce $f$ v bodě $x_0$ zprava.

Nechť $x_0 \in D(f)$. Existuje-li limita
$$ \lim_{x\r x_0^-} \f{f(x)-f(x_0)}{x-x_0} $$
značíme je $f'_+(x_0)$ a nazýváme derivací funkce $f$ v bodě $x_0$ zleva.

\V  Funkce $f$ má derivaci v bodě $x_0$ právě tehdy když exisstují obě jednostrané derivace a jsou si rovny.


\Pr

$f(x)=x^2$:

$$f'(0) = \lim_{x\r 0} \f{x^2-0^2}{x-0} = \lim \f{x^2}{x} = \lim x = 0$$


\V[Bolzanova věta] Má-li funkce derivaci v nějakém bodě, pak je  v tomto bodě spojitá.

\Def Nechť existuje vlastní derivace $f'(x)$  funkce $f(x)$ pro všechna $x\in M$.
Kde $M\subset D(f)$. Pak funkci $f:y=f'(x),x\in M$ nazvěme \emph{derivací funkce $f$ na $M$}.

\Poz Geometricky je derivace $f$ v bodě $x_0$ směrnici sečny grafu funkce $f$ v $x_0$.


\Pr 191/6:
\begin{enumerate}
\item $(x^4+1)' = 4x^3 = 0$
\item $(\sqrt[3]{x^2})'$:
$$f'_+(0) = \lim_{x\r 0^+} \f{\sqrt[3]{x^2} - 0}{x-0} = \f 1 {x^{1/3}} = +\infty $$
$$f'_-(0) = \lim_{x\r 0^-} \f{\sqrt[3]{x^2} - 0}{x-0} = - \f 1 {x^{1/3}} = -\infty $$
Není :-(
\item $\sgn(x)$
Funkce není spojitá a tdy nemuže mít derivaci.
\end{enumerate}



\Pr 218/1:
\begin{enumerate}
	\item $f(x)=x^4-3x^2+2; x_0=0$

	$$\lim_{x\r 0} \f{x^4-3x^2+2-2}{x-0} = 0 $$

\item $f(x)=\sqrt{x^2-1};x_0=\sqrt 5$
	$$
		\lim_{x\r \sqrt 5} \f{\sqrt{x^2-1}-2}{x-\sqrt 5}
		=
		\lim_{x\r \sqrt 5} \f{x^2-1-4}{(x- \sqrt5)(\sqrt{x^2-1}+2)}
		=
		\lim_{x\r \sqrt 5} \f{x+\sqrt 5}{\sqrt{x^2-1}+2}
		=
		\lim_{x\r \sqrt 5} \f{2\sqrt 5}{4}
		=
		\f{\sqrt 5}{2}
	$$
\end{enumerate}


\EndDoc
