\providecommand{\HINCLUDE}{NE}
\if ^\HINCLUDE^
\else
\def\HINCLUDE{}
\global\newdimen\Okraje
\global\Okraje =4cm
\input{$HOME/souteze/_hlavicka/h-.tex}

%\definecolor{colorV}{RGB}{255,127,0}
%\definecolor{colorPoz}{RGB}{153,51,0}
%\definecolor{orangeV}{RGB}{255,127,0}
%\definecolor{colorPr}{RGB}{0,5,255}
%\definecolor{colorDef}{RGB}{0.255,0}

\usepackage[shortlabels]{enumitem}
\setlength{\marginparsep}{2pt}
\setlength{\marginparwidth}{35pt}

\def\st{{\rm st}}
\def\P{{\rm P}}

\def\ISENUM{}
\def\inMargin#1{\End
		
		\hskip0pt \marginpar{{{#1}}}}
\newcounter{V}[section] 
\newcommand{\V}[1][]{\stepcounter{V}\inMargin{\textcolor{green}{V.\arabic{section}.\theV.:}}\ifx^#1^\else\textcolor{green}{\underline{{#1}:}}\addcontentsline{toc}{subsubsection}{V.\arabic{section}.\theV.:$\quad$ {#1}}\\\fi}
\def\Def{\inMargin{\textcolor{red}{Def:}}}
\def\Poz{{\inMargin{\textcolor{brown}{Pozn:}}}}
\def\Pr{{\inMargin{\textcolor{blue}{Př:}}}}
\def\Pozenum
{
	\begin{enumerate}[1)]%, left = 0pt ]
		\item\inMargin{\textcolor{brown}{Pozn:}}\def\ISENUM{a}}
\def\End
{
	\if	^\ISENUM^
	\else \end{enumerate}
	\fi
	\def\ISENUM{}
}
\reversemarginpar

\makeatletter
\renewcommand\thesection{§\arabic{section}.}
\renewcommand\thesubsection{\Alph{subsection})}
\renewcommand\thesubsubsection{\alph{subsubsection})}
\newcounter{chapter}
\setcounter{chapter}{0}
\renewcommand\thechapter{\Alph{chapter})}
\newcounter{roman}
\setcounter{roman}{0}
\renewcommand\theroman{\Roman{roman}.}
\makeatother
\def\sectionnum#1{\setcounter{section}{#1}\addtocounter{section}{-1}}
\def\subsectionnum#1{\setcounter{subsection}{#1}\addtocounter{subsection}{-1}}
\def\subsubsectionnum#1{\setcounter{subsubsection}{#1}\addtocounter{subsubsection}{-1}}
\def\chapternum#1{\setcounter{chapter}{#1}\addtocounter{chapter}{-1}}
\def\chapter#1{

	\addtocounter{chapter}{1}\sectionnum{1}
	\addcontentsline{toc}{section}{\large{\thechapter$\quad${#1}}}
	
	{\LARGE  \textbf{\begin{minipage}[t]{0.1\textwidth}\thechapter\end{minipage}\begin{minipage}[t]{0.95\textwidth}#1\end{minipage}}}

}
\def\ROM{}
\def\Rom#1#2{\setcounter{roman}{#1}\renewcommand\ROM{#2}}

\Rom{6}{Funkce}
\title{\Huge\textbf{\theroman\quad \ROM}}
\author{Jiří Kalvoda}

\newcounter{countOfBegin}
\setcounter{countOfBegin}{0}
\newcommand{\BeginDoc}[1][]
{
	\ifnum\value{countOfBegin}=0
	\begin{document}
		#1
		\fi
	\addtocounter{countOfBegin}{1}
		
}
\def\EndDoc
{
	\addtocounter{countOfBegin}{-1}
	\ifnum\value{countOfBegin}=0
	\end{document}
	\fi
}

\fi
\BeginDoc{}
\def\posloup{$\zs{a_n}_{n=1}^{\infty}$}
\def\pos#1{\zs{#1}_{n=1}^{\infty}}
\def\li{\lim_{n\rightarrow\infty}}
\def\sup{{\rm sup\ }}
\def\sciwinfup{{\rm inf\ }}
\def\su{\sum_{n=1}^{\infty}}
\section{Definice limity}
\subsection{Vlastní limita v vlastním bodě}

\Def Řekneme, že \emph{funkce $f$ má v bodě $x_0\in\R$ limitu $A\in\R$},
jestliže ke každému $\epsilon \in \R^+$ existuje $\sigma \in\R^+$ takové, že pro všechna $x\in(x_0-\sigma,x_0+\sigma)-\zs{x}$,
platí $f(x) \in (A-\epsilon,A+\epsilon)$. Píšeme:
$$\lim_{x\rightarrow x_0} f(x) = A$$
\pdf[0.3]{1-1.pdf}

\subsection{Nevlastní limita v nevlastním bodě}

\Def Řekneme, že \emph{funkce $f$ má v bodě $x_0\in\R$ limitu $+\infty$},
jestliže ke každému $M \in \R^+$ existuje $\sigma \in\R^+$ takové, že pro všechna $x\in(x_0-\sigma,x_0+\sigma)-\zs{x}$,
platí $f(x) > M$. Píšeme:
$$\lim_{x\rightarrow x_0} f(x) = +\infty$$
\pdf[0.3]{1-2.pdf}

\subsection{Vlastní limita v nevlastním bodě}

\Def Řekneme, že \emph{funkce $f$ má v $+\infty$ (nebo podrobněji pro $x$ jdoucí do $+\infty$ limitu $A\in\R$},
jestliže ke každému $\epsilon \in \R^+$ existuje $K \in\R$ takové, že pro všechna $x>K$,
platí $f(x) \in (A-\epsilon,A+\epsilon)$. Píšeme:
$$\lim_{x\rightarrow +\infty} f(x) = A$$
\pdf[0.3]{1-3.pdf}

\subsection{Nevlastní limita v nevlastním bodě}

\Def Řekneme, že \emph{funkce $f$ má v $+\infty$ (nebo podrobněji pro $x$ jdoucí do $+\infty$ limitu $+\infty$},
jestliže ke každému $M \in \R$ existuje $K \in\R$ takové, že pro všechna $x>K$,
platí $f(x) > M$. Píšeme:
$$\lim_{x\rightarrow +\infty} f(x) = +\infty$$
\pdf[0.3]{1-4.pdf}

\subsection{Souhrná definice limity}

\begin{enumerate}
\item
\Def 
\emph{Okolnímu bodu $x_0\in\R$} rozumíme otevřený interval $(x_0-\sigma;x_0+\sigma)$, kde $\sigma$ je kladné reálné číslo.
Značíme je $O(x_0)$.
\item \emph{Okolím bodu $+\infty$} rozumíme každý interval $(k;+\infty)$, kde $k\in\R$.
Značíme je $O(+\infty)$.
\item \emph{Okolím bodu $-\infty$} rozumíme každý interval $(-\infty;k)$, kde $k\in\R$.
Značíme je $O(-\infty)$.
\item \emph{prstencovým okolím bodu $x_0\in\R$} rozumíme množinu $O(x_0)-\zs{x_0}$. Značíme je $P(x_0)$.
\end{enumerate}

\Def Řekneme, že \emph{funkce $f$ má v bodě $x_0\in\R^*$ limitu $A\in\R^*$},
jestliže ke každému okolí $O(A)$ bodu $A$ existuje prstencové okolí $P(x_0)$ bodu $x_0$ takové, že
pro všechna $x\in P(x_0)$ platí $f(x)\in O(A)$.
Píšeme:
$$\lim_{x\rightarrow x_0} f(x) = A$$

\subsection{Jednosměrné limity}
\begin{enumerate}
\item
\Def 
\emph{Levým prstencovým okolím bodu $x_0\in\R$} rozumíme interval $(x_0-\sigma,x_0)$, kde $\sigma$ je kladné reálné číslo.
Značíme je $P^-(x_0)$.
\item \emph{Pravým prstencovým okolím bodu $x_0\in\R$} rozumíme interval $(x_0,x_0+\sigma)$, kde $\sigma$ je kladné reálné číslo.
Značíme je $P^+(x_0)$.
\end{enumerate}

\begin{enumerate}
\item 
\Def
Řekneme, že \emph{funkce $f$ má v bodě $x_0\in\R^*$ limitu zleva} rovnu  $A\in\R^*$,
jestliže ke každému okolí $O(A)$ bodu $A$ existuje levé prstencové okolí $P^-(x_0)$ bodu $x_0$ takové, že
pro všechna $x\in P^-(x_0)$ platí $f(x)\in O(A)$.
Píšeme:
$$\lim_{x\rightarrow x_0^-} f(x) = A$$
\item 
Řekneme, že \emph{funkce $f$ má v bodě $x_0\in\R^*$ limitu zprava} rovnu  $A\in\R^*$,
jestliže ke každému okolí $O(A)$ bodu $A$ existuje pravé prstencové okolí $P^+(x_0)$ bodu $x_0$ takové, že
pro všechna $x\in P^+(x_0)$ platí $f(x)\in O(A)$.
Píšeme:
$$\lim_{x\rightarrow x_0^+} f(x) = A$$
\end{enumerate}

\EndDoc
