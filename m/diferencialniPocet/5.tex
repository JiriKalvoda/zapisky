\providecommand{\HINCLUDE}{NE}
\if ^\HINCLUDE^
\else
\def\HINCLUDE{}
\global\newdimen\Okraje
\global\Okraje =4cm
\input{$HOME/souteze/_hlavicka/h-.tex}

%\definecolor{colorV}{RGB}{255,127,0}
%\definecolor{colorPoz}{RGB}{153,51,0}
%\definecolor{orangeV}{RGB}{255,127,0}
%\definecolor{colorPr}{RGB}{0,5,255}
%\definecolor{colorDef}{RGB}{0.255,0}

\usepackage[shortlabels]{enumitem}
\setlength{\marginparsep}{2pt}
\setlength{\marginparwidth}{35pt}

\def\st{{\rm st}}
\def\P{{\rm P}}

\def\ISENUM{}
\def\inMargin#1{\End
		
		\hskip0pt \marginpar{{{#1}}}}
\newcounter{V}[section] 
\newcommand{\V}[1][]{\stepcounter{V}\inMargin{\textcolor{green}{V.\arabic{section}.\theV.:}}\ifx^#1^\else\textcolor{green}{\underline{{#1}:}}\addcontentsline{toc}{subsubsection}{V.\arabic{section}.\theV.:$\quad$ {#1}}\\\fi}
\def\Def{\inMargin{\textcolor{red}{Def:}}}
\def\Poz{{\inMargin{\textcolor{brown}{Pozn:}}}}
\def\Pr{{\inMargin{\textcolor{blue}{Př:}}}}
\def\Pozenum
{
	\begin{enumerate}[1)]%, left = 0pt ]
		\item\inMargin{\textcolor{brown}{Pozn:}}\def\ISENUM{a}}
\def\End
{
	\if	^\ISENUM^
	\else \end{enumerate}
	\fi
	\def\ISENUM{}
}
\reversemarginpar

\makeatletter
\renewcommand\thesection{§\arabic{section}.}
\renewcommand\thesubsection{\Alph{subsection})}
\renewcommand\thesubsubsection{\alph{subsubsection})}
\newcounter{chapter}
\setcounter{chapter}{0}
\renewcommand\thechapter{\Alph{chapter})}
\newcounter{roman}
\setcounter{roman}{0}
\renewcommand\theroman{\Roman{roman}.}
\makeatother
\def\sectionnum#1{\setcounter{section}{#1}\addtocounter{section}{-1}}
\def\subsectionnum#1{\setcounter{subsection}{#1}\addtocounter{subsection}{-1}}
\def\subsubsectionnum#1{\setcounter{subsubsection}{#1}\addtocounter{subsubsection}{-1}}
\def\chapternum#1{\setcounter{chapter}{#1}\addtocounter{chapter}{-1}}
\def\chapter#1{

	\addtocounter{chapter}{1}\sectionnum{1}
	\addcontentsline{toc}{section}{\large{\thechapter$\quad${#1}}}
	
	{\LARGE  \textbf{\begin{minipage}[t]{0.1\textwidth}\thechapter\end{minipage}\begin{minipage}[t]{0.95\textwidth}#1\end{minipage}}}

}
\def\ROM{}
\def\Rom#1#2{\setcounter{roman}{#1}\renewcommand\ROM{#2}}

\Rom{6}{Funkce}
\title{\Huge\textbf{\theroman\quad \ROM}}
\author{Jiří Kalvoda}

\newcounter{countOfBegin}
\setcounter{countOfBegin}{0}
\newcommand{\BeginDoc}[1][]
{
	\ifnum\value{countOfBegin}=0
	\begin{document}
		#1
		\fi
	\addtocounter{countOfBegin}{1}
		
}
\def\EndDoc
{
	\addtocounter{countOfBegin}{-1}
	\ifnum\value{countOfBegin}=0
	\end{document}
	\fi
}

\fi
\BeginDoc{}
\def\posloup{$\zs{a_n}_{n=1}^{\infty}$}
\def\pos#1{\zs{#1}_{n=1}^{\infty}}
\def\li{\lim_{n\rightarrow\infty}}
\def\lix{\lim_{x\rightarrow x_0}}
\def\r{\rightarrow}
\def\sup{{\rm sup\ }}
\def\sciwinfup{{\rm inf\ }}
\def\su{\sum_{n=1}^{\infty}}
\section{Limity elemantárnich funkcí}
\V
Nechť $x_0\in\R^*$ a nechť existují
$\lim_{x\r x_0} f(x)$ a 
$\lim_{x\r x_0} g(x)$ . Pak platí:
\begin{enumerate}
\item $\lim_{x\r x_0} \[f(x)\pm g(x)\]=\lim_{x\r x_0} f(x) \pm \lim_{x\r x_0} g(x)$
\item $\lim_{x\r x_0} \[f(x) g(x)\]=\lim_{x\r x_0} f(x) \* \lim_{x\r x_0} g(x)$
\item $\lim_{x\r x_0} f(x)g(x)=\f{\lim_{x\r x_0} f(x)}{\lim_{x\r x_0} g(x)}$
\item $\lim_{x\r x_0} |f(x)| = \lim_{x\r x_0} |f(x)|$
\end{enumerate}

\Pr
\begin{enumerate}
\item $\lim_{x\r 1} \(\ln x + x^2 + 3\) = 4$
\item $\lim_{x\r 1} \f{3x^3+x^2-2x+11}{x^2+x+1} = 11$
\item $\lim_{x\r \pi} \f{\cos x}{x} = -\f{1}{\pi} $
\item $\lim_{x\r \pi/2} \sqrt{x\cos x + \tg \f x 2} = 1$
\item $\lim_{x\r 0} \f{e^x+2^x\sin x}{\ln(1+x)+(x+1)\cos x} = 1$
\item $\lim_{x\r \pi/4} x\tan x = \f \pi 4$
\end{enumerate}

\Pr
\begin{enumerate}
\item $\lim_{x\r +\infty}  (e^x+x) = +\infty$
\item $\lim_{x\r -\infty}  (e^x+x) = -\infty$
\item $\lim_{x\r +\infty}  x{\rm arctg} x = +\infty$
\item $\lim_{x\r +\infty}  \f 1{x^2+1} = 0 $
\item $\lim_{x\r -\infty}  (\sqrt{x^2+1}-x) = +\infty$
\end{enumerate}

\Pr
$$\lim_{x\r 0^+} \f 1{x^2} = +\infty$$

\Pr \uv{cvičení 182/1}

Daná strana evidentně neexistuje v daném souboru.

\subsection{Věta o limitě funkcí shodujících se v prstencovém okolí bodu}
\V
Nechť $f,g$ jsou funkce a nechť existuje $P(x_0)$ bodu $x_0\in\R^*$,
takové, že pro každé $x\in P(x_0)$ platí $f(x)=g(x)$.
Nechť existuje $\lix g(x)$.
Pak $\lix f(x) = \lix g(x)$.

\Pr
$$\lim_{x\r -1}\f{x^2-1}{x+1}$$

Kromě $x\neq -1$: $$\f{x^2-1}{x+1} = x-1$$.
Tedy 
$$\lim_{x\r -1}\f{x^2-1}{x+1} = \lim_{x\r -1} x-1=-2$$

\Pr
$$\lim_{x\r 1^+}\f{|x^2-1|}{x-1} = \lim_{x\r 1^+}\f{x^2-1}{x-1} = \lim_{x\r 1} x+1 = 2 $$

$$\lim_{x\r 1^-}\f{|x^2-1|}{x-1} = \lim_{x\r 1^-}-\f{x^2-1}{x-1} = \lim_{x\r 1}-( x+1 )= -2 $$

\Pr
\begin{enumerate}
	\item 
		$$\lim_{x\r+\infty} \f{x^2-x+1}{2x^2+x-3} = \f 12$$
	\item 
		$$\lim_{x\r+\infty} \f{2x^2+3}{\sqrt{3x^4-1}} = \f {2}{\sqrt 3} = \f{2\sqrt 3}3$$
	\item
		$$
		\lim_{x\r-\infty} x \sqrt{\(\sqrt{x^2+9}-\sqrt{x^2-9}\)^2}
		=
		\lim_{x\r-\infty} x\f{(x^2+9)-(x^2-9)}{\sqrt{x^2+9}+\sqrt{x^2-9}}
		=$$$$=
		\lim_{x\r-\infty} \f{18x}{\sqrt{x^2+9}+\sqrt{x^2-9}}
		=
		\lim_{x\r-\infty} \f{18x}{|2x|}
		=
		9
		$$
\end{enumerate}

\subsection{Věta o limitě funkcí shodujících se v prstencovém okolí bodu}
\V
Nechť $f,g,h$ jsou funkce a nechť existuje $P(x_0)$ bodu $x_0\in\R^*$,
		takové, že pro každé $x\in P(x_0)$ platí $g(x)\le f(x)\le h(x)$.
		Nechť existuje $\lix g(x)=h(x)=A$.
Pak $\lix f(x) = A$.

		\pd[0.5]{5-1.pdf}

\subsection{Věta o limitě součtu \uv{nulové} a ohraničené funkce}

\V
Nechť $f,g$ jsou funkce a $\lim_{x\r x_0} f(x)=0$.
Nechť exsistuje $P(x_0)$, takové, že $g$ je na tomto intervalu omezená.
Pak $\lim_{x\r x_0} f(x)g(x)=0$.
\Pr
$$\lim_{x\r 0} x\sin\f1x = 0$$

\Pr
$$
\lim{x\r 2} \f{x-2}{x^2-3x+2}
=
\lim{x\r 2} \f{x-2}{(x-2)(x-1)}
=
\lim{x\r 2} \f{1}{(x-1)}
= 1
$$

$$
\lim_{x\r -3} \f{3x^2+11x+6}{x^3+27}
=
\lim_{x\r -3} \f{3x-2}{1x^2-3x+9} =\f{-11}{27}
$$

$$
\lim_{x\r \pi} \f{\tg x}{\sin 2x} 
=
\lim_{x\r \pi} \f{\sin x}{2\sin x \cos x \cos x} 
=
\lim_{x\r \pi} \f{1}{2\cos x \cos x} = \f 12
$$

$$
\lim_{x\r\f\pi4} \f{\sin x-\cos x}{\cos 2x}
=
\lim_{x\r\f\pi4} \f{\sin x-\cos x}{\cos^2 x-\sin^2 x}
=
\lim_{x\r\f\pi4} \f{-1}{\cos x+\sin x}
$$

\Pr 178/56:
$$
\lim_{x\r 0^+} \f{\sin x+1}{\sin x}
=
1 \*  \f{1}{0^+}
=
+\infty
$$
$$
\lim_{x\r 0^-} \f{\sin x+1}{\sin x}
=
1 \* \f{1}{0^-}
=
-\infty
$$


$$
\lim_{x\r 0} \f{\cos x+1}{\cos x-1}
=
\f{2}{0^-}
=
-\infty
$$

$$
\lim_{x\r -2} \f{x^3}{(x+2)^2}
=
\f{-8}{0^+}
= -\infty
$$

\Pr 182/2:
$$
\lim_{x\r -2} \f{x^2+x-2}{x^2+2x}
=
\lim_{x\r -2} \f{(x-1)(x+2)}{x(x+2)}
=
\f{-3}{-2}=\f32
$$

$$
\lim_{x\r -2} \f{3x^2+3x-6}{2x^2-2x-12}
=
\lim_{x\r -2} \f{3(x-1)(x+2)}{2(x-3)(x+2)}
=
\f{3(-3)}{2(-5)}=\f 9{10}
$$

\Pr 182/3:

$$
\lim_{x\r5}\f{\sqrt{x-1}-2}{x^2-4x-5}
=
\lim_{x\r5}\f{x-1-4}{(x-5)(x+1)(\sqrt{x-1}+2)}
=
\lim_{x\r5}\f{1}{(x+1)(\sqrt{x-1}+2)}
=
\f{1}{6\* 4} = \f1{24}
$$

$$
\lim_{x\r0} \f{\sqrt{ 1 + x^2} -1}{x}
=
\lim_{x\r0} \f{x^2}{(\sqrt{ 1 + x^2} +1)x}
=
\lim_{x\r0} \f{x}{(\sqrt{ 1 + x^2} +1)}
=
0
$$

\Pr 182/4:

$$
\lim_{x\r+\infty} \f{x^3-2x+1}{x^5-2x+2} = 0
$$


$$
\lim_{x\r+\infty} \f{2x^3-2x^2+2}{5x^3+2x-1} = \f 2 5
$$

\Pr 
$$
\lim_{x\r -\infty} \f{\sqrt{x^2+1}}x
= 
\lim_{x\r -\infty} -\sqrt{\f{x^2+1}{x^2}} = -1
$$

$$
\lim_{x\r -\infty} \f{\sqrt{x^2+1}}{\sqrt[3]{x^2-2x^2}}
$$
Výraz není definovaný, protože $x^3-2x^2<0$.

\Pr 182/6:

$$
\lim_{x\r -\infty} \f{\cos(4x-\pi)}{2x}
= 
0
$$

\Pr 183/8:
$$-\infty$$

$$\pm\infty$$

$$+\infty$$

$$\pm \infty$$


\subsection{Autotest}
\begin{enumerate}
	\item nemusí být
	\item je
	\item vlastní
	\item nejvýše
	\item $\sgn(x-1)$
	\item $x$

		$\f 1 {x^2}$
	\item 
		$$
		\lim_{x\r 3} \f{x-3}{(x-3)(x-5)} = \f 1{-2}
		$$

		$$
		\lim_{x\r -\infty} (x^3+x^2)
		=
		\lim_{x\r -\infty} ((x-1)x^2)
		= -\infty \* \infty = -\infty
		$$

		$$
		\lim_{x\r \pm\infty} \f{7x^3-3x^2+5}{4x^3+5}
		=
		\f{\pm 7}{\f pm 4} = \f74
		$$

		$$
		\lim_{x\r 0} \f{\sin 7x}{\sin 2x}
		=
		\lim_{x\r 0} \f{7x}{2x}
		=
		 \f{7}{2}
		 $$

	 \item 
		 $$
		\lim_{x\r 3} \f{x}{x^3-27} = \f3{0^\pm} = \pm \infty
		$$ Tedy neexistuje


		$$
		\lim_{x\r 0} \f{x-1}{x^2} = \f{-1}{0^+} = -\infty
		$$
\end{enumerate}




\EndDoc
