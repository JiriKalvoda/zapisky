\providecommand{\HINCLUDE}{NE}
\if ^\HINCLUDE^
\else
\def\HINCLUDE{}
\global\newdimen\Okraje
\global\Okraje =4cm
\input{$HOME/souteze/_hlavicka/h-.tex}

%\definecolor{colorV}{RGB}{255,127,0}
%\definecolor{colorPoz}{RGB}{153,51,0}
%\definecolor{orangeV}{RGB}{255,127,0}
%\definecolor{colorPr}{RGB}{0,5,255}
%\definecolor{colorDef}{RGB}{0.255,0}

\usepackage[shortlabels]{enumitem}
\setlength{\marginparsep}{2pt}
\setlength{\marginparwidth}{35pt}

\def\st{{\rm st}}
\def\P{{\rm P}}

\def\ISENUM{}
\def\inMargin#1{\End
		
		\hskip0pt \marginpar{{{#1}}}}
\newcounter{V}[section] 
\newcommand{\V}[1][]{\stepcounter{V}\inMargin{\textcolor{green}{V.\arabic{section}.\theV.:}}\ifx^#1^\else\textcolor{green}{\underline{{#1}:}}\addcontentsline{toc}{subsubsection}{V.\arabic{section}.\theV.:$\quad$ {#1}}\\\fi}
\def\Def{\inMargin{\textcolor{red}{Def:}}}
\def\Poz{{\inMargin{\textcolor{brown}{Pozn:}}}}
\def\Pr{{\inMargin{\textcolor{blue}{Př:}}}}
\def\Pozenum
{
	\begin{enumerate}[1)]%, left = 0pt ]
		\item\inMargin{\textcolor{brown}{Pozn:}}\def\ISENUM{a}}
\def\End
{
	\if	^\ISENUM^
	\else \end{enumerate}
	\fi
	\def\ISENUM{}
}
\reversemarginpar

\makeatletter
\renewcommand\thesection{§\arabic{section}.}
\renewcommand\thesubsection{\Alph{subsection})}
\renewcommand\thesubsubsection{\alph{subsubsection})}
\newcounter{chapter}
\setcounter{chapter}{0}
\renewcommand\thechapter{\Alph{chapter})}
\newcounter{roman}
\setcounter{roman}{0}
\renewcommand\theroman{\Roman{roman}.}
\makeatother
\def\sectionnum#1{\setcounter{section}{#1}\addtocounter{section}{-1}}
\def\subsectionnum#1{\setcounter{subsection}{#1}\addtocounter{subsection}{-1}}
\def\subsubsectionnum#1{\setcounter{subsubsection}{#1}\addtocounter{subsubsection}{-1}}
\def\chapternum#1{\setcounter{chapter}{#1}\addtocounter{chapter}{-1}}
\def\chapter#1{

	\addtocounter{chapter}{1}\sectionnum{1}
	\addcontentsline{toc}{section}{\large{\thechapter$\quad${#1}}}
	
	{\LARGE  \textbf{\begin{minipage}[t]{0.1\textwidth}\thechapter\end{minipage}\begin{minipage}[t]{0.95\textwidth}#1\end{minipage}}}

}
\def\ROM{}
\def\Rom#1#2{\setcounter{roman}{#1}\renewcommand\ROM{#2}}

\Rom{6}{Funkce}
\title{\Huge\textbf{\theroman\quad \ROM}}
\author{Jiří Kalvoda}

\newcounter{countOfBegin}
\setcounter{countOfBegin}{0}
\newcommand{\BeginDoc}[1][]
{
	\ifnum\value{countOfBegin}=0
	\begin{document}
		#1
		\fi
	\addtocounter{countOfBegin}{1}
		
}
\def\EndDoc
{
	\addtocounter{countOfBegin}{-1}
	\ifnum\value{countOfBegin}=0
	\end{document}
	\fi
}

\fi
\BeginDoc{}
\def\U{{\mathbb U}}
\section{Kvadratické rovnice v $\C$}
\subsection{Kvadratické rovnice s reálnými koeficienty}
\Poz
S kvadratickou rovnicí s reálnými koeficienty a reálnou neznámou jsme se seznámili ve IV.
kapitole.
\Poz 
Nechť $a \in \R$ . Binomická rovnice 
$z^2 = -a^2; a\neq 0$ má v $\C$ právě dva kořeny.
$z_1=a\* i;z_2 = -a\* i$.

\Def
 Kvadratickou rovnicí s (komplexní) neznámou $z \in\C$ a reálnými koeficienty $a,b,c$ nazýváme
každou rovnici tvaru 
$az^2 + bz + c = 0$, kde $a,b,c \in\R, a\neq 0$.

\Poz
Kvadratickou rovnici řešíme doplněním na čtverec:

$z^2 + \f ba z +\f ca = 0 \imp \(z+\f b{2a}\)^2 = \f{b^2}{a^2} - \f{ca - b^2-4ac}{4a^2}$

$z = \f{-b \pm \sqrt{b^2-4ac}}{2a}$

\V  Nechť $az^1 + bz + z = 0 ; a \neq 0$ (*)
je kvadratická rovnice s reálnými koeficienty a
nechť $D = b^2 -  4ac$
 je její diskriminant. Pak platí: 
\begin{enumerate}
\item $D>0$ $\imp$ (*) má 2 různé reálné kořeny $z_{1,2} = \f{-b\pm\sqrt D}{2a}$
\item $D=0$ $\imp$ (*) má 1 reálný dvonásobný kořeny $z_{1,2} = \f{-b}{2a}$
\item $D<0$ $\imp$ (*) má 2 komplexně sdružené kořeny $z_{1,2} = \f{-b\pm i\sqrt{|D|}}{2a}$
\end{enumerate}

\Pr 
Řešte c $\C$ rovnice:
\begin{enumerate}
\item $z^2 + z + 1 = 0$\\
$z_{1,2} = \f{-1\pm \sqrt{-3}}2 = \f{-1\pm i \sqrt 3}2$

\item $3z^2 - 2z\sqrt{3}-1 = 0$\\
$z_{1,2} = \f{2\sqrt 3 \pm 2\sqrt{3}}6 = \f{\sqrt 3 \pm \sqrt 6}3$
\end{enumerate}

\Pr 155/6:
\begin{enumerate}
\item $x^2 -  4x + 6 = 0$\\
$x = \f{4 \pm \sqrt{16-24}}2 = 2 \pm i \sqrt 2$
\item $5x^2-6x+2 = 0$\\
$x= \f{6\pm\sqrt{36-40}}{10} = \f{3\pm i}5$
\item $x^2-2x+5 = 0$\\
$x = \f{2\pm \sqrt{4-20}}2 = 1\pm 2i$
\item $2x^2 - 11 x + 14 = 0$
$x = \f{11-\sqrt{121-112}}2 = \f{11\pm3}4 \imp x_1=2;x_2=\f 72$
\end{enumerate}
\Def \emph{ Kvadratickou rovnicí s (komplexní) neznámou} $z\in\C$ a komplexními koeficienty $a,b,c$ nazýváme
každou rovnici tvaru $az^2+ bz+ c=0$ , kde $a,b, c \in\C;a \neq 0$. 

\V Každá kvadratická rovnice s komplexními koeficienty má v množině komplexních čísel právě dva
kořeny, počítáme-li dvojnásobný kořen za dva. 
\Pr
\begin{enumerate}
	\item 
		$z^2+2iz+1 = (z+1)^2-i^2+1 =0$

		$t^2 =(z+i)^2 = -1$\\
		$t=\pm i \sqrt 2$

		$z = -i \pm i\sqrt 2$
	\item $z = \f{-2i\pm\sqrt{-8}}2 = -i \pm i\sqrt 2 = i(-1+\sqrt 2)$
	\item
		$z=x+iy$

		$$(x+iy)^2+(x+iy)+2i+1 = x^2+2ixy-y^2+2ix-2y+1 = 0$$
		Porovnání koeficientů:\\
		$i^0:  x^2-y^2-2y+1=0$\\ 
		$i^1:  2xy+2x=x(y+1)=0$\\ 
		\begin{enumerate}
			\item $x=0$:

				$y^2+2y-1=0$
				$y = \f{-2+pm\sqrt{4+4}}2=-1\pm\sqrt 2$

				$z = i(-1\pm\sqrt 2)$
			\item $y=-1$:
				$x^2=-2$ $\imp$ nelze
		\end{enumerate}
\end{enumerate}
\Poz  Pokud vyjde diskriminant D imaginární (s $i$), tak je potřeba vyřešit $\sqrt D$ pomocí binomické
rovnice, nebo III. způsobu – viz následující příklad (spojení II. a III. způsobu). 

\Pr
$z^2+3z+10i = 0$

$ D = 9-40i$\\
$\sqrt D = \sqrt{9-40i} = x+yi$\\
$9-40i = x^2 + 2xyi - y^2$

$i^0: 9 = x^2-y^2$\\
$i^1: -40=2xy$

$9 = x^2 - {400}{x^2}$\\
$0 = x^4 -9x^2-400$\\
$x^2 = \f{9\pm 41}2$

$x^2 = 25$:\\
$x=5; y=-4 \imp \sqrt D = 5-4i$\\
$x=-5; y=4 \imp \sqrt D = -5+4i$

$$z = \f{-3\pm(5-4i)}2$$
$$z_1 = 1-2i$$
$$z_2 = -4+2i$$


\EndDoc
