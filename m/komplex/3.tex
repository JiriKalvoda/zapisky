\providecommand{\HINCLUDE}{NE}
\if ^\HINCLUDE^
\else
\def\HINCLUDE{}
\global\newdimen\Okraje
\global\Okraje =4cm
\input{$HOME/souteze/_hlavicka/h-.tex}

%\definecolor{colorV}{RGB}{255,127,0}
%\definecolor{colorPoz}{RGB}{153,51,0}
%\definecolor{orangeV}{RGB}{255,127,0}
%\definecolor{colorPr}{RGB}{0,5,255}
%\definecolor{colorDef}{RGB}{0.255,0}

\usepackage[shortlabels]{enumitem}
\setlength{\marginparsep}{2pt}
\setlength{\marginparwidth}{35pt}

\def\st{{\rm st}}
\def\P{{\rm P}}

\def\ISENUM{}
\def\inMargin#1{\End
		
		\hskip0pt \marginpar{{{#1}}}}
\newcounter{V}[section] 
\newcommand{\V}[1][]{\stepcounter{V}\inMargin{\textcolor{green}{V.\arabic{section}.\theV.:}}\ifx^#1^\else\textcolor{green}{\underline{{#1}:}}\addcontentsline{toc}{subsubsection}{V.\arabic{section}.\theV.:$\quad$ {#1}}\\\fi}
\def\Def{\inMargin{\textcolor{red}{Def:}}}
\def\Poz{{\inMargin{\textcolor{brown}{Pozn:}}}}
\def\Pr{{\inMargin{\textcolor{blue}{Př:}}}}
\def\Pozenum
{
	\begin{enumerate}[1)]%, left = 0pt ]
		\item\inMargin{\textcolor{brown}{Pozn:}}\def\ISENUM{a}}
\def\End
{
	\if	^\ISENUM^
	\else \end{enumerate}
	\fi
	\def\ISENUM{}
}
\reversemarginpar

\makeatletter
\renewcommand\thesection{§\arabic{section}.}
\renewcommand\thesubsection{\Alph{subsection})}
\renewcommand\thesubsubsection{\alph{subsubsection})}
\newcounter{chapter}
\setcounter{chapter}{0}
\renewcommand\thechapter{\Alph{chapter})}
\newcounter{roman}
\setcounter{roman}{0}
\renewcommand\theroman{\Roman{roman}.}
\makeatother
\def\sectionnum#1{\setcounter{section}{#1}\addtocounter{section}{-1}}
\def\subsectionnum#1{\setcounter{subsection}{#1}\addtocounter{subsection}{-1}}
\def\subsubsectionnum#1{\setcounter{subsubsection}{#1}\addtocounter{subsubsection}{-1}}
\def\chapternum#1{\setcounter{chapter}{#1}\addtocounter{chapter}{-1}}
\def\chapter#1{

	\addtocounter{chapter}{1}\sectionnum{1}
	\addcontentsline{toc}{section}{\large{\thechapter$\quad${#1}}}
	
	{\LARGE  \textbf{\begin{minipage}[t]{0.1\textwidth}\thechapter\end{minipage}\begin{minipage}[t]{0.95\textwidth}#1\end{minipage}}}

}
\def\ROM{}
\def\Rom#1#2{\setcounter{roman}{#1}\renewcommand\ROM{#2}}

\Rom{6}{Funkce}
\title{\Huge\textbf{\theroman\quad \ROM}}
\author{Jiří Kalvoda}

\newcounter{countOfBegin}
\setcounter{countOfBegin}{0}
\newcommand{\BeginDoc}[1][]
{
	\ifnum\value{countOfBegin}=0
	\begin{document}
		#1
		\fi
	\addtocounter{countOfBegin}{1}
		
}
\def\EndDoc
{
	\addtocounter{countOfBegin}{-1}
	\ifnum\value{countOfBegin}=0
	\end{document}
	\fi
}

\fi
\BeginDoc{}
\def\U{{\mathbb U}}
\section{Binomické rovnice}
\Poz Nechť $a'in \R;\sqrt[n]{a}$ v reálném ooru (pokud existuje) je jediné číslo $x$
s vlastností $x^n = a$.
Tzn. v $\R$ existuje nejvýše jedna $\sqrt[n]a$. V množině $\C$ je situace jiná.

\Def \emph{Binomickou rovnicí} s neznámou $z\in\C$ nazýváme každou rovnici tvaru $z^n = a$, kde $a\in\C,n\in\N,n\ge 2$.

Každý komplexní kořen binomické rovnice nazýváme \emph{komplexní $n$-tou odmocninu} z čisla $a$.

\Pr Vypočítejte:
$z = \sqrt{1+\sqrt 3 i}$\\
$z^2 = 1+\sqrt 3 i = 2 e^{i\f\pi 3}$
$|z| = \sqrt 2$
$2\varphi = \f \pi 3 + 2\pi n \imp \varphi = \f\pi 6 + \pi n \imp \varphi = \f\pi 6 \lor \varphi = \f 7 6 \pi $ 

$z_0 = \sqrt 2 e^{i\f\pi6} = \f{\sqrt 6}2 + \f{\sqrt 2}2 i$\\
$z_1 = \sqrt 2 e^{i\f\pi6} = -\f{\sqrt 6}2 -\f{\sqrt 2}2 i$

\Poz  Při řešení binomických rovnic (s využitím goniometrického tvaru komplexních čísel) jde tedy i
o řešení následujícího problému: 

Určit všechna $x\in\R : (\cos x +i\sin x)^n (\cos nx +i\sin nx) = \cos\alpha +i \sin\alpha$

Musí platiti: $nx = \alpha + 2k\pi;k\in\Z \imp x_k = \f{\alpha + 2k\pi}n$

Je zřejmé, že pro $k\in\zs{0,1,\dots,n-1}$ získáme argumenty různých komplexních čisel.

Pro $k = n+h; h \in\N_0 : x_k = \f{\alpha+2h\pi}n+2\pi \imp$ $x_h$ a $x_k$ se liší o $2\pi$, proto jsou to argumenty téhož čísla.

\V Nechť $a\in\C$: pak binomická rovnice $z^n = a$ má v množině $\C$:
\begin{enumerate}
\item $a=0\imp$ jeden kořen
\item $a\neq 0\imp$ právě $n$ kořenů
$$ z_k = \sqrt[n]{|a|} \* e^{i\f{\alpha+2k\pi}n}$$
kde $k\in\zs{0;1;\dots n-1}$, $\alpha$ \dots argument komplexního čísla $a$.
\end{enumerate}

\Pr Řešte v $\C$: $z^3 = i$.

$|z|^3 = 1 \imp |z| = 1$\\
$3\alpha = \f\pi 2 + 2k\pi \in \alpha = \f\pi 6 + \f 23\pi$

$$
\f{\sqrt 3}1+\f 12 i;
-\f{\sqrt 3}1+\f 12 i;
-i
$$

\Poz
Obrazy kořenů binomické rovnice $z^n=a;n\ge 3$
 tvoří vrcholy pravidelného $n$-úhelníka vepsaného
do kružnice se středem v počátku souřadné soustavy a poloměrem $\sqrt[n]{|a|}$, neboť rozdíl dvou
\uv{sousedních} kořenů je konstantní:

$$ x_{i-1} - x_i = \f{\alpha + 2(i+1)\pi}n - \f{\alpha+2i\pi}n = \f{2\pi}n$$



\EndDoc
