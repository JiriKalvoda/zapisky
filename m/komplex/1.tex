\providecommand{\HINCLUDE}{NE}
\if ^\HINCLUDE^
\else
\def\HINCLUDE{}
\global\newdimen\Okraje
\global\Okraje =4cm
\input{$HOME/souteze/_hlavicka/h-.tex}

%\definecolor{colorV}{RGB}{255,127,0}
%\definecolor{colorPoz}{RGB}{153,51,0}
%\definecolor{orangeV}{RGB}{255,127,0}
%\definecolor{colorPr}{RGB}{0,5,255}
%\definecolor{colorDef}{RGB}{0.255,0}

\usepackage[shortlabels]{enumitem}
\setlength{\marginparsep}{2pt}
\setlength{\marginparwidth}{35pt}

\def\st{{\rm st}}
\def\P{{\rm P}}

\def\ISENUM{}
\def\inMargin#1{\End
		
		\hskip0pt \marginpar{{{#1}}}}
\newcounter{V}[section] 
\newcommand{\V}[1][]{\stepcounter{V}\inMargin{\textcolor{green}{V.\arabic{section}.\theV.:}}\ifx^#1^\else\textcolor{green}{\underline{{#1}:}}\addcontentsline{toc}{subsubsection}{V.\arabic{section}.\theV.:$\quad$ {#1}}\\\fi}
\def\Def{\inMargin{\textcolor{red}{Def:}}}
\def\Poz{{\inMargin{\textcolor{brown}{Pozn:}}}}
\def\Pr{{\inMargin{\textcolor{blue}{Př:}}}}
\def\Pozenum
{
	\begin{enumerate}[1)]%, left = 0pt ]
		\item\inMargin{\textcolor{brown}{Pozn:}}\def\ISENUM{a}}
\def\End
{
	\if	^\ISENUM^
	\else \end{enumerate}
	\fi
	\def\ISENUM{}
}
\reversemarginpar

\makeatletter
\renewcommand\thesection{§\arabic{section}.}
\renewcommand\thesubsection{\Alph{subsection})}
\renewcommand\thesubsubsection{\alph{subsubsection})}
\newcounter{chapter}
\setcounter{chapter}{0}
\renewcommand\thechapter{\Alph{chapter})}
\newcounter{roman}
\setcounter{roman}{0}
\renewcommand\theroman{\Roman{roman}.}
\makeatother
\def\sectionnum#1{\setcounter{section}{#1}\addtocounter{section}{-1}}
\def\subsectionnum#1{\setcounter{subsection}{#1}\addtocounter{subsection}{-1}}
\def\subsubsectionnum#1{\setcounter{subsubsection}{#1}\addtocounter{subsubsection}{-1}}
\def\chapternum#1{\setcounter{chapter}{#1}\addtocounter{chapter}{-1}}
\def\chapter#1{

	\addtocounter{chapter}{1}\sectionnum{1}
	\addcontentsline{toc}{section}{\large{\thechapter$\quad${#1}}}
	
	{\LARGE  \textbf{\begin{minipage}[t]{0.1\textwidth}\thechapter\end{minipage}\begin{minipage}[t]{0.95\textwidth}#1\end{minipage}}}

}
\def\ROM{}
\def\Rom#1#2{\setcounter{roman}{#1}\renewcommand\ROM{#2}}

\Rom{6}{Funkce}
\title{\Huge\textbf{\theroman\quad \ROM}}
\author{Jiří Kalvoda}

\newcounter{countOfBegin}
\setcounter{countOfBegin}{0}
\newcommand{\BeginDoc}[1][]
{
	\ifnum\value{countOfBegin}=0
	\begin{document}
		#1
		\fi
	\addtocounter{countOfBegin}{1}
		
}
\def\EndDoc
{
	\addtocounter{countOfBegin}{-1}
	\ifnum\value{countOfBegin}=0
	\end{document}
	\fi
}

\fi
\BeginDoc{}
\section{ Komplexní čísla, algebraický tvar komplexního čísla}

\Poz  Množinu všech komplexních čísel označíme $\C$.

\Def
\emph{Komplexním číslem} $z\in\C$ nazýváme každou uspořadanou dvajici $z=(x,y)$ reálných čísel, tj. kartézského čtverce $\R^2 = \C$, na které jsou definovány rovnost a operace sčítání, odčítání, násobení a dělení takto:
\begin{enumerate}
	\item \emph{Rovnost komplexních čísel} $z_1=(x_1,y_1);z_2=(x_2,y_2)$ definujeme takto:
		$$z_1=z_2 \ekv x_1=x_2 \land y_1 = y_2 $$
		$$z_1\neq z_2 \ekv x_1\neq x_2 \lor y_1\neq y_2 $$

	\item \emph{Součet komplexních čísel} $z_1=(x_1,y_1);z_2=(x_2,y_2)$ definujeme takto:
		$$z_1+z_2 = (x_1+x_2;y_1+y_2)$$

	\item \emph{Rozdíl komplexních čísel} $z_1=(x_1,y_1);z_2=(x_2,y_2)$ :
		Rozdílem rozumíme komplexní číslo $z$, pro které platí $z_1 = z_2 +z$, zapisujeme $z=z_1-z_2$

	\item \emph{Součin komplexních čísel} $z_1=(x_1,y_1);z_2=(x_2,y_2)$ definujeme takto:
		$$z_1\*z_2 = (x_1x_2-y_1y_2;x_1y_2+x_2y_1)$$

	\item \emph{Podíl komplexních čísel} $z_1=(x_1,y_1);z_2=(x_2,y_2)$ :
		Posdílem rozumíme komplexní číslo $z$, pro které platí $z_1 = z_2 \* z$, zapisujeme $z=\f{z_1}{z_2}$
\end{enumerate}

\V Nechť $z_1  =(x_1,y_1),z_2=(x_2,y_2)\in\C$. Pak platí:
\begin{enumerate}
	\item $z_1-z_2 = (x_1-x_2;y_1-y_2)$
	\item $z_2 = \ve 0: \f{z_1}{z_2} = \(\f{x_1x_2+y_1y_2}{x_2^2 + y_2^2};\f{x_2y_1-x_1y_2}{x_2^2 + y_2^2}\)$
\end{enumerate}

[Dk:
\begin{enumerate}
	\item $$z_1=z_2+z$$
		Dosadím:
		$$(x_1,y_1) = (x_2,y_2) + (x_1-x_2;y_1-y_2)$$
		Ekvivalentně upravím:
		$$(x_1,y_1) = (x_1-x_2+x_2;y_1-y_2+y_2)$$
		$$(x_1,y_1) = (x_1,y_2)$$
		Což evidentně platí. \emph{QED}

	\item $$z_1=z_2\*z$$
		Dosadím:
		$$(x_1,y_1) = (x_2,y_2) \* \(\f{x_1x_2+y_1y_2}{x_2^2 + y_2^2};\f{x_2y_1-x_1y_2}{x_2^2 + y_2^2}\)$$
		Ekvivalentně upravím:
		$$(x_1,y_1) = \(
		x_2\f{x_1x_2+y_1y_2}{x_2^2 + y_2^2}-y_2\f{x_2y_1-x_1y_2}{x_2^2 + y_2^2};
		\f{x_1x_2+y_1y_2}{x_2^2 + y_2^2}y_2+x_2\f{x_2y_1-x_1y_2}{x_2^2 + y_2^2}
		\)$$
		$$(x_1,y_1) = \(
		\f{x_1x_2^2+x_2y_1y_2-x_2y_1y_2+x_1y_2^2}{x_2^2 + y_2^2};
		\f{x_1x_2y_2+y_1y_2^2+x_2^2y_2-x_1x_2y_2}{x_2^2 + y_2^2}
		\)$$
		$$(x_1,y_1) = (x_1,y_2)$$
		Což evidentně platí. \emph{QED}

\end{enumerate}

\Pr 11/1:\\
\begin{enumerate}
	\item $5+4i$
	\item $6+3i$
	\item ??? Co to jako má znamenat ???
	\item $6-9i$
	\item analogicky
\end{enumerate}







\EndDoc
