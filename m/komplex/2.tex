\providecommand{\HINCLUDE}{NE}
\if ^\HINCLUDE^
\else
\def\HINCLUDE{}
\global\newdimen\Okraje
\global\Okraje =4cm
\input{$HOME/souteze/_hlavicka/h-.tex}

%\definecolor{colorV}{RGB}{255,127,0}
%\definecolor{colorPoz}{RGB}{153,51,0}
%\definecolor{orangeV}{RGB}{255,127,0}
%\definecolor{colorPr}{RGB}{0,5,255}
%\definecolor{colorDef}{RGB}{0.255,0}

\usepackage[shortlabels]{enumitem}
\setlength{\marginparsep}{2pt}
\setlength{\marginparwidth}{35pt}

\def\st{{\rm st}}
\def\P{{\rm P}}

\def\ISENUM{}
\def\inMargin#1{\End
		
		\hskip0pt \marginpar{{{#1}}}}
\newcounter{V}[section] 
\newcommand{\V}[1][]{\stepcounter{V}\inMargin{\textcolor{green}{V.\arabic{section}.\theV.:}}\ifx^#1^\else\textcolor{green}{\underline{{#1}:}}\addcontentsline{toc}{subsubsection}{V.\arabic{section}.\theV.:$\quad$ {#1}}\\\fi}
\def\Def{\inMargin{\textcolor{red}{Def:}}}
\def\Poz{{\inMargin{\textcolor{brown}{Pozn:}}}}
\def\Pr{{\inMargin{\textcolor{blue}{Př:}}}}
\def\Pozenum
{
	\begin{enumerate}[1)]%, left = 0pt ]
		\item\inMargin{\textcolor{brown}{Pozn:}}\def\ISENUM{a}}
\def\End
{
	\if	^\ISENUM^
	\else \end{enumerate}
	\fi
	\def\ISENUM{}
}
\reversemarginpar

\makeatletter
\renewcommand\thesection{§\arabic{section}.}
\renewcommand\thesubsection{\Alph{subsection})}
\renewcommand\thesubsubsection{\alph{subsubsection})}
\newcounter{chapter}
\setcounter{chapter}{0}
\renewcommand\thechapter{\Alph{chapter})}
\newcounter{roman}
\setcounter{roman}{0}
\renewcommand\theroman{\Roman{roman}.}
\makeatother
\def\sectionnum#1{\setcounter{section}{#1}\addtocounter{section}{-1}}
\def\subsectionnum#1{\setcounter{subsection}{#1}\addtocounter{subsection}{-1}}
\def\subsubsectionnum#1{\setcounter{subsubsection}{#1}\addtocounter{subsubsection}{-1}}
\def\chapternum#1{\setcounter{chapter}{#1}\addtocounter{chapter}{-1}}
\def\chapter#1{

	\addtocounter{chapter}{1}\sectionnum{1}
	\addcontentsline{toc}{section}{\large{\thechapter$\quad${#1}}}
	
	{\LARGE  \textbf{\begin{minipage}[t]{0.1\textwidth}\thechapter\end{minipage}\begin{minipage}[t]{0.95\textwidth}#1\end{minipage}}}

}
\def\ROM{}
\def\Rom#1#2{\setcounter{roman}{#1}\renewcommand\ROM{#2}}

\Rom{6}{Funkce}
\title{\Huge\textbf{\theroman\quad \ROM}}
\author{Jiří Kalvoda}

\newcounter{countOfBegin}
\setcounter{countOfBegin}{0}
\newcommand{\BeginDoc}[1][]
{
	\ifnum\value{countOfBegin}=0
	\begin{document}
		#1
		\fi
	\addtocounter{countOfBegin}{1}
		
}
\def\EndDoc
{
	\addtocounter{countOfBegin}{-1}
	\ifnum\value{countOfBegin}=0
	\end{document}
	\fi
}

\fi
\BeginDoc{}
\def\U{{\mathbb U}}
\section{ Gaussova rovina, goniometrický tvar komplexního čísla}
\Poz
Protože množina $\C$ je definována jako množina uspořádaných dvojic $\R$ čísel, lze každé
komplexní číslo $z = (x, y)$ zobrazit v rovině s kartézskou soustavou souřadnic a ztotožnit s bodem
$z[x, y]$.
Reálná čísla leží na ose $x$ (osa $x$ se nazývá reálná osa).
Ryze imaginární čísla $z = yi, y \neq 0$ leží na ose $y$ (osa $y$ se nazývá imaginární osa).
Rovina s reálnou a imaginární osou se nazývá Gaussova rovina a v ní zobrazujeme komplexní čísla.
Podobně lze interpretovat komplexní čísla jako vektory s pevným počátečním bodem.

\Def
Nechť $z=x+iy\in\C$. Číslo $\overline{z}=x-iy\in\C$ nazýváme \emph{komplexně sdruženým číslem} k číslu $z$.

Číslo $|z| = \sqrt{x^2+y^2} \in \R$ nazýváme \emph{absolutní hodnotou komplexního čísla} $z$.

Komplexní číslo $z$ s vlastností $|z|=1$ nazýváme \emph{komplexní jednotkou}.

\Pozenum
Čísla $z$ a $\overline z$ jsou osově souměrná podle reálné osy (osy $x$)
\item Číslo $|z|$ vyjadřuje vzdálenost bodu $z$ od počátku souřadné soustavy, a tedy délku polohového
vektoru bodu $z$. 
\item Množinu všech komplexních jednotek označíme $\U$. 
\item 
 Protože obrazem množiny komplexních čísel se stejnou nenulovou absolutní hodnotou je kružnice
se středem v počátku souřadné soustavy, je obrazem množiny $\U$ jednotková kružnice se středem
v počátku.
\V
Pro každá dvě komplexně združená čísla $z=x+yi;\overline z = x-yi$ platí:
\begin{enumerate}
	\item $\overline{\(\overline z\)}  =  z$
	\item $z+\overline z  = 2x$
	\item $z\*\overline z = x^2 + y^2 = |z|^2$
	\item $\overline{z_1+z_2} = \overline{z_1}+\overline{z_2}$
	\item $\overline{z_1-z_2} = \overline{z_1}-\overline{z_2}$
	\item $\overline{z_1\*z_2} = \overline{z_1}\*\overline{z_2}$
	\item $\overline{\(\f{z_1}{z_2}\)} = \f{\overline{z_1}}{\overline{z_2}}$
	\item $\overline{\(z^n\)} = \(\overline{z}\)^n$
\end{enumerate}

\V
Pro absolutní hodnoty libovolných komplexních čísel $z,z_1,z_2 \in\C$ platí:
\begin{enumerate}
	\item $|z|\ge0$, přičemž $|z|=0 \equiv z=0$
	\item $|z| = |-z| = |\overline z| = \sqrt{z \* \overline z}$
	\item $|z_1\pm z_2| \le |z_1|+z_2| $
	\item $|z_1\pm z_2| \ge |z_1|-z_2| $
	\item $|z_1\*z_2| = |z_1| \* |z_2|$
	\item $\f{|z_1|}{|z_2|} = \left|\f{z_1}{z_2}\right|\ \ \ (z_2\neq 0)$
	\item ${\rm Re} z \le |z| ; {\rm Im z}\le |z|$
\end{enumerate}

\Poz Všechny vztahy platící pro reálná čísla nelze mechanicky převést do množiny komplexních čísel.
Např.
$|z|^2 \neq z^2$ 
\Poz Na rozdíl od množiny reálných čísel není množina komplexních čísel uspořádaná, tedy komplexní
čísla nelze srovnávat podle velikosti. 

\Def Nechť $z=x+iy \in \C; z \neq 0$. \emph{Argumentem komplexního čilsa} nazýváme orientovaný úhel $\varphi$, který svírá kladná poloosa reálné osy s polohovým vektorem bodu $z$.

\Poz Každé nenulové číslo $z$ má nekonečně mnoho argumentů, přitom každé 2 z nich se liší o
$k \* 2\pi ; k \in \Z$.
Je-li $\varphi \in \<0;2\pi\)$ , pak jej nazýváme \emph{hlavní argument komplexního čísla} $z$ , je jediný. 
\V
Každé komplexní číslo $z = x + iy;z \neq 0$ lze zapsat ve tvaru $z = |z| \*(\cos \varphi + i \sin \varphi)$, kde $|z|$ je
absolutní hodnota $z$ a $\varphi$ je argument $z$, přičemž platí: $\cos \varphi = \f{x}{|z|}; \sin \varphi = \f y{|z|}$.

[Dk.!:

$|z| = \sqrt{x^2+y^2} ; \cos \varphi = \f{x}{|z|}; \sin \varphi = \f y{|z|}$

$ z = x+iy = \sqrt{x^2+y^2} \* \(\f x{\sqrt{x^2+y^2}} + \f{iy}{\sqrt{x^2+y^2}}\) = |z| \* \(\cos \varphi + i \sin \varphi\)$

]

\Def \emph{ Goniometrický tvar komplexního čísla}

Nechť $z\in\C,z\neq 0$. Pak goniometrickým tvarem tohoto čísla nazýváme zápis čísla ve tvaru $z=|z|\*(\cos \varphi + i \sin \varphi)$, kde $|z|$ je absolutní hodnota $z$ a $\varphi$ ja argument $z$, přičemž $\cos \varphi = \f x{|z|}$ a $\sin \varphi = \f y {|z|}$.

\Pr Zapiště v gon. tvaru:
\begin{enumerate}
	\item $z=1+i$\\
		$|z| = \sqrt 2; \varphi = \f\pi4$\\
		$z = \sqrt 2 e^{i\f\pi 4}$
\end{enumerate}
\Pr Zapiště v alg. tvaru:
\begin{enumerate}
	\item $z= 2e^{i\f\pi3} = 2(\f 12 + i \f{\sqrt 3}2 = 1+i\sqrt 3$

\end{enumerate}

\Pr Nechť $z_1,z_2\in\C-\zs{0}; z_1 = |z_1| \* (\cos\varphi_1+i\sqrt\varphi_1);z_2 = |z_2| \* (\cos\varphi_2+i\sqrt\varphi_2)$
pak platí:

$z-1\*z_2 = |z_1|\*|z_2| \* (\cos \varphi_1+i \sin\varphi_1) \* (\cos \varphi_2+i \sin\varphi_2) = |z_1|\*|z_2| \* (\cos(\varphi_1+\varphi_2)+i\sin(\varphi_1+\varphi_2))$

$\f{z_1}{z_2} = \f{|z_1}{|z_2} \* (\cos(\varphi_1-\varphi_2)+i\sin(\varphi_1-\varphi_2))$

\Pr Vypočítejte v algebraickém i goniometrickém tvaru:
\begin{enumerate}
	\item $(i-\sqrt 3 i)(\sqrt 3+i) = 2\sqrt 3 - 2i$

		$2 e^{i\f 53 \pi} \* 2 \* e^{i\f\pi 6} = 4 e^{i\f{11}6\pi}$

	\item $\f{2e^{i\f53\pi}}{2e^{i\f\pi6}} = e^{i\f 32 \pi}$
\end{enumerate}
\V[Moivreova věta]
Nechť máme nenulové kkomplexní číslo $z = |z| e^{i\varphi}; n\in\N$, pak platí:
$$z^n = |z|^n e^{in\varphi}$$

\Poz  Moivreova věta platí i pro celé exponenty. 

\Poz  Součet a rozdíl, součin a podíl komplexních čísel v Gaussově rovině.

Součet a rozdíl jako sčítání a odčítání vektorů. 

Součin $z_1z_2$ (podíl $\f{z_1}{z_2}$):
\begin{enumerate}
	\item zobrazíme $z_1$ ve stejnolehnosti $H_{P,|z_2|}$ ($H_{P,|z_2|^{-1}}$):
		Získáme $|z_1\*z_2|$.
	\item zobrazíme $|z_1\*z_2$ ($\left|\f{z_1}{z_2}\right|)$ v rotaci $R_{P,\arg z_2}$ ($R_{P,-\arg z_2}$)
		Získáme $z_1\*z_2$ ($\f{z_1}{z_2}$).
		\pdf[0.7]{2-1.pdf}
\end{enumerate}
\Pr 
\pdf[0.2]{2-2.pdf}
$z_1 + z_2  = 2 + 0 i$\\
$z_1 - z_2  = 0 - 2 i$\\
$z_1 \* z_2  = 1-i+i+1 = 2$\\
$\f{z_1}{z_2}  = \f{1-i}{1+i}\*\f{1-i}{1-i} = \f{1-2i-1}2=-i$\\
\pdf[0.8]{2-3.pdf}
$z_1 + z_2  = 1+ 6i$\\
$z_1 - z_2  = 1+2i$\\
$z_1 \* z_2  = 2i+4i^2 = -4+2i$\\
$\f{z_1}{z_2}  = \f{1+4i}{2i}\*\f{i}{i} = \f{-4+i}{-2}=2-\f12i$\\

\Pr 147/1:
\begin{enumerate}
	\item[d)] $(1-i)^n = \sqrt{2}^2 e^{-n i\f\pi 4}$
	\item[e)] $(\sqrt 2 + i)^n = 2^n e^{-i\f\pi6}$
	\item[f)] $\(\f{1-i}{1+i}\)^{20} = -i ^{20} = -i ^2 = 1$
\end{enumerate}

\EndDoc
