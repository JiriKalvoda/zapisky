\providecommand{\HINCLUDE}{NE}
\if ^\HINCLUDE^
\else
\def\HINCLUDE{}
\global\newdimen\Okraje
\global\Okraje =4cm
\input{$HOME/souteze/_hlavicka/h-.tex}

%\definecolor{colorV}{RGB}{255,127,0}
%\definecolor{colorPoz}{RGB}{153,51,0}
%\definecolor{orangeV}{RGB}{255,127,0}
%\definecolor{colorPr}{RGB}{0,5,255}
%\definecolor{colorDef}{RGB}{0.255,0}

\usepackage[shortlabels]{enumitem}
\setlength{\marginparsep}{2pt}
\setlength{\marginparwidth}{35pt}

\def\st{{\rm st}}
\def\P{{\rm P}}

\def\ISENUM{}
\def\inMargin#1{\End
		
		\hskip0pt \marginpar{{{#1}}}}
\newcounter{V}[section] 
\newcommand{\V}[1][]{\stepcounter{V}\inMargin{\textcolor{green}{V.\arabic{section}.\theV.:}}\ifx^#1^\else\textcolor{green}{\underline{{#1}:}}\addcontentsline{toc}{subsubsection}{V.\arabic{section}.\theV.:$\quad$ {#1}}\\\fi}
\def\Def{\inMargin{\textcolor{red}{Def:}}}
\def\Poz{{\inMargin{\textcolor{brown}{Pozn:}}}}
\def\Pr{{\inMargin{\textcolor{blue}{Př:}}}}
\def\Pozenum
{
	\begin{enumerate}[1)]%, left = 0pt ]
		\item\inMargin{\textcolor{brown}{Pozn:}}\def\ISENUM{a}}
\def\End
{
	\if	^\ISENUM^
	\else \end{enumerate}
	\fi
	\def\ISENUM{}
}
\reversemarginpar

\makeatletter
\renewcommand\thesection{§\arabic{section}.}
\renewcommand\thesubsection{\Alph{subsection})}
\renewcommand\thesubsubsection{\alph{subsubsection})}
\newcounter{chapter}
\setcounter{chapter}{0}
\renewcommand\thechapter{\Alph{chapter})}
\newcounter{roman}
\setcounter{roman}{0}
\renewcommand\theroman{\Roman{roman}.}
\makeatother
\def\sectionnum#1{\setcounter{section}{#1}\addtocounter{section}{-1}}
\def\subsectionnum#1{\setcounter{subsection}{#1}\addtocounter{subsection}{-1}}
\def\subsubsectionnum#1{\setcounter{subsubsection}{#1}\addtocounter{subsubsection}{-1}}
\def\chapternum#1{\setcounter{chapter}{#1}\addtocounter{chapter}{-1}}
\def\chapter#1{

	\addtocounter{chapter}{1}\sectionnum{1}
	\addcontentsline{toc}{section}{\large{\thechapter$\quad${#1}}}
	
	{\LARGE  \textbf{\begin{minipage}[t]{0.1\textwidth}\thechapter\end{minipage}\begin{minipage}[t]{0.95\textwidth}#1\end{minipage}}}

}
\def\ROM{}
\def\Rom#1#2{\setcounter{roman}{#1}\renewcommand\ROM{#2}}

\Rom{6}{Funkce}
\title{\Huge\textbf{\theroman\quad \ROM}}
\author{Jiří Kalvoda}

\newcounter{countOfBegin}
\setcounter{countOfBegin}{0}
\newcommand{\BeginDoc}[1][]
{
	\ifnum\value{countOfBegin}=0
	\begin{document}
		#1
		\fi
	\addtocounter{countOfBegin}{1}
		
}
\def\EndDoc
{
	\addtocounter{countOfBegin}{-1}
	\ifnum\value{countOfBegin}=0
	\end{document}
	\fi
}

\fi
\def\f{\frac}
\BeginDoc{\sectionnum{14}}
%\section{Posloupnost nezávisle opakovaných pokusů}
\section{Statistika}
\Def
\emph{Statistický soubor} je množina všech objektů statistického pozorování
\Def
\emph{Statistická jednotka} prvek statistického souboru.
\Def
\emph{\underline{Rozsah souboru $n$}} je počet prvků statistického souboru.
\Def
\emph{Statistický znak} je společná vlastnost statistických jednotek, zpravidla se značí $x$.
\Def
\emph{Hodnota znaku} = jednotlivé údaje znaku
Příklady stat. znaků:
Při sčítání obyvatelstva: věk, pohlaví, zaměstnání\dots\\
Hodnoty znaku mohou být vyjádřeny slovy nebo čísly (kvalitativní $\times$ kvantitativní znak)

\Def \emph{\underline{Četnost (absolutní četnost)}} hodnoty znaku je počet statistických jednotek, které mají stejnou hodnotu znaku\\
Značí se $n_j$ a platí $\sum_{j=1}^k n_j=n$.
\Def
\emph{\underline{Relativní četnost hodnoty znaku}} je podíl absolutní četnosti a rozsahu souboru.\\
Značíme $\nu_j = \f{n_j}{n}$ a platí  $\sum_{j=1}^k v_j = 1$.

\Def \emph{Rozdělení četnosti} -- všechny různé hodnoty znaku a jim odpovídající četnosti uspořádáme do tabulky nebo znázorňujeme graficky.
\begin{itemize}
	\item Spojnicový diagram (polygon četnosti)
	\item Sloupkový diagram (histogram)
	\item Kruhový diagram 	
\end{itemize}
\Def \emph{Skupinové (intervalové) rozdělení četnosti} -- blízké hodnoty znaku se sdružují do skupin tvořených obvykle intervaly. Hodnoty znaku, jež se dostali do téhož intervalu, lze potom reprezentovat jednou hodnotou -- střed intervalu = třídní znak.

\subsection{Charakteristiky statistického souboru}
\Poz Čísla, která podávají stručnou souhru informací o uvažovaném statistickém souboru z různých hledisek.

\subsection{Charakteristiky polohy}
\Poz Čísla, kolem nichž jedotlivé hodnoty znaku kolísají 
\Def \emph{Aritmetrický průměr} hodnot $x_1,x_2,\dots,x_n$ kvantitativniho znaku $x$: $\overline{x}=\f{x_1+x_2+\dots+x_n}n$
\Def \emph{Vážený aritmetrický průměr} statistického souboru, kde četnost hodnoty znaku $x_i$ je $n_i$: $\overline{x}=\f{n_1x_1+n_2x_2+\dots+n_kx_n}n$
\Def \emph{Geometrický průměr} hodnot $x_1,x_2,\dots,x_n$ kvantitativniho znaku $x$: $\overline{x}_G=\sqrt[n]{x_1\*x_2\*\dots\*x_n}$
\Def \emph{Harmonický průměr} hodnot $x_1,x_2,\dots,x_n$ kvantitativniho znaku $x$: $\overline{x}_H=\f{n}{\f{1}{x_1}+\f{1}{x_2}+\dots+\f{1}{x_n}}$

\V[AG-nerovnost] $$\overline x \ge \overline x_G$$
\Def \emph{Modus} hodnot $x_1,x_2,\dots,x_n$ \emph{znaku} $x$ je hodnota, která má největší četnost.\\
Značí se ${\rm Mod}(x)$.\\
Modus se užívá k odhadu střední hodnoty znaku souboru, nejčastěji tedy, máme-li sestavenou tabulku rozdělení četnosti pro statistický soubor s velkým rozsahem.
\Def \emph{Medián} hodnot $x_1,x_2,\dots,x_n$ \emph{znaku} $x$ statistického ssouboru, v němž jsou prvky uspořádány dle velikosti hodnot sledovaného znaku je prostřední hodnota:\\
\begin{itemize}
	\item V souboru s lichým rozsahem se rovná prostřednímu členu.
	\item V souboru se sudým počtem znaků se rovná aritmetrickému průměru dvou prostředních znků s indexy $\f n 2 $ a $\f n 2 + 1$.
\end{itemize}

\Pr Ve třídě 1.A je 16 děvčat: Údaje o výšce udává následující tabulka:
$$ \begin{array}{|c||c|c|c|c|c|}\hline
	\text{Výška} & 160-164 & 165-169 & 170-174 & 175-179 & 180-184 \\\hline
	\text{Střed intervalu} & 162 & 167 & 72 & 177 & 182 \\\hline
	\text{Počet} & 2 & 5 & 4 & 3 & 2\\\hline
\end{array} $$
Průměrná výška: $171\j{cm}$\\
Modus: $167\j{cm}$\\
Medián: $\f{172+172}2=172\j{cm}$

\Pr  V testu při zkoušce dostalo 15 studentů znaámku 1, dalších 35 2, 30 3, 15 4 a zbylých 5 studentů dostalo známku 5.\\
Vypočítejte průmernou známku, modus a medián.\\
Průměr: $2.6$\\
Medián: $2.5$\\
Modus:  $2$

\Pr  Vypočítejte pro $n=4,8,12,16$, že v právě $\f 1 2$ hodů padne pana.\\
Celkem možností hodů: $2^n$
Z toho padne pana polovině právě v ${\f n 2\choose n}$ případech.
Celková pravdépodobnost tedy je:
$$\f{n!}{2^n\*\f n 2 !\*\f n 2!}$$
Číselně:\\
$n=4 \Imp 0.375$\\
$n=8 \Imp 0.2734375$\\
$n=12 \Imp 0.2255859375$\\
$n=16 \Imp 0.196380615234375$

\Pr Hážeme 5 kostkami, určete pravděpodobnost toho, že na 3 z nich padnou stejná čísla:

Počet všech možností: $6^5$\\
Šance, že padne $x$ na alespoň 3 kostkách: ${3\choose 5} \* 5^2+ {4\choose 5} \* 5 + {5\choose 5} = 10 + 5 + 1$
\dots

\Pr Co je pravdépodobnéjší? Nad rovnoceným partnerem zvýtězím v $3$ z $4$ nebo z $5$ z 8:

$3$ z 4: $\f{{3 \choose 4}}{2^4}$

Z města $X$ do $Y$ se lze dostat 2 cestami dle mapy:
Jaká je pravděpodobnost, že dojede do $Y$:

$\f 1 2 \* \f 1 2 + \f 1 2 + \f 1 3 = \ 5 12$

\Pr Maturant chce pokračovat na PřF a vybere si jeden z oborů M,F,Ch,Bi s pravděpodobností dle konzultanta 0.4, 0.3, 0.2, 0.1.
Konzultant neví, co si zvolil, ale dozěděl se, že má výbornou.
Pravděpodobnosti na výbornou jsou 0.1, 0.2, 0.3, 0.4.
Vyčíslete pavděpodobnosti:

Pravděpodobnosti na výběr a zároveň 1 jsou: 0.04, 0.06, 0.06, 0.04.
Celkem tedy 0.128. Doplním na celek: 0.3125, 0,46875, 0.46875, 0.3125.

\Pr  Při vyšetřování pacienta je podezdření na 3 navzájem se vylučující choroby s pravděpodobností 0.3 0.5 0.2. Laboratorní zkouška dává pozitivní u 15\%, 30\%, 20\% nemocných danou nemocí. Jaké jsou pravděpodobnosti nemocí po kladné lab. zkoušce?

\Pr $2\times$ kostkou. Jaká je pravděpodobnost, že padle alespoň 10 ok, za předpokladu, že (právě při 1 z hodů, alespoň při 1 z hodů, při 1. hodu) padne 6.

Právě při jednom hodu: $\f 2 5$\\
Alespoň při jednom hodu: $\f 5 {11}$\\
Při prvním hodu: $\f 3 6 = \f 1 2$


\EndDoc
