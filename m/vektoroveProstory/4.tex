\providecommand{\HINCLUDE}{NE}
\if ^\HINCLUDE^
\else
\def\HINCLUDE{}
\global\newdimen\Okraje
\global\Okraje =4cm
\input{$HOME/souteze/_hlavicka/h-.tex}

%\definecolor{colorV}{RGB}{255,127,0}
%\definecolor{colorPoz}{RGB}{153,51,0}
%\definecolor{orangeV}{RGB}{255,127,0}
%\definecolor{colorPr}{RGB}{0,5,255}
%\definecolor{colorDef}{RGB}{0.255,0}

\usepackage[shortlabels]{enumitem}
\setlength{\marginparsep}{2pt}
\setlength{\marginparwidth}{35pt}

\def\st{{\rm st}}
\def\P{{\rm P}}

\def\ISENUM{}
\def\inMargin#1{\End
		
		\hskip0pt \marginpar{{{#1}}}}
\newcounter{V}[section] 
\newcommand{\V}[1][]{\stepcounter{V}\inMargin{\textcolor{green}{V.\arabic{section}.\theV.:}}\ifx^#1^\else\textcolor{green}{\underline{{#1}:}}\addcontentsline{toc}{subsubsection}{V.\arabic{section}.\theV.:$\quad$ {#1}}\\\fi}
\def\Def{\inMargin{\textcolor{red}{Def:}}}
\def\Poz{{\inMargin{\textcolor{brown}{Pozn:}}}}
\def\Pr{{\inMargin{\textcolor{blue}{Př:}}}}
\def\Pozenum
{
	\begin{enumerate}[1)]%, left = 0pt ]
		\item\inMargin{\textcolor{brown}{Pozn:}}\def\ISENUM{a}}
\def\End
{
	\if	^\ISENUM^
	\else \end{enumerate}
	\fi
	\def\ISENUM{}
}
\reversemarginpar

\makeatletter
\renewcommand\thesection{§\arabic{section}.}
\renewcommand\thesubsection{\Alph{subsection})}
\renewcommand\thesubsubsection{\alph{subsubsection})}
\newcounter{chapter}
\setcounter{chapter}{0}
\renewcommand\thechapter{\Alph{chapter})}
\newcounter{roman}
\setcounter{roman}{0}
\renewcommand\theroman{\Roman{roman}.}
\makeatother
\def\sectionnum#1{\setcounter{section}{#1}\addtocounter{section}{-1}}
\def\subsectionnum#1{\setcounter{subsection}{#1}\addtocounter{subsection}{-1}}
\def\subsubsectionnum#1{\setcounter{subsubsection}{#1}\addtocounter{subsubsection}{-1}}
\def\chapternum#1{\setcounter{chapter}{#1}\addtocounter{chapter}{-1}}
\def\chapter#1{

	\addtocounter{chapter}{1}\sectionnum{1}
	\addcontentsline{toc}{section}{\large{\thechapter$\quad${#1}}}
	
	{\LARGE  \textbf{\begin{minipage}[t]{0.1\textwidth}\thechapter\end{minipage}\begin{minipage}[t]{0.95\textwidth}#1\end{minipage}}}

}
\def\ROM{}
\def\Rom#1#2{\setcounter{roman}{#1}\renewcommand\ROM{#2}}

\Rom{6}{Funkce}
\title{\Huge\textbf{\theroman\quad \ROM}}
\author{Jiří Kalvoda}

\newcounter{countOfBegin}
\setcounter{countOfBegin}{0}
\newcommand{\BeginDoc}[1][]
{
	\ifnum\value{countOfBegin}=0
	\begin{document}
		#1
		\fi
	\addtocounter{countOfBegin}{1}
		
}
\def\EndDoc
{
	\addtocounter{countOfBegin}{-1}
	\ifnum\value{countOfBegin}=0
	\end{document}
	\fi
}

\fi
\let\braceru=\relax \let\bracelu=\relax 
\def\o#1{\setbox0=
	\hbox{$\kern2pt\overbrace{\kern-2pt#1\kern-2pt}\kern2pt$}\ht0=2.1ex\box0}
\def\to#1{\hbox{#1\rlap{\t{}}}}
\def\rad{\rm{rad}}
\def\f{\frac}
\BeginDoc{}
\section{Podprostory vektorového prostoru}
\Def Nechť $V$ je vektorový prostor $\ppri{u_1},\ppri{u_2},\dots,\ppri{u_k} \in V$ vektory,
$p_1,p_2,\dots,p_k \in \R$.
Vektor $\ppri x = \sum_{i-1}^k p_i\ppri{u_i}$ nazýváme lineární kombinací vektorů.
Reálná čísla $p_i$ nazyváme \emph{koeficienty lineárni kombinace}.

Lineární kombinaci, kde $\forall i: p_i = 0$, tedy $\ppri x= \ppri 0$ nazyváme triviální lineárni kombinací. 

\Def Podmnožinu $W$ vektorového prostoru $V$ nazýváme \emph{podprostorem vektorového prosotoru} $V$ právě tehdy, když $W$ je vektorovým prostorem vzhledem k operacím sčítání a vnějšího násobení definovaným ve $V$.
\V ňeprázdná množina $W$ je podprostorem vektorového prostoru $V$ právě tehdy, když platí:
\begin{itemize}
	\item $\forall \ppri u,\ppri v \in W : \ppri u+\ppri v \in W$
	\item $\forall p \in \R, \forall \ppri u \in W : p\*\ppri u \in W$
\end{itemize}
[Dk:
\begin{itemize}
	\item[\uv{$\imp$}] Z definice.
	\item[\uv{$\rimp$}] kommutativita a asociativita plyne z komutativity a asociativity ve $V$, platí $\ppri u \in W \imp 0\* \ppri u  = \ppri 0 \in W ^ (-1) \* \ppri u = -\ppri u \in W$.
		Wlastnost 5.-8. z definice vektorového prostoru platí ve $W$, protože platí  ve $V$. 
\end{itemize}
]
\Pr Nechť $\R^{(2)}$ je aritmetrický prostor. Rozhodnéte, zda nálsledujíci množiny jsou podprostory $\R^{(2)}$
\begin{enumerate}
	\item $ S = \{(x,0);x\in\R\}$:\\
		$S$ je podprostorem.
	\item $ S = \{(x,0);x\in\R\}$:\\
		Není: $2\*(1,1) = (2,2)\not\in T$
	\item $U = \{(z,z);z\in\R\}$:\\
		Nechť $\ppri u = (u,u); \ppri v = (v,v)$, pak $\ppri u + \ppri v = \(u+v,u+v\) \in U$.
		je podprostorem.

\end{enumerate}
\V Nechť $S$ je podmnožina vektorového prostoru $V$. Pak množina $\left<S\right>$ všech lineárních kombinací vektorů množiny $S$ je podprostorem $V$.

[Dk: Nechť $S=\{\ppri{u_1},\ppri{u_2},\dots,\ppri{u_k}\}$.
Nechť $\ppri x \in \left< S \right> \imp \exists p_1,p_2,\dots,p_k \in \R : x=p_1\*\ppri{u_1}+p_2\*\ppri{u_2} + \cdots+ p_k\*\ppri{u_k}$.
Nechť $\ppri y \in \left< S \right> \imp \exists q_1,q_2,\dots,q_k \in \R : x=q_1\*\ppri{u_1}+\ppri{q_2}\*u_2 + \cdots+ q_k\*\ppri{u_k}$.
Pak $\ppri x + \ppri y = (p_1+q_1)\*\ppri{u_1} + (p_2+q_2)\*\ppri{u_2} + \dots+(p_k+q_k)\*\ppri{u_k} \in \left< S \right>$. 

Nechť $S=\{\ppri{u_1},\ppri{u_2},\dots,\ppri{u_k}\}$.
Nechť $\ppri x \in \left< S \right> \imp \exists p_1,p_2,\dots,p_k \in \R : x=p_1\*\ppri{u_1}+p_2\*\ppri{u_2} + \cdots+ p_k\*\ppri{u_k}$.
Nechť $q \in \R$.
Pak $q\* \ppri x = (p_1\*q)\*\ppri{u_1} + (p_2\*q)\*\ppri{u_2} + \dots+(p_k\*q)\*\ppri{u_k} \in \left< S \right>$.

$\left<S\right>$ je tedy podprostorem $V$]

\Def Nechť $S\subset V$ je podmnožina vektorového prostoru $V$.
Podprostor $\left<S\right>$ všech lineárních kombinací vektorů množiny $S$ nazýváme \emph{podprostorem generovanym množinou} $S$ nebo \emph{lineárním obalem množiny} $S$.
Množinu $S$ nazýváme \emph{systémem generátorů (množinou generátorů) podprostoru} $\left<S\right>$.

\Pr V geometrickém mmodelu vektorového prostoru je dána množina $S=\{\ppri a\}; a\neq 0$. Nalezněte $\left< S \right>$:

Je to přímka.


\EndDoc

