\providecommand{\HINCLUDE}{NE}
\if ^\HINCLUDE^
\else
\def\HINCLUDE{}
\global\newdimen\Okraje
\global\Okraje =4cm
\input{$HOME/souteze/_hlavicka/h-.tex}

%\definecolor{colorV}{RGB}{255,127,0}
%\definecolor{colorPoz}{RGB}{153,51,0}
%\definecolor{orangeV}{RGB}{255,127,0}
%\definecolor{colorPr}{RGB}{0,5,255}
%\definecolor{colorDef}{RGB}{0.255,0}

\usepackage[shortlabels]{enumitem}
\setlength{\marginparsep}{2pt}
\setlength{\marginparwidth}{35pt}

\def\st{{\rm st}}
\def\P{{\rm P}}

\def\ISENUM{}
\def\inMargin#1{\End
		
		\hskip0pt \marginpar{{{#1}}}}
\newcounter{V}[section] 
\newcommand{\V}[1][]{\stepcounter{V}\inMargin{\textcolor{green}{V.\arabic{section}.\theV.:}}\ifx^#1^\else\textcolor{green}{\underline{{#1}:}}\addcontentsline{toc}{subsubsection}{V.\arabic{section}.\theV.:$\quad$ {#1}}\\\fi}
\def\Def{\inMargin{\textcolor{red}{Def:}}}
\def\Poz{{\inMargin{\textcolor{brown}{Pozn:}}}}
\def\Pr{{\inMargin{\textcolor{blue}{Př:}}}}
\def\Pozenum
{
	\begin{enumerate}[1)]%, left = 0pt ]
		\item\inMargin{\textcolor{brown}{Pozn:}}\def\ISENUM{a}}
\def\End
{
	\if	^\ISENUM^
	\else \end{enumerate}
	\fi
	\def\ISENUM{}
}
\reversemarginpar

\makeatletter
\renewcommand\thesection{§\arabic{section}.}
\renewcommand\thesubsection{\Alph{subsection})}
\renewcommand\thesubsubsection{\alph{subsubsection})}
\newcounter{chapter}
\setcounter{chapter}{0}
\renewcommand\thechapter{\Alph{chapter})}
\newcounter{roman}
\setcounter{roman}{0}
\renewcommand\theroman{\Roman{roman}.}
\makeatother
\def\sectionnum#1{\setcounter{section}{#1}\addtocounter{section}{-1}}
\def\subsectionnum#1{\setcounter{subsection}{#1}\addtocounter{subsection}{-1}}
\def\subsubsectionnum#1{\setcounter{subsubsection}{#1}\addtocounter{subsubsection}{-1}}
\def\chapternum#1{\setcounter{chapter}{#1}\addtocounter{chapter}{-1}}
\def\chapter#1{

	\addtocounter{chapter}{1}\sectionnum{1}
	\addcontentsline{toc}{section}{\large{\thechapter$\quad${#1}}}
	
	{\LARGE  \textbf{\begin{minipage}[t]{0.1\textwidth}\thechapter\end{minipage}\begin{minipage}[t]{0.95\textwidth}#1\end{minipage}}}

}
\def\ROM{}
\def\Rom#1#2{\setcounter{roman}{#1}\renewcommand\ROM{#2}}

\Rom{6}{Funkce}
\title{\Huge\textbf{\theroman\quad \ROM}}
\author{Jiří Kalvoda}

\newcounter{countOfBegin}
\setcounter{countOfBegin}{0}
\newcommand{\BeginDoc}[1][]
{
	\ifnum\value{countOfBegin}=0
	\begin{document}
		#1
		\fi
	\addtocounter{countOfBegin}{1}
		
}
\def\EndDoc
{
	\addtocounter{countOfBegin}{-1}
	\ifnum\value{countOfBegin}=0
	\end{document}
	\fi
}

\fi
\let\braceru=\relax \let\bracelu=\relax 
\def\o#1{\setbox0=
	\hbox{$\kern2pt\overbrace{\kern-2pt#1\kern-2pt}\kern2pt$}\ht0=2.1ex\box0}
\def\to#1{\hbox{#1\rlap{\t{}}}}
\def\rad{\rm{rad}}
\def\f{\frac}
\BeginDoc{}
%\section{Základní pojmy}

\Pr Na množině $\R^{(2)} = \{[x,y]\}$ všech uspořádaných dvojic $\R$ čísel definujeme operaci sčítání a vnějšího násobení takto:

$$ \forall \pri{u_1} = [x_1,y_1]; \forall \pri{u_2} = [x_2,y_2] : \pri{u_1}+\pri{u_2} = \[x_1+x_2 , y_1+y_2\] \in \R^{(2)}$$
$$ \forall \pri{u1} = [x_1,y_1]; \forall p \in \R  : p\*\pri{u_1} = \[[p\*x_1 , p\*y_1\] \in \R^{(2)}$$

Dokažte, že takto definovaná struktura je vektorovým prostorem:
\begin{enumerate}
	\item Komutativita: 
		$$ \pri{u_1} + \pri{u_2} = (x_1 + x_2 m, y_1 + y_2) =  (x_2 + x_1 m, y_2 + y_1) = \pri{u_2} + \pri{u_1} $$
	\item Asociativita:
		$$ \pri{u_1} + \( \pri{u_2} + \pri{u_3}  \) = (x_1+(x_2+x_3),y_1+(y_2+y_3)) = (x_1+x_2+x_3,y_1+y_2+y_3)$$ 
		$$ \( \pri{u_1} +  \pri{u_2} \) + \pri{u_3} = ((x_1+x_2)+x_3,(y_1+y_2)+y_3) = (x_1+x_2+x_3,y_1+y_2+y_3)$$ 
	\item Nulový prvek:
		$ \exists \pri o \in V : \pri u + \pri o =  \pri o + \pri u = u:$
		$$ \pri o = (0,0): (x+y)+(0,0) = (x+0,y+0) = (x+y)$$
	\item Roznásobení
	\item Roznásobení

	\item Exstence neutrálního prvku násobení:

\end{enumerate}
\Poz Analogicky můžeme dokázat, že množina všech uspořádanych $n$-tic reálných čísel tvoří vzhledem k analogickým definicím sčítání a vnějšího násobení vektorový prostor.

{\huge\dots}
\Def Nechť $M$ je množina všech orientovaných úseček v $\E_3$ a nechť $\epsilon \subset M\times M$ je relace ekvipolence. Pak třídu množiny M, která přísluší relaci $\epsilon$, nazvemem \emph{volným vektorem}. 

\Def Nechť $\ppri u \subset M$ je libovolná úsečka, $X\in\E_3$ libovolný bod, pak $\forall! Y \in \E_3 : \ppri{AB} \epsilon \ppri{XY}$.

\Def  Nechť $V$ je množina všech volných vektorů v $\E_3$. Pak \emph{na množiné $V$} definujeme operace \emph{sčítání} a \emph{vnější násobení} takto:
\begin{enumerate}
	\item $\forall \ppri u, \ppri v \subset V : \ppri u + \ppri v = \ppri w$, kde $\ppri w = \{\ppri{XY}:\ppri{XY}\epsilon\ppri{PC}\}$, přitom $\ppri{PA}\in\ppri u$ a $\ppri{PB} \in \ppri v$  a $\ppri{PC} = \ppri{PA} + \ppri{BC}$.
\end{enumerate}
\EndDoc

