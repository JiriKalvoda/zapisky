\documentclass{article}
\usepackage[czech]{babel}
\usepackage[utf8]{inputenc}
\author{Jiří Kalvoda}
\date{}
\title{Moravský kras - oblíbená turistická destinace}
\begin{document}
\maketitle
Moravský kras je oblíbená turistická oblast nejen pro zahraniční turisty, ale v dnešní době si čím dál častěji získává mnoho sympatií mezi Čechy. Nabízí totiž nejen opravdu krásné turisticky oblíbené oblasti, ale i nádherná romantická místa, kam chodí jen pár lidí.

Momentálně stojím u zříceniny hradu Blansek. Cestou sem jsem potkal pouze skupinku tří lidí. Je tu nádherný klid. Pozoruji několik dochovaných zdí hradu a užívám si nádherného čistého osvěžujícího vzduchu. Nikdo by ani nepomyslel, že ani ne deset minut chůze ode mne se nachází nejznámější místo Moravského krasu -- Punkevní jeskyně.

Projížďka na lodičkách i pohled z či do Macochy sice vskutku stojí za zhlédnutí, ale musíte se připravit na to, že se potkáte s velkým počtem turistů a nádherné krápníky budete obdivovat společně se skupinkou šedesáti cestovatelů. Pokud plánujete navštívit Punkevní jeskyně, nezapomeňte se předem objednat. Obzvláště v letních měsících bývají prohlídky obsazené a může se stát, že se nikam nepodíváte. Pokud nejste milovníkem davů, doporučuji tuto atrakci vynechat a podívat se někam jinam.

Krásná vycházka se naskýtá na jihovýchodním úpatí Pustého žlebu či průchod východní částí této skalní prolákliny. Obě cesty se dají spojit do jednoho okruhu. Můžete vyrazit z Nových Dvorů či z městysu Sloup. Cestou budete obdivovat krásné skalní útvary a vyhlídky.
Cesta žlebem je rovněž vhodná pro cestu na kole. Jedná se o starou asfaltovou cestu, která se již dnes nepoužívá, protože vede centrem chráněné oblasti. Cestou jsem potkal skupinku turistů. \uv{Dnes jsme vyrazili na delší procházku krasem. Máme to tu moc rádi, protože je tu hezká příroda a přitom jsme zde z Brna hned. Navíc v létě je tu ve žlebu příjemný chládek, protože sem téměř nesvítí slunce,} oznámil mi jeden z nich. Další chodec mi řekl, že je tu na týdenní dovolené a moc se mu tady líbí. Stejně tak jako mně.
\end{document}