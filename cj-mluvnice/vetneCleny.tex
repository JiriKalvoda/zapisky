\documentclass[a4]{article}
\usepackage[czech]{babel}    % české nastavení
\usepackage[utf8]{inputenc}   % pro unicode UTF-10

\title{Větné členy}
\author{Jiří Kalvoda}
\begin{document}
\maketitle
\section{Přísudek}
Navštívil mě bratr žijíci v Londýně.
-- přívlastek těsný\\
Navštívil mě bratr, žijíci v Londýně.
-- přívlastek volný -- je to jediný bratr.

\section{Příslovečné určení}
\begin{itemize}
	\item času -- kdy, odkdy, dokdy
	\item místa -- kde, kam, odkud, kudy
	\item způsobu -- jak, jakým zpusobem
	\item míry -- jak moc, kolik, kolikrát
	\item účelu -- proč, za jakým důvodem
	\item příčiny (důvodu) -- proč, z jakého důvodu/příčiny
	\item podmínky -- za jaké podnmínky
	\item přípustky -- i přes co, i přes jakou okolnost
	\item zřetele -- vzhledm k čemu (Vzhledem k nemoci nepřídu.)
\end{itemize}


\section{Doplněk}
Děvčata tancovala bosa na louce.

Nadřízený (podmět) jmenoval ( dělníka \b{mistrem}.

Naše škola uspořádala na začátku letošního školního roku závody clapců a děvčat v běhu.

Josef Schulz, náš špičkový architekt, obložil stěny Národního divadla mramorem.



\end{document}
