\documentclass{article}
\usepackage[czech]{babel}
\usepackage[utf8]{inputenc}
\author{Jiří Kalvoda}
\date{}
\title{Propast Macocha}
\begin{document}
\maketitle
Propast Macocha je jednou z dominant Moravského krasu.
Nachází se samém centru této chráněné oblasti poblíž města Blanska.
I přesto, že se nejedná o nejhlubší propast v České republice, je Macocha tou nejnavštěvovanější.
Nejhlubší je Hranická propast se svojí hloubkou minimálně 470 metrů.
Její dno je však zatopené.

Macocha pravděpodobně vznikla prolomením klenby jeskynního dómu.
Úlomky a suť z klenby je stále patrná na jejím dně.
První sestup do propasti uskutečnil již v roce 1723 mnichem Lazarem Schopperem.
Propast je 137 metrů hluboká 170 metrů dlouhá a 70 široká.
Na jejím dně se nachází dvě krasová jezírka charakteristická svojí azurově-zelenou barvou.
Do jezírek dovádí vodu systém systém ponorných říček Amatérské jeskyně.
Z propasti voda odtéká jako řeka Punkva zahloubená v Punkevní jeskyni.
Větší z jezírek má hloubku alespoň 50 metrů.
Její tok je ovšem uměle narušen z důvodu možné plavby po její hladině v jeskyni.
I když na dno slunce svítí v omezeném množství, roste tu zelený travnatý porost.
Stěny propasti jsou tvořeny vápencovými skalními stěnami, které částečně zarůstají mechy a lišejníky.
Severní stěna postupně ve své horní části přechází v lesní porost, zatímco jih propasti strmě stoupá až na okolní úroveň.
Na této straně byl vybudován vyhlídkový můstek.
Výhled do propasti je také umožněn z dolního můstku, pro jehož návštěvu je nutné sestoupit svah ze západní strany.
Dno propasti se dá navštívit při prohlídce navazujících punkevních jeskyň.


Název propasti vznikl zkomolením slova macecha. Dle pověsti měla macecha shodit do propasti svého nevlastního syna.
Ten se však zachytil pod hranou. Poté, co o událostech řekl vesničanům, kteří ho zachránili, svrhli macechu do propasti.

\end{document}
