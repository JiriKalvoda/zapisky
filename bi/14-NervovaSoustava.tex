\documentclass[a4]{article}
\usepackage[czech]{babel}    % české nastavení
\usepackage[utf8]{inputenc}   % pro unicode UTF-10

\title{Nervová soustava}
\author{Jiří Kalvoda}
\begin{document}
\maketitle
{\huge\dots}
\section{Mikroglie}
\begin{itemize}
	\item Imunitní funkce
\end{itemize}
\section{Gliové buňky}
\begin{itemize}
	\item zrychlování vzruchu.
\end{itemize}
\section{Nervové řízení}
\begin{itemize}
	\item Umožnují neurony
	\item Reflexní oblouk
\end{itemize}
\section{Nervové řízení člověka}
\begin{itemize}
	\item Ústřední NS
\begin{itemize}
	\item mozek
	\item mícha páteřní
\end{itemize}
\item Obvodové nervy
	\begin{itemize}
		\item dostředivé
		\item Odstředivé
	\end{itemize}
\end{itemize}
\section{prodloužená mícha}
\begin{itemize}
	\item Retikulárni formace --  dýchání, srdeční frakvence, krevní tlak, pohyby trávícího ústrojí
	\item Vůlí neřízené pokyny -- polykání, kýchání, zvracení \dots
\end{itemize}
\section{Varoluv most}
\begin{itemize}
	\item Vzestupná a ssestupná vlákna -- propojení vyššího mozku
\end{itemize}
\section{Střední mozek}
\begin{itemize}
	\item Nejmenší
	\item Skládán se z čtverhrbolí a dvou stonků mozkových (spojují koncový mozek s nižšími oddíly CNS)
	\item Vyživuje mozek.
\end{itemize}
\section{Mozeček}
\begin{itemize}
	\item 2 polokoule
	\item Purkyňovy buňky
		\begin{itemize}
			\item Velké
			\item Velké axony a rozvétvená síť dendritů
		\end{itemize}
\end{itemize}
\section{Mezimozek}
\begin{itemize}
	\item Řízení hormanální soustavy
	\item Hypotalamus, šišinka
\end{itemize}

\section{Podmíněné reflexy}
\begin{itemize}
	\item I. P. Pavlov
	\item První signální soustava -- konkrétni myšlení (pouze vyšší živočichové)
	\item Druhá signální soustava -- abstraktní myšlení (abstrakce, představiviost, \dots)
\end{itemize}

\section{Nemoci NS}
\begin{itemize}
	\item Meningitída
		-- 
	\item Encefalitida
		-- z klíštěte; hlubší
	\item Epolepsie -- moc podmětů
	\item Mozková mrtvice -- ucpe se céva v mozku
	\item Roztroušená skleróza
		-- zmatenost, částečné zapomínaní
	\item Alzheimer
		-- nejsme schopní ukládat nové informace do paměti
	\item Parkinsova nemoc
		-- třes; nedokonalý přenos na sinapsi
	\item Otřes mozku
		-- nebezpečný otok

\end{itemize}


\section{Smysly}
\subsection{Funkce}
\begin{itemize}
	\item Poskytují informace o vnějším a vnitřiním prostředí
	\item Čidlo
		\begin{itemize}
			\item receptory (exteroreceptor, interoreceptory, proprioreceptory)
		\end{itemize}
\end{itemize}

\subsection{Zrak}



\end{document}
